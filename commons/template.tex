\setVersione{0.2.4}

\thispagestyle{empty}
\pagenumbering{gobble}

\begin{center}

  \includegraphics[width=8.5cm]{logo.png}\\
  {\Large GruppOne --- progetto ``Stalker''}\\
  \vspace{1.5cm}

  {\Huge \thetitle}
  \vspace{1.5cm}

  \begin{table}[H]
    \centering

    \begin{tabular}{r|l} % chktex -2
      \ifthenelse{\equal{\versione}{DISABILITATA}}{}{
      \textbf{Versione}                   & \versione{}             \\
      }
      \textbf{Approvazione}               & \responsabile{}         \\
      \textbf{Redazione}                  & \redattori{}            \\
      \textbf{Verifica}                   & \verificatori{}         \\
      \textbf{Uso}                        & \uso{}                  \\
      \textbf{Destinato a}
                                          & GruppOne                \\
                                          & prof.\ Vardanega Tullio \\
                                          & prof.\ Cardin Riccardo  \\
      \ifthenelse{\equal{\uso}{Esterno}}{
                                          & Imola Informatica       \\
      }{}
    \end{tabular}
  \end{table}

  \vfill
  \textbf{Descrizione}\\
  \descrizione{}\\
  \verb|gruppone.swe@gmail.com|
\end{center}

\newpage
\pagestyle{nopage}

\section*{Registro delle modifiche}%
\label{sec:registro_delle_modifiche}

\rowcolors{2}{lightgray}{white!80!lightgray!100}
\renewcommand{\arraystretch}{2} % allarga le righe con dello spazio sotto e sopra
\begin{longtable}[H]{>{\centering\bfseries}m{2cm} >{\centering}m{5cm} >{\centering}m{2.5cm} >{\centering\arraybackslash}m{6.5cm}}
  \caption{Registro delle modifiche}%
  \label{tab:registro_delle_modifiche}                                                                                                       \\
  \rowcolor{darkgray!90!}
  \color{white}{\textbf{Versione}} & \color{white}{\textbf{Nominativo}} & \color{white}{\textbf{Data}} & \color{white}{\textbf{Descrizione}} \\
  \endfirsthead%
  \rowcolor{darkgray!90!}
  \color{white}{\textbf{Versione}} & \color{white}{\textbf{Nominativo}} & \color{white}{\textbf{Data}} & \color{white}{\textbf{Descrizione}} \\
  \endhead%
  \modifiche{}%
\end{longtable}
% section registro_delle_modifiche (end)

\newpage
\thispagestyle{nopage}
\pagenumbering{roman}
\tableofcontents

\elencoFigure{}

\elencoTabelle{}

\newpage

\pagestyle{usual}
\pagenumbering{arabic}

\newcommand{\scopoDelProdottoEGlossario}{
\subsection{Scopo del prodotto}%
\label{sub:scopo_del_prodotto}
L'obiettivo del progetto è sviluppare un'applicazione mobile distribuita, seguendo il modello client/server.
Il client deve essere in grado di segnalare sia l'ingresso che l'uscita dell'utente dai luoghi (in modalità anonima o meno a seconda delle esigenze), i quali sono definiti dalle organizzazioni.
Il server deve fornire la possibilità di raccogliere ed analizzare i dati relativi alle organizzazioni.
In caso di utenti anonimi l'analisi riguarda solo una stima del numero totale di persone presenti in un dato momento.
In caso di utenti autenticati deve inoltre essere possibile effettuare query di monitoraggio specifiche.
In merito all'ottimizzazione della geolocalizzazione, è richiesto un report che esponga le scelte progettuali, le rispettive motivazioni e i test eseguiti per garantire la rilevazione sufficiente precisa della posizione, considerando le limitazioni dello smartphone.

\subsection{Glossario}%
\label{sub:glossario}

L'uso di vocaboli tecnici e facilmente fraintendibili rende necessaria la realizzazione di un glossario che definirà i termini dai significati più travisabili presenti in ogni documento.
Per garantirne l'inequivocabilità, le parole che possono assumere un significato ambiguo sono evidenziate (ad es., \glossario{way of working}) e riportate nel documento \textit{Glossario (versione \versione)} accompagnate da una breve definizione.
}
