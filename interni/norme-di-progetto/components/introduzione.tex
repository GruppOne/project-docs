\documentclass[../norme-di-progetto.tex]{subfiles}

\begin{document}

\subsection{Scopo del documento}%
\label{sub:scopo_del_documento}

Il presente documento si rivolge ai membri di GruppOne.
Si vogliono definire le regole e le norme a cui ogni membro del gruppo deve attenersi al fine di ottenere un efficace way of working nel corso di ogni attività di processo.
Il documento verrà aggiornato nel corso del tempo, pertanto è ancora incompleto.

\scopoDelProdottoEGlossario{}

% \subsection{IDE/Code Editor}%
% \label{sub:ide_code_editor}

% Lasciamo la scelta degli strumenti da utilizzare a ciascun membro del team, a cui viene quindi assegnata la responsabilità di configurarli coerentemente con quanto descritto nel presente documento.

% In ogni caso, lo strumento utilizzato dalla maggioranza dei membri del team, perlomeno nel periodo iniziale, è Visual Studio Code (vscode), quindi la sua configurazione, contenuta nella cartella \verb|/.vscode/|, verrà inclusa nel sistema di versionamento scelto.

% Qualora si rivelasse insufficiente a coprire i nostri bisogni, soprattutto durante le attività di codifica, valuteremo la necessità di passare a un IDE completo.

% Per semplificare eventuali transizioni, ci impegniamo a mantenere le configurazioni degli strumenti utilizzati in file esterni tipicamente contenuti nella radice delle cartelle di progetto e adeguatamente versionati.

\subsection{Riferimenti}%
\label{sub:riferimenti}

Questi sono riferimenti generali a risorse online relative a standard e strumenti che abbiamo deciso di utilizzare.
All'interno del documento verranno dati dei riferimenti puntuali ove necessario.

\subsubsection{Normativi}%
\label{subs:riferimenti/normativi}

\begin{itemize}
  \item Capitolato d'appalto C5: \href{https://www.math.unipd.it/~tullio/IS-1/2019/Progetto/C5.pdf}{https://www.math.unipd.it/\textasciitilde tullio/IS-1/2019/Progetto/C5.pdf}.
  \item Conventional Commits 1.0.0: \href{https://www.conventionalcommits.org/en/v1.0.0/}{https://www.conventionalcommits.org/en/v1.0.0/}.
  \item OpenAPI specification v3.0.2: \href{https://spec.openapis.org/oas/v3.0.2}{https://spec.openapis.org/oas/v3.0.2}.
\end{itemize}

\subsubsection{Informativi}%
\label{subs:riferimenti/informativi}

\begin{itemize}
  \item Corso di Ingegneria del Software, slide gestione di progetto, diapositive da 1 a 35: \linebreak\href{https://www.math.unipd.it/~tullio/IS-1/2019/Dispense/L06.pdf}{https://www.math.unipd.it/\textasciitilde tullio/IS-1/2019/Dispense/L06.pdf}.
  \item Corso di Ingegneria del Software, slide amministrazione di progetto, diapositive da 1 a 36: \href{https://www.math.unipd.it/~tullio/IS-1/2019/Dispense/FC01.pdf}{https://www.math.unipd.it/\textasciitilde tullio/IS-1/2019/Dispense/FC01.pdf}.
  % \item \href{https://www.pearson.it/opera/pearson/0-6424-ingegneria_del_software}{Ingegneria del software - Ian Sommerville - decima edizione}.
  % \item \href{https://www.computer.org/education/bodies-of-knowledge/software-engineering/v3}{Software Engineering Body of Knowledge v3}.
  \item ISO/IEC 12207:1995, descrizioni di tutti i processi istanziati da GruppOne: \href{https://www.math.unipd.it/~tullio/IS-1/2009/Approfondimenti/ISO_12207-1995.pdf}{https://www.math.unipd.it/\textasciitilde tullio/IS-1/2009/Approfondimenti/ISO\_12207-1995.pdf}.
  \item ISO/IEC 12207:2008, sezione sul processo organizzativo di Risk Management: \href{https://www.iso.org/standard/43447.html}{https://www.iso.org/standard/43447.html}.
\end{itemize}

\subsection{Materiali consigliati per l'autoapprendimento}%
\label{sub:materiali_consigliati_per_l_autoapprendimento}

In questa sezione i membri del gruppo possono trovare dei link a pagine generali di aiuto (in inglese) per gli strumenti da noi utilizzati, e a pagine di approfondimento su aspetti tecnici e teorici che ci auguriamo siano rilevanti.

\begin{itemize}
  \item \LaTeX{}
        \begin{itemize}
          \item Utilizzo del package subfiles: \href{https://www.overleaf.com/learn/latex/Multi-file_LaTeX_projects#The_subfiles_package}{https://www.overleaf.com/learn/latex/Multi-file\_LaTeX\_projects\#The\_subfiles\_package}.
          \item Documentazione chk\TeX:\ \href{https://www.nongnu.org/chktex/ChkTeX.pdf}{https://www.nongnu.org/chktex/ChkTeX.pdf}.
        \end{itemize}

  \item Git
        \begin{itemize}
          \item Il libro Pro Git: \href{https://git-scm.com/book/en/v2}{https://git-scm.com/book/en/v2}.
          \item Tutorial interattivo sulle funzionalità di git: \href{https://learngitbranching.js.org/}{https://learngitbranching.js.org/}.
          \item Come utilizzare in maniera sicura git rebase: \href{https://git-scm.com/book/en/v2/Git-Branching-Rebasing}{https://git-scm.com/book/en/v2/Git-Branching-Rebasing}.
          \item GitHub flow: \href{https://guides.github.com/introduction/flow/}{https://guides.github.com/introduction/flow/}.
          \item Trunk Based Development: \href{https://trunkbaseddevelopment.com/}{https://trunkbaseddevelopment.com/}.
          \item Commitizen: \href{https://commitizen.github.io/cz-cli/}{https://commitizen.github.io/cz-cli/}.
          \item Commitlint: \href{https://commitlint.js.org/}{https://commitlint.js.org/}.
        \end{itemize}

  \item Articoli da vari blog
        \begin{itemize}
          \item Un metodo alternativo al branching per l'implementazione graduale e sicura di feature --- Martin Fowler: \href{https://martinfowler.com/articles/feature-toggles.html}{https://martinfowler.com/articles/feature-toggles.html}.
          \item I test di integrazione --- Martin Fowler: \href{https://martinfowler.com/bliki/IntegrationTest.html}{https://martinfowler.com/bliki/IntegrationTest.html}
          \item I Test Double --- Martin Fowler: \href{https://martinfowler.com/articles/mocksArentStubs.html}{https://martinfowler.com/articles/mocksArentStubs.html}.
          \item Modello di maturità di Richardson --- Martin Fowler: \href{https://martinfowler.com/articles/richardsonMaturityModel.html}{https://martinfowler.com/articles/richardsonMaturityModel.html}.
          \item Sulla disciplina dell'Extreme programming --- Martin Fowler: \href{https://martinfowler.com/bliki/BeckDesignRules.html}{https://martinfowler.com/bliki/BeckDesignRules.html}.
          \item The Clean Architecture --- Robert C. Martin (Uncle Bob): \href{https://blog.cleancoder.com/uncle-bob/2012/08/13/the-clean-architecture.html}{https://blog.cleancoder.com/uncle-bob/2012/08/13/the-clean-architecture.html}.
          \item Una discussione sulla corrispondenza tra triple GIVEN/WHEN/THEN, ARRANGE/ACT/ASSERT e automi a stato finito --- Robert C. Martin (Uncle Bob): \href{https://blog.cleancoder.com/uncle-bob/2018/06/06/PickledState.html}{https://blog.cleancoder.com/uncle-bob/2018/06/06/PickledState.html}.
          \item Una discussione sulla differenza tra oggetti e data structures --- Robert C. Martin (Uncle Bob): \href{https://blog.cleancoder.com/uncle-bob/2019/06/16/ObjectsAndDataStructures.html}{https://blog.cleancoder.com/uncle-bob/2019/06/16/ObjectsAndDataStructures.html}.
          \item The Joel Test: \href{https://www.joelonsoftware.com/2000/08/09/the-joel-test-12-steps-to-better-code/}{https://www.joelonsoftware.com/2000/08/09/the-joel-test-12-steps-to-better-code/}.
          \item The 12-factor app (best practice per lo sviluppo di applicazioni): \href{https://12factor.net/}{https://12factor.net/}.
        \end{itemize}
\end{itemize}

% sub:materiali_consigliati_per_l_autoapprendimento (end)

\end{document}
