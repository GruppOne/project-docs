\documentclass[../../../analisi-dei-requisiti.tex]{subfiles}

\begin{document}


\begin{figure}[H]
  \centering
  \begin{plantuml}
  @startuml
  !include ../../commons/style/use-cases.pu

  actor :amministratore: as A1
  actor :owner: as A2


  rectangle {
    together {
      usecase (AUC10.1) as "AUC10.1\nInserimento email gestore"
    }
  }

  :A1: -- AUC10.1
  :A2: -- AUC10.1

  @enduml
  \end{plantuml}
  \caption{AUC10: Eliminazione gestore}%
  \label{fig:AUC10}
\end{figure}

\begin{description}
  \item[Codice:] AUC10;
  \item[Titolo:] Eliminazione gestore;
  \item[Attori primari:] amministratore, owner;
  \item[Precondizione:] il sistema deve rendere disponibile la pagina di eliminazione gestore;
  \item[Postcondizione:] l'utente non è più gestore;
  \item[Scenario principale:]
  \begin{enumerate}
    \item l'amministratore, oppure l'owner, vuole eliminare i privilegi ad un'utente gestore.
  \end{enumerate}
\end{description}

\subsubsection{AUC10.1: Inserimento email gestore}%
\label{subs:AUC10.1}
\begin{description}
  \item[Codice:] AUC10.1;
  \item[Titolo:] Inserimento email gestore;
  \item[Attori primari:] amministratore, owner;
  \item[Precondizione:] il sistema deve rendere disponibile il campo per l'inserimento della mail del gestore da eliminare;
  \item[Postcondizione:] il campo relativo al gestore viene compilato;
  \item[Scenario principale:]
  \begin{enumerate}
    \item l'amministratore, oppure l'owner, inserisce l'email del gestore da eliminare.
  \end{enumerate}
\end{description}

\end{document}
