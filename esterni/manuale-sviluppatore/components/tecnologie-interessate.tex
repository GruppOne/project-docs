\documentclass[../manuale-sviluppatore.tex]{subfiles}

\begin{document}
\subsection{Back-End}%
\label{sub:back-end}
In questa sezione sono descritte le tecnologie scelte dopo una fase di analisi per lo sviluppo del back-end.

\subsubsection{InfluxDB}%
\label{sub:influxDB}
Abbiamo deciso di utilizzare InfluxDB, un Time Series Database open source, viste le sue alte performance e le sue elevate capacità di scrittura e di sostenere carichi di interrogazioni 
importanti. Usiamo questo database per salvare dati momentanei, non persistenti.
% TODO estendere

\subsubsection{MySql}%
\label{sub:mysql}
Abbiamo deciso di utilizzare MySql per il salvataggio di dati persistenti attraverso il modello relazionale, il quale garantisce affidabilità.
% TODO estendere

\subsubsection{Spring}%
\label{sub:spring}
Per sviluppare il back-end abbiamo deciso di utilizzare Spring, un framework concepito per la realizzazione di applicazioni enterprise, che soddisfa i requisiti
di performance, sicurezza e affidabilità richiesti. \\
Spring contiene molte librerie, che ci hanno consentito di risolvere in modo elegante più di qualche requisito.
Unico lato negativo di questo framework è la sua documentazione, che può risultare molto dispersiva.
% TODO estendere

\subsubsection{Open API}%
\label{sub:open_api}
Grazie alla specifica Open API abbiamo definito le interfacce per la comunicazione, andando quindi a semplificare la sincronizzazione tra documentazione e codice sorgente.
Open API è in grado di generare codice, documentazione e test case dato un file di interfaccia.
% TODO estendere


\end{document}
