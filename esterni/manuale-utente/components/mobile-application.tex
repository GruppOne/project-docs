\documentclass[../manuale-utente.tex]{subfiles}

\begin{document}

\subsection{Requisiti}%
\label{sub:requisiti}

\begin{description}
    \item[Sistema operativo:] Android 5.0 o superiore.
    \item[Processore:] ARM32, ARM64, x86, x86\_64.
    \item[Geolocalizzazione:] A-GPS obbligatorio, Glonass e Galileo consigliati.
    \item[Sensori consigliati:] Giroscopio, Accelerometro.
\end{description}

\subsection{Installazione}

Attualmente \emph{Stalker} non è disponibile per il download tramite \glossario{\emph{Play Store}}, ma è disponibile solo tramite installazione manuale.\\
Vi preghiamo di seguire i seguenti passaggi per l'installazione di \emph{Stalker}:
\begin{itemize}
\item Scaricare l'\glossario{\emph{APK}} di \emph{Stalker} dall'apposito link dal browser del vostro smartphone.
\item Accedere ad un \glossario{file browser}.
\item Accedere alla cartella download dello smartphone.
\item Selezionare l'applicazione di \emph{Stalker} appena scaricata.
\item A questo punto si avvierà la procedura di installazione e avrete installato \emph{Stalker} con successo.
\end{itemize}
Nel caso ci fossero problemi durante l'installazione di \emph{Stalker} vi preghiamo di consultare la sezione \emph{Risoluzione dei problemi}.\\
Nel caso non abbiate un file browser nel vostro dispositivo vi preghiamo di procedere installandone uno dal \emph{Play Store}.
\newpage

\subsection{Manuale d'uso}%
\label{sub:manuale_uso_mobile}

\subsubsection{Pagina di accesso}%
\label{sub:pagina_di_accesso}

\begin{figure}[H]
    \centering
    \includegraphics[width=70mm]{img/mobile-app/pagina-di-accesso.jpg}
    \caption{Pagina di accesso}%
    \label{fig:mobile_app_pagina_di_accesso}
\end{figure}

Al primo avvio di Stalker app, l'utente visualizza la pagina di login. 
Per accedere al servizio l'utente dovrà seguire questi tre passaggi:
\begin{itemize}
    \item inserire correttamente l'email.
    \item inserire la password, e possibilmente visualizzarla durante la sua digitazione cliccando sulla figura a fine riga, che deve contenere almeno 8 caratteri alfanumerici e al massimo ne può contenere 32.
    \item selezionare il pulsante \textbf{Login} per inviare le credenziali al server.
\end{itemize} 

Se i dati inseriti sono corretti allora l'utente potrà entrare e cominciare ad essere monitorato!
In caso contrario, se l'utente inserisce delle credenziali non valide, l'accesso a Stalker sarà per forza negato.\\
Le cause possono essere le seguenti:
\begin{itemize}
    \item l'email inserita non rispetta i vincoli imposti dal sistema di Stalker.
    \item la password non rispetta i vincoli imposti dal sistema di Stalker.
    \item l'email e la password inserite non rispettano i vincoli imposti dal sistema di Stalker.
    \item l'email inserita rispetta i vincoli, ma non è presente all'interno del database di Stalker.
    \item la password inserita rispetta i vincoli, ma non è presente all'interno del database di Stalker.
    \item l'email e la password inserite rispettano i vincoli, ma non sono presenti all'interno del database di Stalker.
\end{itemize}

Se l'utente non possiede delle credenziali, ha la possibilità di registrarsi, grazie all'apposito form disponibile dopo aver cliccato su \textit{Not yet registered? Sign up}.
\newpage

\subsubsection{Pagina di registrazione}%
\label{sub:pagina_di_registrazione}

\begin{figure}[H]
    \centering
    \includegraphics[width=70mm]{img/mobile-app/pagina-di-registrazione.jpg}
    \caption{Pagina di registrazione}%
    \label{fig:mobile_app_pagina_di_registrazione}
\end{figure}

Se l'utente non si è mai registrato in Stalker, questo form è l'unico modo per farlo!\\
L'utente per registrarsi deve inserire delle semplici informazioni personali:

\begin{itemize}
    \item inserire il nome.
    \item inserire il cognome.
    \item inserire la data di nascita.
    \item inserire l'email, non precedentemente utilizzata in Stalker.
    \item inserire la password, che deve essere alfanumerica con almeno 8 caratteri e al massimo 32.
    \item premere il pulsante sign up per effettuare la registrazione.
\end{itemize}

Si avvisa inoltre che non è possibile registrarsi più volte con la stessa email.\\
Se questi vincoli non saranno rispettati l'utente non potrà registrarsi.\\
Se l'utente possiede delle credenziali, ha la possibilità di accedere, grazie all'apposito form disponibile dopo aver cliccato su \textit{Already have an account? Login}.
\newpage

\subsubsection{Pagina iniziale}%
\label{sub:pagina_iniziale}

% \begin{figure}[H]
%     \centering
%     \includegraphics{img/mobile-app/pagina-iniziale.png}
%     \caption{Pagina iniziale}%
%     \label{fig:mobile_app_pagina_iniziale}
% \end{figure}

Lorem ipsum

% add other functionalities 
\newpage
\subsection{Risoluzione dei problemi}

\subsubsection{Accesso all'applicazione non disponibile}

Nel caso ci fossero problemi con il collegamento all'app, vi consigliamo di controllare se al momento è attiva una rete internet nel vostro smartphone Android.
Nel caso in cui la rete sia abilitata vi preghiamo di riprovare l’accesso in un secondo momento.
Se il problema dovesse persistere vi preghiamo di segnalarcelo nelle modalità discusse nella sezione \emph{Supporto tecnico}.

\subsubsection{Errore nel reperimento della posizione}
Nel caso il vostro smartphone non riuscisse a reperire la vostra posizione, vi consigliamo di verificare se al momento è attiva la geolocalizzazione nel vostro smartphone.
Se la geolocalizzazione dovesse essere attiva, vi consigliamo di provare a seguire i seguenti passaggi:
\begin{itemize}
    \item Accedere alle impostazioni del vostro smartphone, tramite l'apposita icona oppure tramite la tendina delle notifiche.
    \item Accedere alla pagina \emph{applicazioni} oppure \emph{app e notifiche}. Nel caso non troviate queste pagine dalle impostazioni, vi consigliamo di consultare
        il manuale del vostro smartphone e ricercare la pagina rilevante le applicazioni installate.
    \item Cercare tra le applicazioni in lista \emph{Stalker} e selezionarlo.
    \item Accedere alla sezione \emph{autorizzazioni} o \emph{permessi} e verificare se i permessi riguardante la geolocalizzazione sono attivi.
    \item Nel caso non fossero attivi, permettere la geolocalizzazione a \emph{Stalker}.
\end{itemize}
Nel caso nemmeno i passi indicati siano stati utili per il funzionamento di \emph{Stalker} verificare il corretto funzionamento della geolocalizzazione attraverso
altre applicazioni. \\
Se il problema persiste solamente in \emph{Stalker}  vi preghiamo di segnalarcelo nelle modalità discusse nella sezione \emph{Supporto tecnico}.\\
Se il problema è generale probabilmente avete un problema tecnico e vi preghiamo di contattare il vostro \glossario{\emph{vendor}} al più presto.

\subsubsection{Errore durante l'installazione}
Nel caso \emph{Stalker} non dovesse installarsi nel dispositivo in modo corretto, preghiamo gli utenti di verificare i requisiti, descritti nella sua apposita sezione.\\
Se il problema dovesse persistere anche in caso di requisiti soddisfatti vi preghiamo di segnalarcelo nelle modalità discusse nella sezione \emph{Supporto tecnico}.
\end{document}