\paragraph{Amministratore}
Di seguito sono riportati tutti i casi d'uso che coinvolgono il \glossario{super utente} \glossario{amministratore}.

%inserire UML generale amministratore

\subsubsection{UC1 - Sistema di autenticazione amministratore}

\begin{itemize}
\item Attori primari: amministratore;
\item Descrizione: il possibile amministratore tenta di autenticarsi al sistema;
\item Scenario principale: il possibile amministratore non è ancora autenticato e vuole eseguire il login;
\item Precondizione: il possibile amministratore non è autenticato alla piattaforma;
\item Postcondizione: l'amministratore ha effettuato correttamente il login nel sistema.

\end{itemize}

\subsubsection{UC1.1 - Autenticazione}

\begin{itemize}
\item Attori primari: amministratore;
\item Descrizione: l'amministratore visualizza la pagina di login, dove poter inserire le proprie credenziali. 
\item Scenario principale: ll possibile amministratore accede alla pagina di login, e visualizza tutti i campi da compilare;
\item Estensioni: \\\emph{UC1.3} Visualizzazione messaggio di credenziali errate;
\item Inclusioni: \\\emph{UC1.2} Verifica credenziali;
\item Precondizione: il sistema è raggiungibile e funzionante, il possibile amministratore deve poter visualizzare la pagina di login;
\item Postcondizione: il possibile amministratore ha inserito le possibili credenziali e sta tentando di effettuare il login.Ogni volta che cercherà di effettuare
login sarà verificato che le credenziali inserite siano corrette. In caso contrario verrà visualizzato un messaggio di errore, e l'accesso sarà negato;

\end{itemize}


\subsubsection{UC1.2 - Verifica Credenziali}

\begin{itemize}
\item Attori primari: amministratore;
\item Descrizione: vengono verificate le credenziali immesse dal possibile amministratore;
\item Scenario principale: il possibile amministratore sta tentando di effettuare l'accesso e sta attendendo la verifica delle credenziali immesse;
\item Precondizione: il possibile amministratore ha inviato al server le sue credenziali per tentare il login;
\item Postcondizione: il possibile amministratore deve poter accedere alla sua area riservata, nel caso in cui le credenziali siano corrette. In caso
contrario deve essere visualizzato un messaggio di credenziali sbagliate. (\emph{UC1.3}).

\end{itemize}


\subsubsection{UC1.3 - Visualizzazione credenziali errate}

\begin{itemize}
\item Attori primari: amministratore;
\item Descrizione: viene visualizzato un errore di login;
\item Scenario principale: il possibile amministratore cerca di effettuare il login con delle credenziali sbagliate;
\item Precondizione: il possibile amministratore ha inviato al server le sue credenziali per tentare il login, e le credenziali sono state verificate;
\item Postcondizione: il possibile amministraotre visualizza un messaggio di credenziali sbagliate. (\emph{UC1.3}).

\end{itemize}

\subsubsection{UC2 - Approvazione owner}

\begin{itemize}
\item Attori primari: amministratore;
\item Descrizione: l'amministratore approva il compito di owner ad un gestore di una organizzazione;
\item Scenario principale: deve essere assegnato ad un'organizzazione un owner;
\item Precondizione: l'organizzazione scelta non deve avere nessun Owner approvato;
\item Postcondizione: viene approvata la richiesta, e l'organizzazione avrà un nuovo owner.

\end{itemize}

\subsubsection{UC3 - Approvazione creazione organizzazione}

\begin{itemize}
\item Attori primari: amministratore;
\item Descrizione: l'amministratore approva la richiesta di creazione di una nuova organizzazione;
\item Scenario principale: viene richiesta la creazione di una nuova organizzazione, che vuole utilizzare \glossario{Stalker};
\item Precondizione: l'organizzazione non deve essere già creata dal sistema, devono essere stati specificati i dati necessari per la creazione;
\item Postcondizione: viene approvata la richiesta, e l'organizzazione viene creata.

\end{itemize}

\subsubsection{UC4 - Approvazione modifica organizzazione}

\begin{itemize}
\item Attori primari: amministratore;
\item Descrizione: l'amministratore approva la richiesta di modifica di una organizzazione;
\item Scenario principale: viene richiesta la modifica di una organizzazione, presente in \glossario{Stalker};
\item Precondizione: l'organizzazione deve essere già stata creata dal sistema, devono essere state apportate modifiche;
\item Postcondizione: viene approvata la richiesta, e l'organizzazione viene modificata.

\end{itemize}

\subsubsection{UC5 - Approvazione aggiunta luogo}

\begin{itemize}
\item Attori primari: amministratore;
\item Descrizione: l'amministratore approva la richiesta di aggiunta di un luogo all'interno di un'organizzazione;
\item Scenario principale: viene richiesta l'aggiunta di un luogo all'interno di un'organizzazione;
\item Precondizione: l'organizzazione deve essere già stata creata dal sistema, e deve essere stata fatta la richiesta di aggiunta luogo;
\item Postcondizione: viene approvata la richiesta, e viene aggiunto il luogo all'organizzazione interessata.

\end{itemize}

\subsubsection{UC6 - Approvazione modifica luogo}

\begin{itemize}
\item Attori primari: amministratore;
\item Descrizione: l'amministratore approva la richiesta di una modifica di un luogo all'interno di un'organizzazione;
\item Scenario principale: viene richiesta una modifica di un luogo all'interno di un'organizzazione;
\item Precondizione: l'organizzazione e il luogo devono essere già stati creati dal sistema, e deve essere stata fatta la richiesta di modifica luogo;
\item Postcondizione: viene approvata la richiesta, e viene modificato il luogo dell'organizzazione interessata.

\end{itemize}