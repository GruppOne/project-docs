\documentclass[../../../analisi-dei-requisiti.tex]{subfiles}

\begin{document}

\begin{figure}[H]
  \centering
  \begin{plantuml}
  @startuml
  !include ../../commons/style/use-cases.pu

  actor :utente collegato: as A3


  rectangle {
    together {
      usecase (UUC9) as "UUC9\nVisualizzazione tempo \nnell'organizzazione corrente\n--\nExtension points:\nVisualizzazione errore in\ncaso di richiesta tempo trascorso\nin mancanza di rete"
      usecase (UUC12) as "UUC12\nVisualizzazione errore\nrete mancante"
      note right of (UUC9): implementazione opzionale
    }
  }

  :A3: -- (UUC9)

  (UUC12) .up.> (UUC9) : <<extends>>

  @enduml
  \end{plantuml}
  \caption{UUC9: Visualizzazione tempo trascorso nell'organizzazione corrente}%
  \label{fig:UUC9}
\end{figure}

\begin{description}
  \item[Caso d’uso:] UUC9;
  \item[Titolo:] Visualizzazione tempo trascorso nell'organizzazione corrente;
  \item[Attori primari:] utente autenticato, in particolare utente collegato;
  \item[Precondizione:] l'utente deve poter accedere al pulsante del tempo;
  \item[Postcondizione:] l'utente visualizza il tempo trascorso nell'organizzazione;
  \item[Scenario principale:]
        \begin{enumerate}
          \item l'utente visualizza in tempo reale il tempo trascorso all'interno di un'organizzazione.
        \end{enumerate}
  \item[Estensioni:]
        \begin{enumerate}
          \item se l'utente accede ad una schermata per la visualizzazione del tempo trascorso nell'organizzazione in assenza di rete, si visualizzerà un errore \emph{[UUC12]}.
        \end{enumerate}
\end{description}


\end{document}
