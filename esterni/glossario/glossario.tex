\documentclass{article}

\usepackage[italian]{babel}
\usepackage[margin=2cm, footskip=5mm]{geometry}
% questi package non sono necessari in lualatex; ref https://tex.stackexchange.com/a/413046
% \usepackage[utf8]{inputenc}
% \usepackage[T1]{fontenc}
\usepackage{enumitem}
\usepackage{hyperref}
\usepackage{titlesec}
\usepackage{soulutf8}
\usepackage{contour}
\usepackage{float}
\usepackage{graphicx}
\usepackage{fancyhdr}
\usepackage{longtable}
\usepackage[table]{xcolor}
\usepackage{titling}
\usepackage{lastpage}
\usepackage{ifthen}
\usepackage{calc}
\usepackage{minted}
\usepackage{pgfgantt}
\usepackage{subfiles}

\newlength{\imgwidth}

\newcommand\scalegraphics[1]{%
    \settowidth{\imgwidth}{\includegraphics{#1}}%
    \setlength{\imgwidth}{\minof{\imgwidth}{\textwidth}}%
    \includegraphics[width=\imgwidth]{#1}%
}

% XXX definizione dei percorsi in cui cercare immagini
\graphicspath{ {./}
    {./img/}
}

% esempio di utilizzo: \appendToGraphicspath{./img/} (un comando diverso per ogni path da includere)
% N.B.: ci DEVE essere un forward slash alla fine del path, a indicare che è una cartella.
\makeatletter
\newcommand\appendToGraphicspath[1]{%
  \g@addto@macro\Ginput@path{{#1}}%
}
\makeatother

% setup della sottolineatura
\setuldepth{Flat}
\contourlength{0.8pt}

\newcommand{\uline}[1]{%
  \ul{{\phantom{#1}}}%
  \llap{\contour{white}{#1}}%
}

% setup dei link
\hypersetup{
  colorlinks=true, % set true if you want colored links
  linktoc=all,     % set to all if you want both sections and subsections linked
  linkcolor=black, % choose some color if you want links to stand out
}

% setup di header e footer
\pagestyle{fancy}

\fancyhf{}
\fancyhead[L]{\includegraphics[width=1cm]{logo.png}}
\fancyhead[R]{\thetitle}
\fancyfoot[R]{\thepage\ di~\pageref{LastPage}}

\fancypagestyle{nopage}{%
  \fancyfoot{}%
}

\setlength{\headheight}{1.2cm}

% setup forma \paragraph e \subparagraph
\titleformat{\paragraph}[hang]{\normalfont\normalsize\bfseries}{\theparagraph}{1em}{}
\titleformat{\subparagraph}[hang]{\normalfont\normalsize\bfseries}{\thesubparagraph}{1em}{}

% setup profondità indice di default
\setcounter{secnumdepth}{5}
\setcounter{tocdepth}{5}

% shortcut per i placeholder
\newcommand{\plchold}[1]{\textit{\{#1\}}} % chktex 20

% hook per lo script che genera il glossario
\newcommand{\glossario}[1]{\underline{#1}\textsubscript{g}}

% definizione dei comandi \uso e \stato
\makeatletter
\newcommand{\setUso}[1]{%
  \newcommand{\@uso}{#1}%
}
\newcommand{\uso}{\@uso}

\newcommand{\setStato}[1]{%
  \newcommand{\@stato}{#1}%
}
\newcommand{\stato}{\@stato}

\newcommand{\setVersione}[1]{%
  \newcommand{\@versione}{#1}%
}
\newcommand{\versione}{\@versione}

\newcommand{\setResponsabile}[1]{%
  \newcommand{\@responsabile}{#1}%
}
\newcommand{\responsabile}{\@responsabile}

\newcommand{\setRedattori}[1]{%
  \newcommand{\@redattori}{#1}%
}
\newcommand{\redattori}{\@redattori}

\newcommand{\setVerificatori}[1]{%
  \newcommand{\@verificatori}{#1}%
}
\newcommand{\verificatori}{\@verificatori}

\newcommand{\setDescrizione}[1]{%
  \newcommand{\@descrizione}{#1}%
}
\newcommand{\descrizione}{\@descrizione}

\newcommand{\setModifiche}[1]{%
  \newcommand{\@modifiche}{#1}%
}

\newcommand{\modifiche}{\@modifiche}

\makeatother

% setup delle description
\setlist[description,1]{font=$\bullet$\hspace{1.5mm}, leftmargin=*,labelindent=12.5mm}
\setlist[description,2]{font=$\bullet$\hspace{1.5mm}, leftmargin=*,labelindent=12.5mm}
\appendToGraphicspath{../../commons/img/}

\title{Glossario}

\setVersione{\plchold{versione}}
\setResponsabile{Alessandro Rizzo}
\setRedattori{Alberto Gobbo}
\setVerificatori{
  Riccardo Agatea \\ &
  Luca Ercole
}
\setStato{WIP}
\setUso{Esterno}
\setDescrizione{Questo \glossario{documento} ha lo scopo di definire tutti i termini comparsi nella documentazione che possono causare ambiguità.}
\setModifiche{}

\setcounter{secnumdepth}{0}

\begin{document}

\thispagestyle{empty}
\pagenumbering{gobble}
\begin{center}
	\includegraphics[width=8.5cm]{\commons/img/logo.png}\\
	{\Large GruppOne - progetto "Stalker"}\\
	\vspace{1.5cm}
	{\Huge \thetitle}
	\vspace{1.5cm}
	\begin{table}[H]
		\centering
		\begin{tabular}{r|l}
			\textbf{Versione}&\versione\\
			\textbf{Approvazione}&\responsabile\\
			\textbf{Redazione}&\redattori\\
			\textbf{Verifica}&\verificatori\\
			\textbf{Stato}&\stato\\
			\textbf{Uso}&\uso\\
			\textbf{Destinato a}&Imola Informatica\\
			&GruppOne\\
			&Prof. Tullio Vardanega\\
			&Prof. Riccardo Cardin\\
		\end{tabular}
	\end{table}
	\vspace{3cm}
	\textbf{Descrizione}\\
	\descrizione\\
	\vfill
	\verb|gruppone.swe@gmail.com|
\end{center}
\newpage
\thispagestyle{nopage}
\section*{Registro delle modifiche}
\label{sec:registro_delle_modifiche}
\begin{table}[H]
	\label{tab:registro_delle_modifiche}
	\centering
	\rowcolors{2}{lightgray}{white!80!lightgray!100}
	\begin{longtable}[c]{c c c c l}
		\rowcolor{darkgray!90!}\color{white}{\textbf{Versione}}&\color{white}{\textbf{Data}}&\color{white}{\textbf{Nominativo}}&\color{white}{\textbf{Ruolo}}&\color{white}{\textbf{Descrizione}}\\\endhead
	\end{longtable}
\end{table}
% section registro_delle_modifiche (end)
\newpage
\thispagestyle{nopage}
\pagenumbering{roman}
\tableofcontents
\newpage
\pagenumbering{arabic}

	\section{A}
	\begin{description}
		\item[Account] il complesso dei dati identificativi di un utente, che gli consentono l'accesso all'applicazione mobile oppure alla web application.
		\item[Amazon Web Services] piattaforma di Amazon che offre servizi in cloud computing su richiesta.
		\item[Amministratore] utente con privilegi avanzati su tutto il sistema di Stalker, esegue operazioni amministrative e gestisce le richieste che provengono da utenti con permessi limitati.
		\item[Apache Kafka] piattaforma open source per il data streaming che permette di archiviare, pubblicare ed elaborare flussi di dati in tempo reale.
		\item[Applicazione mobile] applicazione software dedicata ai dispositivi di tipo mobile, quali smartphone o tablet. 
		\item[Attività] parte di un processo che indica l'insieme di compiti da realizzare per realizzare il prodotto finale e soddisfarne i requisiti.
		\item[amministratore] toDelete
		\item[applicazione mobile] toDelete
		\item[aree d'interesse] toDelete
		\item[attività] toDelete
	\end{description}
	\section{B}
	\begin{description}
		\item[Back-end] parte di un software con il quale l'utente interagisce indirettamente, ma che risulta essenziale per il funzionamento del sistema.
		\item[Bootstrap] toolkit open source per la creazione e lo sviluppo di siti e application web, che contiene modelli di progettazione basati su HTML, CSS e JavaScript.
	\end{description}
	\section{C}
	\begin{description}
		\item[CSV] acronimo di Comma-Separated Values,è un formato file utilizzato per lo scambio di dati fra fogli di calcolo o database.
		\item[Capitolato] documento tecnico redatto dall'azienda committente, contenente tutte le specifiche tecniche del prodotto software che devono essere rispettate ed eseguite dall'azienda fornitrice per effetto di un contratto d'appalto.
		\item[Caso d'uso] insieme di scenari, ovvero sequenze di azioni, che hanno in comune uno scopo finale per un attore, ovvero un utente. Descrivono l'insieme di funzionalità del sistema percepite dall'utente.
		\item[ClickHouse] DBMS column-oriented open source che consente di generare report di dati analitici in tempo reale.
		\item[Client] componente che accede ai servizi o alle risorse di un'altra componente detta server.
		\item[Copertura] nelle telecomunicazioni, è la zona geografica servita da una rete di telefonia cellulare o stazione radio.
		\item[client] toDelete
		\item[copertura] toDelete
	\end{description}
	\section{D}
	\begin{description}
		\item[Dato] valore privo di significato, che lo assume se associato in determinati contesti significativi.
		\item[Docker] piattaforma software open source che permette di automatizzare la distribuzione di un'applicazione.
		\item[Documento] entità fisica o digitale atta a registrare informazioni significative.
		\item[dati] toDelete
	\end{description}
	\section{E}
	\begin{description}
		\item[ESlint] strumento di analisi statica del codice per identificare modelli problematici trovati nel codice JavaScript.
		\item[EULA] acronimo di End-User License Agreement, è il contratto tra il fornitore di un programma software e l'utente finale. Se l'utente accetta tale contratto, gli verrà assegnata la licenza d'uso del programma nei termini stabiliti dal contratto stesso.
		\item[Ethereum] piattaforma open source globale per applicazioni decentralizzate, nella quale è possibile scrivere codice che controlla il valore digitale.
	\end{description}
	\section{F}
	\begin{description}
		\item[Fase] segmento temporale delimitato da un inizio e una fine, nel quale avvengono delle transazioni coerenti tra stati.
		\item[Fase di incremento] fase nella quale la parte da integrare nel software da aggiungere viene verificata e, solo in caso di accettazione, questa viene integrata nel software, rispettando la logica del modello incrementale.
		\item[Front-end] parte di un software visibile all'utente e con cui egli può interagire tramite un interfaccia utente.
		\item[fasi] toDelete
		\item[fasi di incremento] toDelete
	\end{description}
	\section{G}
	\begin{description}
		\item[GPS] acronimo di Global Positioning System, è un sistema di posizionamento satellitare che permette in ogni istante di localizzare la latitudine e la longitudine di un oggetto, grazie all'utilizzo di satelliti che stazionano nell'orbita terrestre.
		\item[Gestore] utente appartenente ad un'organizzazione abilitato a gestire i luoghi dell'organizzazione. Un singolo utente può essere gestore di più organizzazioni distinte.
		\item[Gherkin] linguaggio che descrive collettivamente i casi d'uso relativi a un sistema software con l'utilizzo di un linguaggio non tecnico.
		\item[Grafana] software open source di analisi dei dati che consente di generare grafici e dashboard per il monitoraggio di ambienti e di sistemi.
		\item[gestore] toDelete
	\end{description}
	\section{H}
	\begin{description}
		\item[Hangouts] web application di Google utilizzata da GruppOne per la messaggistica istantanea e videoconferenza.
	\end{description}
	\section{I}
	\begin{description}
		\item[Imola Informatica] proponente del capitolato per il progetto Stalker.
		\item[Internet of Things] tradotto come Internet delle cose, è l'insieme di tecnologie che permettono di collegare a Internet qualunque tipo di apparato.
	\end{description}
	\section{J}
	\begin{description}
		\item[Javascript] linguaggio di scripting orientato agli oggetti e agli eventi, comunemente utilizzato nella programmazione web lato client per gestire gli effetti dinamici interattivi.
		\item[javascript] toDelete
	\end{description}
	\section{K}
	\begin{description}
		\item[Kubernetes] piattaforma open source per automatizzare la distribuzione, la scalabilità e la gestione di applicazioni containerizzate.
	\end{description}
	\section{L}
	\begin{description}
		\item[LDAP] acronimo di Lightweight Directory Access Protocol, LDAP è un protocollo d'accesso a servizi di directory, basato sul modello client-server, che opera su TCP/IP o su altre connessioni orientate al servizio.
		\item[Link] collegamento ad un'altra pagina.
		\item[Luogo] spazio fisico indentificato da un insieme di coordinate geografiche. Ciascun luogo è riconducibile ad un'organizzazione.
	\end{description}
	\section{M}
	\begin{description}
		\item[Modello a V] modello di sviluppo del software che dimostra la relazione tra ogni fase del ciclo di vita dello sviluppo del software e la sua fase di testing.
		\item[Modello client/server] modello architetturale di sistemi distribuiti, organizzato come un insieme di servizi associati a uno o più server, e di client che accedono e utilizzano tali servizi.
		\item[Modello incrementale] modello di sviluppo del software atto a procedere per incrementi continui del sistema, con la caratteristica che tutto il software sviluppato prima di un incremento è valido. L'incremento deve essere di qualità per attenersi ai requisiti del prodotto finale.
		\item[modello incrementale] toDelete
		\item[modello server/client] toDelete
	\end{description}
	\section{N}
	\begin{description}
		\item[NodeJS] ambiente di runtime JavaScript open source multipiattaforma orientato agli eventi per l'esecuzione di codice JavaScript.
		\item[Notifica] messaggio che viene spedito all'utente direttamente dal server, a causa del verificarsi di un evento indipendente dalle richieste dell'utente.
		\item[Npm] gestore di pacchetti per il linguaggio di programmazione JavaScript.
	\end{description}
	\section{O}
	\begin{description}
		\item[Orange Canvas] software open-source di programmazione visiva che offre un front-end grafico per l'analisi dei dati.
		\item[Organizzazione] soggetto che ha interesse a tracciare le presenze delle persone all'interno dei propri luoghi, in maniera anonima o autenticata.
		\item[Owner] utente proprietario di una o più organizzazioni, ha privilegi di gestione dell'organizzazione su cui opera e dei suoi gestori e visualizzatori. Un singolo utente può essere owner di più organizzazioni distinte.
		\item[organizzazioni] toDelete
		\item[owner] toDelete
	\end{description}
	\section{P}
	\begin{description}
		\item[Periodo] Spazio di tempo caratterizzato da particolari condizioni o avvenimenti.
		\item[Postcondizione] condizione che deve essere sempre vera immediatamente dopo l'uscita dallo scenario di un caso d'uso specifico.
		\item[PostgreSQL] DBMS (Database Management System) ad oggetti rilasciato con licenza libera.
		\item[Precondizione] condizione che deve essere sempre vera prima di accedere allo scenario di un caso d'uso specifico.
		\item[Prodotto] entità software progettata per essere rilasciata all'utente.
		\item[Progetto] insieme di processi, attività e compiti progettato e svolto dal fornitore, per raggiungere gli obiettivi prestabiliti dai requisiti obbligatori, in un arco temporale fissato nel contratto stipulato con il proponente e con risorse limitate che si esauriscono con l'avanzare del tempo.
		\item[Proof of concept] realizzazione incompleta di un determinato progetto, allo scopo di provare il funzionamento base dell'applicativo software e il conseguimento di alcuni requisiti.
		\item[Proponente] azienda che presenta una proposta di contratto.
		\item[Python] linguaggio di programmazione dinamico orientato agli oggetti utilizzabile per molti tipi di sviluppo software. Offre un forte supporto all'integrazione con altri linguaggi e programmi.
		\item[periodi] toDelete
		\item[postcondizioni] toDelete
		\item[precondizioni] toDelete
		\item[progetto] toDelete
		\item[proof of concept] toDelete
	\end{description}
	\section{Q}
	\begin{description}
		\item[Query] interrogazione da parte di un utente di un database, per estrarre o aggiornare i dati che soddisfano un certo criterio di ricerca.
		\item[query] toDelete
	\end{description}
	\section{R}
	\begin{description}
		\item[Report] insieme di dati opportunamente estrapolati e/o elaborati, organizzati sotto forma tabellare.
		\item[Requisiti] necessità del committente, legate al prodotto software da realizzare, che devono essere obbligatoriamente soddisfatte dal fornitore.
		\item[regressione lineare] metodo classico di previsione di un valore numerico, che sottende ad una formula di previsione determinata da una linea retta.
		\item[report] toDelete
	\end{description}
	\section{S}
	\begin{description}
		\item[Sage Maker] servizio offerto da Amazon che permette di sviluppare e formare in modo semplice e rapido modelli di Machine Learning.
		\item[Scalabilità] parametro di qualità, che indica la capacità di un sistema di aumentare o diminuire in modo dinamico le risorse a sua disposizione in funzione delle necessità e disponibilità. Ai fini di questo progetto si parla di scalabilità di carico, ovvero la capacità di un sistema di incrementare le proprie prestazioni, e può essere di due tipi: verticale, che consiste in una scalabilità hardware in cui si aumentano le risorse fisiche a disposizione, e orizzontale, che consiste nell’aggiunta di più istanze dell’applicazione.
		\item[Server] componente o sistema informatico di elaborazione e gestione del traffico di informazioni che fornisce servizi ad altre componenti dette client.
		\item[Server LDAP] server che utilizza il protocollo standard LDAP per l'autenticazione degli utenti ad un'organizzazione, se questa la prevede.
		\item[Serverless Framework] è un framework web gratuito e open source sviluppato per la creazione di applicazioni su AWS Lambda, una piattaforma di elaborazione senza server fornita da Amazon come parte di Amazon Web Services.
		\item[Smartphone] apparecchio elettronico che combina le funzioni di un telefono cellulare e di un computer palmare.
		\item[Software] insieme delle procedure e delle istruzioni in un sistema di elaborazione dati.
		\item[Stalker] nome del progetto proposto dall'azienda Imola Informatica.
		\item[Stima] previsione e valutazione del valore economico e del tempo da impiegare nell'ambito di un progetto.
		\item[Support Vector Machine] modello di apprendimento automatico, mirato a istruire un sistema informatico con algoritmi per la regressione e la classificazione.
		\item[scalabilità] toDelete
		\item[server] toDelete
		\item[smartphone] toDelete
		\item[software] toDelete
		\item[stima] toDelele
	\end{description}
	\section{T}
	\begin{description}
		\item[Telegram] servizio di messaggistica istantanea e broadcasting basato sul cloud.
		\item[TimescaleDB] database open source creato per analizzare dati di serie temporali che utilizza l'SQL standard.
		\item[Typescript] linguaggio di programmazione open source sviluppato da Microsoft che estende la sintassi di JavaScript.
	\end{description}
	\section{U}
	\begin{description}
		\item[UML] acronimo di Unified Modeling Language, rappresenta un linguaggio visuale che si basa su un insieme di regole, vincoli e teorie utilizzate per la modellazione di una classe di problemi. Serve a supportare la descrizione dei sistemi software
		\item[Utente] soggetto che ha interesse ad utilizzare il software prodotto da GruppOne.
		\item[Utente autenticato] utente non autenticato che ha effettuato l'autenticazione alla piattaforma dall'applicazione mobile oppure alla web application.
		\item[Utente collegato] utente autenticato che è collegato ad almeno un'organizzazione;
		\item[Utente collegato incognito] utente collegato ad un'organizzazione in modalità incognito, in quanto la sua identità non è conosciuta.
		\item[Utente collegato noto] utente collegato ad un'organizzazione in modalità noto, in quanto la sua identità è conosciuta.
		\item[Utente non autenticato] utente che non ha ancora effettuato l'autenticazione all'applicazione mobile oppure alla web application.
		\item[Utente non collegato] utente autenticato che non ha effettuato alcun collegamento ad un'organizzazione.
		\item[utente] toDelete
		\item[utente autenticato] toDelete
		\item[utente collegato] toDelete
		\item[utente generico] toDelete
		\item[utente non autenticato] toDelete
		\item[utenti anonimi] toDelete
		\item[utenti autenticati] toDelete
	\end{description}
	\section{V}
	\begin{description}
		\item[Visualizzatore] utente appartenente ad un'organizzazione abilitato esclusivamente a visualizzare i luoghi dell'organizzazione. Un singolo utente può essere visualizzatore di più organizzazioni distinte.
		\item[visualizzatore] toDelete
	\end{description}
	\section{W}
	\begin{description}
		\item[Web application] applicazione distribuita accessibile in una rete Internet per mezzo di un'architettura client/server.
	\end{description}
\end{document}
