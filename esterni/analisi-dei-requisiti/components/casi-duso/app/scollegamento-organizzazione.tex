\documentclass[../../../analisi-dei-requisiti.tex]{subfiles}

\begin{document}

\begin{figure}[H]
  \centering
  \begin{plantuml}
  @startuml
  !include ../../commons/style/use-cases.pu

  actor :utente autenticato: as A3
  actor :utente collegato: as A3.1
  :A3.1: -up-|> :A3:

  rectangle {
    together {
      usecase (UUC7) as "UUC7\nScollegamento organizzazione\n--\nExtension points:\nVisualizzazione errore in caso di\nscollegamento dall'organizzazione\n con rete mancante"
      usecase (UUC5) as "UUC5\nVisualizzazione errore\nrete mancante"
    }
  }

  :A3.1: -- (UUC7)

  (UUC5) .up.|> (UUC7) : <<extends>>

  @enduml
    \end{plantuml}
  \caption{UUC7: Scollegamento organizzazione}
  \label{fig:uuc7}
\end{figure}

\begin{description}
  \item[Caso d’uso:] UUC7
  \item[Titolo:] Scollegamento organizzazione
  \item[Attori primari:] utente autenticato, in particolare utente collegato
  \item[Precondizione:] l'utente si trova sulla schermata specifica dell'organizzazione a cui è collegato.
  \item[Postcondizione:] l'utente non è più collegato all'organizzazione.
  \item[Scenario principale:]
        \begin{enumerate}
          \item l'utente vuole scollegarsi da una organizzazione a cui è collegato, e lo fa dalla schermata dell'organizzazione.
        \end{enumerate}
  \item[Estensioni:]
        \begin{enumerate}
          \item se l'utente cerca di scollegarsi da un'organizzazione in caso di rete mancante, si visualizzerà un errore \emph{[UUC5]}.
        \end{enumerate}
\end{description}


\end{document}
