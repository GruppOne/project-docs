\documentclass{article}

\usepackage[italian]{babel}
\usepackage[margin=2cm, footskip=5mm]{geometry}
% questi package non sono necessari in lualatex; ref https://tex.stackexchange.com/a/413046
% \usepackage[utf8]{inputenc}
% \usepackage[T1]{fontenc}
\usepackage{enumitem}
\usepackage{hyperref}
\usepackage{titlesec}
\usepackage{soulutf8}
\usepackage{contour}
\usepackage{float}
\usepackage{graphicx}
\usepackage{fancyhdr}
\usepackage{longtable}
\usepackage[table]{xcolor}
\usepackage{titling}
\usepackage{lastpage}
\usepackage{ifthen}
\usepackage{calc}
\usepackage{minted}
\usepackage{pgfgantt}
\usepackage{subfiles}

\newlength{\imgwidth}

\newcommand\scalegraphics[1]{%
    \settowidth{\imgwidth}{\includegraphics{#1}}%
    \setlength{\imgwidth}{\minof{\imgwidth}{\textwidth}}%
    \includegraphics[width=\imgwidth]{#1}%
}

% XXX definizione dei percorsi in cui cercare immagini
\graphicspath{ {./}
    {./img/}
}

% esempio di utilizzo: \appendToGraphicspath{./img/} (un comando diverso per ogni path da includere)
% N.B.: ci DEVE essere un forward slash alla fine del path, a indicare che è una cartella.
\makeatletter
\newcommand\appendToGraphicspath[1]{%
  \g@addto@macro\Ginput@path{{#1}}%
}
\makeatother

% setup della sottolineatura
\setuldepth{Flat}
\contourlength{0.8pt}

\newcommand{\uline}[1]{%
  \ul{{\phantom{#1}}}%
  \llap{\contour{white}{#1}}%
}

% setup dei link
\hypersetup{
  colorlinks=true, % set true if you want colored links
  linktoc=all,     % set to all if you want both sections and subsections linked
  linkcolor=black, % choose some color if you want links to stand out
}

% setup di header e footer
\pagestyle{fancy}

\fancyhf{}
\fancyhead[L]{\includegraphics[width=1cm]{logo.png}}
\fancyhead[R]{\thetitle}
\fancyfoot[R]{\thepage\ di~\pageref{LastPage}}

\fancypagestyle{nopage}{%
  \fancyfoot{}%
}

\setlength{\headheight}{1.2cm}

% setup forma \paragraph e \subparagraph
\titleformat{\paragraph}[hang]{\normalfont\normalsize\bfseries}{\theparagraph}{1em}{}
\titleformat{\subparagraph}[hang]{\normalfont\normalsize\bfseries}{\thesubparagraph}{1em}{}

% setup profondità indice di default
\setcounter{secnumdepth}{5}
\setcounter{tocdepth}{5}

% shortcut per i placeholder
\newcommand{\plchold}[1]{\textit{\{#1\}}} % chktex 20

% hook per lo script che genera il glossario
\newcommand{\glossario}[1]{\underline{#1}\textsubscript{g}}

% definizione dei comandi \uso e \stato
\makeatletter
\newcommand{\setUso}[1]{%
  \newcommand{\@uso}{#1}%
}
\newcommand{\uso}{\@uso}

\newcommand{\setStato}[1]{%
  \newcommand{\@stato}{#1}%
}
\newcommand{\stato}{\@stato}

\newcommand{\setVersione}[1]{%
  \newcommand{\@versione}{#1}%
}
\newcommand{\versione}{\@versione}

\newcommand{\setResponsabile}[1]{%
  \newcommand{\@responsabile}{#1}%
}
\newcommand{\responsabile}{\@responsabile}

\newcommand{\setRedattori}[1]{%
  \newcommand{\@redattori}{#1}%
}
\newcommand{\redattori}{\@redattori}

\newcommand{\setVerificatori}[1]{%
  \newcommand{\@verificatori}{#1}%
}
\newcommand{\verificatori}{\@verificatori}

\newcommand{\setDescrizione}[1]{%
  \newcommand{\@descrizione}{#1}%
}
\newcommand{\descrizione}{\@descrizione}

\newcommand{\setModifiche}[1]{%
  \newcommand{\@modifiche}{#1}%
}

\newcommand{\modifiche}{\@modifiche}

\makeatother

% setup delle description
\setlist[description,1]{font=$\bullet$\hspace{1.5mm}, leftmargin=*,labelindent=12.5mm}
\setlist[description,2]{font=$\bullet$\hspace{1.5mm}, leftmargin=*,labelindent=12.5mm}
\appendToGraphicspath{../../commons/img/}

\title{Verbale esterno --- 11/03/2020}

\setResponsabile{Riccardo Agatea}
\setRedattori{Alessandro Rizzo}
\setVerificatori{
  Tobia Apolloni \\ &
  Fabio Scettro
}
\setUso{Esterno}
\setDescrizione{Verbale dell'incontro di GruppOne del 11/03/2020}
\setModifiche{%
\cellcolor{white!80!lightgray!100} & Riccardo Agatea & 2020--03--12 & approva documento \\%
\cellcolor{white!80!lightgray!100} & Verificatori & 2020--03--11 & verifica verbale \\%
\multirow{-3}{*}{0.1.2} \cellcolor{white!80!lightgray!100} & Alessandro Rizzo & 2020--03--11 & stendi verbale %
}

\disabilitaVersione{}
\disabilitaElencoFigure{}
\disabilitaElencoTabelle{}

\begin{document}

\thispagestyle{empty}
\pagenumbering{gobble}
\begin{center}
	\includegraphics[width=8.5cm]{\commons/img/logo.png}\\
	{\Large GruppOne - progetto "Stalker"}\\
	\vspace{1.5cm}
	{\Huge \thetitle}
	\vspace{1.5cm}
	\begin{table}[H]
		\centering
		\begin{tabular}{r|l}
			\textbf{Versione}&\versione\\
			\textbf{Approvazione}&\responsabile\\
			\textbf{Redazione}&\redattori\\
			\textbf{Verifica}&\verificatori\\
			\textbf{Stato}&\stato\\
			\textbf{Uso}&\uso\\
			\textbf{Destinato a}&Imola Informatica\\
			&GruppOne\\
			&Prof. Tullio Vardanega\\
			&Prof. Riccardo Cardin\\
		\end{tabular}
	\end{table}
	\vspace{3cm}
	\textbf{Descrizione}\\
	\descrizione\\
	\vfill
	\verb|gruppone.swe@gmail.com|
\end{center}
\newpage
\thispagestyle{nopage}
\section*{Registro delle modifiche}
\label{sec:registro_delle_modifiche}
\begin{table}[H]
	\label{tab:registro_delle_modifiche}
	\centering
	\rowcolors{2}{lightgray}{white!80!lightgray!100}
	\begin{longtable}[c]{c c c c l}
		\rowcolor{darkgray!90!}\color{white}{\textbf{Versione}}&\color{white}{\textbf{Data}}&\color{white}{\textbf{Nominativo}}&\color{white}{\textbf{Ruolo}}&\color{white}{\textbf{Descrizione}}\\\endhead
	\end{longtable}
\end{table}
% section registro_delle_modifiche (end)
\newpage
\thispagestyle{nopage}
\pagenumbering{roman}
\tableofcontents
\newpage
\pagenumbering{arabic}

\section{Informazioni logistiche}%
\label{sec:informazioni_logistiche}

\begin{description}
  \item [Luogo] Telegram
  \item [Data] 7/03/2020 \symbol{8594} 11/03/2020
  \item [Ora] 11:00 \symbol{8594} 13:00
\end{description}

\subsection{Membri del gruppo presenti}%
\label{sub:membri_del_gruppo_presenti}

\begin{enumerate}
  \item Riccardo Agatea
  \item Tobia Apolloni
  \item Riccardo Cestaro
  \item Alberto Cocco
  \item Luca Ercole
  \item Alberto Gobbo
  \item Alessandro Rizzo
  \item Fabio Scettro
\end{enumerate}

% sub:membri_del_gruppo_presenti (end)

\subsection{Altri partecipanti}%
\label{sub:altri_partecipanti}

\begin{enumerate}
  \item Davide Zanetti (Imola Informatica, proponente del capitolato)
  \item Andrea Sabbioni (Imola Informatica, referente per il server)
\end{enumerate}

% sub:altri_partecipanti (end)
% sec:informazioni_logistiche (end)

\section{Introduzione}%
\label{sec:introduzione}

L'incontro è avvenuto tramite chat su un gruppo Telegram dedicato.

Lo scopo principale era discutere della possibilità per il gruppo di utilizzare l'infrastruttura server di \glossario{Imola Informatica}.

\section{Ordine del giorno}%
\label{sec:ordine_del_giorno}

\begin{itemize}
  \item Conversazione Telegram
\end{itemize}

\section{Conversazione Telegram}%
\label{sec:conversazione_telegram}

Riportiamo per completezza la conversazione avvenuta su Telegram:

Andrea Sabbioni, tecnico di Imola Informatica:
\begin{enumerate}

\item Ciao ragazzi, vi va di spiegarmi un attimo che risorse vi servono e cosa ci andate a fare sopra così cerco di darvi qualche consiglio. Tenete conto che le risorse che vi possiamo dare non hanno garanzie di \glossario{SLA} quindi cerchiamo di pianificare bene le cose
\end{enumerate}

A cui il gruppo ha risposto:

\begin{enumerate}
  \item Ciao! Siamo ancora agli inizi per cui non farti problemi a dirci che le nostre idee sono insensate.

  Avevamo pensato di utilizzare le vostre risorse come dev environment remoto per il nostro server \glossario{Spring Boot} che espone un'API \glossario{RESTful}. Quindi ci basterebbe fare ssh sul vostro server e:

  git clone —depth=1 repo-nostro-server

  ./mvnw compile jib:build \ldots %chktex 26

  docker run something

  e poi fare richieste HTTP dai nostri pc usando prima \glossario{curl} e più avanti la web app e app che svilupperemo, in modo da verificare se le cose funzionano come dovrebbero.
  Una delle prime cose che dobbiamo fare (appena capiamo come farlo) è il meccanismo di autenticazione/autorizzazione dell'API (dato che abbiamo tipologie diverse di utenti che hanno accesso a funzionalità diverse).

  Abbiamo scoperto questo tool qui che dovrebbe semplificarci build e deploy dell'immagine: https://github.com/GoogleContainerTools/jib. Lo conoscevi? È qualcosa che viene utilizzato in pratica?  %chktex 44
  Alternativamente possiamo provare a pubblicare l'immagine o su docker hub o sul GitHub package registry in modo da poter fare direttamente un docker pull

  Infine: tutto questo setup ti sembra ragionevole nell'ottica di (eventualmente) usare Kubernetes per far scalare il server?
\end{enumerate}

Il consiglio del tecnico è stato:

\begin{enumerate}
  \item Ciao scusate il ritardo
  Come primo consiglio visto le precedenti esperienze vi consiglio di poter avere l'intero ciclo dell'applicazione prima in locale
  Ad esempio con un docker compose che simula sia i client che lato server
  \item Questo permette anche di avere un fall back in caso di problemi durante la demo
  \item Per il resto vi invito sulla mail che mi avete dato e vi creo gli account per creare delle VM su di un nostro cluster interno
  \item Ma sul principio di creare l'intera infrastruttura in locale mi sembra una cosa sulla quale dovreste spingere
  \item Anche in un ottica \glossario{Dev ops} e vi facilità poi a approcciare \glossario{Kubernetes}
\end{enumerate}

Il suggerimento del tecnico dunque è stato quello di avere a disposizione l'intera applicazione in locale e di non appoggiarci unicamente a \glossario{Imola Informatica}.

Per questo e per quanto riferitoci nel primo messaggio il gruppo ha deciso di iniziare a configurare una versione locale dell'applicazione con l'utilizzo di \glossario{Docker} prima di accedere al server dell'azienda.

% sec:conversazione_telegram (end)

\newpage
\section{Registro delle decisioni}%
\label{sec:registro_delle_decisioni}

\begin{description}
  \item[20200311-ext-001] Creare una versione locale del ciclo dell'applicazione con docker, da poter poi utilizzare con \glossario{Kubernetes}
\end{description}

\end{document}
