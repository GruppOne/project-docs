\documentclass[../piano-di-progetto.tex]{subfiles}

\begin{document}
Considerate le scadenze scelte in sottosezione~\ref{sub:scadenze_fissate} ed il modello di sviluppo scelto in sezione~\ref{sec:modello_di_sviluppo}, GruppOne ha individuato cinque \glossario{fasi} di sviluppo:
\begin{itemize}
  \item Analisi preliminare
  \item Preparazione in entrata
  \item Progettazione architetturale
  \item Sviluppo e verifica incrementi
  \item Verifica e collaudo finali
\end{itemize}
Ciascuna fase è caratterizzata da delle \glossario{precondizioni} e \glossario{postcondizioni} e divisa in \glossario{periodi}.
Per ciascun periodo sono fissate le \glossario{attività} da svolgervisi.
\subsection[Analisi preliminare]{Analisi preliminare {\normalsize\normalfont\itshape(14/11/2019 \symbol{8594} 14/01/2020)}}%
\label{sub:analisi_preliminare}
Nella fase di analisi preliminare il gruppo si forma (scegliendo, ad esempio, nome e logo) e si prepara alla Revisione dei Requisiti studiando i capitolati proposti e svolgendo l'analisi preliminare dei requisiti e la pianificazione globale del lavoro da svolgere.
I ruoli coinvolti sono:
\begin{itemize}
  \item Responsabile
  \item Amministratore
  \item Analista
  \item Verificatore
\end{itemize}
Le precondizioni sono:
\begin{itemize}
  \item I componenti del gruppo sono stati individuati
  \item I capitolati sono stati presentati
\end{itemize}
Le postcondizioni sono:
\begin{itemize}
  \item Il gruppo ha scelto il capitolato per cui concorrere, raccogliendo le considerazioni su cui si è basato nel documento \textit{Studio di fattibilità}
  \item Il gruppo ha deciso quali standard seguire e quali strumenti utilizzare per la comunicazione e per la produzione dei documenti, e ha raccolto queste nozioni nel documento \textit{Norme di progetto}
  \item Il gruppo ha definito i requisiti del sistema in via preliminare, e li ha raccolti nel documento \textit{Analisi dei requisiti}
  \item Il gruppo ha pianificato il lavoro da svolgere, ed ha esposto la pianificazione nel documento \textit{Piano di progetto} assieme al preventivo delle spese ed al consuntivo delle ore investite.
  \item \plchold{piano di qualifica}
  \item Il gruppo ha raccolto eventuali termini dal significato ambiguo o poco chiaro nel documento \textit{Glossario}, accompagnati dalla rispettiva definizione
\end{itemize}
Questa fase è divisa in quattro periodi.
Per ciascuno si elencano le attività coinvolte e i rispettivi compiti che devono essere svolti.
\subsubsection[Formazione del gruppo]{Formazione del gruppo {\normalsize\normalfont\itshape(14/11/2019 \symbol{8594} 14/01/2020)}}%
\label{subs:formazione_del_gruppo}
\begin{itemize}
  \item Caratterizzazione del gruppo
  \begin{itemize}
    \item Scelta del nome del gruppo
    \item Scelta del logo del gruppo
    \item Definizione dell'identità digitale del gruppo
  \end{itemize}
\end{itemize}
% subs:formazione_del_gruppo (end)
% sub:analisi_preliminare (end)
\subsection[Preparazione in entrata]{Preparazione in entrata {\normalsize\normalfont\itshape(15/01/2020 \symbol{8594} 21/01/2020)}}%
\label{sub:preparazione_in_entrata}

% sub:preparazione_in_entrata (end)
\subsection[Progettazione architetturale]{Progettazione architetturale {\normalsize\normalfont\itshape(22/01/2020 \symbol{8594} DD/MM/2020)}}%
\label{sub:progettazione_architetturale}

% sub:progettazione_architetturale (end)
\subsection[Sviluppo e verifica incrementi]{Sviluppo e verifica incrementi {\normalsize\normalfont\itshape(DD/MM/2020 \symbol{8594} DD/MM/2020)}}%
\label{sub:sviluppo_e_verifica_incrementi}

% sub:sviluppo_e_verifica_incrementi (end)
\subsection[Verifica e collaudo finali]{Verifica e collaudo finali {\normalsize\normalfont\itshape(DD/MM/2020 \symbol{8594} DD/MM/2020)}}%
\label{sub:verifica_e_collaudo_finali}

% sub:verifica_e_collaudo_finali (end)
\end{document}
