\documentclass{article}

\usepackage[italian]{babel}
\usepackage[margin=2cm, footskip=5mm]{geometry}
% questi package non sono necessari in lualatex; ref https://tex.stackexchange.com/a/413046
% \usepackage[utf8]{inputenc}
% \usepackage[T1]{fontenc}
\usepackage{enumitem}
\usepackage{hyperref}
\usepackage{titlesec}
\usepackage{soulutf8}
\usepackage{contour}
\usepackage{float}
\usepackage{graphicx}
\usepackage{fancyhdr}
\usepackage{longtable}
\usepackage[table]{xcolor}
\usepackage{titling}
\usepackage{lastpage}
\usepackage{ifthen}
\usepackage{calc}
\usepackage{minted}
\usepackage{pgfgantt}
\usepackage{subfiles}

\newlength{\imgwidth}

\newcommand\scalegraphics[1]{%
    \settowidth{\imgwidth}{\includegraphics{#1}}%
    \setlength{\imgwidth}{\minof{\imgwidth}{\textwidth}}%
    \includegraphics[width=\imgwidth]{#1}%
}

% XXX definizione dei percorsi in cui cercare immagini
\graphicspath{ {./}
    {./img/}
}

% esempio di utilizzo: \appendToGraphicspath{./img/} (un comando diverso per ogni path da includere)
% N.B.: ci DEVE essere un forward slash alla fine del path, a indicare che è una cartella.
\makeatletter
\newcommand\appendToGraphicspath[1]{%
  \g@addto@macro\Ginput@path{{#1}}%
}
\makeatother

% setup della sottolineatura
\setuldepth{Flat}
\contourlength{0.8pt}

\newcommand{\uline}[1]{%
  \ul{{\phantom{#1}}}%
  \llap{\contour{white}{#1}}%
}

% setup dei link
\hypersetup{
  colorlinks=true, % set true if you want colored links
  linktoc=all,     % set to all if you want both sections and subsections linked
  linkcolor=black, % choose some color if you want links to stand out
}

% setup di header e footer
\pagestyle{fancy}

\fancyhf{}
\fancyhead[L]{\includegraphics[width=1cm]{logo.png}}
\fancyhead[R]{\thetitle}
\fancyfoot[R]{\thepage\ di~\pageref{LastPage}}

\fancypagestyle{nopage}{%
  \fancyfoot{}%
}

\setlength{\headheight}{1.2cm}

% setup forma \paragraph e \subparagraph
\titleformat{\paragraph}[hang]{\normalfont\normalsize\bfseries}{\theparagraph}{1em}{}
\titleformat{\subparagraph}[hang]{\normalfont\normalsize\bfseries}{\thesubparagraph}{1em}{}

% setup profondità indice di default
\setcounter{secnumdepth}{5}
\setcounter{tocdepth}{5}

% shortcut per i placeholder
\newcommand{\plchold}[1]{\textit{\{#1\}}} % chktex 20

% hook per lo script che genera il glossario
\newcommand{\glossario}[1]{\underline{#1}\textsubscript{g}}

% definizione dei comandi \uso e \stato
\makeatletter
\newcommand{\setUso}[1]{%
  \newcommand{\@uso}{#1}%
}
\newcommand{\uso}{\@uso}

\newcommand{\setStato}[1]{%
  \newcommand{\@stato}{#1}%
}
\newcommand{\stato}{\@stato}

\newcommand{\setVersione}[1]{%
  \newcommand{\@versione}{#1}%
}
\newcommand{\versione}{\@versione}

\newcommand{\setResponsabile}[1]{%
  \newcommand{\@responsabile}{#1}%
}
\newcommand{\responsabile}{\@responsabile}

\newcommand{\setRedattori}[1]{%
  \newcommand{\@redattori}{#1}%
}
\newcommand{\redattori}{\@redattori}

\newcommand{\setVerificatori}[1]{%
  \newcommand{\@verificatori}{#1}%
}
\newcommand{\verificatori}{\@verificatori}

\newcommand{\setDescrizione}[1]{%
  \newcommand{\@descrizione}{#1}%
}
\newcommand{\descrizione}{\@descrizione}

\newcommand{\setModifiche}[1]{%
  \newcommand{\@modifiche}{#1}%
}

\newcommand{\modifiche}{\@modifiche}

\makeatother

% setup delle description
\setlist[description,1]{font=$\bullet$\hspace{1.5mm}, leftmargin=*,labelindent=12.5mm}
\setlist[description,2]{font=$\bullet$\hspace{1.5mm}, leftmargin=*,labelindent=12.5mm}
\appendToGraphicspath{../../commons/img/}

\title{Verbale esterno --- 09/03/2020}

\setResponsabile{Riccardo Agatea}
\setRedattori{Alberto Cocco}
\setVerificatori{Tobia Apolloni}
\setUso{Esterno}
\setDescrizione{Verbale dell'incontro di GruppOne del 09/03/2020}
\setModifiche{%
\cellcolor{white!80!lightgray!100} & Riccardo Agatea & 2020--03--11 & approva documento \\%
\cellcolor{white!80!lightgray!100} & Verificatori & 2020--03--10 & verifica verbale \\%
\multirow{-3}{*}{0.1.6} \cellcolor{white!80!lightgray!100} & Alberto Cocco & 2020--03--09 & stendi verbale %
}

\disabilitaVersione{}
\disabilitaElencoFigure{}
\disabilitaElencoTabelle{}

\begin{document}

\thispagestyle{empty}
\pagenumbering{gobble}
\begin{center}
	\includegraphics[width=8.5cm]{\commons/img/logo.png}\\
	{\Large GruppOne - progetto "Stalker"}\\
	\vspace{1.5cm}
	{\Huge \thetitle}
	\vspace{1.5cm}
	\begin{table}[H]
		\centering
		\begin{tabular}{r|l}
			\textbf{Versione}&\versione\\
			\textbf{Approvazione}&\responsabile\\
			\textbf{Redazione}&\redattori\\
			\textbf{Verifica}&\verificatori\\
			\textbf{Stato}&\stato\\
			\textbf{Uso}&\uso\\
			\textbf{Destinato a}&Imola Informatica\\
			&GruppOne\\
			&Prof. Tullio Vardanega\\
			&Prof. Riccardo Cardin\\
		\end{tabular}
	\end{table}
	\vspace{3cm}
	\textbf{Descrizione}\\
	\descrizione\\
	\vfill
	\verb|gruppone.swe@gmail.com|
\end{center}
\newpage
\thispagestyle{nopage}
\section*{Registro delle modifiche}
\label{sec:registro_delle_modifiche}
\begin{table}[H]
	\label{tab:registro_delle_modifiche}
	\centering
	\rowcolors{2}{lightgray}{white!80!lightgray!100}
	\begin{longtable}[c]{c c c c l}
		\rowcolor{darkgray!90!}\color{white}{\textbf{Versione}}&\color{white}{\textbf{Data}}&\color{white}{\textbf{Nominativo}}&\color{white}{\textbf{Ruolo}}&\color{white}{\textbf{Descrizione}}\\\endhead
	\end{longtable}
\end{table}
% section registro_delle_modifiche (end)
\newpage
\thispagestyle{nopage}
\pagenumbering{roman}
\tableofcontents
\newpage
\pagenumbering{arabic}

\section{Informazioni logistiche}%
\label{sec:informazioni_logistiche}

\begin{description}
  \item [Luogo] video-conferenza tramite applicazione Zoom
  \item [Data] 09/03/2020
  \item [Ora] 11:00 \symbol{8594} 11:45
\end{description}

\subsection{Membri del gruppo presenti}%
\label{sub:membri_del_gruppo_presenti}

\begin{enumerate}
  \item Riccardo Agatea
  % \item Tobia Apolloni
  \item Riccardo Cestaro
  \item Alberto Cocco
  \item Luca Ercole
  \item Alberto Gobbo
  \item Alessandro Rizzo
  \item Fabio Scettro
\end{enumerate}

% sub:membri_del_gruppo_presenti (end)

\subsection{Altri partecipanti}%
\label{sub:altri_partecipanti}

\begin{enumerate}
  \item Prof.\ Tullio Vardanega (committente)
\end{enumerate}

% sub:altri_partecipanti (end)

\section{Introduzione}%
\label{sec:introduzione}

L'incontro è avvenuto tramite la piattaforma di videochiamate Zoom.
Lo scopo dell'incontro era di discutere col committente della nostra nuova pianificazione e di porre alcune domande.

\section{Ordine del giorno}%
\label{sec:ordine_del_giorno}

\begin{itemize}
  \item Chiarimenti sulla pianificazione e sullo slittamento della revisione di progettazione.
  \item Incontri col proponente.
  \item Registro delle modifiche.
  \item Chiarimenti sulla progettazione architetturale.
\end{itemize}

\section{Chiarimenti sulla pianificazione e sullo slittamento della revisione di progettazione}%
\label{sec:chiarimenti_sulla_pianificazione_e_sullo_slittamento_della_revisione_di_progettazione}

\textbf{Domanda:} La correzione dei documenti ci ha occupato molto più tempo di quanto avessimo previsto, e abbiamo dovuto posticipare l'inizio della progettazione architetturale.
Di conseguenza, ci era rimasta circa una settimana per la progettazione e la codifica del POC, prima dell'eventuale call con il prof Cardin. Abbiamo considerato la possibilità di accelerare e concentrare il periodo in quella settimana, ma abbiamo velocemente scartato l'idea, e deciso di posticipare la RP ad aprile.
Tutto ciò, unito a quanto emerso dall'ultimo incontro che abbiamo fatto con lei, ci ha portato a modificare drasticamente la pianificazione (ovviamente il preventivo totale è rimasto quello dichiarato post correzione RR, in versione 0.1.1), sfruttando il mese aggiuntivo per riorganizzare il lavoro, aggiungendo esplicitamente la produzione di un POC incrementale, e pianificando (non nel Piano di progetto, ma nella vita reale) il lavoro in settimane, come ci aveva suggerito.
Comunque, nel consuntivo di periodo che consegneremo in RP sarà presente una spiegazione precisa con date di riferimento di quello che le stiamo dicendo ora.

\textbf{Risposta:} Secondo il committente la ragione del nostro ritardo rispetto alla pianificazione non dovrebbe essere la scrittura dei documenti.
Abbiamo ancora una percezione sbagliata dei documenti e non riusciamo a comprenderne a fondo l'importanza.
È necessario che per la prossima revisione i documenti presentino una struttura e dei contenuti di maggiore qualità anche a discapito di una presentazione e di un aspetto di qualità inferiore.
Dobbiamo imparare dagli errori commessi e cercare non solo di pensare ma anche di procedere seguendo la pianificazione che ci siamo imposti di rispettare.\\

Successivamente il Professore ha posto delle domande e ha consultato due volte tutti i componenti del gruppo.
Durante il primo giro di consultazione ci ha chiesto se la nuova pianificazione ha sconvolto i nostri piani di laurea e se abbiamo accettato i cambiamenti all`unanimità' e con serenità.
I membri di GruppOne non hanno espresso alcune perplessità e hanno assicurato che tale modifica alla pianificazione gioverà anche alla qualità del prodotto che intendiamo realizzare.
Nel corso del secondo giro di consultazione, invece, il Professore ci ha chiesto se era possibile prevedere già a Gennaio che ci sarebbero stati dei problemi a rispettare la pianificazione.
La maggior parte dei componenti del gruppo ha espresso come fosse difficile prevedere così in anticipo dei ritardi, dovuti in particolare al sopraggiungere della sessione invernale e ha rassicurato che da ora in avanti si seguirà con impegno quanto previsto nel \textit{Piano di Progetto}.
È stato poi effettuato un terzo giro di consultazione in cui il Professore ha chiesto se l'impossibilità di vedersi fisicamente e di riunirsi in strutture universitarie data l'emergenza sanitaria stia in qualche modo rallentando il progetto. Ogni componente di GruppOne poteva opzionalmente rispondere: i membri del team sono comunque tutti d'accordo sul fatto che sia possibile lavorare abbastanza agevolmente anche in remoto.

\section{Incontri col proponente}%
\label{sec:incontri_col_proponente}

Il professore ci ha invitato a continuare a comunicare con il proponente.
È fondamentale discutere con lui mediante i mezzi che preferiamo per decidere come suddividere gli incrementi.
Non è necessario rendere conto al committente delle nostre discussioni col proponente ma è importante fare in modo che tali interazioni ci siano.

\section{Registro delle modifiche}%
\label{sec:registro_delle_modifiche}

\textbf{Domanda:} Abbiamo ``raggruppato'' i numeri nella colonna a sinistra per rendere esplicito che un set di modifiche contribuisce a una versione specifica. È soddisfacente?\\

\textbf{Risposta:} Il Professore non si è esposto ma ha detto che valuterà alla prossima revisione se le modifiche che abbiamo attuato siano da ritenersi soddisfacenti.

\section{Chiarimenti sulla progettazione architetturale}%
\label{sec:chiarimenti_sulla_progettazione_architetturale}

\textbf{Domanda:} La nostra progettazione quanto deve essere precisa? Tenendo conto del fatto che poi andremo ad implementare man mano nuovi dettagli secondo il modello incrementale. Dovremmo quindi limitarci a pianificare nel dettaglio le componenti del PoC e lasciare il resto per gli incrementi?\\

\textbf{Risposta:} Il committente ci ha assicurato che nella Technology Baseline lui e il Professor Cardin non si aspettano una progettazione già completa, tuttavia valutano la presenza di un impianto progettuale che ammetta dei raffinamenti nei successivi incrementi.
È compito nostro, quindi, comprendere quanto ``alzare l'asticella'' e perciò capire a che grana di dettaglio il nostro \textit{Proof of Concept} deve essere.
Il Professore ci ha inoltre chiesto di dimostrare come le tecnologie che abbiamo scelto valorizzino l'architettura che abbiamo progettato su carta.
Al giorno d'oggi, infatti, le tecnologie stesse impongono l'adozione di un certo tipo di architettura.

\newpage
\section{Registro delle decisioni}%
\label{sec:registro_delle_decisioni}

Non sono state prese decisioni.

\end{document}
