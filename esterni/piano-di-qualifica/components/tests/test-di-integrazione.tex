\documentclass[../../piano-di-qualifica.tex]{subfiles}
\appendToGraphicspath{../../../../commons/img/}

\begin{document}



I test di integrazione verificano la corretta interazione e collaborazione tra un insieme di unità. I test verranno identificati nel seguente modo:
\begin{center}
  TI[codice]
\end{center}
Dove \textit{codice} rappresenta un numero progressivo che identifica il test di integrazione.

\subsubsection{Tabella dei test di integrazione}
\label{subsub:tabella_test_di_integrazione}


\rowcolors{2}{lightgray}{white!80!lightgray!100}
\renewcommand{\arraystretch}{2}
\begin{longtable}[H]{>{\centering\bfseries}m{3cm} >{}m{10cm} >{\centering\arraybackslash}m{3cm}}
  \rowcolor{darkgray!90!}
  \color{white}
  {\textbf{ID test}} & \color{white}{\textbf{Descrizione}}                                    & \color{white}{\textbf{Esito}} \\
  \endhead\rowcolor{white}%
  \multicolumn{3}{r}{\textit{Continua alla pagina seguente}}
  \endfoot%
  \endlastfoot%

  TI1                & Il sistema deve garantire l'integrazione tra API e web application.
                     & NS                                                                                                     \\

  TI2                & Il sistema deve garantire l'integrazione tra API e mobile application.
                     & NS                                                                                                     \\
  \rowcolor{white}
  \caption{Tabella dei test di integrazione}%
  \label{tab:test_integrazione}
\end{longtable}
%sub:test_di_integrazione (end)
\end{document}
