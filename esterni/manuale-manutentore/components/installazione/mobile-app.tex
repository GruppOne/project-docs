\documentclass[../manuale-sviluppatore.tex]{subfiles}

\begin{document}

\subsection{Mobile application}%
\label{sub:mobile_application}


\subsubsection{Installazione IDE}%
\label{subs:installazione_ide}

Per configurare l'ambiente di lavoro dell'applicazione mobile di Stalker, è necessario installare un \glossarioLocale{IDE} come \glossarioLocale{IntelliJ IDEA} o \glossarioLocale{Android Studio}.

Il team ha deciso di utilizzare IntelliJ IDEA Community versione 2019.3.x come IDE di riferimento, ma è possibile utilizzare allo stesso modo Android Studio.

L'utente deve installare \href{https://www.jetbrains.com/toolbox-app/}{JetBrains Toolbox}, e poi selezionare per l'installazione IntelliJ IDEA e, facoltativamente, Android Studio.

\subsubsection{Debug ed esecuzione applicazione}%
\label{subs:debug_ed_esecuzione_applicazione}

Per il \glossarioLocale{debug} dell'applicazione mobile, l'utente deve aprire il codice sorgente e avviare l'esecuzione tramite \glossarioLocale{ADB}.

%TODO aggiungere come eseguire l'applicazione utilizzando l'emulatore oppure lo smartphone collegandolo con cavo USB (e come attivare il Debug USB da Android!! --> attivare Opzioni Sviluppatore)

Per avere un quadro generale delle specifiche operazioni da eseguire su IntelliJ IDEA, visitare il seguente \href{https://www.jetbrains.com/help/idea/running-and-debugging-android-applications.html}{indirizzo}.

\end{document}
