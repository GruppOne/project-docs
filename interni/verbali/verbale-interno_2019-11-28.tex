\documentclass{article}

\usepackage[italian]{babel}
\usepackage[margin=2cm, footskip=5mm]{geometry}
% questi package non sono necessari in lualatex; ref https://tex.stackexchange.com/a/413046
% \usepackage[utf8]{inputenc}
% \usepackage[T1]{fontenc}
\usepackage{enumitem}
\usepackage{hyperref}
\usepackage{titlesec}
\usepackage{soulutf8}
\usepackage{contour}
\usepackage{float}
\usepackage{graphicx}
\usepackage{fancyhdr}
\usepackage{longtable}
\usepackage[table]{xcolor}
\usepackage{titling}
\usepackage{lastpage}
\usepackage{ifthen}
\usepackage{calc}
\usepackage{minted}
\usepackage{pgfgantt}
\usepackage{subfiles}

\newlength{\imgwidth}

\newcommand\scalegraphics[1]{%
    \settowidth{\imgwidth}{\includegraphics{#1}}%
    \setlength{\imgwidth}{\minof{\imgwidth}{\textwidth}}%
    \includegraphics[width=\imgwidth]{#1}%
}

% XXX definizione dei percorsi in cui cercare immagini
\graphicspath{ {./}
    {./img/}
}

% esempio di utilizzo: \appendToGraphicspath{./img/} (un comando diverso per ogni path da includere)
% N.B.: ci DEVE essere un forward slash alla fine del path, a indicare che è una cartella.
\makeatletter
\newcommand\appendToGraphicspath[1]{%
  \g@addto@macro\Ginput@path{{#1}}%
}
\makeatother

% setup della sottolineatura
\setuldepth{Flat}
\contourlength{0.8pt}

\newcommand{\uline}[1]{%
  \ul{{\phantom{#1}}}%
  \llap{\contour{white}{#1}}%
}

% setup dei link
\hypersetup{
  colorlinks=true, % set true if you want colored links
  linktoc=all,     % set to all if you want both sections and subsections linked
  linkcolor=black, % choose some color if you want links to stand out
}

% setup di header e footer
\pagestyle{fancy}

\fancyhf{}
\fancyhead[L]{\includegraphics[width=1cm]{logo.png}}
\fancyhead[R]{\thetitle}
\fancyfoot[R]{\thepage\ di~\pageref{LastPage}}

\fancypagestyle{nopage}{%
  \fancyfoot{}%
}

\setlength{\headheight}{1.2cm}

% setup forma \paragraph e \subparagraph
\titleformat{\paragraph}[hang]{\normalfont\normalsize\bfseries}{\theparagraph}{1em}{}
\titleformat{\subparagraph}[hang]{\normalfont\normalsize\bfseries}{\thesubparagraph}{1em}{}

% setup profondità indice di default
\setcounter{secnumdepth}{5}
\setcounter{tocdepth}{5}

% shortcut per i placeholder
\newcommand{\plchold}[1]{\textit{\{#1\}}} % chktex 20

% hook per lo script che genera il glossario
\newcommand{\glossario}[1]{\underline{#1}\textsubscript{g}}

% definizione dei comandi \uso e \stato
\makeatletter
\newcommand{\setUso}[1]{%
  \newcommand{\@uso}{#1}%
}
\newcommand{\uso}{\@uso}

\newcommand{\setStato}[1]{%
  \newcommand{\@stato}{#1}%
}
\newcommand{\stato}{\@stato}

\newcommand{\setVersione}[1]{%
  \newcommand{\@versione}{#1}%
}
\newcommand{\versione}{\@versione}

\newcommand{\setResponsabile}[1]{%
  \newcommand{\@responsabile}{#1}%
}
\newcommand{\responsabile}{\@responsabile}

\newcommand{\setRedattori}[1]{%
  \newcommand{\@redattori}{#1}%
}
\newcommand{\redattori}{\@redattori}

\newcommand{\setVerificatori}[1]{%
  \newcommand{\@verificatori}{#1}%
}
\newcommand{\verificatori}{\@verificatori}

\newcommand{\setDescrizione}[1]{%
  \newcommand{\@descrizione}{#1}%
}
\newcommand{\descrizione}{\@descrizione}

\newcommand{\setModifiche}[1]{%
  \newcommand{\@modifiche}{#1}%
}

\newcommand{\modifiche}{\@modifiche}

\makeatother

% setup delle description
\setlist[description,1]{font=$\bullet$\hspace{1.5mm}, leftmargin=*,labelindent=12.5mm}
\setlist[description,2]{font=$\bullet$\hspace{1.5mm}, leftmargin=*,labelindent=12.5mm}
\appendToGraphicspath{../../commons/img/}

\title{Verbale interno --- 28/11/2019}

\setVersione{0.0.8}
\setResponsabile{Alessandro Rizzo}
\setRedattori{Riccardo Agatea}
\setVerificatori{
  Tobia Apolloni \\ &
  Riccardo Cestaro
}
\setUso{Interno}
\setDescrizione{Verbale dell'incontro di GruppOne del 28/11/2019}
\setModifiche{%
- &Alessandro Rizzo & 2019--11--29 & approva documento \\%
- &Tobia Apolloni, Riccardo Cestaro & 2019--11--29 & verifica verbale \\%
- &Alberto Gobbo & 2019--11--28 & stendi verbale %
}

\begin{document}

\thispagestyle{empty}
\pagenumbering{gobble}
\begin{center}
	\includegraphics[width=8.5cm]{\commons/img/logo.png}\\
	{\Large GruppOne - progetto "Stalker"}\\
	\vspace{1.5cm}
	{\Huge \thetitle}
	\vspace{1.5cm}
	\begin{table}[H]
		\centering
		\begin{tabular}{r|l}
			\textbf{Versione}&\versione\\
			\textbf{Approvazione}&\responsabile\\
			\textbf{Redazione}&\redattori\\
			\textbf{Verifica}&\verificatori\\
			\textbf{Stato}&\stato\\
			\textbf{Uso}&\uso\\
			\textbf{Destinato a}&Imola Informatica\\
			&GruppOne\\
			&Prof. Tullio Vardanega\\
			&Prof. Riccardo Cardin\\
		\end{tabular}
	\end{table}
	\vspace{3cm}
	\textbf{Descrizione}\\
	\descrizione\\
	\vfill
	\verb|gruppone.swe@gmail.com|
\end{center}
\newpage
\thispagestyle{nopage}
\section*{Registro delle modifiche}
\label{sec:registro_delle_modifiche}
\begin{table}[H]
	\label{tab:registro_delle_modifiche}
	\centering
	\rowcolors{2}{lightgray}{white!80!lightgray!100}
	\begin{longtable}[c]{c c c c l}
		\rowcolor{darkgray!90!}\color{white}{\textbf{Versione}}&\color{white}{\textbf{Data}}&\color{white}{\textbf{Nominativo}}&\color{white}{\textbf{Ruolo}}&\color{white}{\textbf{Descrizione}}\\\endhead
	\end{longtable}
\end{table}
% section registro_delle_modifiche (end)
\newpage
\thispagestyle{nopage}
\pagenumbering{roman}
\tableofcontents
\newpage
\pagenumbering{arabic}

\section{Informazioni logistiche}%
\label{sec:informazioni_logistiche}

\begin{description}
  \item [Luogo] Complesso Paolotti, laboratorio P036
  \item [Data] 28/11/2019
  \item [Ora] 14:30 \symbol{8594} 17:30
\end{description}

\subsection{Membri del gruppo presenti}%
\label{sub:membri_del_gruppo_presenti}

\begin{enumerate}
  \item Riccardo Agatea
  % \item Tobia Apolloni
  \item Riccardo Cestaro
  \item Alberto Cocco
  \item Luca Ercole
  \item Alberto Gobbo
  \item Alessandro Rizzo
  \item Fabio Scettro
\end{enumerate}
% sub:membri_del_gruppo_presenti (end)

% sec:informazioni_logistiche (end)

\section{Ordine del giorno}%
\label{sec:ordine_del_giorno}

\begin{itemize}
  \item Ruoli
  \item Configurazione repo
        \begin{itemize}
          \item Template documenti
          \item Scansione date e milestone
          \item Standard messaggi di commit
          \item Versionamento
        \end{itemize}
  \item Studio di fattibilità
\end{itemize}

% section Ordine_del_giorno (end)
\section{Ruoli}%
\label{sec:ruoli}

In prima analisi, abbiamo scelto di dividere i ruoli come segue:

\begin{description}
  \item[Responsabile] Alberto Cocco
  \item[Amministratore] Luca Ercole
  \item[Analista] Alberto Gobbo
  \item[Analista] Riccardo Agatea
  \item[Analista] Fabio Scettro
  \item[Analista] Alessandro Rizzo
  \item[Verificatore] Tobia Apolloni
  \item[Verificatore] Riccardo Cestaro
\end{description}

% section ruoli (end)
\section{Configurazione repo}%
\label{sec:configurazione_repo}

Abbiamo scelto di utilizzare repo separate per documentazione e prodotto, anche considerando che i capitolati verso cui siamo orientati prevedono lo sviluppo, ad esempio, di un server centrale e di un app separata.
La repo di documentazione è divisa nelle cartelle \verb|interni| e \verb|esterni|, per i rispettivi documenti, e ciascuna contiene una sottocartella per ogni singolo documento da produrre.
Ciascuna cartella contiene il file principale del documento ed una sottocartella contenente i file relativi alle sue singole parti.
Abbiamo rimandato la definizione della struttura delle repo di codice alla stesura delle Norme di Progetto.

\subsection{Template documenti}%
\label{sub:template_documenti}

I documenti saranno scritti in LaTeX, usando una struttura a più file separati (come evidenziato dall'organizzazione della repo).
Abbiamo deciso di utilizzare dei ``file di configurazione'' di alto livello e dei file template per uniformare la struttura dei documenti.
Per la stesura del registro delle modifiche abbiamo deciso di utilizzare dei tool automatici, legati ai messaggi di commit (vedi sezione~\ref{sub:standard_messaggi_di_commit}).

% subsection template_documenti (end)
\subsection{Scansione date e milestone}%
\label{sub:scansione_date_e_milestone}

\begin{description}
  \item[da definire] Incontro con ImolaInformatica
  \item[da definire] Incontro con ImolaInformatica (+ \varepsilon)
  \item[20/12/19] Ultimo incontro prima delle vacanze + primo cambio ruoli
  \item[03/01/20] Incontro intermedio
  \item[07/01/20] Completamento documenti per RR
  \item[da definire] Incontro con ImolaInformatica
\end{description}

% subsection scansione_date_e_milestone (end)
\subsection{Standard messaggi di commit}%
\label{sub:standard_messaggi_di_commit}

Abbiamo scelto di scrivere i messaggi di commit seguendo lo standard dato da \href{https://www.conventionalcommits.org/en/v1.0.0/}{conventional commits v1.0.0};
i commit relativi alla repo di documentazione avranno commit message in italiano per facilitare la generazione automatica del registro delle modifiche.

% subsection standard_messaggi_di_commit (end)
\subsection{Versionamento}%
\label{sub:versionamento}
Abbiamo deciso di utilizzare il \href{https://semver.org/}{semantic versioning 2.0.0} per il versionamento, considerandone una versione modificata per il versionamento dei documenti.
Per le specifiche, rimandiamo alle Norme di Progetto, ma l'idea sarà di incrementare MAJOR con le approvazioni, MINOR con le verifiche e PATCH con le modifiche.

% subsection versionamento (end)
% section configurazione_repo (end)
\section{Studio di fattibilità}%
\label{sec:studio_di_fattibilità}

Abbiamo riunito le informazioni raccolte dai singoli componenti del gruppo sui singoli capitolati in un unico file, studio-di-fattibilita.tex
% section studio_di_fattibilità (end)
\newpage
\section{Registro delle decisioni}%
\label{sec:registro_delle_decisioni}
\begin{table}[H]
  \centering
  \rowcolors{2}{lightgray}{white!80!lightgray!100}
  \renewcommand{\arraystretch}{2}
  \begin{tabular}{c b{13cm}}
    \rowcolor{darkgray!90!}\color{white}{\textbf{Codice}} & \color{white}{\textbf{Decisione}}\\
    D003I2019--11--28&Scelta prima suddivisione dei ruoli\\
    D004I2019--11--28&Scelto di separare la repo di documentazione da quella di codice\\
    D005I2019--11--28&Scelta struttura della repo di documentazione\\
    D006I2019--11--28&Scelto standard per i messaggi di commit\\
    D007I2019--11--28&Scelto standard per il numero di versione\\
  \end{tabular}
  \caption{registro delle decisioni}%
~~\label{tab:registro delle decisioni}
\end{table}
% sec:registro_delle_decisioni (end)
\end{document}
