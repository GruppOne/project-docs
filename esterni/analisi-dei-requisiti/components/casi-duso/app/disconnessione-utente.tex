\documentclass[../../../analisi-dei-requisiti.tex]{subfiles}

\begin{document}

\begin{figure}[H]
  \centering
  \begin{plantuml}
  @startuml
  !include ../../commons/style/use-cases.pu

  actor :utente autenticato: as A3

  rectangle {
    together {
      usecase (UUC10) as "UUC10\nDisconnessione utente\n--\nExtension points:\nVisualizzazione errore in\ncaso di disconnessione\nin mancanza di rete"
      usecase (UUC12) as "UUC12\nVisualizzazione errore\nrete mancante"
    }
  }

  :A3: -- (UUC10)

  (UUC12) .up.> (UUC10) : <<extends>>

  @enduml
  \end{plantuml}
  \caption{UUC10: Disconnessione}%
  \label{fig:UUC10}
\end{figure}

  \begin{description}
  \item[Caso d’uso:] UUC10;
  \item[Titolo:] Disconnessione;
  \item[Attori primari:] utente autenticato;
  \item[Precondizione:] l'utente deve poter accedere al pulsante di disconnessione;
  \item[Postcondizione:] l'utente non è più autenticato;
  \item[Scenario principale:]
        \begin{enumerate}
          \item l'utente esegue la disconnessione dal servizio.
        \end{enumerate}
  \item[Estensioni:]
        \begin{enumerate}
          \item se l'utente cerca di disconnettersi in assenza di rete, si visualizzerà un errore \emph{[UUC12]}.
        \end{enumerate}
\end{description}


\end{document}
