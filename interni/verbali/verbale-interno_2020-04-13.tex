\documentclass{article}

\usepackage[italian]{babel}
\usepackage[margin=2cm, footskip=5mm]{geometry}
% questi package non sono necessari in lualatex; ref https://tex.stackexchange.com/a/413046
% \usepackage[utf8]{inputenc}
% \usepackage[T1]{fontenc}
\usepackage{enumitem}
\usepackage{hyperref}
\usepackage{titlesec}
\usepackage{soulutf8}
\usepackage{contour}
\usepackage{float}
\usepackage{graphicx}
\usepackage{fancyhdr}
\usepackage{longtable}
\usepackage[table]{xcolor}
\usepackage{titling}
\usepackage{lastpage}
\usepackage{ifthen}
\usepackage{calc}
\usepackage{minted}
\usepackage{pgfgantt}
\usepackage{subfiles}

\newlength{\imgwidth}

\newcommand\scalegraphics[1]{%
    \settowidth{\imgwidth}{\includegraphics{#1}}%
    \setlength{\imgwidth}{\minof{\imgwidth}{\textwidth}}%
    \includegraphics[width=\imgwidth]{#1}%
}

% XXX definizione dei percorsi in cui cercare immagini
\graphicspath{ {./}
    {./img/}
}

% esempio di utilizzo: \appendToGraphicspath{./img/} (un comando diverso per ogni path da includere)
% N.B.: ci DEVE essere un forward slash alla fine del path, a indicare che è una cartella.
\makeatletter
\newcommand\appendToGraphicspath[1]{%
  \g@addto@macro\Ginput@path{{#1}}%
}
\makeatother

% setup della sottolineatura
\setuldepth{Flat}
\contourlength{0.8pt}

\newcommand{\uline}[1]{%
  \ul{{\phantom{#1}}}%
  \llap{\contour{white}{#1}}%
}

% setup dei link
\hypersetup{
  colorlinks=true, % set true if you want colored links
  linktoc=all,     % set to all if you want both sections and subsections linked
  linkcolor=black, % choose some color if you want links to stand out
}

% setup di header e footer
\pagestyle{fancy}

\fancyhf{}
\fancyhead[L]{\includegraphics[width=1cm]{logo.png}}
\fancyhead[R]{\thetitle}
\fancyfoot[R]{\thepage\ di~\pageref{LastPage}}

\fancypagestyle{nopage}{%
  \fancyfoot{}%
}

\setlength{\headheight}{1.2cm}

% setup forma \paragraph e \subparagraph
\titleformat{\paragraph}[hang]{\normalfont\normalsize\bfseries}{\theparagraph}{1em}{}
\titleformat{\subparagraph}[hang]{\normalfont\normalsize\bfseries}{\thesubparagraph}{1em}{}

% setup profondità indice di default
\setcounter{secnumdepth}{5}
\setcounter{tocdepth}{5}

% shortcut per i placeholder
\newcommand{\plchold}[1]{\textit{\{#1\}}} % chktex 20

% hook per lo script che genera il glossario
\newcommand{\glossario}[1]{\underline{#1}\textsubscript{g}}

% definizione dei comandi \uso e \stato
\makeatletter
\newcommand{\setUso}[1]{%
  \newcommand{\@uso}{#1}%
}
\newcommand{\uso}{\@uso}

\newcommand{\setStato}[1]{%
  \newcommand{\@stato}{#1}%
}
\newcommand{\stato}{\@stato}

\newcommand{\setVersione}[1]{%
  \newcommand{\@versione}{#1}%
}
\newcommand{\versione}{\@versione}

\newcommand{\setResponsabile}[1]{%
  \newcommand{\@responsabile}{#1}%
}
\newcommand{\responsabile}{\@responsabile}

\newcommand{\setRedattori}[1]{%
  \newcommand{\@redattori}{#1}%
}
\newcommand{\redattori}{\@redattori}

\newcommand{\setVerificatori}[1]{%
  \newcommand{\@verificatori}{#1}%
}
\newcommand{\verificatori}{\@verificatori}

\newcommand{\setDescrizione}[1]{%
  \newcommand{\@descrizione}{#1}%
}
\newcommand{\descrizione}{\@descrizione}

\newcommand{\setModifiche}[1]{%
  \newcommand{\@modifiche}{#1}%
}

\newcommand{\modifiche}{\@modifiche}

\makeatother

% setup delle description
\setlist[description,1]{font=$\bullet$\hspace{1.5mm}, leftmargin=*,labelindent=12.5mm}
\setlist[description,2]{font=$\bullet$\hspace{1.5mm}, leftmargin=*,labelindent=12.5mm}

\appendToGraphicspath{../../commons/img/}

\title{Verbale interno --- 13/04/2020}

\setResponsabile{Luca Ercole}
\setRedattori{Alberto Cocco}
\setVerificatori{
  Alberto Gobbo
}
\setUso{Interno}
\setDescrizione{Verbale dell'incontro di GruppOne del 13/04/2020}
\setModifiche{%
\cellcolor{white!80!lightgray!100} & Luca Ercole & 2020--04--14 & approva documento \\%
\cellcolor{white!80!lightgray!100} & Alberto Gobbo & 2020--03--13 & verifica verbale \\%
\multirow{-3}{*}{-} & Alberto Cocco & 2020--04--13 & stendi verbale%
}

\disabilitaVersione{}
\disabilitaElencoFigure{}
\disabilitaElencoTabelle{}

\begin{document}

\thispagestyle{empty}
\pagenumbering{gobble}
\begin{center}
	\includegraphics[width=8.5cm]{\commons/img/logo.png}\\
	{\Large GruppOne - progetto "Stalker"}\\
	\vspace{1.5cm}
	{\Huge \thetitle}
	\vspace{1.5cm}
	\begin{table}[H]
		\centering
		\begin{tabular}{r|l}
			\textbf{Versione}&\versione\\
			\textbf{Approvazione}&\responsabile\\
			\textbf{Redazione}&\redattori\\
			\textbf{Verifica}&\verificatori\\
			\textbf{Stato}&\stato\\
			\textbf{Uso}&\uso\\
			\textbf{Destinato a}&Imola Informatica\\
			&GruppOne\\
			&Prof. Tullio Vardanega\\
			&Prof. Riccardo Cardin\\
		\end{tabular}
	\end{table}
	\vspace{3cm}
	\textbf{Descrizione}\\
	\descrizione\\
	\vfill
	\verb|gruppone.swe@gmail.com|
\end{center}
\newpage
\thispagestyle{nopage}
\section*{Registro delle modifiche}
\label{sec:registro_delle_modifiche}
\begin{table}[H]
	\label{tab:registro_delle_modifiche}
	\centering
	\rowcolors{2}{lightgray}{white!80!lightgray!100}
	\begin{longtable}[c]{c c c c l}
		\rowcolor{darkgray!90!}\color{white}{\textbf{Versione}}&\color{white}{\textbf{Data}}&\color{white}{\textbf{Nominativo}}&\color{white}{\textbf{Ruolo}}&\color{white}{\textbf{Descrizione}}\\\endhead
	\end{longtable}
\end{table}
% section registro_delle_modifiche (end)
\newpage
\thispagestyle{nopage}
\pagenumbering{roman}
\tableofcontents
\newpage
\pagenumbering{arabic}

\section{Informazioni logistiche}%
\label{sec:informazioni_logistiche}

\begin{description}
  \item [Luogo] videochiamata su Google Hangouts
  \item [Data] 13/04/2020
  \item [Ora] 18:30 \symbol{8594} 20:00
\end{description}

\subsection{Membri del gruppo presenti}%
\label{sub:membri_del_gruppo_presenti}

\begin{enumerate}
  \item Riccardo Agatea
  \item Tobia Apolloni
  \item Riccardo Cestaro
  \item Alberto Cocco
  \item Luca Ercole
  \item Alberto Gobbo
  \item Alessandro Rizzo
  \item Fabio Scettro
\end{enumerate}
% sub:membri_del_gruppo_presenti (end)

\section{Introduzione}%
\label{sec:introduzione}

L'incontro si è svolto su Google Hangouts per riorganizzare il lavoro in vista della revisione di progettazione.

\section{Ordine del giorno}%
\label{sec:ordine_del_giorno}
\begin{itemize}
  \item Istituzione di un documento di tracciamento dei requisiti.
  \item Creazione di una kanban board.
\end{itemize}

\section{Documento di tracciamento dei requisiti}%
\label{sec:documento_di_tracciamento_dei_requisiti}

Per avere un controllo maggiore sullo stato di avanzamento del progetto si è deciso di creare un documento per il tracciamento dei requisiti.
Ogni requisito è stato riportato all'interno di un Google docs  con la rispettiva sigla.
Per ogni requisito vengono indicati i componenti coinvolti (mobile application, web application, server) e l'incremento in cui il requisito deve essere soddisfatto.
Ogniqualvolta un requisito è in corso di soddisfacimento lo si deve indicare sulla colonna \textbf{WIP}, mentre se un requisito è soddisfatto lo si indica nella colonna \textbf{soddisfatto}.
Al link \href{https://docs.google.com/spreadsheets/d/1J-RbNrb1yN_X1rVlKzpP9rjaiGst8B62k1K6EHY-grU/edit#gid=0}{Google Docs} è disponibile il foglio Google condiviso tra i membri del gruppo.

\section{Creazione di una kanban board}%
\label{sec:creazione_di_una_kanban_board}

Si è deciso di creare una kanban Board su GitHub per avere controllo delle attività che ogni componente del gruppo sta svolgendo.
La Kanban board contiene tutti gli issue del progetto con le rispettive label. Ogni issue o pull request può trovarsi in uno dei seguenti stati:
\begin{description}
  \item [To do -- Requirements] la issue è marcata come requirement e rappresenta uno dei requisiti definiti nel foglio ore.
  \item [To do] la issue deve ancora essere presa in incarico da qualcuno.
  \item [In progress] la issue è già assegnata a un componente del gruppo che la sta portando a termine.
  \item [Ready review] la pull request è pronta per la revisione da parte dei verificatori.
  \item [Reviewer approved] la pull request è stata approvata dal responsabile.
  \item [Done] l'issue/pull request è stata chiusa.
\end{description}
Ogni componente del gruppo dovrà fare in modo di mantenere la Kanban board aggiornata.


\newpage
\section{Registro delle decisioni}%
\label{sec:registro_delle_decisioni}
\begin{description}
  \item[20200413-int-001] Istituzione di una kanban board per un miglior tracciamento delle attività che ogni componente del gruppo sta svolgendo.
  \item[20200413-int-002] Creazione di un documento per tracciare in maniera precisa i requisiti del capitolato.
\end{description}

% sec:registro_delle_decisioni (end)
\end{document}
