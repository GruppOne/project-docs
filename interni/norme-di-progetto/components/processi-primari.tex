\documentclass[../norme-di-progetto.tex]{subfiles}

\begin{document}

\subsection{Descrizione}%
\label{sub:processi_primari/descrizione}
Sono i processi essenziali durante il ciclo di vita del software. ISO 12207:1995 ne definisce cinque:

\begin{itemize}
  \item Acquisizione
  \item Fornitura
  \item Sviluppo
  \item Gestione operativa
  \item Manutenzione.
\end{itemize}

Di questi, noi istanziamo in maniera significativa solo i processi di fornitura e sviluppo.

\subsection{Fornitura}%
\label{sub:fornitura}

\subsubsection{Finalità}%
\label{subs:fornitura/finalita}

GruppOne istanzia il processo di fornitura per potersi aggiudicare il capitolato attraverso un accordo con il committente che ha valore contrattuale.

\subsubsection{Descrizione}%
\label{subs:fornitura/descrizione}

Il processo di fornitura concerne il rapporto tra il fornitore e il cliente, e l'amministrazione delle procedure e risorse necessarie per lo sviluppo del \textit{Piano di progetto}.

È composto da diverse attività:

\begin{itemize}
  \item Inizializzazione
  \item Preparazione della risposta
  \item Contratto
  \item Pianificazione
  \item Esecuzione e controllo
  \item Revisione
  \item Consegna e completamento.
\end{itemize}

\subsubsection{Attività}%
\label{subs:attivita}

\paragraph{Inizializzazione}%
\label{par:inizializzazione}

Il team effettua delle attente valutazioni per ogni capitolato  e si candida presso il committente per realizzare quello che reputa migliore.

\subparagraph{Studio di Fattibilità}%
\label{subp:studio_di_fattibilita}

GruppOne si impegna a consegnare un documento sintetico contenente una descrizione, i pregi e le criticità riscontrati in ciascuno dei capitolati.

Lo \textit{Studio di fattibilità} è redatto dall'\glossario{amministratore} con l'aiuto degli \glossario{analisti} e ha l'obiettivo di esporre quali capitolati sono presi in considerazione e le relative motivazioni.

Il documento è così articolato:
\begin{description}
  \item [Descrizione] una breve descrizione del problema esposto dal proponente
  \item [Finalità del progetto] che cosa bisogna realizzare
  \item [Tecnologie interessate] le tecnologie da considerare in fase di sviluppo
  \item [Aspetti positivi] i pregi del capitolato
  \item [Criticità e fattori di rischio] difficoltà che potremmo incontrare.
\end{description}

\paragraph{Preparazione della risposta}%
\label{subs:preparazione_della_risposta}

GruppOne si occupa di fornire al committente una proposta di candidatura per il capitolato che possa soddisfare le sue richieste.

\paragraph{Contratto}%
\label{par:contratto}

GruppOne stipula un contratto con il committente per la realizzazione di un prodotto software.

\paragraph{Pianificazione}%
\label{par:pianificazione}

Il team prepara una pianificazione che si occuperà di:
\begin{itemize}
  \item Scegliere il modello di sviluppo che meglio si adatta al progetto.
  \item Organizzare le attività dei processi secondo il modello di sviluppo scelto.
  \item Definire degli obiettivi di qualità che il prodotto dovrà obbligatoriamente soddisfare.
\end{itemize}

\subparagraph{Piano di progetto}%
\label{subp:piano_di_progetto}
Il piano di progetto è redatto dal responsabile e contiene l'organizzazione dell'attività allo scopo di raggiungere l'economicità.
Esso rappresenta, anche mediante l'utilizzo di diagrammi, le risorse fondamentali a disposizione e come andrebbero impiegate per organizzare un efficiente pianificazione delle attività.
Il documento è così strutturato:
\begin{description}
  \item [Introduzione] si presentano lo scopo e la struttura del documento.
  \item [Analisi dei rischi] si realizza un'indagine per scoprire quali sono i maggiori rischi con cui il gruppo deve confrontarsi nel corso della realizzazione del progetto. Si illustra la gravità di ogni rischio, i danni che ognuno potrebbe arrecare, ed eventuali contromisure per riuscire a contrastarli.
  \item [Pianificazione] si stabiliscono quali sono le risorse temporali e umane a disposizione e si decide come suddividerle nelle diverse attività.
  \item [Preventivo e consuntivo] Il preventivo presenta una stima dei costi totali di progetto, mentre il consuntivo illustra un bilancio totale delle attività di processo compiute in un determinato periodo di tempo.
\end{description}

\subparagraph{Piano di qualifica}%
\label{subp:piano_di_qualifica}
Il \textit{Piano di qualifica} è redatto dai progettisti nella sua parte programmatica e dai verificatori nella parte che illustra le verifiche effettuate.
I suoi scopi principali sono:

\begin{itemize}
  \item Definire gli attributi di qualità che il nostro prodotto deve avere, e dettagliare le specifiche dei test di accettazione, sistema, integrazione e unità.
  \item Definire quali sono i valori minimi e quelli ottimali delle metriche riportate nelle \textit{Norme di progetto} per ciascuno dei processi istanziati da GruppOne che sono soggetti al sistema di \glossario{qualità}.
  \item Documentare le strategie e procedure attraverso cui intendiamo perseguire gli obiettivi definiti nei punti precedenti.
  \item Analizzare in retrospettiva i risultati delle prove e verifiche effettuate in accordo con quanto definito.
\end{itemize}

La struttura del documento vuole riflettere le due visioni perpendicolari del sistema qualità.

\begin{description}
  \item [Introduzione] esplora più in dettaglio lo scopo del documento.
  \item [Qualità di prodotto] riguarda la visione verticale e specifica a ciò che stiamo realizzando.
  \item [Qualità di processo] esplicita la visione orizzontale trasversale a processi, attività e compiti istanziati dal team.
        % \item [Report su attività di verifica] contiene brevi riassunti delle attività di verifica
\end{description}

Il \textit{Piano di qualifica} è redatto dai progettisti nella sua parte programmatica e dai verificatori nella parte che illustra le verifiche effettuate. Esso descrive la strategia complessiva di verifica e validazione adottata da GruppOne per garantire qualità ai processi e ai prodotti. Il documento è così strutturato:
\begin{description}
  \item [Introduzione]: si presentano lo scopo del prodotto e del documento.
  \item [Qualità di prodotto]: si delineano gli attributi di qualità che il nostro prodotto deve avere. Per ognuno degli attributi si indicano le metriche definite nelle \textit{Norme di progetto} per quantificare la qualità e si illustrano dei valori soglia da rispettare. Si definiscono inoltre i test di unità, integrazione, sistema e accettazione.
  \item [Qualità di processo]: per ogni processo istanziato nelle \textit{Norme di progetto} si presentano le metriche per misurare la qualità e si descrivono i valori ottimali e minimali.
\end{description}

\paragraph{Esecuzione e controllo}%
\label{par:esecuzione e controllo}

GruppOne si impegna a sviluppare e consegnare il prodotto nei tempi previsti nella pianificazione.

\paragraph{Revisione}%
\label{par:revisione}

Il team organizza riunioni con l'acquirente per chiedere chiarimenti e informarlo sullo stato del progetto.

\subparagraph{Incontri con il proponente}%
\label{subp:incontri_con_il_proponente}

GruppOne intende mantenere uno stretto rapporto di collaborazione con i proponenti del capitolato Stalker.
Tale rapporto si mantiene attraverso incontri che si svolgono fisicamente presso le aule del Dipartimento di Matematica ``Tullio Levi Civita'' e virtualmente utilizzando Google Hangouts.

Gli obiettivi degli incontri sono:
\begin{description}
  \item [Comprensione e perfezionamento dei requisiti] il team discute col proponente i problemi e i dubbi riscontrati durante l'analisi del capitolato in modo da comprendere incrementalmente i requisiti.
  \item [Ricerca e valutazione del software] il team chiede al proponente se le componenti e i software proposti soddisfino le funzionalità richieste.
  \item [Convalida dei documenti e del prodotto] il team si rivolge al proponente per avere conferme sul lavoro svolto siano essi documenti, \glossario{prototipi} appena abbozzati o prodotti software ad uno stato avanzato.
\end{description}
Gli esiti degli argomenti discussi durante gli incontri saranno poi riportati nei verbali.

\subparagraph{Incontri con il committente}%
\label{subp:incontri_con_il_committente}

GruppOne intende fare riferimento al committente del progetto anche al di fuori delle revisioni obbligatorie.
Compatibilmente con la disponibilità del committente, chiederemo di fissare degli incontri per discutere eventuali aspetti che non ci sono chiari.

Ciò che emerge da questi incontri verrà messo a verbale e reso disponibile nella cartella \verb|esterni/verbali/| della documentazione di progetto.

\paragraph{Consegna e completamento}%
\label{par:consegna e completamento}

Il gruppo si impegna a consegnare il prodotto software secondo le modalità definite nel contratto. In particolare, si forniscono anche un manuale utente e un manuale sviluppatore.

% TODO scrivi metriche

% subs:metriche (end)

\subsection{Sviluppo}%
\label{sub:sviluppo}

\subsubsection{Finalità}%
\label{subs:sviluppo/finalita}

GruppOne istanzia il processo di sviluppo per realizzare il prodotto richiesto dal proponente.

\subsubsection{Descrizione}%
\label{subs:sviluppo/descrizione}

Il processo di sviluppo fissa quali sono gli obiettivi dello sviluppo, dalla creazione alla consegna del prodotto finale.

Raggruppa le seguenti attività:
\begin{itemize}
  \item Analisi dei requisiti
  \item Progettazione
  \item Codifica
  \item Testing
  \item Installazione.
\end{itemize}

\subsubsection{Attività}%
\label{subs:sviluppo/attivita}

\paragraph{Analisi dei requisiti}%
\label{par:analisi_dei_requisiti}
L'analisi dei requisiti è l'attività che studia e comprende il dominio applicativo del problema e ha come scopo quello di offrire i requisiti che dovranno essere soddisfatti dal software.

Si articola nei seguenti compiti:

\begin{itemize}
  \item Delineare i casi d'uso del sistema a partire dal capitolato e dallo \textit{studio di fattibilità}.
  \item Realizzare i diagrammi UML dei casi d'uso.
  \item Determinare i requisiti impliciti ed espliciti del sistema.
  \item Classificare i requisiti.
  \item Ricavare i requisiti atomici da fornire ai progettisti perché possano iniziare l'attività di progettazione.
\end{itemize}

Il documento di riferimento, \textit{Analisi dei requisiti}, è redatto dagli analisti.

\subparagraph{Classificazione dei requisiti}%
\label{subp:classificazione_dei_requisiti}

Per identificare i requisiti in maniera univoca, GruppOne ha deciso di adottare delle norme per la classificazione dei requisiti.
Ogni requisito è caratterizzato da un codice alfanumerico così formato:
\begin{center}
  \textbf{R[numero][tipo][priorità]}
\end{center}
in cui ogni elemento ha un diverso significato:
\begin{description}
  \item [numero] indica quale numero di caso d’uso si sta esaminando. È un numero di tre cifre progressivo a partire da 1, con eventuali 0 di riempimento a partire dalla cifra più significativa.
  \item [tipo] individua la tipologia di requisito. Esso può essere:
        \begin{description}
          \item [F (funzionale)] indica servizi che il sistema dovrebbe fornire.
          \item [P (prestazionale)] indica le prestazioni che il programma deve fornire: velocità in esecuzione e memoria occupata.
          \item [D (dichiarativo)] indica requisiti definiti dall'esterno, ad esempio requisiti di vincolo esposti nel capitolato.
        \end{description}
  \item [priorità] determina la priorità del requisito con un numero da 1 a 3:
        \begin{enumerate}
          \item requisito obbligatorio che deve essere assolutamente soddisfatto dal sistema.
          \item requisito desiderabile il cui soddisfacimento è apprezzato dal committente.
          \item requisito facoltativo la cui decisione è lasciata al team.
        \end{enumerate}
\end{description}

\subparagraph{Casi d'uso}%
\label{subp:casi_d'uso}
I diagrammi dei casi d'uso descrivono le funzioni e i servizi offerti dal prodotto agli attori che interagiscono con il sistema. Ogni caso d'uso ha:

\begin{itemize}
  \item Una rappresentazione testuale
  \item Una rappresentazione grafica.
\end{itemize}

Nel presente paragrafo si descrive la prima mentre nel successivo verrà descritta la seconda.
Un generico caso d'uso è caratterizzato da:
\begin{description}
  \item [Sigla del caso d'uso] per riflettere più chiaramente il sistema che stiamo sviluppando anteponiamo una lettera maiuscola al caso d'uso.
        Useremo:
        \begin{description}
          \item [A] per distinguere i casi d'uso validi nell'interfaccia web.
          \item [U] per distinguere i casi d'uso validi nell'applicazione.
        \end{description}
  \item [Titolo] breve titolo del caso d'uso.
  \item [Codice] codice identificativo del caso d'uso. Ogni caso d'uso può essere suddiviso in altri sottocasi. La denominazione convenzionale è la seguente:
        \begin{description}
          \item [Caso d'uso] [A/U]UC[codice numerico]
          \item [Sottocaso d'uso] [A/U]UC[codice numerico del genitore].[codice numerico del sottocaso].
        \end{description}
        Ad esempio il primo caso d'uso interfaccia web avrà codice identificativo AUC1 mentre i relativi sottocasi saranno AUC1.1 AUC1.2 AUC1.3..
  \item [Attore primario] attore principale coinvolto nel caso d'uso.
  \item [Attori secondari (opzionale)] attori secondari coinvolti nel caso d'uso.
  \item [Precondizione] condizione in cui si trovano gli attori prima del verificarsi del caso d'uso.
  \item [Postcondizione] condizione in cui si trovano gli attori dopo il verificarsi del caso d'uso.
  \item [Scenario principale] sequenza di azioni svolte dall'attore per portare a compimento il caso d'uso.
  \item [Inclusioni (opzionale)] il caso d'uso incluso è incondizionatamente eseguito dal caso d'uso che lo include.
  \item [Estensioni (opzionale)] sequenza di possibilità dell'attore al verificarsi di eventi anomali o di situazioni di errore.
\end{description}

\subparagraph{Diagrammi UML dei casi d'uso}%
\label{subp:diagrammi_UML_dei_casi_d'uso}
I diagrammi dei casi d'uso forniscono una rappresentazione grafica del caso d'uso che si sta descrivendo. I principali elementi di un diagramma UML sono:
\begin{itemize}
  \item Attori
  \item Scenario
  \item Use case.
\end{itemize}
Gli attori che interagiscono con il sistema si trovano fuori dallo scenario, mentre gli use case sono parte integrante dello scenario.
I collegamenti tra attori e casi d'uso, e tra quest'ultimi e altri casi d'uso, avvengono tramite legami rappresentati mediante linee.
Possono essere di quattro differenti tipi:
\begin{description}
  \item [Associazione] l'associazione è la comunicazione diretta tra attore e use case. Rappresenta la partecipazione dell'attore al caso d'uso a cui è legato.
  \item [Inclusione] L'inclusione è un legame diretto stretto tra due use case. Dati due casi d'uso A e B, si dice che A include B se ogni istanza di A esegue B. B è incluso nell'esecuzione di A e la responsabilità di esecuzione di B è unicamente di A.
  \item [Estensione] L'estensione aumenta le funzionalità di uno use case. Dati due casi d'uso A e B, si dice che B estende A se A esegue B solo a determinate condizioni. L'esecuzione di B interrompe A e per questo motivo viene utilizzata prevalentemente per gestire errori e eccezioni.
  \item [Generalizzazione] La generalizzazione è un legame tra attori o più raramente tra use case. Dati due casi d'uso A e B, A è generalizzazione di B se condivide almeno le funzionalità di A. B può modificare le funzionalità di A, mentre tutte le funzionalità non ridefinite si mantengono identiche a quelle di A.
\end{description}

\paragraph{Progettazione}%
\label{par:progettazione}
L'attività di progettazione si occupa di organizzare le componenti del sistema determinandone la struttura.
Essa procede inversamente rispetto all'analisi dei requisiti la quale adotta un approccio investigativo, di studio del dominio applicativo e di scomposizione dei requisiti.
La progettazione cerca, invece, di fissare l'architettura del prodotto scegliendo una soluzione che possa soddisfare tutti gli \glossario{stakeholder}.

Essa si pone i seguenti obiettivi:
\begin{itemize}
  \item Soddisfare i requisiti con un sistema di qualità.
  \item Ricercare una buona soluzione architetturale.
  \item Suddividere il problema in sotto-problemi più semplici da risolvere.
\end{itemize}

\subparagraph{Technology Baseline}%
\label{subp:technology_baseline}
La \glossario{Technology Baseline} è un documento tecnico parte integrante della revisione di progettazione. Essa presenta:

\begin{itemize}
  \item Le tecnologie
  \item I framework
  \item Le librerie.
\end{itemize}
Deve contenere inoltre una \glossario{Proof of Concept} che ha lo scopo di evidenziare come le tecnologie utilizzate possano servire allo sviluppo del prodotto.

\subparagraph{Product Baseline}%
\label{subp:product_baseline}
La \glossario{Product Baseline} è un documento tecnico parte integrante della revisione di qualifica. Ha il compito di mostrare l'architettura del prodotto attraverso la creazione di:

\begin{itemize}
  \item Diagrammi delle classi
  \item Diagrammi di sequenza
  \item Diagrammi delle attività
  \item Diagrammi di package.
\end{itemize}

\subsubsection{Strumenti}%
\label{subs:strumenti}

\paragraph{PlantUML}%
\label{par:plantuml}
Per la costruzione dei diagrammi UML il team ha deciso di utilizzare \glossario{PlantUML}\@.
È un software open source che permette la costruzione di diagrammi UML a partire dalla scrittura di codice in un linguaggio di markup dedicato. GruppOne ha valutato positivamente tale strumento in quanto:

\begin{itemize}
  \item Gli aspetti grafici di costruzione dei diagrammi sono demandati al software sottostante.
  \item La sua natura testuale e dichiarativa permette di attuare un versionamento efficace dei diagrammi attraverso strumenti che già utilizziamo.
  \item Permette di scrivere agevolmente i diagrammi dei casi d'uso, delle classi, di sequenza e di package.
\end{itemize}

In seguito a difficoltà nell'utilizzo del package apposito di \LaTeX{}, abbiamo deciso di pre-compilare separatamente i diagrammi utilizzando le funzionalità esposte dalla \glossario{Command Line Interface} di PlantUML\@.

Posizionarsi nella cartella contenente la documentazione di progetto ed eseguire il comando:

\begin{minted}{bash}
  java \
    -jar \
    $PLANTUML_JAR \
    -checkmetadata \
    -charset UTF-8 \
    -x **/commons/style/*.pu \
    -o img \
    **/**.pu
\end{minted}

La corretta esecuzione del comando richiede di aver impostato la variabile d'ambiente \verb|PLANTUML_JAR| al percorso completo del file \verb|plantuml.jar| reperito dal sito del progetto.

In questo modo vengono generate le immagini che sono effettivamente incluse nei documenti dal comando dato in~\ref{par:LaTeX}.

\subparagraph{Esempio diagramma PlantUML}%
\label{subp:esempio_diagramma_plantuml}

\begin{minted}{text}
  @startuml
  left to right direction
  skinparam packageStyle rectangle
  actor customer
  actor clerk
  rectangle checkout {
    customer -- (checkout)
    (checkout) .> (payment) : include
    (help) .> (checkout) : extends
    (checkout) -- clerk
  }
  @enduml
\end{minted}

Il risultato grafico del codice è presentato in~\ref{fig:esempio_caso_duso}

\begin{figure}[H]%
  \label{fig:esempio_caso_duso}
  \includegraphics[width=8cm]{use-case-example.png}
  \centering
  \caption{diagramma dei casi d'uso realizzato con PlantUML}
\end{figure}

% subp:esempio_diagramma_plantuml (end)

\paragraph{Package pgfgantt}%
\label{par:pgfgantt}

Per i diagrammi di Gantt il gruppo ha scelto di utilizzare pgfgantt, un package disponibile per \LaTeX{} che sfrutta il linguaggio \glossario{PGF} e il rispettivo omonimo interprete.
PGF è in grado di produrre rappresentazioni grafiche vettoriali, e pgfgantt fornisce dei comandi \TeX{} che fungono da astrazioni per le direttive di basso livello.
Un diagramma di Gantt viene quindi costruito attraverso uno o più comandi di titolo, seguiti da una combinazione di comandi specifici per inserire gruppi, ``barre'' (cioè attività) o milestone, oppure per collegare due elementi.

L'integrazione dei diagrammi nel documento avviene durante la compilazione del documento effettuata attraverso il comando dato in~\ref{par:LaTeX}.

% par:pgfgantt (end)

\subsubsection{Metriche di processo}%
\label{subs:sviluppo/metriche_di_processo}

\paragraph{MPS-ROS\@: Requisiti obbligatori soddisfatti}%
\label{par:MPS-ROS_requisiti_obbligatori_soddisfatti}

La metrica requisiti obbligatori soddisfatti indica la percentuale di requisiti obbligatori soddisfatti sul numero totale di requisiti obbligatori. La formula è la seguente:
\[
  \frac{requisiti\ obbligatori\ soddisfatti}{requisiti\ obbligatori\ totali}\cdot 100
\]
% TODO scrivi metriche di codifica (grosso)
% TODO basta scrivere statement/branch/decision coverage per ora?
% Per il momento le lasciamo perdere le rimandiamo alla progettazione architetturale

\end{document}
