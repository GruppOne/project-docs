\documentclass{article}

\usepackage[italian]{babel}
\usepackage[margin=2cm, footskip=5mm]{geometry}
% questi package non sono necessari in lualatex; ref https://tex.stackexchange.com/a/413046
% \usepackage[utf8]{inputenc}
% \usepackage[T1]{fontenc}
\usepackage{enumitem}
\usepackage{hyperref}
\usepackage{titlesec}
\usepackage{soulutf8}
\usepackage{contour}
\usepackage{float}
\usepackage{graphicx}
\usepackage{fancyhdr}
\usepackage{longtable}
\usepackage[table]{xcolor}
\usepackage{titling}
\usepackage{lastpage}
\usepackage{ifthen}
\usepackage{calc}
\usepackage{minted}
\usepackage{pgfgantt}
\usepackage{subfiles}

\newlength{\imgwidth}

\newcommand\scalegraphics[1]{%
    \settowidth{\imgwidth}{\includegraphics{#1}}%
    \setlength{\imgwidth}{\minof{\imgwidth}{\textwidth}}%
    \includegraphics[width=\imgwidth]{#1}%
}

% XXX definizione dei percorsi in cui cercare immagini
\graphicspath{ {./}
    {./img/}
}

% esempio di utilizzo: \appendToGraphicspath{./img/} (un comando diverso per ogni path da includere)
% N.B.: ci DEVE essere un forward slash alla fine del path, a indicare che è una cartella.
\makeatletter
\newcommand\appendToGraphicspath[1]{%
  \g@addto@macro\Ginput@path{{#1}}%
}
\makeatother

% setup della sottolineatura
\setuldepth{Flat}
\contourlength{0.8pt}

\newcommand{\uline}[1]{%
  \ul{{\phantom{#1}}}%
  \llap{\contour{white}{#1}}%
}

% setup dei link
\hypersetup{
  colorlinks=true, % set true if you want colored links
  linktoc=all,     % set to all if you want both sections and subsections linked
  linkcolor=black, % choose some color if you want links to stand out
}

% setup di header e footer
\pagestyle{fancy}

\fancyhf{}
\fancyhead[L]{\includegraphics[width=1cm]{logo.png}}
\fancyhead[R]{\thetitle}
\fancyfoot[R]{\thepage\ di~\pageref{LastPage}}

\fancypagestyle{nopage}{%
  \fancyfoot{}%
}

\setlength{\headheight}{1.2cm}

% setup forma \paragraph e \subparagraph
\titleformat{\paragraph}[hang]{\normalfont\normalsize\bfseries}{\theparagraph}{1em}{}
\titleformat{\subparagraph}[hang]{\normalfont\normalsize\bfseries}{\thesubparagraph}{1em}{}

% setup profondità indice di default
\setcounter{secnumdepth}{5}
\setcounter{tocdepth}{5}

% shortcut per i placeholder
\newcommand{\plchold}[1]{\textit{\{#1\}}} % chktex 20

% hook per lo script che genera il glossario
\newcommand{\glossario}[1]{\underline{#1}\textsubscript{g}}

% definizione dei comandi \uso e \stato
\makeatletter
\newcommand{\setUso}[1]{%
  \newcommand{\@uso}{#1}%
}
\newcommand{\uso}{\@uso}

\newcommand{\setStato}[1]{%
  \newcommand{\@stato}{#1}%
}
\newcommand{\stato}{\@stato}

\newcommand{\setVersione}[1]{%
  \newcommand{\@versione}{#1}%
}
\newcommand{\versione}{\@versione}

\newcommand{\setResponsabile}[1]{%
  \newcommand{\@responsabile}{#1}%
}
\newcommand{\responsabile}{\@responsabile}

\newcommand{\setRedattori}[1]{%
  \newcommand{\@redattori}{#1}%
}
\newcommand{\redattori}{\@redattori}

\newcommand{\setVerificatori}[1]{%
  \newcommand{\@verificatori}{#1}%
}
\newcommand{\verificatori}{\@verificatori}

\newcommand{\setDescrizione}[1]{%
  \newcommand{\@descrizione}{#1}%
}
\newcommand{\descrizione}{\@descrizione}

\newcommand{\setModifiche}[1]{%
  \newcommand{\@modifiche}{#1}%
}

\newcommand{\modifiche}{\@modifiche}

\makeatother

% setup delle description
\setlist[description,1]{font=$\bullet$\hspace{1.5mm}, leftmargin=*,labelindent=12.5mm}
\setlist[description,2]{font=$\bullet$\hspace{1.5mm}, leftmargin=*,labelindent=12.5mm}
\appendToGraphicspath{../../commons/img/}

\title{Verbale esterno --- 16/06/2020}

\setResponsabile{Fabio Scettro}
\setRedattori{Alberto Gobbo}
\setVerificatori{
  Alberto Cocco
}
\setUso{Esterno}
\setDescrizione{Verbale dell'incontro di GruppOne del 20/05/2020}
\setModifiche{%
\cellcolor{white!80!lightgray!100} & Fabio Scettro & 2020--05--21 & approva documento \\%
\cellcolor{white!80!lightgray!100} & Alberto Cocco & 2020--05--20 & verifica verbale \\%
\multirow{-3}{*}{0.1.2} \cellcolor{white!80!lightgray!100} & Alessandro Rizzo & 2020--05--20 & stendi verbale %
}

\disabilitaVersione{}
\disabilitaElencoFigure{}
\disabilitaElencoTabelle{}

\begin{document}

\thispagestyle{empty}
\pagenumbering{gobble}
\begin{center}
	\includegraphics[width=8.5cm]{\commons/img/logo.png}\\
	{\Large GruppOne - progetto "Stalker"}\\
	\vspace{1.5cm}
	{\Huge \thetitle}
	\vspace{1.5cm}
	\begin{table}[H]
		\centering
		\begin{tabular}{r|l}
			\textbf{Versione}&\versione\\
			\textbf{Approvazione}&\responsabile\\
			\textbf{Redazione}&\redattori\\
			\textbf{Verifica}&\verificatori\\
			\textbf{Stato}&\stato\\
			\textbf{Uso}&\uso\\
			\textbf{Destinato a}&Imola Informatica\\
			&GruppOne\\
			&Prof. Tullio Vardanega\\
			&Prof. Riccardo Cardin\\
		\end{tabular}
	\end{table}
	\vspace{3cm}
	\textbf{Descrizione}\\
	\descrizione\\
	\vfill
	\verb|gruppone.swe@gmail.com|
\end{center}
\newpage
\thispagestyle{nopage}
\section*{Registro delle modifiche}
\label{sec:registro_delle_modifiche}
\begin{table}[H]
	\label{tab:registro_delle_modifiche}
	\centering
	\rowcolors{2}{lightgray}{white!80!lightgray!100}
	\begin{longtable}[c]{c c c c l}
		\rowcolor{darkgray!90!}\color{white}{\textbf{Versione}}&\color{white}{\textbf{Data}}&\color{white}{\textbf{Nominativo}}&\color{white}{\textbf{Ruolo}}&\color{white}{\textbf{Descrizione}}\\\endhead
	\end{longtable}
\end{table}
% section registro_delle_modifiche (end)
\newpage
\thispagestyle{nopage}
\pagenumbering{roman}
\tableofcontents
\newpage
\pagenumbering{arabic}

\section{Informazioni logistiche}%
\label{sec:informazioni_logistiche}

\begin{description}
  \item [Luogo] chiamata Hangouts
  \item [Data] 16/06/2020
  \item [Ora] 17:00 \symbol{8594} 18:10
\end{description}

\subsection{Membri del gruppo presenti}%
\label{sub:membri_del_gruppo_presenti}

\begin{enumerate}
  \item Riccardo Agatea
  % \item Tobia Apolloni
  \item Riccardo Cestaro
  \item Alberto Cocco
  \item Luca Ercole
  \item Alberto Gobbo
  \item Alessandro Rizzo
  \item Fabio Scettro
\end{enumerate}

% sub:membri_del_gruppo_presenti (end)

\subsection{Altri partecipanti}%
\label{sub:altri_partecipanti}

\begin{enumerate}
  \item Davide Zanetti (Imola Informatica, proponente del capitolato)
  \item Lorenzo Patera (Imola Informatica)
  \item Luca Cappelletti (Imola Informatica)
\end{enumerate}

% sub:altri_partecipanti (end)
% sec:informazioni_logistiche (end)

\section{Introduzione}%
\label{sec:introduzione}

L'incontro è avvenuto tramite chiamata Hangouts.
Lo scopo principale era di mostrare al proponente il prodotto ormai finito, e avere un suo feedback sul risultato.

\section{Ordine del giorno}%
\label{sec:ordine_del_giorno}

\begin{itemize}
  \item Dubbi sollevati dal gruppo e dal proponente in merito alla situazione attuale del prodotto
  \item Visione del prodotto e giudizio personale del proponente
\end{itemize}

\section{Dubbi sollevati dal gruppo e dal proponente e in merito alla situazione attuale del prodotto}%
\label{sec:dubbi_sollevati_dal_gruppo_e_dal_proponente_in_merito_alla_situazione_attuale_del_prodotto}

I membri del gruppo hanno posto alcune domande al proponente riguardo lo stato attuale del prodotto:
\begin{itemize}
  \item Il gruppo ha chiesto se per il proponente fosse soddisfacente avere un report complessivo dello stato del componente nella repository del componente specifico. La risposta è stata affermativa.
  \item Un collega ha poi chiesto se il proponente fosse d'accordo a mostrare durante la demo una versione modificata dell'applicazione mobile che emettesse più di frequente update della posizione così da facilitare lo svolgimento della demo nei tempi previsti. Anche in questo caso il proponente ha ritenuto ragionevole la nostra proposta.
\end{itemize}

\section{Visione del prodotto e giudizio personale del proponente}%
\label{sec:visione_del_prodotto_e_giudizio_personale_del_proponente}

Dopo aver chiarito i dubbi rimasti, il gruppo ha mostrato il prodotto al proponente seguendo il flusso della demo che verrà presentata in sede di revisione di accettazione.
In particolare, sono stati esposti tutti i passaggi della web application, dall'autenticazione dell'utente fino alla visualizzazione, modifica dei dettagli delle organizzazioni e visualizzazione dei report.
Sono stati mostrati anche i passaggi principali della mobile application, ossia l'autenticazione, la connessione ad una nuova organizzazione e il report degli accessi.
Il giudizio del proponente è stato nel complesso positivo nonostante qualche problema tecnico, soprattutto si è complimentato per l'impianto grafico delle due componenti mostrate e per la decisione di dare la possibilità ad ogni utente di creare una organizzazione.


\newpage
\section{Registro delle decisioni}%
\label{sec:registro_delle_decisioni}

\begin{description}
  \item[20200616-ext-001] In accordo con il proponente, il gruppo ha deciso di creare dei report finali per ogni componente posizionati ognuno nella repository corrispondente.
\end{description}

\end{document}
