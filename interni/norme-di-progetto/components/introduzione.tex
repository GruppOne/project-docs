\documentclass[../norme-di-progetto.tex]{subfiles}

\begin{document}

\subsection{Scopo del documento}%
\label{sub:scopo_del_documento}

Il presente documento si rivolge ai membri di GruppOne.
Si vogliono definire le regole e le norme a cui ogni membro del gruppo deve attenersi al fine di ottenere un efficace way of working nel corso di ogni attività di processo.
Il documento verrà aggiornato nel corso del tempo, pertanto è ancora incompleto.

\subsection{Scopo del prodotto}%
\label{sub:scopo_del_prodotto}
L'obiettivo del \glossario{progetto} è sviluppare un'\glossario{applicazione mobile} distribuita, seguendo il \glossario{modello client/server}.
Il \glossario{client} deve essere in grado di segnalare sia l'ingresso che l'uscita dell'\glossario{utente} dai \glossario{luoghi} (in modalità anonima o meno a seconda delle esigenze), i quali sono definiti dalle \glossario{organizzazioni}.
Il \glossario{server} deve fornire la possibilità di raccogliere ed analizzare i \glossario{dati} relativi alle organizzazioni.
In caso di utenti anonimi l'analisi riguarda solo una \glossario{stima} del numero totale di persone presenti in un dato momento.
In caso di utenti autenticati deve inoltre essere possibile effettuare \glossario{query} di monitoraggio specifiche.
In merito all'ottimizzazione della geolocalizzazione, è richiesto un \glossario{report} che esponga le scelte progettuali, le rispettive motivazioni e i test eseguiti per garantire la rilevazione sufficiente precisa della posizione, considerando le limitazioni dello \glossario{smartphone}.


\subsection{Glossario}%
\label{sub:glossario}

L'uso di vocaboli tecnici e facilmente fraintendibili rende necessaria la realizzazione di un glossario che definirà i termini dai significati più travisabili presenti in ogni documento.
Per garantirne l'inequivocabilità, le parole che possono assumere un significato ambiguo sono evidenziate (ad es., \glossario{way of working}) e riportate nel documento \textit{Glossario (versione \versione)} accompagnate da una breve definizione.

\subsection{IDE/Code Editor}%
\label{sub:ide_code_editor}

Lasciamo la scelta degli strumenti da utilizzare a ciascun membro del team, a cui viene quindi assegnata la responsabilità di configurarli coerentemente con quanto descritto nel presente documento.

In ogni caso, lo strumento utilizzato dalla maggioranza dei membri del team, perlomeno nel periodo iniziale è Visual Studio Code (vscode), quindi la sua configurazione, contenuta nella cartella \verb|/.vscode/|, verrà inclusa nel sistema di versionamento scelto.

Qualora si rivelasse insufficiente a coprire i nostri bisogni, soprattutto durante le attività di codifica, valuteremo la necessità di passare a un IDE completo.

Per semplificare eventuali transizioni, ci impegniamo a mantenere le configurazioni degli strumenti utilizzati in file esterni tipicamente contenuti nella radice delle cartelle di progetto e adeguatamente versionati.

\subsection{Riferimenti}%
\label{sub:riferimenti}

Questi sono riferimenti generali a risorse online relative a standard e strumenti che abbiamo deciso di utilizzare.
All'interno del documento verranno dati dei riferimenti puntuali ove necessario.

\subsubsection{Normativi}%
\label{subs:riferimenti/normativi}

\begin{itemize}
  \item \href{https://www.math.unipd.it/~tullio/IS-1/2019/Progetto/C5.pdf}{Capitolato d'appalto C5}.
  \item \href{https://www.conventionalcommits.org/en/v1.0.0/}{Conventional Commits 1.0.0}.
  \item \href{https://spec.openapis.org/oas/v3.0.2}{OpenAPI specification v3.0.2}
\end{itemize}

\subsubsection{Informativi}%
\label{subs:riferimenti/informativi}

\begin{itemize}
  \item \href{https://www.math.unipd.it/~tullio/IS-1/2019/Dispense/L06.pdf}{Corso di Ingegneria del Software, slide gestione di progetto}, diapositive da 1 a 35.
  \item \href{https://www.math.unipd.it/~tullio/IS-1/2019/Dispense/FC01.pdf}{Corso di Ingegneria del Software, slide amministrazione di progetto}, diapositive da 1 a 36.
  % \item \href{https://www.pearson.it/opera/pearson/0-6424-ingegneria_del_software}{Ingegneria del software - Ian Sommerville - decima edizione}.
  % \item \href{https://www.computer.org/education/bodies-of-knowledge/software-engineering/v3}{Software Engineering Body of Knowledge v3}.
  \item \href{https://www.math.unipd.it/~tullio/IS-1/2009/Approfondimenti/ISO_12207-1995.pdf}{ISO/IEC 12207:1995, descrizioni di processi e attività istanziati da GruppOne}.
  \item \href{https://www.iso.org/standard/43447.html}{ISO/IEC 12207:2008, sezione sul processo organizzativo di Risk Management}.
  % TODO esplicitare sezioni di interesse, o linkare praxiom?
  % \item \href{https://www.iso.org/standard/66240.html}{ISO 90003:2014}.
\end{itemize}

\subsection{Materiali consigliati per l'autoapprendimento}%
\label{sub:materiali_consigliati_per_l_autoapprendimento}

In questa sezione i membri del gruppo possono trovare dei link a pagine generali di aiuto (in inglese) per gli strumenti da noi utilizzati, e a pagine di approfondimento su aspetti tecnici e teorici che ci auguriamo siano rilevanti.

\begin{itemize}
  \item \LaTeX{}
        \begin{itemize}
          \item \href{https://www.overleaf.com/learn/latex/Multi-file_LaTeX_projects#The_subfiles_package}{Utilizzo del package subfiles}.
          \item \href{https://www.nongnu.org/chktex/ChkTeX.pdf}{Documentazione chk\TeX}.
        \end{itemize}

  \item Git
        \begin{itemize}
          \item \href{https://git-scm.com/book/en/v2}{il libro Pro Git}.
          \item \href{https://learngitbranching.js.org/}{Tutorial interattivo sulle funzionalità di git}.
          \item \href{https://git-scm.com/book/en/v2/Git-Branching-Rebasing}{Come utilizzare in maniera sicura git rebase}.
          \item \href{https://guides.github.com/introduction/flow/}{GitHub flow}.
          \item \href{https://trunkbaseddevelopment.com/}{Trunk Based Development}.
          \item \href{https://commitizen.github.io/cz-cli/}{Commitizen}.
          \item \href{https://commitlint.js.org/}{Commitlint}.
        \end{itemize}

  \item Articoli da vari blog
        \begin{itemize}
          \item \href{https://martinfowler.com/articles/feature-toggles.html}{Un metodo alternativo al branching per l'implementazione graduale e sicura di feature --- Martin Fowler}.
          \item \href{https://martinfowler.com/bliki/IntegrationTest.html}{I test di integrazione --- Martin Fowler}
          \item\href{https://martinfowler.com/articles/mocksArentStubs.html}{I Test Double --- Martin Fowler}.
          \item \href{https://martinfowler.com/articles/richardsonMaturityModel.html}{Modello di maturità di Richardson --- Martin Fowler}.
          \item \href{https://martinfowler.com/bliki/BeckDesignRules.html}{Sulla disciplina dell'Extreme programming --- Martin Fowler}.
          \item \href{https://blog.cleancoder.com/uncle-bob/2012/08/13/the-clean-architecture.html}{The Clean Architecture --- Robert C. Martin (Uncle Bob)}.
          \item \href{https://blog.cleancoder.com/uncle-bob/2018/06/06/PickledState.html}{Una discussione sulla corrispondenza tra triple GIVEN/WHEN/THEN, ARRANGE/ACT/ASSERT e automi a stato finito --- Robert C. Martin (Uncle Bob)}.
          \item \href{https://blog.cleancoder.com/uncle-bob/2019/06/16/ObjectsAndDataStructures.html}{Una discussione sulla differenza tra oggetti e data structures --- Robert C. Martin (Uncle Bob)}.
          \item \href{https://www.joelonsoftware.com/2000/08/09/the-joel-test-12-steps-to-better-code/}{The Joel Test}.
          \item \href{https://12factor.net/}{The 12-factor app (best practice per lo sviluppo di applicazioni)}.
        \end{itemize}
\end{itemize}

% sub:materiali_consigliati_per_l_autoapprendimento (end)

\end{document}
