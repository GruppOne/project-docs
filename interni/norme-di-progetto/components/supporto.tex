\documentclass[../norme-di-progetto.tex]{subfiles}
\begin{document}
\subsection{Descrizione}
\label{sub:descrizione}
I processo di supporto operano a sostegno degli altri processi con lo scopo di garantire il successo e fornire qualità. Essi non esistono autonomamente ma hanno la necessità di appoggiarsi ad altri processi. ISO 11207-1995 ne distingue otto:
\begin{itemize}
	\item Documentazione
	\item Gestione della configurazione
	\item Accertamento della qualità
	\item Verifica
	\item Validazione
	\item Revisioni congiunte
	\item Verifiche interne
	\item Risoluzione dei problemi
\end{itemize} 
\subsection{Documentazione}
\label{sub:documentazione}
\subsubsection{Finalità}
\label{subs:finalità}
Grupp0ne istanza il processo di documentazione per illustrare in maniera chiara e coerente le attività di processo svolte e i prodotti da esse ottenuti.
\subsubsection{Descrizione}
\label{subs:descrizione}
La documentazione è essenziale durante ogni attività del ciclo di vita del software. Essa costituisce un importante mezzo di comunicazione sia internamente per il team che esternamente verso il committente. Si vogliono quindi definire delle norme che possano standardizzare il processo di documentazione e renderlo più accessibile a tutti i componenti del gruppo.
\subsubsection{Implementazione}
\label{subs:implementazione}
\paragraph{Ciclo di vita}
\label{par:ciclo di vita}
Il ciclo di vita di rappresenta gli stadi in cui il documento si può trovare nel corso della sua esistenza. Ne distinguiamo cinque:
\begin{itemize}
	\item \textbf{Creazione del template del documento}: si crea il template del documento il quale conterrà solamente una pagina di frontespizio. Inizialmente sono configurate una serie di impostazioni di base per la formattazione delle pagine grazie all'utilizzo di alcuni package di latex.
	\item \textbf{Scrittura del documento}: si scrive il documento incrementalmente e lo si termina entro la scadenza della \glossario{milestone} fissata.
	\item \textbf{Verifica del documento}: i \glossario{verificatori} effettuano una revisione del documento segnalando eventuali errori o discordanze che verranno segnalate all'autore del documento che sarà incaricato di correggere.
	\item \textbf{Approvazione del documento}: il documento viene approvato dal \glossario{responsabile} di progetto.
	\item \textbf{Archiviazione}: il documento viene archiviato in un repository pubblico su \glossario{github} ed è pronto per la consegna.
\end{itemize}
\subsubsection{Struttura}
\label{subs:struttura}
\paragraph{Frontespizio}
\label{par:frontespizio}
Il frontespizio è la prima pagina di ogni documento. È diviso in due parti: nella prima è presente l'intestazione contenente logo, nome del gruppo e nome del documento, mentre la seconda consiste di uno schema contenente alcune informazioni essenziali sul documento descritto. In esso compaiono, ordinatamente, dall'alto verso il basso:
\begin{itemize}
	\item \textbf{Intestazione}
	\begin{itemize}
		\item \textbf{Logo}: rappresenta il logo del team.
		\item \textbf{Nome del gruppo}: rappresenta il nome del gruppo.
		\item \textbf{Nome del documento}: rappresenta nome del documento.
	\end{itemize}
	\item \textbf{Schema}
	\begin{itemize}
		\item \textbf{Versione}: indica la versione attuale del documento (e.g. glossario v1.0.0).
		\item \textbf{Approvazione}: indica chi ha approvato il documento.
		\item \textbf{Redazione}: indica la lista dei redattori del documento.
		\item \textbf{Verifica}: indica la lista dei verificatori del documento.
		\item \textbf{Stato}: indica lo stato attuale in cui si trova il documento.
		\item \textbf{Uso}: indica l'uso finale del documento (interno o esterno).
		\item \textbf{Destinato a}: indica lo stato attuale del documento.
		\item \textbf{Descrizione}: indica una breve descrizione del documento. 
	\end{itemize}
\end{itemize}
\paragraph{Registro delle modifiche}
\label{par:registro delle modifiche}
Il registro delle modifiche è la seconda pagina di ogni documento. Ha lo scopo di presentare quali cambiamenti sono stati effettuati e da parte di quale componente del gruppo. Consiste di quattro colonne ed è così articolato:
\begin{itemize}
	\item \textbf{Versione} : indica la versione del documento in cui viene realizzata la modifica.
	\item \textbf{Data} : indica la data di modifica del documento con formato yyyy/mm/dd.
	\item \textbf{Nominativo} : indica nome e cognome del componente del team che ha effettuato la modifica.
	\item \textbf{Ruolo} : indica il ruolo del componente del team che ha realizzato la modifica.
	\item \textbf{Descrizione}: indica il tipo e il luogo in cui è avvenuta la modifica.
\end{itemize}
\paragraph{Indice}
\label{par:indice}
L'indice inizia nella terza pagine del documento e presenta tutti gli argomenti trattati. Grupp0ne ha deciso di utilizzare la tipica suddivisione del testo offerta da latex che distingue cinque diversi blocchi testuali:
\begin{itemize}
	\item Sezioni
	\item Sottosezioni
	\item Capitoli
	\item Paragrafi
	\item Sottoparagrafi 
\end{itemize}
Ogni blocco testuale ha un numero identificativo univoco che dipende dal grado di annidamento del blocco definito da Latex ( e.g. : la sezione Introduzione è indicata con 1, la relativa sottosezione quattro è indicata con 1.4 e i capitoli contenuti in quella sottosezione sono indicati con 1.4.1 e 1.4.2).
\newline Ogni riga dell'indice contiene il numero di pagina in cui il blocco di testo è riferito e cliccando sopra il nome del blocco è possibile raggiungerlo direttamente attraverso un link.
\paragraph{Contenuto}
\label{par:contenuto}
Le pagine di contenuto sono suddivise in tre parti:
\begin{itemize}
	\item \textbf{Header}: contiene in alto a sinistra il logo del gruppo, mentre in alto a destra il nome del documento.
	\item \textbf{Contenuto}: contiene il testo del documento.
	\item \textbf{Footer}: contiene l'indicazione della pagina attuale rispetto il totale (e.g 1/6).
\end{itemize}
\paragraph{Norme per la redazione dei documenti}
\label{par:norme per la redazione dei documenti}
\subparagraph{Stile del testo}
\label{subp:stile del testo}
In questo paragrafo Grupp0ne definisce le norme che uniformano lo stile di scrittura dei documenti:
\begin{itemize}
		\item \textbf{Verbi in forma attiva}: i verbi devono essere in forma attiva e al tempo presente indicativo o passato prossimo. È ammesso l'uso del futuro per esprimere azioni che devono ancora avvenire.
		\item \textbf{Struttura del testo chiara}: la suddivisione del testo in sezioni, sottosezioni e paragrafi aiuta la coerenza e la coesione.
		\item \textbf{Frasi brevi e poco complesse}: i periodi devono essere il più possibile semplici per non generare incomprensioni.
		\item \textbf{Uso degli elenchi puntati}: per evitare lunghe digressioni ed eccessiva verbosità si vogliono utilizzare gli elenchi puntati laddove è possibile.
		\item \textbf{Brevi blocchi testuali}: si preferisce l'utilizzo di brevi paragrafi.
		\item \textbf{Termini di glossario in maiuscolo}: il testo è scritto in minuscolo. I termini di glossario, invece, sono indicati in maiuscolo con una g a pedice nel nome. Questa regola vale per la prima occorrenza di ogni termine di glossario.
\end{itemize}
\subparagraph{Elenchi puntati}
\label{subp:elenchi puntati}
Gli elenchi puntati sono un mezzo ottimo per la scrittura di documentazione. Essi permettono di riordinare il testo e di elencare una serie di elementi correlati. Pertanto, in questo paragrafo, si stabiliscono le norme per il corretto uso degli elenchi puntati:
\begin{enumerate}
	\item Per indicare gli elementi di un elenco puntato non innestato utilizzare il simbolo •. Gli elementi innestati vengono preceduti da -.
	\item Ogni elemento di elenco puntato inizia con una lettera maiuscola.
	\item Se un elenco puntato ha elementi composti da etichetta e descrizione, l'etichetta deve essere scritta in grassetto e la descrizione va inserita dopo i due punti.
	\item Gli elenchi puntati semplici non hanno bisogno di punti fermi per terminare la frase.Gli elenchi puntati complessi in particolare quelli formati da etichetta e descrizione richiedono un punto al termine della descrizione.
\end{enumerate}
\subparagraph{Nomi dei file}
\label{nomi dei file}
\subparagraph{Sigle e convenzioni}
\label{sigle e convenzioni}
\subparagraph{Immagini}
\subsubsection{Produzione}
\paragraph{Suddivisione dei documenti}
\subparagraph{Interni}
\subparagraph{Esterni}
\subparagraph{Verbali}
\subsubsection{Strumenti}
\paragraph {Latex}
\subsection{Gestione della configurazione}
\subsubsection{Finalità}
\subsubsection{Gestione versionamento}
\paragraph{Repository}
\paragraph{Struttura dei commit}
\subsubsection{Strumenti}
\subsection{Accertamento della qualità}
\subsubsection{Finalità}
\subsubsection{Descrizione}
\subsubsection{Attività}
\subsection{Verifica}
\subsubsection{Finalità}
\subsubsection{Descrizione}
\subsubsection{Implementazione del processo}
\paragraph{Analisi dei rischi}
\subsubsection{Verifica}
\paragraph{Analisi statica}
\paragraph{Analisi dinamica}
\subparagraph{Test di unità}
\subparagraph{Test di integrazione}
\subparagraph{Test di sistema}
\subparagraph{Test di regressione}
\subparagraph{Test di accettazione}
\paragraph{Strumenti}
\subparagraph{Controllo ortografico e della sintassi}
\subparagraph{Code owners}
\subsection{Validazione}
\subsubsection{Finalità}
\subsubsection{Descrizione}
\subsubsection{Attività}
\end{document}