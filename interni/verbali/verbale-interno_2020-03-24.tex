\documentclass{article}

\usepackage[italian]{babel}
\usepackage[margin=2cm, footskip=5mm]{geometry}
% questi package non sono necessari in lualatex; ref https://tex.stackexchange.com/a/413046
% \usepackage[utf8]{inputenc}
% \usepackage[T1]{fontenc}
\usepackage{enumitem}
\usepackage{hyperref}
\usepackage{titlesec}
\usepackage{soulutf8}
\usepackage{contour}
\usepackage{float}
\usepackage{graphicx}
\usepackage{fancyhdr}
\usepackage{longtable}
\usepackage[table]{xcolor}
\usepackage{titling}
\usepackage{lastpage}
\usepackage{ifthen}
\usepackage{calc}
\usepackage{minted}
\usepackage{pgfgantt}
\usepackage{subfiles}

\newlength{\imgwidth}

\newcommand\scalegraphics[1]{%
    \settowidth{\imgwidth}{\includegraphics{#1}}%
    \setlength{\imgwidth}{\minof{\imgwidth}{\textwidth}}%
    \includegraphics[width=\imgwidth]{#1}%
}

% XXX definizione dei percorsi in cui cercare immagini
\graphicspath{ {./}
    {./img/}
}

% esempio di utilizzo: \appendToGraphicspath{./img/} (un comando diverso per ogni path da includere)
% N.B.: ci DEVE essere un forward slash alla fine del path, a indicare che è una cartella.
\makeatletter
\newcommand\appendToGraphicspath[1]{%
  \g@addto@macro\Ginput@path{{#1}}%
}
\makeatother

% setup della sottolineatura
\setuldepth{Flat}
\contourlength{0.8pt}

\newcommand{\uline}[1]{%
  \ul{{\phantom{#1}}}%
  \llap{\contour{white}{#1}}%
}

% setup dei link
\hypersetup{
  colorlinks=true, % set true if you want colored links
  linktoc=all,     % set to all if you want both sections and subsections linked
  linkcolor=black, % choose some color if you want links to stand out
}

% setup di header e footer
\pagestyle{fancy}

\fancyhf{}
\fancyhead[L]{\includegraphics[width=1cm]{logo.png}}
\fancyhead[R]{\thetitle}
\fancyfoot[R]{\thepage\ di~\pageref{LastPage}}

\fancypagestyle{nopage}{%
  \fancyfoot{}%
}

\setlength{\headheight}{1.2cm}

% setup forma \paragraph e \subparagraph
\titleformat{\paragraph}[hang]{\normalfont\normalsize\bfseries}{\theparagraph}{1em}{}
\titleformat{\subparagraph}[hang]{\normalfont\normalsize\bfseries}{\thesubparagraph}{1em}{}

% setup profondità indice di default
\setcounter{secnumdepth}{5}
\setcounter{tocdepth}{5}

% shortcut per i placeholder
\newcommand{\plchold}[1]{\textit{\{#1\}}} % chktex 20

% hook per lo script che genera il glossario
\newcommand{\glossario}[1]{\underline{#1}\textsubscript{g}}

% definizione dei comandi \uso e \stato
\makeatletter
\newcommand{\setUso}[1]{%
  \newcommand{\@uso}{#1}%
}
\newcommand{\uso}{\@uso}

\newcommand{\setStato}[1]{%
  \newcommand{\@stato}{#1}%
}
\newcommand{\stato}{\@stato}

\newcommand{\setVersione}[1]{%
  \newcommand{\@versione}{#1}%
}
\newcommand{\versione}{\@versione}

\newcommand{\setResponsabile}[1]{%
  \newcommand{\@responsabile}{#1}%
}
\newcommand{\responsabile}{\@responsabile}

\newcommand{\setRedattori}[1]{%
  \newcommand{\@redattori}{#1}%
}
\newcommand{\redattori}{\@redattori}

\newcommand{\setVerificatori}[1]{%
  \newcommand{\@verificatori}{#1}%
}
\newcommand{\verificatori}{\@verificatori}

\newcommand{\setDescrizione}[1]{%
  \newcommand{\@descrizione}{#1}%
}
\newcommand{\descrizione}{\@descrizione}

\newcommand{\setModifiche}[1]{%
  \newcommand{\@modifiche}{#1}%
}

\newcommand{\modifiche}{\@modifiche}

\makeatother

% setup delle description
\setlist[description,1]{font=$\bullet$\hspace{1.5mm}, leftmargin=*,labelindent=12.5mm}
\setlist[description,2]{font=$\bullet$\hspace{1.5mm}, leftmargin=*,labelindent=12.5mm}

\appendToGraphicspath{../../commons/img/}

\title{Verbale interno --- 25/03/2020}

\setResponsabile{Alessandro Rizzo}
\setRedattori{Riccardo Agatea}
\setVerificatori{
  Tobia Apolloni \\ &
  Riccardo Cestaro
}
\setUso{Interno}
\setDescrizione{Verbale dell'incontro di GruppOne del 11/03/2020}
\setModifiche{%
\cellcolor{white!80!lightgray!100} & \plchold{Alessandro Rizzo} & 2020--03--27 & approva documento \\%
\cellcolor{white!80!lightgray!100} & \plchold{Tobia Apolloni, Riccardo Cestaro} & 2020--03--26 & verifica verbale \\%
\multirow{-3}{*}{0.0.3} & \plchold{Riccardo Agatea} & 2020--03--25 & stendi verbale%
}

\disabilitaVersione{}
\disabilitaElencoFigure{}
\disabilitaElencoTabelle{}

\begin{document}

\thispagestyle{empty}
\pagenumbering{gobble}
\begin{center}
	\includegraphics[width=8.5cm]{\commons/img/logo.png}\\
	{\Large GruppOne - progetto "Stalker"}\\
	\vspace{1.5cm}
	{\Huge \thetitle}
	\vspace{1.5cm}
	\begin{table}[H]
		\centering
		\begin{tabular}{r|l}
			\textbf{Versione}&\versione\\
			\textbf{Approvazione}&\responsabile\\
			\textbf{Redazione}&\redattori\\
			\textbf{Verifica}&\verificatori\\
			\textbf{Stato}&\stato\\
			\textbf{Uso}&\uso\\
			\textbf{Destinato a}&Imola Informatica\\
			&GruppOne\\
			&Prof. Tullio Vardanega\\
			&Prof. Riccardo Cardin\\
		\end{tabular}
	\end{table}
	\vspace{3cm}
	\textbf{Descrizione}\\
	\descrizione\\
	\vfill
	\verb|gruppone.swe@gmail.com|
\end{center}
\newpage
\thispagestyle{nopage}
\section*{Registro delle modifiche}
\label{sec:registro_delle_modifiche}
\begin{table}[H]
	\label{tab:registro_delle_modifiche}
	\centering
	\rowcolors{2}{lightgray}{white!80!lightgray!100}
	\begin{longtable}[c]{c c c c l}
		\rowcolor{darkgray!90!}\color{white}{\textbf{Versione}}&\color{white}{\textbf{Data}}&\color{white}{\textbf{Nominativo}}&\color{white}{\textbf{Ruolo}}&\color{white}{\textbf{Descrizione}}\\\endhead
	\end{longtable}
\end{table}
% section registro_delle_modifiche (end)
\newpage
\thispagestyle{nopage}
\pagenumbering{roman}
\tableofcontents
\newpage
\pagenumbering{arabic}

\section{Informazioni logistiche}%
\label{sec:informazioni_logistiche}

\begin{description}
  \item [Luogo] videochiamata su Google Hangouts
  \item [Data] 25/03/2020
  \item [Ora] 15:00 \symbol{8594} 17:00
\end{description}

\subsection{Membri del gruppo presenti}%
\label{sub:membri_del_gruppo_presenti}

\begin{enumerate}
  \item Riccardo Agatea
  % \item Tobia Apolloni
  \item Riccardo Cestaro
  \item Alberto Cocco
  \item Luca Ercole
  \item Alberto Gobbo
  \item Alessandro Rizzo
  \item Fabio Scettro
\end{enumerate}
% sub:membri_del_gruppo_presenti (end)

\section{Ordine del giorno}%
\label{sec:ordine_del_giorno}

\begin{itemize}
  \item Punto della situazione
  \item Finalizzazione POC
\end{itemize}

\section{Punto della situazione}%
\label{sec:punto_della_situazione}

Ciascuno dei tre team in cui ci siamo suddivisi per parallelizzare il lavoro nelle tre componenti del prodotto ha aggiornato gli altri sui propri avanzamenti.
Gli endpoint del server sono definiti, rimane da decidere quali implementare e in che modo (cioè se implementarli effettivamente, oppure se ricorrere a dei mock) per il POC\@.
Per la webapp è stato definito lo stile utilizzando angular material e i servizi più rappresentativi sono quasi pronti.
Per l'app mobile è pronta la base strutturale; lo sviluppo delle activity principali è in corso, e bisogna decidere quali componenti presentare per la Technology baseline.
Abbiamo discusso sull'autenticazione, valutando se fosse il caso di implementarne almeno una bozza oppure se rimandarla interamente al primo incremento.
Abbiamo accennato ad alcuni framework da tenere in considerazione, fra cui \href{https://www.keycloak.org/}{Keycloak} e \href{https://spring.io/projects/spring-security}{Spring security}.
Abbiamo deciso di continuare a studiare le tecnologie e di posporre la decisione.

\section{Finalizzazione POC}%
\label{sec:finalizzazione_poc}

Visto lo stato di avanzamento, abbiamo è deciso di implementare (o completare):
\begin{itemize}
  \item Per quanto riguarda il server, gli endpoint relativi alla richiesta delle informazioni su un'organizzazione, l'aggiunta di un luogo ad un'organizzazione, la segnalazione di entrata e/o uscita di un utente da un luogo ed eventualmente la richiesta di login (in base ai risultati dell'autoapprendimento); l'implementazione sarà provvisoria, perché la progettazione del database è ancora in corso.
  \item Per quanto riguarda la webapp, la visualizzazione delle informazioni relative ad un'organizzazione, l'aggiunta di un luogo ad un'organizzazione ed eventualmente il login; l'implementazione sarà possibilmente definitiva.
  \item Per quanto riguarda l'app mobile, la visualizzazione dell'elenco di organizzazioni a cui un utente è collegato e la rilevazione dell'entrata e/o uscita da un luogo di interesse; anche in questo caso l'implementazione sarà possibilmente definitiva.
\end{itemize}

\newpage
\section{Registro delle decisioni}%
\label{sec:registro_delle_decisioni}

\begin{enumerate}
  \item Abbiamo deciso di posporre la scelta della tecnologia di autenticazione.
  \item Abbiamo deciso la composizione del POC (per il dettaglio, §\ref{sec:finalizzazione_poc}).
\end{enumerate}

% sec:registro_delle_decisioni (end)
\end{document}
