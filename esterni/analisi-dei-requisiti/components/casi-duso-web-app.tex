\documentclass[casi-duso]{subfiles}

\begin{document}

\subsubsection{AUC1 - Sistema di autenticazione}%
\label{subsub:AUC1}

\begin{figure}[h!]
  \centering
  \begin{plantuml}
  @startuml
  !include ../../commons/style/use-cases.pu
  scale 3/4

  actor :utente non autenticato: as A

  rectangle {
    together {
      usecase (AUC1.1) as "AUC1.1\nAutenticazione\n--\nExtension points:\nVisualizzazione errore se\nle credenziali inserite\n non sono corrette"
      usecase (AUC1.2) as "AUC1.2\nVerifica credenziali"
      usecase (AUC1.3) as "AUC1.3\nVisualizzazione credenziali errate"
    }
  }

  :A: -- AUC1.1

  (AUC1.2) .up.|> (AUC1.1) : <<include>>
  (AUC1.3) .up.|> (AUC1.1) : <<extends>>

  @enduml
  \end{plantuml}
  \caption{AUC1: Sistema di autenticazione}
  \label{fig:auc1}
\end{figure}

\begin{description}
  \item[Codice:] AUC1
  \item[Titolo:] Sistema di autenticazione
  \item[Attori primari:] utente non autenticato
  \item[Precondizione:] l'utente non autenticato non è autenticato alla piattaforma.
  \item[Postcondizione:] l'\glossario{utente} ha effettuato correttamente il login nel sistema.
  \item[Scenario principale:]
  \begin{enumerate}
    \item l'utente non autenticato non è ancora autenticato e vuole eseguire il login.
  \end{enumerate}
\end{description}
% subsub:AUC1 (end)

\subsubsection{AUC1.1 - Autenticazione}%
\label{subsub:AUC1.1}

\begin{figure}[h!]
  \centering
  \begin{plantuml}
  @startuml
  !include ../../commons/style/use-cases.pu
  scale 3/4

  actor :utente non autenticato: as A

  rectangle {
    together {
    usecase (AUC1.1.1) as "AUC1.1.1\nInserimento email"
    usecase (AUC1.1.2) as "AUC1.1.2\nInserimento password"
    }
  }

  :A: -- AUC1.1.1
  :A: -- AUC1.1.2

  @enduml
  \end{plantuml}
  \caption{AUC1.1: Autenticazione}
  \label{fig:auc1_1}
\end{figure}

\begin{description}
  \item[Codice:] AUC1.1
  \item[Titolo:] autenticazione
  \item[Attori primari:] Utente non autenticato
  \item[Precondizione:] il sistema è raggiungibile e funzionante, l'utente non autenticato deve poter visualizzare la pagina di login.
  \item[Postcondizione:] l'utente non autenticato ha inserito le possibili credenziali e sta tentando di effettuare l'autenticazione. Ogni volta che cercherà di effettuare
        login sarà verificato che le credenziali inserite siano corrette. In caso contrario verrà visualizzato un messaggio di errore, e l'accesso sarà negato.
  \item[Scenario principale:]
  \begin{enumerate}
    \item  l'utente non autenticato accede alla pagina di login, e visualizza tutti i campi da compilare:
    \begin{enumerate}
      \item inserisce l’email associata all’account \emph{[AUC1.1.1]};
      \item inserisce la password associata all’account \emph{[AUC1.1.2]}.
    \end{enumerate}
  \end{enumerate}
  \item[Inclusioni:]
  \begin{enumerate}
    \item Verifica credenziali \emph{[AUC1.2]}.
  \end{enumerate}
  \item[Estensioni:]
  \begin{enumerate}
    \item Visualizzazione messaggio di credenziali errate \emph{[AUC1.3]}.
  \end{enumerate}
\end{description}
% subsub:AUC1.1 (end)

\subsubsection{AUC1.1.1 - Inserimento e-mail}%
\label{subsub:AUC1.1.1}
\begin{description}
  \item[Codice:] AUC1.1.1
  \item[Titolo:] Inserimento e-mail
  \item[Attori primari:] utente non autenticato
  \item[Precondizione:] il sistema ha reso disponibile il campo per l'inserimento della propria e-mail.
  \item[Postcondizione:] l'utente non autenticato ha compilato il campo relativo alla propria e-mail.
  \item[Scenario principale:]
  \begin{enumerate}
    \item l'utente non autenticato compila il campo relativo al proprio username di registrazione.
  \end{enumerate}
\end{description}
% subsub:AUC1.1.1 (end)

\subsubsection{AUC1.1.2 - Inserimento password}%
\label{subsub:AUC1.1.2}
\begin{description}
  \item[Codice:] AUC1.1.2
  \item[Titolo:] Inserimento password
  \item[Attori primari:] utente non autenticato
  \item[Precondizione:] il sistema ha reso disponibile il campo per l'inserimento della password.
  \item[Postcondizione:] l'utente non autenticato ha compilato il campo relativo alla sua password.
  \item[Scenario principale:]
  \begin{enumerate}
    \item l'utente non autenticato compila il campo relativo alla sua password di registrazione.
  \end{enumerate}
\end{description}
% subsub:AUC1.1.2 (end)

\subsubsection{AUC1.2 - Verifica Credenziali}%
\label{subsub:AUC1.2}
\begin{description}
  \item[Codice:] AUC1.2
  \item[Titolo:] Verifica credenziali
  \item[Attori primari:] utente non autenticato
  \item[Precondizione:] l'utente non autenticato ha inviato al server le sue credenziali per tentare il login
  \item[Postcondizione:] l'utente non autenticato deve poter accedere alla sua area riservata, nel caso in cui le credenziali siano corrette. In caso
  contrario deve essere visualizzato un messaggio di credenziali sbagliate \emph{[AUC1.3]}.
  \item[Scenario principale:]
  \begin{enumerate}
    \item l'utente non autenticato sta tentando di effettuare l'accesso e sta attendendo la verifica delle credenziali immesse.
  \end{enumerate}
\end{description}
% subsub:AUC1.2 (end)

\subsubsection{AUC1.3 - Visualizzazione credenziali errate}%
\label{subsub:AUC1.3}
\begin{description}
  \item[Codice:] AUC1.3
  \item[Titolo:] Visualizzazione credenziali errate
  \item[Attori primari:] utente non autenticato
  \item[Precondizione:] l'utente non autenticato ha inviato al server le sue credenziali per tentare il login, e le credenziali sono state verificate.
  \item[Postcondizione:] l'utente non autenticato visualizza un messaggio di credenziali sbagliate.
  \item[Scenario principale:]
  \begin{enumerate}
    \item l'utente non autenticato cerca di effettuare il login con delle credenziali sbagliate.
  \end{enumerate}
\end{description}
% subsub:AUC1.3 (end)


\subsubsection{AUC2 - Richiesta owner}%
\label{subsub:AUC2}

\begin{figure}[h!]
  \centering
  \begin{plantuml}
  @startuml
  !include ../../commons/style/use-cases.pu
  scale 3/4

  actor :utente autenticato: as A

  rectangle {
    together {
      usecase (AUC2.3) as "AUC2.3\nRichiesta rifiutata"
      usecase (AUC2.2) as "AUC2.2\nInvio richiesta owner"
      usecase (AUC2.1) as "AUC2.1\nVerifica tipologia utente"
    }
  }

  :A: -- AUC2.1
  :A: -- AUC2.2
  :A: -- AUC2.3

  @enduml
  \end{plantuml}
  \caption{AUC2: Richiesta owner}
  \label{fig:auc2}
\end{figure}

\begin{description}
  \item[Codice:] AUC2
  \item[Titolo:] Richiesta owner
  \item[Attori primari:] utente autenticato
  \item[Precondizione:] l'utente autenticato non deve essere un \glossario{owner}.
  \item[Postcondizione:] se viene approvata la richiesta, l'utente autenticato diventerà un owner.
  \item[Scenario principale:]
  \begin{enumerate}
    \item l'utente autenticato vuole iniziare ad utilizzare \emph{Stalker} creando una sua organizzazione.
  \end{enumerate}
\end{description}
% subsub:AUC2 (end)

\subsubsection{AUC2.1 - Verifica tipologia utente}%
\label{subsub:AUC2.1}
\begin{description}
  \item[Codice:] AUC2.1
  \item[Titolo:] Verifica tipologia utente
  \item[Attori primari:] utente autenticato
  \item[Precondizione:] l'utente autenticato visualizza l'apposito form per inviare la richiesta per diventare owner.
  \item[Postcondizione:] il form relativo alla richiesta di owner viene abilitato o disabilitato di conseguenza.
  \item[Scenario principale:]
  \begin{enumerate}
    \item l'utente autenticato può richiedere di diventare un owner, nel caso in cui sia:
    \begin{description}
      \item un \glossario{amministratore},
      \item un \glossario{visualizzatore},
      \item un \glossario{gestore} oppure un utente senza alcun privilegio;
    \end{description}
    \item ogni tentativo di richiesta da un già owner viene rifiutato;
    \item \glossario{root} è già un owner;
    \item l'amministratore potrà accettare la propria richiesta inviata.
  \end{enumerate}
\end{description}
% subsub:AUC2.1 (end)

\subsubsection{AUC2.2 - Invio richiesta owner}%
\label{subsub:AUC2.2}
\begin{description}
  \item[Codice:] AUC2.2
  \item[Titolo:] Invio richiesta owner
  \item[Attori primari:] utente autenticato
  \item[Precondizione:] l'utente autenticato utilizza l'apposito form per inviare la richiesta per diventare owner.
  \item[Postcondizione:] viene inviata la richiesta, in attesa di essere approvata da un amministratore.
  \item[Scenario principale:]
  \begin{enumerate}
    \item l'utente autenticato invia la richiesta per diventare owner.
  \end{enumerate}
  \item[Inclusioni:]
  \begin{enumerate}
    \item Verifica tipologia utente \emph{[AUC2.1]};
  \end{enumerate}
  \item[Estensioni:]
  \begin{enumerate}
    \item Richiesta rifiutata \emph{[AUC.2.3]}.
  \end{enumerate}
\end{description}
% subsub:AUC2.2 (end)

\subsubsection{AUC2.3 - Richiesta rifiutata}%
\label{subsub:AUC2.3}
\begin{description}
  \item[Codice:] AUC2.3
  \item[Titolo:] Richiesta rifiutata
  \item[Attori primari:] utente autenticato
  \item[Precondizione:] l'utente autenticato ha inviato la propria richiesta per diventare owner, l'amministratore ha rifiutato la richiesta.
  \item[Postcondizione:] l'utente autenticato non ha ottenuto l'abilitazione per un owner, visualizza un errore.
  \item[Scenario principale:]
  \begin{enumerate}
    \item l'amministratore non ha accettato la richiesta posta dall'utente autenticato.
  \end{enumerate}
\end{description}
% subsub:AUC2.2 (end)

\subsubsection{AUC3 - Disconnessione}%
\label{subsub:AUC3}

\begin{figure}[h!]
  \centering
  \begin{plantuml}
  @startuml
  !include ../../commons/style/use-cases.pu
  scale 3/4

  actor :utente autenticato: as A

  rectangle {
    together {
      usecase (AUC3.1) as "AUC3.1\nDisconnessione utente"
    }
  }

  :A: -- AUC3.1

  @enduml
  \end{plantuml}
  \caption{AUC3: Disconnessione}
  \label{fig:auc3}
\end{figure}

\begin{description}
  \item[Codice:] AUC3
  \item[Titolo:] Disconnessione
  \item[Attori primari:] utente autenticato
  \item[Precondizione:] il sistema ha reso disponibile la possibilità di effettuare la disconnessione.
  \item[Postcondizione:] l'utente autenticato diventerà un utente non autenticato, effettuando la disconnessione.
  \item[Scenario principale:]
  \begin{enumerate}
    \item l'utente autenticato vuole effettuare il logout dalla piattaforma;
  \end{enumerate}
\end{description}
% subsub:AUC3 (end)

\subsubsection{AUC4 - Gestione richieste}%
\label{subsub:AUC4}

\begin{figure}[h!]
  \centering
  \begin{plantuml}
  @startuml
  !include ../../commons/style/use-cases.pu
  scale 3/4

  actor :amministratore: as A

  rectangle {
    together {
      usecase (AUC4.8) as "AUC4.8\nGestione richiesta nuovo gestore"
      usecase (AUC4.7) as "AUC4.7\nGestione richiesta nuovo visualizzatore"
      usecase (AUC4.6) as "AUC4.6\nGestione richiesta trasferimento di proprietà organizzazione"
      usecase (AUC4.5) as "AUC4.5\nGestione richiesta modifica luogo"
      usecase (AUC4.4) as "AUC4.4\nGestione richiesta aggiunta luogo"
      usecase (AUC4.3) as "AUC4.3\nGestione richiesta modifica organizzazione"
      usecase (AUC4.2) as "AUC4.2\nGestione richiesta creazione organizzazione"
      usecase (AUC4.1) as "AUC4.1\nGestione richieste owner"
    }
  }

  :A: -- AUC4.1
  :A: -- AUC4.2
  :A: -- AUC4.3
  :A: -- AUC4.4
  :A: -- AUC4.5
  :A: -- AUC4.6
  :A: -- AUC4.7
  :A: -- AUC4.8

  @enduml

  \end{plantuml}
  \caption{AUC4: Gestione richieste}
  \label{fig:auc4}
\end{figure}

\begin{description}
  \item[Codice:] AUC4
  \item[Titolo:] Gestione richieste
  \item[Attori primari:] amministratore
  \item[Precondizione:] il sistema deve rendere disponibili le richieste effettuate dai super utenti.
  \item[Postcondizione:] vengono approvate o rifiutate le richieste.
  \item[Scenario principale:]
  \begin{enumerate}
    \item l'amministratore gestisce le richieste.
    \item le richieste vengono accettate o rifiutate.
  \end{enumerate}
\end{description}

\subsubsection{AUC4.1 - Gestione richieste owner}%
\label{subsub:AUC4.1}

\begin{figure}[h!]
  \centering
  \begin{plantuml}
  @startuml
  !include ../../commons/style/use-cases.pu
  scale 3/4

  actor :amministratore: as A

  rectangle {
    together {
      usecase (AUC4.1.2) as "AUC4.1.2\nRifiuta richiesta owner"
      usecase (AUC4.1.1) as "AUC4.1.1\nAccetta richiesta owner"
    }
  }

  :A: -- AUC4.1.1
  :A: -- AUC4.1.2

  @enduml
  \end{plantuml}
  \caption{AUC4.1: Gestione richieste owner}
  \label{fig:auc4_1}
\end{figure}

\begin{description}
  \item[Codice:] AUC4.1
  \item[Titolo:] Gestione richieste owner
  \item[Attori primari:] amministratore
  \item[Precondizione:] l'utente autenticato deve aver richiesto di diventare un owner.
  \item[Postcondizione:] viene approvata o rifiutata la richiesta, e l'utente autenticato sarà informato.
  \item[Scenario principale:]
  \begin{enumerate}
    \item l'amministratore gestisce le richieste inviate dagli utenti autenticati di diventare owner;
    \item le richieste vengono accettate o rifiutate.
  \end{enumerate}
\end{description}
% subsub:AUC4.1 (end)

\subsubsection{AUC4.1.1 - Accetta richiesta owner}%
\label{subsub:AUC4.1.1}
\begin{description}
  \item[Codice:] AUC4.1.1
  \item[Titolo:] Accetta richiesta owner
  \item[Attori primari:] amministratore
  \item[Precondizione:] l'utente autenticato deve aver richiesto di diventare un owner.
  \item[Postcondizione:] viene approvata la richiesta da parte di un amministratore.
  \item[Scenario principale:]
  \begin{enumerate}
    \item l'amministratore accetta la richiesta inviata dall'utente autenticato di diventare owner;
  \end{enumerate}
\end{description}
% subsub:AUC4.1.1 (end)

\subsubsection{AUC4.1.2 - Rifiuta richiesta owner}%
\label{subsub:AUC4.1.2}
\begin{description}
  \item[Codice:] AUC4.1.2
  \item[Titolo:] Rifiuta richiesta owner
  \item[Attori primari:] amministratore
  \item[Precondizione:] l'utente autenticato deve aver richiesto di diventare un owner.
  \item[Postcondizione:] viene rifiutata la richiesta da parte di un amministratore.
  \item[Scenario principale:]
  \begin{enumerate}
    \item l'amministratore rifiuta la richiesta inviata dall'utente autenticato di diventare owner;
  \end{enumerate}
\end{description}
% subsub:AUC4.1.2 (end)

\subsubsection{AUC4.2 - Gestione richiesta creazione organizzazione}%
\label{subsub:AUC4.2}

\begin{figure}[h!]
  \centering
  \begin{plantuml}
  @startuml
  !include ../../commons/style/use-cases.pu
  scale 3/4

  actor :amministratore: as A

  rectangle {
    together {
      usecase (AUC4.2.2) as "AUC4.2.2\nRifiuta richiesta\ncreazione organizzazione"
      usecase (AUC4.2.1) as "AUC4.2.1\nAccetta richiesta\ncreazione organizzazione"
    }
  }

  :A: -- AUC4.2.1
  :A: -- AUC4.2.2

  @enduml
  \end{plantuml}
  \caption{AUC4.2: Gestione richiesta creazione organizzazione}
  \label{fig:auc4_2}
\end{figure}

\begin{description}
  \item[Codice:] AUC4.2
  \item[Titolo:] Gestione richiesta creazione organizzazione
  \item[Attori primari:] amministratore
  \item[Precondizione:] l'owner deve aver richiesto di creare un organizzazione.
  \item[Postcondizione:] viene approvata o rifiutata la richiesta, e l'owner viene informato.
  \item[Scenario principale:]
  \begin{enumerate}
    \item viene richiesta la creazione di una nuova organizzazione, che vuole utilizzare \emph{Stalker};
    \item L'amministratore può accettare o rifiutare la richiesta.
  \end{enumerate}
\end{description}
% subsub:AUC4.2 (end)

\subsubsection{AUC4.2.1 - Accetta richiesta creazione organizzazione}%
\label{subsub:AUC4.2.1}
\begin{description}
  \item[Codice:] AUC4.2.1
  \item[Titolo:] Accetta richiesta creazione organizzazione
  \item[Attori primari:] amministratore
  \item[Precondizione:] l'owner deve aver richiesto di creare un organizzazione.
  \item[Postcondizione:] viene approvata la richiesta, e l'owner viene informato.
  \item[Scenario principale:]
  \begin{enumerate}
    \item  l'amministratore accetta la richiesta inviata dall'owner di creare una nuova organizzazione;
  \end{enumerate}
\end{description}
% subsub:AUC4.2.1 (end)

\subsubsection{AUC4.2.2 - Rifiuta richiesta creazione organizzazione}%
\label{subsub:AUC4.2.2}
\begin{description}
  \item[Codice:] AUC4.2.2
  \item[Titolo:] Rifiuta richiesta creazione organizzazione
  \item[Attori primari:] amministratore
  \item[Precondizione:] l'owner deve aver richiesto di creare un organizzazione;
  \item[Postcondizione:] viene rifiutata la richiesta, e l'owner viene informato;
  \item[Scenario principale:]
  \begin{enumerate}
    \item  l'amministratore rifiutala richiesta inviata dall'owner di creare una nuova organizzazione;
  \end{enumerate}
\end{description}
% subsub:AUC4.2.2 (end)

\subsubsection{AUC4.3 - Gestione richiesta modifica organizzazione}%
\label{subsub:AUC4.3}

\begin{figure}[h!]
  \centering
  \begin{plantuml}
  @startuml
  !include ../../commons/style/use-cases.pu
  scale 3/4

  actor :amministratore: as A

  rectangle {
    together {
      usecase (AUC4.3.2) as "AUC4.3.2\nRifiuta richiesta\nmodifica organizzazione"
      usecase (AUC4.3.1) as "AUC4.3.1\nAccetta richiesta\nmodifica organizzazione"
    }
  }

  :A: -- AUC4.3.1
  :A: -- AUC4.3.2

  @enduml
  \end{plantuml}
  \caption{AUC4.3: Gestione richiesta modifica organizzazione}
  \label{fig:auc4_3}
\end{figure}

\begin{description}
  \item[Codice:] AUC4.3
  \item[Titolo:] Gestione richiesta modifica organizzazione
  \item[Attori primari:] amministratore
  \item[Precondizione:] un gestore deve aver richiesto di modificare un'organizzazione.
  \item[Postcondizione:] viene approvata o rifiutata la richiesta, e il gestore viene informato.
  \item[Scenario principale:]
  \begin{enumerate}
    \item viene richiesta la modifica di una organizzazione;
    \item L'amministratore può accettare o rifiutare la richiesta.
  \end{enumerate}
\end{description}

\subsubsection{AUC4.3.1 - Accetta richiesta modifica organizzazione}%
\label{subsub:AUC4.3.1}
\begin{description}
  \item[Codice:] AUC4.3.1
  \item[Titolo:] Accetta richiesta modifica organizzazione
  \item[Attori primari:] amministratore
  \item[Precondizione:] il gestore deve aver richiesto di modificare un'organizzazione.
  \item[Postcondizione:] viene approvata la richiesta, e il gestore viene informato.
  \item[Scenario principale:]
  \begin{enumerate}
    \item  l'amministratore accetta la richiesta inviata dal gestore di modificare un'organizzazione;
  \end{enumerate}
\end{description}
% subsub:AUC4.3.1 (end)

\subsubsection{AUC4.3.2 - Rifiuta richiesta modifica organizzazione}%
\label{subsub:AUC4.3.2}
\begin{description}
  \item[Codice:] AUC4.3.2
  \item[Titolo:] Rifiuta richiesta modifica organizzazione
  \item[Attori primari:] amministratore
  \item[Precondizione:] il gestore deve aver richiesto di modificare un'organizzazione.
  \item[Postcondizione:] viene rifiutata la richiesta, e il gestore viene informato.
  \item[Scenario principale:]
  \begin{enumerate}
    \item l'amministratore rifiuta la richiesta inviata dal gestore di modificareun'organizzazione;
  \end{enumerate}
\end{description}
% subsub:AUC4.3.2 (end)
% subsub:AUC4.3 (end)

\subsubsection{AUC4.4 - Gestione richiesta aggiunta luogo}%
\label{subsub:AUC4.4}

\begin{figure}[h!]
  \centering
  \begin{plantuml}
  @startuml
  !include ../../commons/style/use-cases.pu
  scale 3/4

  actor :amministratore: as A

  rectangle {
    together {
      usecase (AUC4.4.2) as "AUC4.4.2\nRifiuta richiesta\naggiungi luogo"
      usecase (AUC4.4.1) as "AUC4.4.1\nAccetta richiesta\naggiungi luogo"
    }
  }

  :A: -- AUC4.4.1
  :A: -- AUC4.4.2

  @enduml
  \end{plantuml}
  \caption{AUC4.4: Gestione richiesta aggiunta luogo}
  \label{fig:auc4_4}
\end{figure}

\begin{description}
  \item[Codice:] AUC4.4
  \item[Titolo:] Gestione richiesta aggiunta luogo
  \item[Attori primari:] amministratore
  \item[Precondizione:] un gestore deve aver richiesto di aggiungere un luogo.
  \item[Postcondizione:] viene approvata o rifiutata la richiesta, e il gestore viene informato.
  \item[Scenario principale:]
  \begin{enumerate}
    \item viene richiesto di aggiungere un luogo;
    \item L'amministratore può accettare o rifiutare la richiesta.
  \end{enumerate}
\end{description}
% subsub:AUC4.4 (end)

\subsubsection{AUC4.4.1 - Accetta richiesta aggiungi luogo}%
\label{subsub:AUC4.4.1}
\begin{description}
  \item[Codice:] AUC4.4.1
  \item[Titolo:] Accetta richiesta aggiungi luogo
  \item[Attori primari:] amministratore
  \item[Precondizione:] il gestore deve aver richiesto di aggiungere un luogo.
  \item[Postcondizione:] viene approvata la richiesta, e il gestore viene informato.
  \item[Scenario principale:]
  \begin{enumerate}
    \item  l'amministratore accetta la richiesta inviata dal gestore di aggiungere un luogo.
  \end{enumerate}
\end{description}
% subsub:AUC4.4.1 (end)

\subsubsection{AUC4.4.2 - Rifiuta richiesta aggiungi luogo}%
\label{subsub:AUC4.4.2}
\begin{description}
  \item[Codice:] AUC4.4.2
  \item[Titolo:] Rifiuta richiesta aggiungi luogo
  \item[Attori primari:] amministratore
  \item[Precondizione:] il gestore deve aver richiesto di aggiungere un luogo.
  \item[Postcondizione:] viene rifiutata la richiesta, e il gestore viene informato.
  \item[Scenario principale:]
  \begin{enumerate}
    \item l'amministratore rifiuta la richiesta inviata dal gestore di aggiungere un luogo.
  \end{enumerate}
\end{description}
% subsub:AUC4.4.2 (end)

\subsubsection{AUC4.5 - Gestione richiesta modifica luogo}%
\label{subsub:AUC4.5}

\begin{figure}[h!]
  \centering
  \begin{plantuml}
  @startuml
  !include ../../commons/style/use-cases.pu
  scale 3/4

  actor :amministratore: as A

  rectangle {
    together {
      usecase (AUC4.5.2) as "AUC4.5.2\nRifiuta richiesta modifica luogo"
      usecase (AUC4.5.1) as "AUC4.5.1\nAccetta richiesta modifica luogo"
    }
  }

  :A: -- AUC4.5.1
  :A: -- AUC4.5.2

  @enduml
  \end{plantuml}
  \caption{AUC4.5: Gestione richiesta modifica luogo}
  \label{fig:auc }
\end{figure}

\begin{description}
  \item[Codice:] AUC4.5
  \item[Titolo:] Gestione richiesta modifica luogo
  \item[Attori primari:] amministratore
  \item[Precondizione:] un gestore deve aver richiesto di modificare un luogo.
  \item[Postcondizione:] viene approvata o rifiutata la richiesta, e il gestore viene informato.
  \item[Scenario principale:]
  \begin{enumerate}
    \item viene richiesto di modificare un luogo;
    \item L'amministratore può accettare o rifiutare la richiesta.
  \end{enumerate}
\end{description}
% subsub:AUC4.5 (end)

\subsubsection{AUC4.5.1 - Accetta richiesta modifica luogo}%
\label{subsub:AUC4.5.1}
\begin{description}
  \item[Codice:] AUC4.5.1
  \item[Titolo:] Accetta richiesta modifica luogo
  \item[Attori primari:] amministratore
  \item[Precondizione:] il gestore deve aver richiesto di modificare un luogo.
  \item[Postcondizione:] viene approvata la richiesta, e il gestore viene informato.
  \item[Scenario principale:]
  \begin{enumerate}
    \item  l'amministratore accetta la richiesta inviata dal gestore di modificare un luogo.
  \end{enumerate}
\end{description}
% subsub:AUC4.5.1 (end)

\subsubsection{AUC4.5.2 - Rifiuta richiesta modifica luogo}%
\label{subsub:AUC4.5.2}
\begin{description}
  \item[Codice:] AUC4.5.2
  \item[Titolo:] Rifiuta richiesta modifica luogo
  \item[Attori primari:] amministratore
  \item[Precondizione:] il gestore deve aver richiesto di modificare un luogo.
  \item[Postcondizione:] viene rifiutata la richiesta, e il gestore viene informato.
  \item[Scenario principale:]
  \begin{enumerate}
    \item l'amministratore rifiuta la richiesta inviata dal gestore di modificare un luogo.
  \end{enumerate}
\end{description}
% subsub:AUC4.5.2 (end)

\subsubsection{AUC4.6 - Gestione richiesta trasferimento di proprietà organizzazione}%
\label{subsub:AUC4.6}

\begin{figure}[h!]
  \centering
  \begin{plantuml}
  @startuml
  !include ../../commons/style/use-cases.pu
  scale 3/4

  actor :amministratore: as A

  rectangle {
    together {
      usecase (AUC4.6.2) as "AUC4.6.2\nRifiuta richiesta trasferimento\ndi proprietà organizzazione"
      usecase (AUC4.6.1) as "AUC4.6.1\nAccetta richiesta trasferimento\ndi proprietà organizzazione"
    }
  }

  :A: -- AUC4.6.1
  :A: -- AUC4.6.2

  @enduml
  \end{plantuml}
  \caption{AUC4.6: Gestione richiesta trasferimento di proprietà organizzazione}
  \label{fig:auc4_6}
\end{figure}

\begin{description}
  \item[Codice:] AUC4.6
  \item[Titolo:] Gestione richiesta trasferimento di proprietà organizzazione
  \item[Attori primari:] amministratore
  \item[Precondizione:] l'organizzazione deve essere già stata creata dal sistema, e deve essere stata fatta la richiesta trasferimento di proprietà.
  \item[Postcondizione:] viene approvata o rifiutata la richiesta, e l'owner viene informato.
  \item[Scenario principale:]
  \begin{enumerate}
    \item l'amministratore approva o rifiuta la richiesta di un trasferimento di proprietà di un'organizzazione.
  \end{enumerate}
\end{description}


\subsubsection{AUC4.6.1 - Accetta richiesta trasferimento di proprietà organizzazione}%
\label{subsub:AUC4.6.1}
\begin{description}
  \item[Codice:] AUC4.6.1
  \item[Titolo:] Accetta richiesta trasferimento di proprietà organizzazione
  \item[Attori primari:] amministratore
  \item[Precondizione:] l'owner deve aver fatto richiesta di trasferimento di proprietà di organizzazione.
  \item[Postcondizione:] viene approvata la richiesta, e l'owner viene informato.
  \item[Scenario principale:]
  \begin{enumerate}
    \item l'amministratore approva la richiesta di un trasferimento di proprietà di un'organizzazione.
  \end{enumerate}
\end{description}
% subsub:AUC4.6.1 (end)

\subsubsection{AUC4.6.2 - Rifiuta richiesta trasferimento di proprietà organizzazione}%
\label{subsub:AUC4.6.2}
\begin{description}
  \item[Codice:] AUC4.6.2
  \item[Titolo:] Rifiuta richiesta trasferimento di proprietà organizzazione
  \item[Attori primari:] amministratore
  \item[Precondizione:] l'owner deve aver fatto richiesta di trasferimento di proprietà di organizzazione.
  \item[Postcondizione:] viene rifiutata la richiesta, e l'owner viene informato.
  \item[Scenario principale:]
  \begin{enumerate}
    \item  l'amministratore rifiuta la richiesta di un trasferimento di proprietà di un'organizzazione.
  \end{enumerate}
\end{description}
% subsub:AUC4.6.2 (end)
% subsub:AUC4.6 (end)

\subsubsection{AUC4.7 - Gestione richiesta nuovo visualizzatore}%
\label{subsub:AUC4.7}

\begin{figure}[h!]
  \centering
  \begin{plantuml}
  @startuml
  !include ../../commons/style/use-cases.pu
  scale 3/4

  actor :amministratore: as A

  rectangle {
    together {
      usecase (AUC4.7.2) as "AUC4.7.2\nRifiuta richiesta\nnuovo visualizzatore"
      usecase (AUC4.7.1) as "AUC4.7.1\nAccetta richiesta\nnuovo visualizzatore"
    }
  }

  :A: -- AUC4.7.1
  :A: -- AUC4.7.2

  @enduml
  \end{plantuml}
  \caption{AUC4.7: Gestione richiesta nuovo visualizzatore}
  \label{fig:auc4_7}
\end{figure}

\begin{description}
  \item[Codice:] AUC4.7
  \item[Titolo:] Gestione richiesta nuovo visualizzatore
  \item[Attori primari:] amministratore
  \item[Precondizione:] deve essere stata fatta la richiesta di nuovo visualizzatore da parte di un owner.
  \item[Postcondizione:] viene approvata o rifiutata la richiesta, e l'owner viene informato.
  \item[Scenario principale:]
  \begin{enumerate}
    \item l'amministratore approva o rifiuta la richiesta di nuovo visualizzatore.
  \end{enumerate}
\end{description}

\subsubsection{AUC4.7.1 - Accetta richiesta nuovo visualizzatore}%
\label{subsub:AUC4.7.1}
\begin{description}
  \item[Codice:] AUC4.7.1
  \item[Titolo:] Accetta richiesta nuovo visualizzatore
  \item[Attori primari:] amministratore
  \item[Precondizione:] l'owner deve aver fatto richiesta di nuovo visualizzatore.
  \item[Postcondizione:] viene approvata la richiesta, e l'owner viene informato.
  \item[Scenario principale:]
  \begin{enumerate}
    \item l'amministratore approva la richiesta di nuovo visualizzatore.
  \end{enumerate}
\end{description}
% subsub:AUC4.7.1 (end)

\subsubsection{AUC4.7.2 - Rifiuta richiesta nuovo visualizzatore}%
\label{subsub:AUC4.7.2}
\begin{description}
  \item[Codice:] AUC4.7.2
  \item[Titolo:] Rifiuta richiesta nuovo visualizzatore
  \item[Attori primari:] amministratore
  \item[Precondizione:] l'owner deve aver fatto richiesta di nuovo visualizzatore.
  \item[Postcondizione:] viene rifiutata la richiesta, e l'owner viene informato.
  \item[Scenario principale:]
  \begin{enumerate}
    \item l'amministratore rifiuta la richiesta di nuovo visualizzatore.
  \end{enumerate}
\end{description}
% subsub:AUC4.7.2 (end)
% subsub:AUC4.7

\subsubsection{AUC4.8 - Gestione richiesta nuovo gestore}%
\label{subsub:AUC4.8}

\begin{figure}[h!]
  \centering
  \begin{plantuml}
  @startuml
  !include ../../commons/style/use-cases.pu
  scale 3/4

  actor :amministratore: as A

  rectangle {
    together {
      usecase (AUC4.8.2) as "AUC4.8.2\nRifiuta richiesta\nnuovo gestore"
      usecase (AUC4.8.1) as "AUC4.8.1\nAccetta richiesta\nnuovo gestore"
    }
  }

  :A: -- AUC4.8.1
  :A: -- AUC4.8.2

  @enduml
  \end{plantuml}
  \caption{AUC4.8: Gestione richiesta nuovo gestore}
  \label{fig:auc4_8}
\end{figure}

\begin{description}
  \item[Codice:] AUC4.8
  \item[Titolo:] Gestione richiesta nuovo gestore
  \item[Attori primari:] amministratore
  \item[Precondizione:] deve essere stata fatta la richiesta di nuovo gestore da parte di un owner.
  \item[Postcondizione:] viene approvata o rifiutata la richiesta, e l'owner viene informato.
  \item[Scenario principale:]
  \begin{enumerate}
    \item l'amministratore approva o rifiuta la richiesta di nuovo gestore.
  \end{enumerate}
\end{description}

\subsubsection{AUC4.8.1 - Accetta richiesta nuovo gestore}%
\label{subsub:AUC4.8.1}
\begin{description}
  \item[Codice:] AUC4.8.1
  \item[Titolo:] Accetta richiesta nuovo gestore
  \item[Attori primari:] amministratore
  \item[Precondizione:] l'owner deve aver fatto richiesta di nuovo gestore.
  \item[Postcondizione:] viene approvata la richiesta, e l'owner viene informato.
  \item[Scenario principale:]
  \begin{enumerate}
    \item l'amministratore approva la richiesta di nuovo gestore.
  \end{enumerate}
\end{description}
% subsub:AUC4.8.1 (end)

\subsubsection{AUC4.8.2 - Rifiuta richiesta nuovo gestore}%
\label{subsub:AUC4.8.2}
\begin{description}
  \item[Codice:] AUC4.8.2
  \item[Titolo:] Rifiuta richiesta nuovo gestore
  \item[Attori primari:] amministratore
  \item[Precondizione:] l'owner deve aver fatto richiesta di nuovo gestore.
  \item[Postcondizione:] viene rifiutata la richiesta, e l'owner viene informato.
  \item[Scenario principale:]
  \begin{enumerate}
    \item l'amministratore rifiuta la richiesta di nuovo gestore.
  \end{enumerate}
\end{description}
% subsub:AUC4.8.2 (end)
% subsub:AUC4.8

\subsubsection{AUC5 - Creazione amministratore}%
\label{subsub:AUC5}

\begin{figure}[h!]
  \centering
  \begin{plantuml}
  @startuml
  !include ../../commons/style/use-cases.pu
  scale 3/4

  actor :root: as A

  rectangle {
    together {
      usecase (AUC5.1) as "AUC5.1\nInserimento credenziali\n--\nExtension points:\nvisualizzazione errore se non\nvengono rispettati i requisiti"
      usecase (AUC5.2) as "AUC5.2\nVerifica credenziali"
      usecase (AUC5.3) as "AUC5.3\nVisualizza creazione fallita"
    }
  }

  :A: -- AUC5.1

  (AUC5.2) .up.|> (AUC5.1) : <<include>>
  (AUC5.3) .up.|> (AUC5.1) : <<extends>>

  @enduml
  \end{plantuml}
  \caption{AUC5: Creazione amministratore}
  \label{fig:auc5}
\end{figure}

\begin{description}
  \item[Codice:] AUC5
  \item[Titolo:] Creazione amministratore
  \item[Attori primari:] root
  \item[Precondizione:] devono essere specificate le credenziali del nuovo amministratore, che devono essere univoche.
  \item[Postcondizione:] l'amministratore viene creato.
  \item[Scenario principale:]
  \begin{enumerate}
    \item sorge la necessità di creare un nuovo amministratore per gestire \emph{Stalker}.
  \end{enumerate}
\end{description}


\subsubsection{AUC5.1 - Inserimento credenziali}%
\label{subsub:AUC5.1}

\begin{figure}[h!]
  \centering
  \begin{plantuml}
  @startuml
  !include ../../commons/style/use-cases.pu
  scale 3/4

  actor :root: as A

  rectangle {
    together {
      usecase (AUC5.1.3) as "AUC5.1.3\nConferma password"
      usecase (AUC5.1.2) as "AUC5.1.2\nInserimento nuova password"
      usecase (AUC5.1.1) as "AUC5.1.1\nInserimento nuova email"
    }
  }

  :A: -- AUC5.1.1
  :A: -- AUC5.1.2
  :A: -- AUC5.1.3

  @enduml
  \end{plantuml}
  \caption{AUC5.1: Inserimento credenziali}
  \label{fig:auc5_1}
\end{figure}

\begin{description}
  \item[Codice:] AUC5.1
  \item[Titolo:] Inserimento credenziali
  \item[Attori primari:] root
  \item[Precondizione:] il sistema deve rendere disponibile la pagina di creazione nuovo amministratore.
  \item[Postcondizione:] root ha inserito le credenziali dell'amministratore che vuole creare.
  \item[Scenario principale:]
  \begin{enumerate}
    \item \glossario{Root} inserisce le credenziali dell'amministratore che vuole creare.
  \end{enumerate}
  \item[Inclusioni:]
  \begin{enumerate}
    \item verifica credenziali\emph{[AUC5.2]}.
  \end{enumerate}
  \item[Estensioni:]
  \begin{enumerate}
    \item  visualizza creazione fallita\emph{[AUC5.3]}.
  \end{enumerate}
\end{description}
% subsub:AUC5.1 (end)

\subsubsection{AUC5.1.1 - Inserimento nuova email}%
\label{subsub:AUC5.1.1}
\begin{description}
  \item[Codice:] AUC5.1.1
  \item[Titolo:] Inserimento nuova email
  \item[Attori primari:] root
  \item[Precondizione:] il sistema ha reso disponibile il campo per l'inserimento dell'email.
  \item[Postcondizione:] root ha inserito l'email dell'amministratore che vuole creare.
  \item[Scenario principale:]
  \begin{enumerate}
    \item \glossario{Root} inserisce l'email dell'amministratore che vuole creare.
  \end{enumerate}
\end{description}
% subsub:AUC5.1.1 (end)

\subsubsection{AUC5.1.2 - Inserimento nuova password}%
\label{subsub:AUC5.1.2}
\begin{description}
  \item[Codice:] AUC5.1.2
  \item[Titolo:] Inserimento nuova password
  \item[Attori primari:] root
  \item[Precondizione:] il sistema ha reso disponibile il campo per l'inserimento della password.
  \item[Postcondizione:] root ha inserito la password dell'amministratore che vuole creare.
  \item[Scenario principale:]
  \begin{enumerate}
    \item \glossario{Root} inserisce la password dell'amministratore che vuole creare.
  \end{enumerate}
\end{description}
% subsub:AUC5.1.2 (end)

\subsubsection{AUC5.1.3 - Conferma password}%
\label{subsub:AUC5.1.3}
\begin{description}
  \item[Codice:] AUC5.1.3
  \item[Titolo:] Conferma password
  \item[Attori primari:] root
  \item[Precondizione:] il sistema ha reso disponibile il campo per l'inserimento del campo conferma password.
  \item[Postcondizione:] root ha inserito la password nel campo di conferma dell'amministratore che vuole creare.
  \item[Scenario principale:]
  \begin{enumerate}
    \item \glossario{Root} inserisce la password nel campo di conferma dell'amministratore che vuole creare.
  \end{enumerate}
\end{description}
% subsub:AUC5.1.3 (end)

\subsubsection{AUC5.2 - Verifica credenziali}%
\label{subsub:AUC5.2}
\begin{description}
  \item[Codice:] AUC5.2
  \item[Titolo:] Verifica credenziali
  \item[Attori primari:] root
  \item[Precondizione:] root invia richiesta di creazione nuovo amministratore.
  \item[Postcondizione:] il nuovo amministratore viene creato solo se i requisiti sono stati rispettati.
  \item[Scenario principale:]
  \begin{enumerate}
    \item il sistema verifica se le credenziali immesse rispettano i requisiti:
    \begin{enumerate}
      \item e-mail e password non siano vuoti;
      \item e-mail e password contengano solo i caratteri consentiti;
      \item lo username non esista.
    \end{enumerate}
  \end{enumerate}
\end{description}
% subsub:AUC5.2 (end)

\subsubsection{AUC5.3 - Visualizza creazione fallita}%
\label{subsub:AUC5.3}
\begin{description}
  \item[Codice:] AUC5.3
  \item[Titolo:] Visualizza creazione fallita
  \item[Attori primari:] root
  \item[Precondizione:] la verifica delle credenziali è fallita.
  \item[Postcondizione:] root visualizza un messaggio di creazione fallita.
  \item[Scenario principale:]
  \begin{enumerate}
    \item root cerca di creare un nuovo amministratore che non rispetta i requisiti.
  \end{enumerate}
\end{description}
% subsub:AUC5.3 (end)
% subsub:AUC5 (end)

\subsubsection{AUC6 - Gestione organizzazione}%
\label{subsub:AUC6}

\begin{figure}[h!]
  \centering
  \begin{plantuml}
  @startuml
  !include ../../commons/style/use-cases.pu
  scale 3/4

  actor :root: as A

  rectangle {
    together {
      usecase (AUC6.5) as "AUC6.5\nInvio richiesta aggiornamento\nlista organizzazioni"
      usecase (AUC6.4) as "AUC6.4\nSeleziona organizzazione"
      usecase (AUC6.3) as "AUC6.3\nModifica organizzazione"
      usecase (AUC6.2) as "AUC6.2\nEliminazione organizzazione"
      usecase (AUC6.1) as "AUC6.1\nCreazione organizzazione"
    }
  }

  :A: -- AUC6.1
  :A: -- AUC6.2
  :A: -- AUC6.3

  (AUC6.4) .up.|> (AUC6.2) : <<include>>
  (AUC6.4) .up.|> (AUC6.3) : <<include>>

  (AUC6.5) .up.|> (AUC6.1) : <<include>>
  (AUC6.5) .up.|> (AUC6.2) : <<include>>

  @enduml
  \end{plantuml}
  \caption{AUC6: Gestione organizzazione}
  \label{fig:auc6}
\end{figure}

\begin{description}
  \item[Codice:] AUC6
  \item[Titolo:] Gestione organizzazione
  \item[Attori primari:] root
  \item[Precondizione:] il sistema deve rendere disponibile la pagina di gestione organizzazione.
  \item[Postcondizione:] vengono gestite una o più organizzazioni.
  \item[Scenario principale:]
  \begin{enumerate}
    \item sorge la necessità di effettuare operazioni su una o più organizzazioni;
  \end{enumerate}
\end{description}

  \subsubsection{AUC6.1 - Creazione organizzazione}%
  \label{subsub:AUC6.1}

  \begin{figure}[h!]
    \centering
    \begin{plantuml}
    @startuml
    !include ../../commons/style/use-cases.pu
    scale 3/4

    actor :root: as A1
    actor :owner: as A2


    rectangle {
      together {
        usecase (AUC6.1.4) as "AUC6.1.4\nConfigurazione dettagli server LDAP"
        usecase (AUC6.1.3) as "AUC6.1.3\nInserisci descrizione organizzazione"
        usecase (AUC6.1.2) as "AUC6.1.2\nInserisci indirizzo organizzazione"
        usecase (AUC6.1.1) as "AUC6.1.1\nInserisci nome organizzazione"
      }
    }

    :A1: -- AUC6.1.1
    :A1: -- AUC6.1.2
    :A1: -- AUC6.1.3
    :A1: -- AUC6.1.4

    :A2: -- AUC6.1.1
    :A2: -- AUC6.1.2
    :A2: -- AUC6.1.3
    :A2: -- AUC6.1.4

    @enduml
    \end{plantuml}
    \caption{AUC6.1: Creazione organizzazione}
    \label{fig:auc6_1}
  \end{figure}

  \begin{description}
    \item[Codice:] AUC6.1
    \item[Titolo:] Creazione organizzazione
    \item[Attori primari:] root;
    \item[Precondizione:] l'organizzazione non deve esistere nella lista di \emph{Stalker}, deve essere specificato il suo nome.
    \item[Postcondizione:] l'organizzazione viene creata.
    \item[Scenario principale:] sorge la necessità di creare un'organizzazione, senza essere effettivamente richiesta;
    \item[Inclusioni:]
    \begin{enumerate}
      \item Invio richiesta aggiornamento lista organizzazioni \emph{[AUC6.5]};
    \end{enumerate}
  \end{description}
  % subsub:AUC6.1 (end)

  \subsubsection{AUC6.1.1 - Inserisci nome organizzazione}%
  \label{subsub:AUC6.1.1}
  \begin{description}
    \item[Codice:] AUC6.1.1
    \item[Titolo:] Inserisci nome organizzazione
    \item[Attori primari:] root,owner
    \item[Precondizione:] il sistema fornisce il campo di inserimento nome.
    \item[Postcondizione:] il nome viene opportunamente inserito.
    \item[Scenario principale:]
    \begin{enumerate}
      \item si vuole inserire il nome di un'organizzazione.
    \end{enumerate}

  \end{description}
  % subsub:AUC6.1.1 (end)

  \subsubsection{AUC6.1.2 - Inserisci indirizzo organizzazione}%
  \label{subsub:AUC6.1.2}
  \begin{description}
    \item[Codice:] AUC6.1.2
    \item[Titolo:] Inserisci indirizzo organizzazione
    \item[Attori primari:] root,owner
    \item[Precondizione:] il sistema fornisce il campo di inserimento indirizzo organizzazione.
    \item[Postcondizione:] l'indirizzo viene opportunamente inserito.
    \item[Scenario principale:]
    \begin{enumerate}
      \item si vuole inserire l'indirizzo di un'organizzazione.
    \end{enumerate}
  \end{description}
  % subsub:AUC6.1.2 (end)

  \subsubsection{AUC6.1.3 - Inserisci descrizione organizzazione}%
  \label{subsub:AUC6.1.3}
  \begin{description}
    \item[Codice:] AUC6.1.3
    \item[Titolo:] Inserisci descrizione organizzazione
    \item[Attori primari:] root,owner
    \item[Precondizione:] il sistema fornisce il campo di inserimento descrizione organizzazione.
    \item[Postcondizione:] la descrizione viene opportunamente inserito.
    \item[Scenario principale:]
    \begin{enumerate}
      \item si vuole inserire la descrizione di un'organizzazione.
    \end{enumerate}
  \end{description}
  % subsub:AUC6.1.3 (end)

  \subsubsection{AUC6.1.4 - Configurazione dettagli server LDAP}%
  \label{subsub:AUC6.1.4}
  \begin{description}
    \item[Codice:] AUC6.1.4
    \item[Titolo:] Configurazione dettagli server LDAP
    \item[Attori primari:] root,owner
    \item[Precondizione:] il sistema fornisce i campi per la configurazione del server LDAP.
    \item[Postcondizione:] il server LDAP è stato configurato.
    \item[Scenario principale:]
    \begin{enumerate}
      \item si vuole configurare i dettagli del \glossario{server LDAP} che le applicazioni dovranno utilizzare per registrarsi ad un'organizzazione
      \item se l'organizzazione è segnata come pubblica, i parametri del server LDAP non verranno configurati.
    \end{enumerate}
  \end{description}
  % subsub:AUC6.1.4 (end)

\subsubsection{AUC6.2 - Eliminazione organizzazione}%
\label{subsub:AUC6.2}
\begin{description}
  \item[Codice:] AUC6.2
  \item[Titolo:] Eliminazione organizzazione
  \item[Attori primari:] root
  \item[Precondizione:] deve essere stata selezionata l'organizzazione da eliminare, presente nella lista di \emph{Stalker}.
  \item[Postcondizione:] l'organizzazione viene eliminata.
  \item[Scenario principale:]
  \begin{enumerate}
    \item sorge la necessità di eliminare un'organizzazione, senza interagire con il suo owner;
  \end{enumerate}
  \item[Inclusioni:]
  \begin{enumerate}
    \item Seleziona organizzazione\emph{[AUC6.4]};
    \item Invio richiesta aggiornamento lista organizzazioni \emph{[AUC6.5]};
  \end{enumerate}
\end{description}
% subsub:AUC6.2 (end)

\subsubsection{AUC6.3 - Modifica organizzazione}%
\label{subsub:AUC6.3}

\begin{figure}[h!]
  \centering
  \begin{plantuml}
  @startuml
  !include ../../commons/style/use-cases.pu
  scale 3/4

  actor :root: as A

  rectangle {
    together {
      usecase (AUC6.3.5) as "AUC6.3.5\nModifica configurazione dettagli server LDAP"
      usecase (AUC6.3.4) as "AUC6.3.4\nGestione luoghi"
      usecase (AUC6.3.3) as "AUC6.3.3\nModifica descrizione organizzazione"
      usecase (AUC6.3.2) as "AUC6.3.2\nModifica indirizzo organizzazione"
      usecase (AUC6.3.1) as "AUC6.3.1\nModifica nome organizzazione"
    }
  }

  :A: -- AUC6.3.1
  :A: -- AUC6.3.2
  :A: -- AUC6.3.3
  :A: -- AUC6.3.4
  :A: -- AUC6.3.5

  @enduml
  \end{plantuml}
  \caption{AUC6.3: Modifica organizzazione}
  \label{fig:auc6_3}
\end{figure}

\begin{description}
  \item[Codice:] AUC6.3
  \item[Titolo:] Modifica organizzazione
  \item[Attori primari:] root
  \item[Precondizione:] deve essere stata selezionata l'organizzazione da modificare, presente nella lista di \emph{Stalker}.
  \item[Postcondizione:] l'organizzazione viene modificata.
  \item[Scenario principale:]
  \begin{enumerate}
    \item sorge la necessità di modificare un'organizzazione, senza interagire con il suo owner;
  \end{enumerate}
  \item[Inclusioni:]
  \begin{enumerate}
    \item Seleziona organizzazione\emph{[AUC6.4]};
  \end{enumerate}
\end{description}
% subsub:AUC6.3 (end)

\subsubsection{AUC6.3.1 - Modifica nome organizzazione}%
\label{subsub:AUC6.3.1}
\begin{description}
  \item[Codice:] AUC6.3.1
  \item[Titolo:] Modifica nome organizzazione
  \item[Attori primari:] root
  \item[Precondizione:] il sistema fornisce il campo di modifica nome.
  \item[Postcondizione:] il nome viene opportunamente modificato.
  \item[Scenario principale:]
  \begin{enumerate}
    \item si vuole modificare il nome di un'organizzazione.
  \end{enumerate}
\end{description}
% subsub:AUC6.3.1 (end)

\subsubsection{AUC6.3.2 - Modifica indirizzo organizzazione}%
\label{subsub:AUC6.3.2}
\begin{description}
  \item[Codice:] AUC6.3.2
  \item[Titolo:] Modifica indirizzo organizzazione
  \item[Attori primari:] root
  \item[Precondizione:] il sistema fornisce il campo di modifica indirizzo organizzazione.
  \item[Postcondizione:] l'indirizzo viene opportunamente modificato.
  \item[Scenario principale:]
  \begin{enumerate}
    \item si vuole modificare l'indirizzo di un'organizzazione.
  \end{enumerate}
\end{description}
% subsub:AUC6.3.2 (end)

\subsubsection{AUC6.3.3 - Modifica descrizione organizzazione}%
\label{subsub:AUC6.3.3}
\begin{description}
  \item[Codice:] AUC6.3.3
  \item[Titolo:] Modifica descrizione organizzazione
  \item[Attori primari:] root
  \item[Precondizione:] il sistema fornisce il campo di modifica descrizione organizzazione.
  \item[Postcondizione:] la descrizione viene opportunamente modificata.
  \item[Scenario principale:]
  \begin{enumerate}
    \item si vuole modificare la descrizione di un'organizzazione.
  \end{enumerate}
\end{description}
% subsub:AUC6.3.3 (end)

\subsubsection{AUC6.3.4 - Gestione luoghi}%
\label{subsub:AUC6.3.4}

\begin{figure}[h!]
  \centering
  \begin{plantuml}
  @startuml
  !include ../../commons/style/use-cases.pu
  scale 3/4

  actor :root: as A

  rectangle {
    together {
      usecase (AUC6.3.4.4) as "AUC6.3.4.4\nAUC6.3.4.4\nSeleziona luogo"
      usecase (AUC6.3.4.3) as "AUC6.3.4.3\nModifica luogo"
      usecase (AUC6.3.4.2) as "AUC6.3.4.2\nEliminazione luogo"
      usecase (AUC6.3.4.1) as "AUC6.3.4.1\nAggiungi luogo"
    }
  }

  :A: -- AUC6.3.4.1
  :A: -- AUC6.3.4.2
  :A: -- AUC6.3.4.3
  :A: -- AUC6.3.4.4

  @enduml
  \end{plantuml}
  \caption{AUC6.3.4: Gestione luoghi}
  \label{fig:auc6_3_4}
\end{figure}

\begin{description}
  \item[Codice:] AUC6.3.4
  \item[Titolo:] Gestione luoghi
  \item[Attori primari:] root
  \item[Precondizione:] il sistema deve rendere disponibile la pagina di gestione dei luoghi di un'organizzazione.
  \item[Postcondizione:] vengono gestiti i luoghi di un'organizzazione.
  \item[Scenario principale:]
  \begin{enumerate}
    \item sorge la necessità di effettuare operazioni sul luogo di un'organizzazione, e viene offerta la possibilità di selezionarlo;
  \end{enumerate}
\end{description}
% subsub:AUC6.3.4 (end)

\subsubsection{AUC6.3.4.1 - Aggiungi luogo}%
\label{subsub:AUC6.3.4.1}

\begin{figure}[h!]
  \centering
  \begin{plantuml}
  @startuml
  !include ../../commons/style/use-cases.pu
  scale 3/4

  actor :root: as A

  rectangle {
    together {
      usecase (AUC6.3.4.1.1) as "AUC6.3.4.1.1\nInserisci coordinate geografiche"
    }
  }

  :A: -- AUC6.3.4.1.1

  @enduml
  \end{plantuml}
  \caption{AUC6.3.4.1: Aggiungi luogo}
  \label{fig:auc6_3_4_1}
\end{figure}

\begin{description}
  \item[Codice:] AUC6.3.4.1
  \item[Titolo:] aggiungi luogo
  \item[Attori primari:] root
  \item[Precondizione:] il \glossario{luogo} dell'organizzazione deve non esistere.
  \item[Postcondizione:] il luogo dell'organizzazione viene aggiunto.
  \item[Scenario principale:]
  \begin{enumerate}
    \item sorge la necessità di aggiungere un luogo ad un'organizzazione, senza interagire con il suo owner;
  \end{enumerate}
\end{description}
% subsub:AUC6.3.4.1 (end)

\subsubsection{AUC6.3.4.1.1 - Inserisci coordinate geografiche}%
\label{subsub:AUC6.3.4.1.1}
\begin{description}
  \item[Codice:] AUC6.3.4.1.1
  \item[Titolo:] Inserisci coordinate geografiche
  \item[Attori primari:] root
  \item[Precondizione:] il sistema deve fornire i campi relativi alle coordinate geografiche.
  \item[Postcondizione:] i campi relativi alle coordinate geografiche sono stati riempiti.
  \item[Scenario principale:]
  \begin{enumerate}
    \item si vogliono inserire le coordinate di un nuovo luogo;
  \end{enumerate}
\end{description}
% subsub:AUC6.3.4.1.1 (end)

\subsubsection{AUC6.3.4.2 - Eliminazione luogo}%
\label{subsub:AUC6.3.4.2}
\begin{description}
  \item[Codice:] AUC6.3.4.2
  \item[Titolo:] Eliminazione luogo
  \item[Attori primari:] root
  \item[Precondizione:] il luogo dell'organizzazione deve essere presente in \emph{Stalker}.
  \item[Postcondizione:] il luogo dell'organizzazione viene eliminato.
  \item[Scenario principale:]
  \begin{enumerate}
    \item sorge la necessità di eliminare un luogo di un'organizzazione, senza interagire con il suo owner;
  \end{enumerate}
  \item[Inclusioni:]
  \begin{enumerate}
    \item Seleziona luogo\emph{[AUC6.3.4.4]};
  \end{enumerate}
\end{description}
% subsub:AUC6.3.4.2 (end)

\subsubsection{AUC6.3.4.3 - Modifica luogo}%
\label{subsub:AUC6.3.4.3}

\begin{figure}[h!]
  \centering
  \begin{plantuml}
  @startuml
  !include ../../commons/style/use-cases.pu
  scale 3/4

  actor :root: as A

  rectangle {
    together {
      usecase (AUC6.3.4.3.1) as "AUC6.3.4.3.1\nModifica coordinate geografiche"
    }
  }

  :A: -- AUC6.3.4.3.1

  @enduml
  \end{plantuml}
  \caption{AUC6.3.4.3: Modifica luogo}
  \label{fig:auc6_3_4_3}
\end{figure}

\begin{description}
  \item[Codice:] AUC6.3.4.3
  \item[Titolo:] Modifica luogo
  \item[Attori primari:] root
  \item[Precondizione:] il luogo dell'organizzazione deve essere presente in \emph{Stalker}.
  \item[Postcondizione:] il luogo dell'organizzazione viene modificato.
  \item[Scenario principale:]
  \begin{enumerate}
    \item sorge la necessità di modificare un luogo di un'organizzazione, senza interagire con il suo owner;
  \end{enumerate}
  \item[Inclusioni:]
  \begin{enumerate}
    \item Seleziona luogo\emph{[AUC6.3.4.4]};
  \end{enumerate}
\end{description}
% subsub:AUC6.3.4.3 (end)

\subsubsection{AUC6.3.4.3.1 - Modifica coordinate geografiche}%
\label{subsub:AUC6.3.4.3.1}
\begin{description}
  \item[Codice:] AUC6.3.4.3.1
  \item[Titolo:] Modifica coordinate geografiche
  \item[Attori primari:] root
  \item[Precondizione:] il sistema deve fornire i campi relativi alle coordinate geografiche.
  \item[Postcondizione:] i campi relativi alle coordinate geografiche sono stati modificati.
  \item[Scenario principale:]
  \begin{enumerate}
    \item si vogliono modificare le coordinate di un luogo;
  \end{enumerate}
\end{description}
% subsub:AUC6.3.4.3.1 (end)

\subsubsection{AUC6.3.4.4 - Seleziona luogo}%
\label{subsub:AUC6.3.4.4}
\begin{description}
  \item[Codice:] AUC6.3.4.4
  \item[Titolo:] Seleziona luogo
  \item[Attori primari:] root
  \item[Precondizione:] il sistema deve mostrare la lista dei luoghi all'interno di una organizzazione.
  \item[Postcondizione:] viene scelto il luogo desiderato.
  \item[Scenario principale:]
  \begin{enumerate}
    \item sorge la necessità di effettuare operazioni sul luogo di un'organizzazione, e viene offerta la possibilità di selezionarlo;
  \end{enumerate}
\end{description}
% subsub:AUC6.3.4.4 (end)
% subsub:AUC6.3.4 (end)

\subsubsection{AUC6.3.5 - Modifica configurazione dettagli server LDAP}%
  \label{subsub:AUC6.3.5}
  \begin{description}
    \item[Codice:] AUC6.3.5
    \item[Titolo:] Configurazione dettagli server LDAP
    \item[Attori primari:] root,owner
    \item[Precondizione:] il sistema fornisce i campi per la configurazione del server LDAP.
    \item[Postcondizione:] il server LDAP è stato configurato.
    \item[Scenario principale:]
    \begin{enumerate}
      \item si vuole modificare la configurazione del server LDAP.
    \end{enumerate}
  \end{description}
  % subsub:AUC6.3.5 (end)

\subsubsection{AUC6.4 - Seleziona organizzazione}%
\label{subsub:AUC6.4}
\begin{description}
  \item[Codice:] AUC6.4
  \item[Titolo:] Seleziona organizzazione
  \item[Attori primari:] root
  \item[Precondizione:] il sistema deve mostrare la lista di organizzazioni in \emph{Stalker}.
  \item[Postcondizione:] viene scelta l'organizzazione desiderata.
  \item[Scenario principale:]
  \begin{enumerate}
    \item sorge la necessità di effettuare operazioni su un'organizzazione, e viene offerta la possibilità di selezionarla;
  \end{enumerate}
\end{description}
% subsub:AUC6.4 (end)

\subsubsection{AUC6.5 - Invio richiesta aggiornamento lista organizzazioni}%
\label{subsub:AUC6.5}
\begin{description}
  \item[Codice:] AUC6.5
  \item[Titolo:] Invio richiesta aggiornamento lista organizzazioni
  \item[Attori primari:] root
  \item[Precondizione:] il sistema mostra la pagina di creazione o eliminazione di un'organizzazione.
  \item[Postcondizione:] la nuova lista delle organizzazioni viene inviata a tutte le applicazioni mobile.
  \item[Scenario principale:]
  \begin{enumerate}
    \item si vuole creare oppure eliminare un'organizzazione specifica
    \item ad operazione avvenuta, la lista delle organizzazione viene aggiornata e inviata a tutti gli utenti che hanno installato l'applicazione mobile.
  \end{enumerate}
\end{description}
% subsub:AUC6.5 (end)
% subsub:AUC6 (end)

\subsubsection{AUC7 - Eliminazione account}%
\label{subsub:AUC7}

\begin{figure}[h!]
  \centering
  \begin{plantuml}
  @startuml
  !include ../../commons/style/use-cases.pu
  scale 3/4

  actor :root: as A

  rectangle {
    together {
      usecase (AUC7.1) as "AUC7.1\nSeleziona account"
    }
  }

  :A: -- AUC7.1

  @enduml
  \end{plantuml}
  \caption{AUC7: Eliminazione account}
  \label{fig:auc7}
\end{figure}

\begin{description}
  \item[Codice:] AUC7
  \item[Titolo:] Eliminazione account
  \item[Attori primari:] root;
  \item[Precondizione:] deve essere stato selezionato l'\glossario{account} da eliminare, che deve esistere in \emph{Stalker};
  \item[Postcondizione:] l'account selezionato è stato eliminato.
  \item[Scenario principale:]
  \begin{enumerate}
    \item sorge la necessità di eliminare un account per sconosciuti motivi;
  \end{enumerate}
\end{description}

\subsubsection{AUC7.1 - Seleziona account}%
\label{subsub:AUC7.1}
\begin{description}
  \item[Codice:] AUC7.1
  \item[Titolo:] Seleziona account
  \item[Attori primari:] root;
  \item[Precondizione:] il sistema deve rendere disponibile la lista degli account registrati a \emph{Stalker}.
  \item[Postcondizione:] l'account è stato selezionato.
  \item[Scenario principale:]
  \begin{enumerate}
    \item \glossario{Root} seleziona l'account da eliminare;
  \end{enumerate}
\end{description}
% subsub:AUC7.1 (end)
% subsub:AUC7 (end)

\subsubsection{AUC8 - Query sull'organizzazione}%
\label{subsub:AUC8}

\begin{figure}[h!]
  \centering
  \begin{plantuml}
  @startuml
  !include ../../commons/style/use-cases.pu
  scale 3/4

  actor :visualizzatore: as A

  rectangle {
    together {
      usecase (AUC8) as "AUC8\nQuery sull'organizzazione"
    }
  }

  :A: -- AUC8

  @enduml
  \end{plantuml}
  \caption{AUC8: Query sull'organizzazione}
  \label{fig:auc8}
\end{figure}

\begin{description}
  \item[Codice:] AUC8
  \item[Titolo:] Query sull'organizzazione
  \item[Attori primari:] visualizzatore;
  \item[Precondizione:] il sistema risponde correttamente alle interrogazioni;
  \item[Postcondizione:] il visualizzatore ottiene le informazioni di cui ha bisogno.
  \item[Scenario principale:]
  \begin{enumerate}
    \item il visualizzatore vuole avere delle informazioni riguardanti l'organizzazione su cui opera;
  \end{enumerate}
\end{description}
% subsub:AUC8 (end)

\subsubsection{AUC9 - Query sul dipendente}%
\label{subsub:AUC9}

\begin{figure}[h!]
  \centering
  \begin{plantuml}
  @startuml
  !include ../../commons/style/use-cases.pu
  scale 3/4

  actor :visualizzatore: as A

  rectangle {
    together {
      usecase (AUC9) as "AUC9\nQuery sul dipendente"
    }
  }

  :A: -- AUC9

  @enduml
  \end{plantuml}
  \caption{AUC9: Query sul dipendente}
  \label{fig:auc9}
\end{figure}

\begin{description}
  \item[Codice:] AUC9
  \item[Titolo:] Query sul dipendente
  \item[Attori primari:] visualizzatore;
  \item[Precondizione:] il sistema risponde correttamente alle interrogazioni;
  \item[Postcondizione:] il visualizzatore ottiene le informazioni di cui ha bisogno.
  \item[Scenario principale:]
  \begin{enumerate}
    \item il visualizzatore vuole avere delle informazioni riguardanti il dipendente dell'organizzazione su cui opera;
  \end{enumerate}
  \item[Inclusioni:]
  \begin{enumerate}
    \item Seleziona organizzazione\emph{[AUC6.4]};
  \end{enumerate}
\end{description}
% subsub:AUC9 (end)

\subsubsection{AUC10 - Richiesta su luoghi}%
\label{subsub:AUC10}

\begin{figure}[h!]
  \centering
  \begin{plantuml}
  @startuml
  !include ../../commons/style/use-cases.pu
  scale 3/4

  actor :gestore: as A

  rectangle {
    together {
      usecase (AUC10.1) as "AUC10.1\nRichiesta aggiunta luogo"
      usecase (AUC10.2) as "AUC10.2\nRichiesta rimozione luogo"
      usecase (AUC10.3) as "AUC10.3\nRichiesta modifica luogo"
      usecase (AUC10.4) as "AUC10.4\nRichiesta modifica organizzazione"
    }
    together {
      usecase (AUC6.3.1) as "AUC6.3.1\nModifica nome organizzazione"
      usecase (AUC6.3.2) as "AUC6.3.2\nModifica indirizzo organizzazione"
      usecase (AUC6.3.3) as "AUC6.3.3\nModifica descrizione organizzazione"
      usecase (AUC6.3.4.3.1) as "AUC6.3.4.3.1\nModifica coordinate geografiche"
      usecase (AUC6.3.4.1.1) as "AUC6.3.4.1.1\nInserisci coordinate geografiche"
      usecase (AUC6.3.5) as "AUC6.3.5\nModifica configurazione dettagli server LDAP"
    }
  }

  :A: -- AUC10.1
  :A: -- AUC10.2
  :A: -- AUC10.3
  :A: -- AUC10.4


  (AUC6.3.1) .up.|> (AUC10.4) : <<include>>
  (AUC6.3.2) .up.|> (AUC10.4) : <<include>>
  (AUC6.3.3) .up.|> (AUC10.4) : <<include>>
  (AUC6.3.4.3.1) .up.|> (AUC10.3) : <<include>>
  (AUC6.3.4.1.1) .up.|> (AUC10.1) : <<include>>
  (AUC6.3.5) .up.|> (AUC10.4) : <<include>>

  @enduml
  \end{plantuml}
  \caption{AUC10: Richiesta su luoghi}
  \label{fig:auc10}
\end{figure}

\begin{description}
  \item[Codice:] AUC10
  \item[Titolo:] Richiesta su luoghi
  \item[Attori primari:] gestore
  \item[Precondizione:] il sistema deve rendere disponibile la pagina delle richieste sui luoghi.
  \item[Postcondizione:] la richiesta di aggiunta di un nuovo luogo viene posta.
  \item[Scenario principale:]
  \begin{enumerate}
    \item il gestore vuole effettuare richieste sui luoghi dell'organizzazione su cui opera;
  \end{enumerate}
\end{description}

\subsubsection{AUC10.1 - Richiesta aggiunta luogo}%
\label{subsub:AUC10.1}
\begin{description}
  \item[Codice:] AUC10.1
  \item[Titolo:] Richiesta aggiunta luogo
  \item[Attori primari:] gestore;
  \item[Precondizione:] il luogo da aggiungere non deve già esistere;
  \item[Postcondizione:] la richiesta di aggiunta di un nuovo luogo viene posta.
  \item[Scenario principale:]
  \begin{enumerate}
    \item il gestore vuole aggiungere un nuovo luogo all'organizzazione su cui opera;
  \end{enumerate}
  \item[Inclusioni:]
  \begin{enumerate}
    \item Inserisci coordinate geografiche \emph{[AUC6.3.4.1.1]};
  \end{enumerate}
\end{description}
% subsub:AUC10.1 (end)

\subsubsection{AUC10.2 - Richiesta rimozione luogo}%
\label{subsub:AUC10.2}
\begin{description}
  \item[Codice:] AUC10.2
  \item[Titolo:] Richiesta rimozione luogo
  \item[Attori primari:] gestore
  \item[Precondizione:] il luogo da eliminare deve esistere.
  \item[Postcondizione:] la richiesta di eliminazione di un luogo viene posta.
  \item[Scenario principale:]
  \begin{enumerate}
    \item il gestore vuole eliminare un luogo all'organizzazione su cui opera;
  \end{enumerate}
\end{description}
% subsub:AUC10.2 (end)

\subsubsection{AUC10.3 - Richiesta modifica luogo}%
\label{subsub:AUC10.3}
\begin{description}
  \item[Codice:] AUC10.3
  \item[Titolo:] Richiesta modifica luogo
  \item[Attori primari:] gestore
  \item[Precondizione:] il luogo da modificare deve esistere.
  \item[Postcondizione:] la richiesta di modifica di un luogo viene posta.
  \item[Scenario principale:]
  \begin{enumerate}
    \item il gestore vuole modificare un luogo dell'organizzazione su cui opera;
  \end{enumerate}
  \item[Inclusioni:]
  \begin{enumerate}
    \item Modifica coordinate geografiche \emph{[AUC6.3.4.3.1]};
  \end{enumerate}
\end{description}
% subsub:AUC10.3 (end)
% subsub:AUC10 (end)

\subsubsection{AUC10.4 - Richiesta modifica organizzazione}%
\label{subsub:AUC10.4}
\begin{description}
  \item[Codice:] AUC10.4
  \item[Titolo:] Richiesta modifica organizzazione
  \item[Attori primari:] gestore
  \item[Precondizione:] i parametri devono essere effettivamente modificati.
  \item[Postcondizione:] la richiesta di modifica viene posta.
  \item[Scenario principale:]
  \begin{enumerate}
    \item il gestore vuole modificare i parametri dell'organizzazione su cui opera;
  \end{enumerate}
  \item[Inclusioni:]
  \begin{enumerate}
    \item Modifica nome organizzazione \emph{[AUC6.3.1]}.
    \item Modifica indirizzo organizzazione \emph{[AUC6.3.2]}.
    \item Modifica descrizione organizzazione \emph{[AUC6.3.3]}.
    \item Gestione luoghi \emph{[AUC6.3.4]}
    \item Modifica configurazione dettagli server LDAP \emph{[AUC6.3.5]}
  \end{enumerate}
\end{description}
% subsub:AUC10.4 (end)

\subsubsection{AUC11 - Richiesta creazione organizzazione}%
\label{subsub:AUC11}

\begin{figure}[h!]
  \centering
  \begin{plantuml}
  @startuml
  !include ../../commons/style/use-cases.pu
  scale 3/4

  actor :owner: as A

  rectangle {
    together {
      usecase (AUC11) as "AUC11\nRichiesta creazione organizzazione"
    }
    together {
      usecase (AUC6.1.1) as "AUC6.1.1\nInserisci nome organizzazione"
      usecase (AUC6.1.2) as "AUC6.1.2\nInserisci indirizzo organizzazione"
      usecase (AUC6.1.3) as "AUC6.1.3\nInserisci descrizione organizzazione"
      usecase (AUC6.1.4) as "AUC6.1.4\nConfigurazione dettagli server LDAP"
    }
  }

  :A: -- AUC11

  (AUC6.1.1) .up.|> (AUC11) : <<include>>
  (AUC6.1.2) .up.|> (AUC11) : <<include>>
  (AUC6.1.3) .up.|> (AUC11) : <<include>>
  (AUC6.1.4) .up.|> (AUC11) : <<include>>

  @enduml
  \end{plantuml}
  \caption{AUC11: Richiesta creazione organizzazione}
  \label{fig:auc11}
\end{figure}

\begin{description}
  \item[Codice:] AUC11
  \item[Titolo:] Richiesta creazione organizzazione
  \item[Attori primari:] owner
  \item[Precondizione:] l' organizzazione non deve già esistere.
  \item[Postcondizione:] la richiesta di creare una nuova organizzazione è stata posta.
  \item[Scenario principale:]
  \begin{enumerate}
    \item l'owner vuole creare una nuova organizzazione.
  \end{enumerate}
  \item[Inclusioni:]
  \begin{enumerate}
    \item Inserimento nome organizzazione \emph{[AUC6.1.1]}.
    \item Inserimento indirizzo organizzazione \emph{[AUC6.1.2]}.
    \item Inserimento descrizione organizzazione \emph{[AUC6.1.3]}.
    \item Configurazione dettagli server LDAP \emph{[AUC6.1.4]}
  \end{enumerate}
\end{description}
% subsub:AUC11 (end)

\subsubsection{AUC12 - Richiesta cessione proprietà organizzazione}%
\label{subsub:AUC12}

\begin{figure}[h!]
  \centering
  \begin{plantuml}
  @startuml
  !include ../../commons/style/use-cases.pu
  scale 3/4

  actor :owner: as A

  rectangle {
    together {
      usecase (AUC12.1) as "AUC12.1\nInserimento e-mail nuovo owner"
    }
  }

  :A: -- AUC12.1

  @enduml
  \end{plantuml}
  \caption{AUC12: Richiesta cessione proprietà organizzazione}
  \label{fig:auc12}
\end{figure}

\begin{description}
  \item[Codice:] AUC12
  \item[Titolo:] Richiesta cessione proprietà organizzazione
  \item[Attori primari:] owner
  \item[Precondizione:] il futuro owner deve esistere.
  \item[Postcondizione:] la richiesta di cedere l'organizzazione è stata posta.
  \item[Scenario principale:]
  \begin{enumerate}
    \item l' owner vuole cedere l'organizzazione ad un'altro owner;
  \end{enumerate}
\end{description}
% subsub:AUC12 (end)

\subsubsection{AUC12.1 - Inserimento e-mail nuovo owner}%
\label{subsub:AUC12.1}
\begin{description}
  \item[Codice:] AUC12.1
  \item[Titolo:] Inserimento e-mail nuovo owner
  \item[Attori primari:] owner
  \item[Precondizione:] il sistema deve rendere disponibile il campo per l'inserimento del nuovo owner.
  \item[Postcondizione:] il campo relativo al nuovo owner viene riempito.
  \item[Scenario principale:]
  \begin{enumerate}
    \item l' owner inserisce l'e-mail del nuovo owner;
  \end{enumerate}
\end{description}
% subsub:AUC12.1 (end)

\subsubsection{AUC13 - Richiesta aggiunta visualizzatore}%
\label{subsub:AUC13}

\begin{figure}[h!]
  \centering
  \begin{plantuml}
  @startuml
  !include ../../commons/style/use-cases.pu
  scale 3/4

  actor :owner: as A

  rectangle {
    together {
      usecase (AUC13.1) as "AUC13.1\nInserimento e-mail visualizzatore"
    }
  }

  :A: -- AUC13.1

  @enduml
  \end{plantuml}
  \caption{AUC13: Richiesta aggiunta visualizzatore}
  \label{fig:auc13}
\end{figure}

\begin{description}
  \item[Codice:] AUC13
  \item[Titolo:] Richiesta aggiunta visualizzatore
  \item[Attori primari:] owner
  \item[Precondizione:] il nuovo super utente non deve già esistere come visualizzatore.
  \item[Postcondizione:] viene inviata la richiesta di nomina visualizzatore all'amministratore.
  \item[Scenario principale:] l' owner vuole aggiungere un visualizzatore alla sua organizzazione.
\end{description}
% subsub:AUC13 (end)

\subsubsection{AUC13.1 - Inserimento e-mail visualizzatore}%
\label{subsub:AUC13.1}
\begin{description}
  \item[Codice:] AUC13.1
  \item[Titolo:] Inserimento e-mail visualizzatore
  \item[Attori primari:] owner
  \item[Precondizione:] il sistema deve rendere disponibile il campo per l'inserimento del nuovo visualizzatore.
  \item[Postcondizione:] il campo relativo al visualizzatore viene riempito.
  \item[Scenario principale:]
  \begin{enumerate}
    \item l' owner inserisce l'e-mail del nuovo visualizzatore;
  \end{enumerate}
\end{description}
% subsub:AUC13.1 (end)

\subsubsection{AUC14 - Richiesta aggiunta gestore}%
\label{subsub:AUC14}

\begin{figure}[h!]
  \centering
  \begin{plantuml}
  @startuml
  !include ../../commons/style/use-cases.pu
  scale 3/4

  actor :owner: as A

  rectangle {
    together {
      usecase (AUC14.1) as "AUC14.1\nInserimento e-mail gestore"
    }
  }

  :A: -- AUC14.1

  @enduml
  \end{plantuml}
  \caption{AUC14: Richiesta aggiunta gestore}
  \label{fig:auc14}
\end{figure}

\begin{description}
  \item[Codice:] AUC14
  \item[Titolo:] Richiesta aggiunta gestore
  \item[Attori primari:] owner
  \item[Precondizione:] il nuovo super utente non deve già esistere come gestore.
  \item[Postcondizione:] viene inviata la richiesta di nomina gestore all'amministratore.
  \item[Scenario principale:] l' owner vuole aggiungere un gestore alla sua organizzazione, la nomina deve essere approvata da un
  amministratore.
\end{description}
% subsub:AUC14 (end)

\subsubsection{AUC14.1 - Inserimento e-mail gestore}%
\label{subsub:AUC14.1}
\begin{description}
  \item[Codice:] AUC14.1
  \item[Titolo:] Inserimento e-mail gestore
  \item[Attori primari:] owner
  \item[Precondizione:] il sistema deve rendere disponibile il campo per l'inserimento del nuovo gestore.
  \item[Postcondizione:] il campo relativo al gestore viene riempito.
  \item[Scenario principale:]
  \begin{enumerate}
    \item l' owner inserisce l'e-mail del nuovo gestore;
  \end{enumerate}
\end{description}
% subsub:AUC14.1 (end)

% par:Owner (end)
\end{document}
