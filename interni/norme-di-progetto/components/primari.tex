\documentclass[../norme-di-progetto.tex]{subfiles}
\begin{document}
\subsection{Descrizione}
\label{sub:descrizione}
Sono i processi essenziali durante il ciclo di vita del software. ISO 12207-1995 ne definisce cinque:
\begin{itemize}
	\item Acquisizione
	\item Fornitura
	\item Sviluppo
	\item Gestione operativa
	\item Manutenzione
\end{itemize}
\subsection{Fornitura}
\label{sub:fornitura}
\subsubsection{Finalità}
\label{subs:finalità}
Tale processo viene istanziato da Grupp0ne per potersi aggiudicare il capitolato attraverso un accordo con il committente e la successiva stipulazione di un contratto.
\subsubsection{Descrizione}
\label{subs:descrizione}
Il processo di fornitura contiene le attività e i compiti del fornitore. Esso gestisce il rapporto tra il fornitore e il cliente ed inoltre amministra le procedure e le risorse necessarie per lo sviluppo del piano di progetto. È composto da diverse attività:
\begin{itemize}
\item Inizializzazione
\item Preparazione della risposta
\item Contratto
\item Pianificazione
\item Esecuzione e controllo
\item Revisione e valutazione
\item Consegna e completamento
\end{itemize}
\subsubsection{Studio di Fattibilità}
\label{par:studio di fattibilità}
Grupp0ne si impegna a consegnare un sintetico documento contenente una breve descrizione, i pregi e le criticità riscontrate in ciascuno dei capitolati. Lo studio di fattibilità è redatto dall'\glossario{\textit{amministratore}}
con l'aiuto degli \glossario{\textit{analisti}} e ha l'obiettivo di esporre quali capitolati sono stati presi in considerazione con le relative motivazioni. Il documento è così articolato:
\begin{itemize}
\item \textbf{Descrizione}: si fornisce una breve descrizione del problema esposto dal proponente.
\item \textbf{Finalità del progetto}: si descrive che cosa bisogna realizzare.
\item \textbf{Tecnologie interessate}: si mostrano le tecnologie da considerare in fase di sviluppo.
\item \textbf{Aspetti positivi}: si espongono i pregi del capitolato.
\item \textbf{Criticità e fattori di rischio}: si illustrano i difetti osservati studiando il capitolato.
\end{itemize}
\subsubsection{Piano di progetto}
\label{par:piano di progetto}
\subsubsection{Piano di qualifica}
\label{par:piano di qualifica}
\subsubsection{Incontri con il proponente}
\label{subs:incontri con il proponente}
\subsection{Sviluppo}
\label{sub:sviluppo}
\subsubsection{Finalità}
\label{subs:finalità}
\subsubsection{Descrizione}
\label{subs:descrizione}
\subsubsection{Attività}
\label{subs:attività}
\paragraph{Analisi dei requisiti}
\label{par:analisi dei requisiti}
\subparagraph{Classificazione dei requisiti}
\label{subp:classificazione dei requisiti}
\subparagraph{Diagrammi UML}
\label{subp:diagrammi UML}
\paragraph{Progettazione}
\label{par:progettazione}
\subsubsection{Strumenti}
\label{subs:strumnti}
\end{document}