\documentclass[casi-duso]{subfiles}

%\renewcommand{\commons}{../../../commons}

\begin{document}

%UC1 (start)
\subsubsection{UC1 - Apertura applicazione mobile}
\label{subsub:uc1utente}
%inserire diagramma UC1
\begin{itemize}
  \item \textbf{Caso d’uso:} UC1 
  \item \textbf{Titolo:} Apertura applicazione mobile
  \item \textbf{Attori primari:} utente non registrato, utente non autenticato
  \item \textbf{Precondizioni:} l'utente deve aprire l'\glossario{applicazione mobile} e non è riconosciuto dal sistema.
  \item \textbf{Postcondizioni:} se l'utente non è registrato al \glossario{servizio} offerto dall'applicazione, allora si deve registrare e diventa un utente non autenticato; 
  altrimenti se lo è già, allora si deve autenticare e può utilizzare le funzionalità messe a disposizione dall'applicazione. 
  \item \textbf{Scenario principale:} 
  \begin{enumerate}
    \item 
  \end{enumerate}  
\end{itemize}
% subsub:uc1utente (end)

%UC2 (start)
\subsubsection{UC2 - Registrazione}
\label{subsub:uc2utente}
%inserire diagramma UC2
\begin{itemize}
  \item \textbf{Caso d’uso:} UC1 
  \item \textbf{Titolo:} Registrazione
  \item \textbf{Attori primari:} utente non registrato
  \item \textbf{Precondizioni:} 
  \item \textbf{Postcondizioni:} 
  \item \textbf{Scenario principale:} 
  \begin{enumerate}
    \item 
  \end{enumerate}  
  \item \textbf{Inclusioni:} 
  \begin{enumerate}
    \item 
  \end{enumerate}  
\end{itemize}
% subsub:uc2utente (end)

%UC2.1 (start)
\subsubsection{UC2.1 - Visualizzazione EULA}
\label{subsub:uc2.1utente}
\begin{itemize}
  \item \textbf{Caso d’uso:} UC2.1 
  \item \textbf{Titolo:} Visualizzazione EULA
  \item \textbf{Attori primari:} utente non registrato
  \item \textbf{Precondizioni:} 
  \item \textbf{Postcondizioni:}  
  \item \textbf{Scenario principale:} 
  \begin{enumerate}
    \item 
  \end{enumerate}  
  \item \textbf{Inclusioni:} 
  \begin{enumerate}
    \item 
  \end{enumerate}   
\end{itemize}
% subsub:uc2.1utente (end)

%UC2.2 (start)
\subsubsection{UC2.1.1 - Conferma EULA}
\label{subsub:uc2.1.1utente}
\begin{itemize}
  \item \textbf{Caso d’uso:} UC2.1.1 
  \item \textbf{Titolo:} Conferma EULA
  \item \textbf{Attori primari:} utente non registrato
  \item \textbf{Precondizioni:} 
  \item \textbf{Postcondizioni:}  
  \item \textbf{Scenario principale:} 
  \begin{enumerate}
    \item 
  \end{enumerate}    
\end{itemize}
% subsub:uc2.1.1utente (end)

%UC2.2 (start)
\subsubsection{UC2.2 - Inserimento dati registrazione}
\label{subsub:uc2.2utente}
%inserire diagramma UC2.2
\begin{itemize}
  \item \textbf{Caso d’uso:} UC2.1 
  \item \textbf{Titolo:} Visualizzazione EULA
  \item \textbf{Attori primari:} utente non registrato
  \item \textbf{Precondizioni:} 
  \item \textbf{Postcondizioni:}  
  \item \textbf{Scenario principale:} 
  \begin{enumerate}
    \item 
  \end{enumerate}
  \item \textbf{Estensioni:} 
  \begin{enumerate}
    \item 
  \end{enumerate}     
\end{itemize}
% subsub:uc2.2utente (end)

%UC2.2.1 (start)
\subsubsection{UC2.2.1 - Registrazione email}
\label{subsub:uc2.2.1utente}
\begin{itemize}
  \item \textbf{Caso d’uso:} UC2.2.1 
  \item \textbf{Titolo:} Registrazione email
  \item \textbf{Attori primari:} utente non registrato
  \item \textbf{Precondizioni:} 
  \item \textbf{Postcondizioni:}  
  \item \textbf{Scenario principale:} 
  \begin{enumerate}
    \item 
  \end{enumerate}
  \item \textbf{Estensioni:} 
  \begin{enumerate}
    \item 
  \end{enumerate}     
\end{itemize}
% subsub:uc2.2.1utente (end)

%UC2.2.2 (start)
\subsubsection{UC2.2.2 - Registrazione password}
\label{subsub:uc2.2.2utente}
\begin{itemize}
  \item \textbf{Caso d’uso:} UC2.2.2 
  \item \textbf{Titolo:} Registrazione password
  \item \textbf{Attori primari:} utente non registrato
  \item \textbf{Precondizioni:} 
  \item \textbf{Postcondizioni:}  
  \item \textbf{Scenario principale:} 
  \begin{enumerate}
    \item 
  \end{enumerate}
  \item \textbf{Estensioni:} 
  \begin{enumerate}
    \item 
  \end{enumerate}     
\end{itemize}
% subsub:uc2.2.2utente (end)

%UC2.2.3 (start)
\subsubsection{UC2.2.3 - Conferma password}
\label{subsub:uc2.2.3utente}
\begin{itemize}
  \item \textbf{Caso d’uso:} UC2.2.1 
  \item \textbf{Titolo:} Conferma password
  \item \textbf{Attori primari:} utente non registrato
  \item \textbf{Precondizioni:} 
  \item \textbf{Postcondizioni:}  
  \item \textbf{Scenario principale:} 
  \begin{enumerate}
    \item 
  \end{enumerate}
  \item \textbf{Estensioni:} 
  \begin{enumerate}
    \item 
  \end{enumerate}     
\end{itemize}
% subsub:uc2.2.3utente (end)

%UC2.2.4 (start)
\subsubsection{UC2.2.4 - Inserimento dati anagrafici}
\label{subsub:uc2.2.4utente}
\begin{itemize}
  \item \textbf{Caso d’uso:} UC2.2.4 
  \item \textbf{Titolo:} Inserimento dati anagrafici
  \item \textbf{Attori primari:} utente non registrato
  \item \textbf{Precondizioni:} 
  \item \textbf{Postcondizioni:}  
  \item \textbf{Scenario principale:} 
  \begin{enumerate}
    \item 
  \end{enumerate}
  \item \textbf{Estensioni:} 
  \begin{enumerate}
    \item 
  \end{enumerate}     
\end{itemize}
% subsub:uc2.2.4utente (end)

%UC2.2.5 (start)
\subsubsection{UC2.2.5 - Informazioni registrazione non valide}
\label{subsub:uc2.2.5utente}
\begin{itemize}
  \item \textbf{Caso d’uso:} UC2.2.5 
  \item \textbf{Titolo:} informazioni registrazione non valide
  \item \textbf{Attori primari:} utente non registrato
  \item \textbf{Precondizioni:} 
  \item \textbf{Postcondizioni:}  
  \item \textbf{Scenario principale:} 
  \begin{enumerate}
    \item 
  \end{enumerate}
  \item \textbf{Estensioni:} 
  \begin{enumerate}
    \item 
  \end{enumerate}     
\end{itemize}
% subsub:uc2.2.5utente (end)

%UC3 (start)
\subsubsection{UC3 - Autenticazione}
\label{subsub:uc2utente}
%inserire diagramma UC3
\begin{itemize}
  \item \textbf{Caso d’uso:} UC3 
  \item \textbf{Titolo:} Autenticazione
  \item \textbf{Attori primari:} utente non autenticato
  \item \textbf{Precondizioni:} 
  \item \textbf{Postcondizioni:}
  \item \textbf{Scenario principale:} 
  \begin{enumerate}
    \item 
  \end{enumerate}  
  \item \textbf{Estensioni:} 
  \begin{enumerate}
    \item 
  \end{enumerate}  
\end{itemize}
% subsub:uc3utente (end)

%UC3.1 (start)
\subsubsection{UC3.1 - Inserimento email}
\label{subsub:uc2.1utente}
\begin{itemize}
  \item \textbf{Caso d’uso:} UC3.1 
  \item \textbf{Titolo:} Inserimento email
  \item \textbf{Attori primari:} utente non autenticato
  \item \textbf{Precondizioni:} 
  \item \textbf{Postcondizioni:}
  \item \textbf{Scenario principale:} 
  \begin{enumerate}
    \item 
  \end{enumerate}  
  \item \textbf{Estensioni:} 
  \begin{enumerate}
    \item 
  \end{enumerate}  
\end{itemize}
% subsub:uc3.1utente (end)

%UC3.2 (start)
\subsubsection{UC3.2 - inserimento password}
\label{subsub:uc3.2utente}
\begin{itemize}
  \item \textbf{Caso d’uso:} UC3.2 
  \item \textbf{Titolo:} Inserimento password
  \item \textbf{Attori primari:} utente non autenticato
  \item \textbf{Precondizioni:} 
  \item \textbf{Postcondizioni:}
  \item \textbf{Scenario principale:} 
  \begin{enumerate}
    \item 
  \end{enumerate}  
  \item \textbf{Estensioni:} 
  \begin{enumerate}
    \item 
  \end{enumerate}  
\end{itemize}
% subsub:uc3.2utente (end)

%UC3.3 (start)
\subsubsection{UC3.3 - Informazioni autenticazione non valide}
\label{subsub:uc3.3utente}
\begin{itemize}
  \item \textbf{Caso d’uso:} UC3.3 
  \item \textbf{Titolo:} Informazioni autenticazione non valide
  \item \textbf{Attori primari:} utente non autenticato
  \item \textbf{Precondizioni:} 
  \item \textbf{Postcondizioni:}
  \item \textbf{Scenario principale:} 
  \begin{enumerate}
    \item 
  \end{enumerate}  
  \item \textbf{Estensioni:} 
  \begin{enumerate}
    \item 
  \end{enumerate}  
\end{itemize}
% subsub:uc3.3utente (end)

%UC4 (start)
\subsubsection{UC4 - Recupero credenziali}
\label{subsub:uc4utente}
%inserire diagramma UC4
\begin{itemize}
  \item \textbf{Caso d’uso:} UC4 
  \item \textbf{Titolo:} Recupero credenziali
  \item \textbf{Attori primari:} utente non autenticato
  \item \textbf{Precondizioni:} 
  \item \textbf{Postcondizioni:}
  \item \textbf{Scenario principale:} 
  \begin{enumerate}
    \item 
  \end{enumerate} 
\end{itemize}
% subsub:uc4utente (end)

%UC4.1 (start)
\subsubsection{UC4.1 - Recupero password}
\label{subsub:uc4.1utente}
\begin{itemize}
  \item \textbf{Caso d’uso:} UC4.1 
  \item \textbf{Titolo:} Recupero password
  \item \textbf{Attori primari:} utente non autenticato
  \item \textbf{Precondizioni:} 
  \item \textbf{Postcondizioni:}
  \item \textbf{Scenario principale:} 
  \begin{enumerate}
    \item 
  \end{enumerate} 
\end{itemize}
% subsub:uc4.1utente (end)

%UC5 (start)
\subsubsection{UC5 - Visualizzazione organizzazioni}
\label{subsub:uc5utente}
%inserire diagramma UC5
\begin{itemize}
  \item \textbf{Caso d’uso:} UC5 
  \item \textbf{Titolo:} Visualizzazione organizzazioni
  \item \textbf{Attori primari:} utente autenticato
  \item \textbf{Precondizioni:} 
  \item \textbf{Postcondizioni:}
  \item \textbf{Scenario principale:} 
  \begin{enumerate}
    \item 
  \end{enumerate}  
  \item \textbf{Inclusioni:} 
  \begin{enumerate}
    \item 
  \end{enumerate}
  \item \textbf{Estensioni:} 
  \begin{enumerate}
    \item 
  \end{enumerate}  
\end{itemize}
% subsub:uc5utente (end)

%UC5.1 (start)
\subsubsection{UC5.1 - Visualizzazione lista organizzazioni}
\label{subsub:uc5utente}
\begin{itemize}
  \item \textbf{Caso d’uso:} UC5 
  \item \textbf{Titolo:} Visualizzazione lista organizzazioni
  \item \textbf{Attori primari:} utente autenticato
  \item \textbf{Precondizioni:} 
  \item \textbf{Postcondizioni:}
  \item \textbf{Scenario principale:} 
  \begin{enumerate}
    \item 
  \end{enumerate}  
  \item \textbf{Inclusioni:} 
  \begin{enumerate}
    \item 
  \end{enumerate}
  \item \textbf{Estensioni:} 
  \begin{enumerate}
    \item 
  \end{enumerate}  
\end{itemize}
% subsub:uc5.1utente (end)

%UC5.1.1 (start)
\subsubsection{UC5.1.1 - Aggiornamento lista organizzazioni}
\label{subsub:uc5utente}
\begin{itemize}
  \item \textbf{Caso d’uso:} UC5 
  \item \textbf{Titolo:} Aggiornamento lista organizzazioni
  \item \textbf{Attori primari:} utente autenticato
  \item \textbf{Precondizioni:} 
  \item \textbf{Postcondizioni:}
  \item \textbf{Scenario principale:} 
  \begin{enumerate}
    \item 
  \end{enumerate}  
\end{itemize}
% subsub:uc5.1.1utente (end)

%UC6 (start)
\subsubsection{UC6 - Visualizzazione errore rete mancante}
\label{subsub:uc6utente}
\begin{itemize}
  \item \textbf{Caso d’uso:} UC6
  \item \textbf{Titolo:} Visualizzazione errore rete mancante
  \item \textbf{Attori primari:} utente autenticato
  \item \textbf{Precondizioni:} 
  \item \textbf{Postcondizioni:}
  \item \textbf{Scenario principale:} 
  \begin{enumerate}
    \item 
  \end{enumerate}  
\end{itemize}
% subsub:uc6utente (end)

%UC7 (start)
\subsubsection{UC7 - Collegamento organizzazione}
\label{subsub:uc7utente}
%inserire diagramma UC7
\begin{itemize}
  \item \textbf{Caso d’uso:} UC7
  \item \textbf{Titolo:} Collegamento organizzazione
  \item \textbf{Attori primari:} utente autenticato, in particolare utente non collegato
  \item \textbf{Precondizioni:} 
  \item \textbf{Postcondizioni:}
  \item \textbf{Scenario principale:} 
  \begin{enumerate}
    \item 
  \end{enumerate}  
  \item \textbf{Estensioni:} 
  \begin{enumerate}
    \item 
  \end{enumerate}  
\end{itemize}
% subsub:uc7utente (end)

%UC7.1 (start)
\subsubsection{UC7.1 - Selezionamento di un'organizzazione}
\label{subsub:uc7utente}
\begin{itemize}
  \item \textbf{Caso d’uso:} UC7.1
  \item \textbf{Titolo:} Selezionamento di un'organizzazione
  \item \textbf{Attori primari:} utente autenticato, in particolare utente non collegato
  \item \textbf{Precondizioni:} 
  \item \textbf{Postcondizioni:}
  \item \textbf{Scenario principale:} 
  \begin{enumerate}
    \item 
  \end{enumerate}  
  \item \textbf{Estensioni:} 
  \begin{enumerate}
    \item 
  \end{enumerate}  
\end{itemize}
% subsub:uc7.1utente (end)

%UC8 (start)
\subsubsection{UC8 - Scollegamento organizzazione}
\label{subsub:uc8utente}
%inserire diagramma UC8
\begin{itemize}
  \item \textbf{Caso d’uso:} UC8
  \item \textbf{Titolo:} Scollegamento organizzazione
  \item \textbf{Attori primari:} utente autenticato, in particolare utente collegato
  \item \textbf{Precondizioni:} 
  \item \textbf{Postcondizioni:}
  \item \textbf{Scenario principale:} 
  \begin{enumerate}
    \item 
  \end{enumerate}  
  \item \textbf{Estensioni:} 
  \begin{enumerate}
    \item 
  \end{enumerate}  
\end{itemize}
% subsub:uc8utente (end)

%UC8.1 (start)
\subsubsection{UC8.1 - Uscita da un'organizzazione}
\label{subsub:uc8.1utente}
\begin{itemize}
  \item \textbf{Caso d’uso:} UC8.1
  \item \textbf{Titolo:} Uscita da un'organizzazione
  \item \textbf{Attori primari:} utente autenticato, in particolare utente collegato
  \item \textbf{Precondizioni:} 
  \item \textbf{Postcondizioni:}
  \item \textbf{Scenario principale:} 
  \begin{enumerate}
    \item 
  \end{enumerate}  
  \item \textbf{Estensioni:} 
  \begin{enumerate}
    \item 
  \end{enumerate}  
\end{itemize}
% subsub:uc8.1utente (end)

%UC9 (start)
\subsubsection{UC9 - Passaggio noto/incognito}
\label{subsub:uc9utente}
%inserire diagramma UC9
\begin{itemize}
  \item \textbf{Caso d’uso:} UC9
  \item \textbf{Titolo:} Passaggio noto/incognito
  \item \textbf{Attori primari:} utente autenticato, in particolare utente collegato
  \item \textbf{Precondizioni:} 
  \item \textbf{Postcondizioni:}
  \item \textbf{Scenario principale:} 
  \begin{enumerate}
    \item 
  \end{enumerate}  
  \item \textbf{Estensioni:} 
  \begin{enumerate}
    \item 
  \end{enumerate}  
\end{itemize}
% subsub:uc9utente (end)

%UC9.1 (start)
\subsubsection{UC9.1 - Scelta noto o incognito}
\label{subsub:uc9.1utente}
%inserire diagramma UC9.1
\begin{itemize}
  \item \textbf{Caso d’uso:} UC9.1
  \item \textbf{Titolo:} Scelta noto o incognito
  \item \textbf{Attori primari:} utente collegato, in particolare utente noto ed utente incognito
  \item \textbf{Precondizioni:} 
  \item \textbf{Postcondizioni:}
  \item \textbf{Scenario principale:} 
  \begin{enumerate}
    \item 
  \end{enumerate}
\end{itemize}
% subsub:uc9.1utente (end)

%UC9.1.1 (start)
\subsubsection{UC9.1.1 - Passaggio a incognito}
\label{subsub:uc9.1.1utente}
\begin{itemize}
  \item \textbf{Caso d’uso:} UC9.1.1
  \item \textbf{Titolo:} Passaggio a incognito
  \item \textbf{Attori primari:} utente collegato, in particolare utente noto
  \item \textbf{Precondizioni:} 
  \item \textbf{Postcondizioni:}
  \item \textbf{Scenario principale:} 
  \begin{enumerate}
    \item 
  \end{enumerate}
\end{itemize}
% subsub:uc9.1.1utente (end)

%UC9.1.2 (start)
\subsubsection{UC9.1.2 - Passaggio a noto}
\label{subsub:uc9.1.2utente}
\begin{itemize}
  \item \textbf{Caso d’uso:} UC9.1.2
  \item \textbf{Titolo:} Passaggio a noto
  \item \textbf{Attori primari:} utente collegato, in particolare utente incognito
  \item \textbf{Precondizioni:} 
  \item \textbf{Postcondizioni:}
  \item \textbf{Scenario principale:} 
  \begin{enumerate}
    \item 
  \end{enumerate}
\end{itemize}
% subsub:uc9.1.2utente (end)

%UC10 (start)
\subsubsection{UC10 - Storico utente}
\label{subsub:uc10utente}
%inserire diagramma UC10
\begin{itemize}
  \item \textbf{Caso d’uso:} UC10
  \item \textbf{Titolo:} Storico utente
  \item \textbf{Attori primari:} utente autenticato
  \item \textbf{Precondizioni:} 
  \item \textbf{Postcondizioni:}
  \item \textbf{Scenario principale:} 
  \begin{enumerate}
    \item 
  \end{enumerate}  
  \item \textbf{Estensioni:} 
  \begin{enumerate}
    \item 
  \end{enumerate}  
\end{itemize}
% subsub:uc10utente (end)

%UC10.1 (start)
\subsubsection{UC10.1 - Visualizzazione storico accessi}
\label{subsub:uc10.1utente}
\begin{itemize}
  \item \textbf{Caso d’uso:} UC10.1
  \item \textbf{Titolo:} Visualizzazione storico accessi
  \item \textbf{Attori primari:} utente autenticato
  \item \textbf{Precondizioni:} 
  \item \textbf{Postcondizioni:}
  \item \textbf{Scenario principale:} 
  \begin{enumerate}
    \item 
  \end{enumerate}  
  \item \textbf{Estensioni:} 
  \begin{enumerate}
    \item 
  \end{enumerate}  
\end{itemize}
% subsub:uc10.1utente (end)

%UC11 (start)
\subsubsection{UC11 - Tempo utente nell'organizzazione}
\label{subsub:uc11utente}
%inserire diagramma UC11
\begin{itemize}
  \item \textbf{Caso d’uso:} UC11
  \item \textbf{Titolo:} Tempo utente nell'organizzazione
  \item \textbf{Attori primari:} utente autenticato, in particolare utente collegato
  \item \textbf{Precondizioni:} 
  \item \textbf{Postcondizioni:}
  \item \textbf{Scenario principale:} 
  \begin{enumerate}
    \item 
  \end{enumerate}  
  \item \textbf{Estensioni:} 
  \begin{enumerate}
    \item 
  \end{enumerate}  
\end{itemize}
% subsub:uc11utente (end)

%UC11.1 (start)
\subsubsection{UC11.1 - Visualizzazione tempo trascorso nell'organizzazione corrente}
\label{subsub:uc11utente}
\begin{itemize}
  \item \textbf{Caso d’uso:} UC11
  \item \textbf{Titolo:} Visualizzazione tempo trascorso nell'organizzazione corrente
  \item \textbf{Attori primari:} utente autenticato, in particolare utente collegato
  \item \textbf{Precondizioni:} 
  \item \textbf{Postcondizioni:}
  \item \textbf{Scenario principale:} 
  \begin{enumerate}
    \item 
  \end{enumerate}  
  \item \textbf{Estensioni:} 
  \begin{enumerate}
    \item 
  \end{enumerate}  
\end{itemize}
% subsub:uc11.1utente (end)



%-------------------------------------------------------------------prendere spunto dai casi d'uso precedenti-------------------------------------------------------

%UC1 (start)
\subsubsection{UC1 - Apertura applicazione mobile}
\label{subsub:uc1utente}

%inserire diagramma UC1

\begin{itemize}
  \item \textbf{Caso d’uso:} UC1 
  \item \textbf{Titolo:} apertura applicazione mobile
  \item \textbf{Attori primari:} utente non autenticato, utente non registrato
  \item \textbf{Precondizioni:} l'utente deve aprire l'\glossario{applicazione mobile} e non è riconosciuto dal sistema.
  \item \textbf{Postcondizioni:} se l'utente non è registrato al \glossario{servizio} offerto dall'applicazione, allora si deve registrare e diventa un utente non autenticato; 
  altrimenti se lo è già, allora si deve autenticare e può utilizzare le funzionalità messe a disposizione dall'applicazione. 
  \item \textbf{Scenario principale:} 
  \begin{enumerate}
    \item l'utente non registrato deve accedere all'apposito form di registrazione per poter registrare le proprie credenziali nel sistema.
    \item l'utente non autenticato è già registrato al sistema, ma non è ancora collegato al sistema: deve accedere all'apposito form di autenticazione per usufruire delle funzionalità
    messe a disposizione dal servizio.
  \end{enumerate}  
  \item \textbf{Inclusioni:} 
  \begin{enumerate}
    \item al momento della registrazione, l'utente visualizzerà l'\glossario{EULA} e dovrà accettarla per continuare il processo di registrazione.
    \item a registrazione ultimata, l'utente registrato sarà rilanciato alla schermata iniziale, dove avrà la possiblità di autenticarsi al servizio.
  \end{enumerate}  
  \item \textbf{Estensioni:} 
  \begin{enumerate}
    \item se l'utente registrato, nella registrazione non inserisce correttamente le credenziali username e password, verrà visualizzato un errore. 
    \item se l'utente non autenticato inserisce erroneamente le credenziali nel form di autenticazione, allora verrà visualizzato un errore.
  \end{enumerate}  
\end{itemize}
% subsub:uc1utente (end)

\subsubsection{UC1.1 - Registrazione}
\label{subsub:uc1.1utente}
\begin{itemize}
  \item \textbf{Codice:} UC1.1
  \item \textbf{Titolo:} registrazione
  \item \textbf{Attori primari:} utente non registrato
  \item \textbf{Precondizione:} l'utente non deve essere registrato al servizio offerto dall'applicazione.
  \item \textbf{Postcondizione:} l'utente ha le sue credenziali per poter autenticarsi al sistema.
  \item \textbf{Scenario Principale:} 
  \begin{enumerate}
    \item l'utente accede alla schermata di registrazione e inserisce le credenziali richieste dal servizio, ovvero username e password.
  \end{enumerate} 
  \item \textbf{Estensioni:}
  \begin{enumerate}
    \item se l'utente viola le regole d'inserimento delle credenziali, allora l'applicazione mobile presenterà un messaggio d'errore e la registrazione al servizio non procede fintantochè
    le credenziali non sono valide.
  \end{enumerate}
  \item \textbf{Inclusioni:}
  \begin{enumerate}
    \item prima della registrazione, l'utente visualizza l'EULA e deve accettarla se vuole registrarsi al servizio.
    \item se la registrazione è ultimata e va a buon fine, l'utente visualizzerà la schermata iniziale dell'applicazione mobile e potrà autenticarsi al servizio.
  \end{enumerate}
\end{itemize}
% subsub:uc1.1utente (end)

\subsubsection{UC1.1.1 - Visualizzazione EULA}
\label{subsub:uc1.1.2utente}
\begin{itemize}
  \item \textbf{Codice:} UC1.1.1
  \item \textbf{Titolo:} visualizzazione EULA
  \item \textbf{Attori primari:} utente non registrato
  \item \textbf{Precondizione:} l'utente si trova nella schermata di registrazione e visualizza l'EULA\@.
  \item \textbf{Postcondizione:} l'utente accetta l'EULA e la registrazione prosegue.
  \item \textbf{Scenario Principale:}
  \begin{enumerate}
    \item l'utente visualizza l'EULA e ha la possiblità di accettarla oppure non accettarla: nel caso la accetti, allora la registrazione può proseguire.
  \end{enumerate}
\end{itemize}
% subsub:uc1.1.1utente (end)

\subsubsection{UC1.1.2 - Informazioni non valide}
\label{subsub:uc1.1.2utente}
\begin{itemize}
  \item \textbf{Codice:} UC1.1.2
  \item \textbf{Titolo:} informazioni non valide
  \item \textbf{Attori primari:} utente non registrato
  \item \textbf{Precondizione:} l'utente si trova nella schermata di registrazione e non rispetta le regole d'inserimento delle credenziali.
  \item \textbf{Postcondizione:} l'applicazione mobile comunica all'utente l'errore.
  \item \textbf{Scenario Principale:} 
  \begin{enumerate}
    \item l'utente visualizza il messaggio d'errore, in quanto le informazioni inserite non sono valide.
  \end{enumerate}
\end{itemize}
% subsub:uc1.1.2utente (end)

\subsubsection{UC1.2 - Autenticazione}
\label{subsub:uc1.2utente}
\begin{itemize}
  \item \textbf{Codice:} UC1.2
  \item \textbf{Titolo:} autenticazione
  \item \textbf{Attori primari:} utente non autenticato
  \item \textbf{Precondizione:} l'utente è registrato al sistema e si trova nella schermata iniziale dell'applicazione mobile.
  \item \textbf{Postcondizione:} l'utente ha inserito le credenziali corrette ed è un utente autenticato.
  \item \textbf{Scenario Principale:}
  \begin{enumerate}
    \item l'utente deve inserire le proprie credenziali per autenticarsi al servizio.
    \item l'utente ha inserito correttamente le credenziali, quindi visualizza la schermata successiva, con tutte le funzionalità messe a disposizione dall'applicazione mobile.
  \end{enumerate}
  \item \textbf{Estensioni:}
  \begin{enumerate}
    \item se l'utente inserisce erroneamente le credenziali per l'autenticazione, allora visualizzerà un messaggio d'errore.
  \end{enumerate}
  \item \textbf{Inclusioni:}
  \begin{enumerate}
    \item se l'utente è la prima volta che accede all'applicazione mobile ed effettua la registrazione delle proprie credenziali, se tutto va a buon fine accede direttamente alla schermata
    iniziale per l'autenticazione.
  \end{enumerate}
\end{itemize}
% subsub:uc1.2utente (end)

\subsubsection{UC1.2.1 - Credenziali errate}
\label{subsub:uc1.2.1utente}
\begin{itemize}
  \item \textbf{Codice:} UC1.2.1
  \item \textbf{Titolo:} credenziali errate
  \item \textbf{Attori primari:} utente non autenticato
  \item \textbf{Precondizione:} l'utente inserisce erroneamente le credenziali per l'autenticazione.
  \item \textbf{Postcondizione:} l'applicazione mobile comunica all'utente l'errore.
  \item \textbf{Scenario Principale:}
  \begin{enumerate}
    \item l'utente visualizza il messaggio d'errore, in quanto le credenziali inserite non sono valide.
  \end{enumerate}
\end{itemize}
% subsub:uc1.2.1utente (end)


%UC2 (start)
\subsubsection{UC2 - Schermata utente autenticato}
\label{subsub:uc2utente}

% integrare diagramma UC2

\begin{itemize}
  \item \textbf{Caso d’uso:} UC2
  \item \textbf{Titolo:} schermata utente autenticato
  \item \textbf{Attore primario:} utente autenticato, che può essere utente non collegato oppure utente collegato
  \item \textbf{Precondizioni:} l'utente deve essere registrato al servizio offerto dall'applicazione mobile.
  \item \textbf{Postcondizioni:} l'utente è autenticato.
  \item \textbf{Scenario principale:} 
  \begin{enumerate}
    \item l'\glossario{utente autenticato}, inizialmente, è un \glossario{utente non collegato} ad alcuna organizzazione, quindi visualizza la lista delle organizzazioni che hanno 
    l'autorizzazione da parte di Stalker a rintracciare gli utenti registrati al servizio.
    \item l'utente non collegato ha l'autorizzazione a \glossario{collegarsi} ad un'organizzazione e diventa un \glossario{utente collegato}, rintracciabile dall'organizzazione selezionata.
    \item se l'utente è collegato, può collegarsi ad altre organizzazioni, provocando lo scollegamento da quella corrente; altrimenti, può scollegarsi e non essere più rintracciato da
    alcuna organizzazione.
    \item se l'utente è collegato, ha la possibilità di rendersi incognito oppure noto all'interno di un'organizzazione.
    \item un'utente autenticato ha la possibilità di visualizzare il suo personale \glossario{storico degli accessi}.
    \item un'utente autenticato ha la possibilità di visualizzare un cronometro che indica il tempo trascorso all'interno di un'organizzazione.
  \end{enumerate}
  \item \textbf{Inclusioni:}
  \begin{enumerate}
    \item la lista delle organizzazioni può aggiornarsi.
  \end{enumerate}
  \item \textbf{Estensioni:} 
  \begin{enumerate}
    \item se l'utente effettua un'operazione in assenza di \glossario{rete}, allora l'applicazione mobile comunica l'errore.
  \end{enumerate}
\end{itemize}
% subsub:uc2utente (end)

\subsubsection{UC2.1 - Visualizzazione lista organizzazioni}
\label{subsub:uc2.1utente}
\begin{itemize}
  \item \textbf{Codice:} UC2.1
  \item \textbf{Titolo:} visualizzazione lista organizzazioni
  \item \textbf{Attori primari:} utente autenticato
  \item \textbf{Precondizione:} l'utente deve essere registrato al servizio offerto dall'applicazione mobile.
  \item \textbf{Postcondizione:} l'utente è autenticato.
  \item \textbf{Scenario Principale:}
  \begin{enumerate}
    \item l'utente autenticato visualizza la lista delle organizzazioni alla quale si può collegare ed autorizzare il rintracciamento.
  \end{enumerate}
  \item \textbf{Inclusioni:}
  \begin{enumerate}
    \item la lista delle organizzazioni può aggiornarsi.
  \end{enumerate}
  \item \textbf{Estensioni:} 
  \begin{enumerate}
    \item se l'utente aggiorna la lista delle organizzazioni in assenza di \glossario{rete}, allora l'applicazione mobile comunica l'errore.
  \end{enumerate}
\end{itemize}
% subsub:uc2.1utente (end)

\subsubsection{UC2.1.1 - Aggiornamento lista organizzazioni}
\label{subsub:uc2.1.1utente}
\begin{itemize}
  \item \textbf{Codice:} UC2.1.1
  \item \textbf{Titolo:} aggiornamento lista organizzazioni
  \item \textbf{Attori primari:} utente autenticato
  \item \textbf{Precondizione:} l'utente visualizza la lista delle organizzazioni.
  \item \textbf{Postcondizione:} l'utente ha la lista aggiornata delle organizzazioni.
  \item \textbf{Scenario Principale:}
  \begin{enumerate}
    \item l'utente richiede l'aggiornamento della lista delle organizzazioni.
  \end{enumerate}
\end{itemize}
% subsub:uc2.1.1utente (end)

\subsubsection{UC2.2 - Collegarsi ad un'organizzazione}
\label{subsub:uc2.2utente}
\begin{itemize}
  \item \textbf{Codice:} UC2.2
  \item \textbf{Titolo:} collegarsi ad un'organizzazione
  \item \textbf{Attori primari:} utente collegato, utente non collegato
  \item \textbf{Precondizione:} l'utente è autenticato.
  \item \textbf{Postcondizione:} l'utente è collegato ad un'organizzazione.
  \item \textbf{Scenario Principale:}
  \begin{enumerate}
    \item se l'utente non è ancora collegato ad alcuna organizzazione, allora può chiedere il collegamento ad una delle organizzazioni della lista.
    \item se l'utente è già collegato ad un'organizzazione, ma necessita di passarne ad un'altra, allora avviene il cambio di organizzazione, scollegandosi dalla precedente e
    collegandosi a quella nuova.
  \end{enumerate}
  \item \textbf{Estensioni:} 
  \begin{enumerate}
    \item se l'utente cerca di collegarsi ad un'organizzazione in assenza di \glossario{rete}, allora l'applicazione mobile comunica l'errore.
  \end{enumerate}
\end{itemize}
% subsub:uc2.2utente (end)

\subsubsection{UC2.3 - Scollegarsi da un'organizzazione}
\label{subsub:uc2.3utente}
\begin{itemize}
  \item \textbf{Codice:} UC2.3
  \item \textbf{Titolo:} scollegarsi da un'organizzazione
  \item \textbf{Attori primari:} utente collegato 
  \item \textbf{Precondizione:} l'utente deve essere collegato ad un'organizzazione.
  \item \textbf{Postcondizione:} l'utente è autenticato ed è non collegato ad alcuna organizzazione.
  \item \textbf{Scenario Principale:}
  \begin{enumerate}
    \item l'utente non vuole più essere rintracciato dall'organizzazione corrente, quindi si scollega.
  \end{enumerate}
  \item \textbf{Estensioni:} 
  \begin{enumerate}
    \item se l'utente cerca di scollegarsi ad un'organizzazione in assenza di rete, allora l'applicazione mobile comunica l'errore.
  \end{enumerate}
\end{itemize}
% subsub:uc2.3utente (end)

\subsubsection{UC2.4 - Schermata scelta noto o incognito}
\label{subsub:uc2.4utente}
% integrare diagramma UC2.4
\begin{itemize}
  \item \textbf{Caso d’uso:} UC2.4
  \item \textbf{Titolo:} schermata scelta noto o incognito
  \item \textbf{Attore primario:} utente collegato, che può essere \glossario{utente noto} oppure \glossario{utente incognito}
  \item \textbf{Precondizioni:} l'utente deve essere collegato.
  \item \textbf{Postcondizioni:} l'utente è noto oppure incognito.
  \item \textbf{Scenario principale:} 
  \begin{enumerate}
    \item l'utente è collegato ad un'organizzazione in cui la sua presenza è nota; l'utente ha la possibilità di passare in modalità incognito.
    \item l'utente è collegato ad un'organizzazione in cui la sua presenza è incognita; l'utente ha la possibilità di passare in modalità nota.
  \end{enumerate}
  \item \textbf{Estensioni:} 
\end{itemize}
% subsub:uc2.4utente (end)

\subsubsection{UC2.4.1 - Passaggio a incognito}
\label{subsub:uc2.4.1utente}
\begin{itemize}
  \item \textbf{Codice:} UC2.4.1
  \item \textbf{Titolo:} passaggio a incognito
  \item \textbf{Attori primari:} utente noto
  \item \textbf{Precondizione:} l'utente deve essere collegato e noto.
  \item \textbf{Postcondizione:} l'utente è collegato ed incognito.
  \item \textbf{Scenario Principale:}
  \begin{enumerate}
    \item l'utente è noto in un'organizzazione e vuole diventare incognito: in questo modo la sua presenza è nota, ma non lo è la sua identità.
  \end{enumerate}
  \item \textbf{Estensioni:} 
  \begin{enumerate}
    \item se l'utente cerca di passare da noto ad incognito in assenza di rete, allora l'applicazione mobile comunica l'errore.
  \end{enumerate}
\end{itemize}
% subsub:uc2.4.1utente (end)

\subsubsection{UC2.4.2 - Passaggio a noto}
\label{subsub:uc2.4.2utente}
\begin{itemize}
  \item \textbf{Codice:} UC2.4.2
  \item \textbf{Titolo:} passaggio a noto
  \item \textbf{Attori primari:} utente incognito
  \item \textbf{Precondizione:} l'utente deve essere collegato ed incognito.
  \item \textbf{Postcondizione:} l'utente è collegato e noto.
  \item \textbf{Scenario Principale:}
  \begin{enumerate}
    \item l'utente è incognito in un'organizzazione e vuole diventare noto: in questo modo, sia la sua presenza che la sua identità sono note.
  \end{enumerate}
  \item \textbf{Estensioni:} 
  \begin{enumerate}
    \item se l'utente cerca di passare da incognito a noto in assenza di rete, allora l'applicazione mobile comunica l'errore.
  \end{enumerate}
\end{itemize}
% subsub:uc2.4.2utente (end)

\subsubsection{UC2.5 - Visualizzazione storico accessi}
\label{subsub:uc2.5utente}
\begin{itemize}
  \item \textbf{Codice:} UC2.5
  \item \textbf{Titolo:} visualizzazione storico accessi
  \item \textbf{Attori primari:} utente autenticato
  \item \textbf{Precondizione:} l'utente deve essere registrato al servizio offerto dall'applicazione mobile.
  \item \textbf{Postcondizione:} l'utente è autenticato.
  \item \textbf{Scenario Principale:}
  \begin{enumerate}
    \item l'utente accede ad una schermata per la visualizzaione dello storico personale degli accessi.
  \end{enumerate}
  \item \textbf{Estensioni:} 
  \begin{enumerate}
    \item se l'utente cerca di visualizzare lo storico degli accessi in assenza di rete, allora l'applicazione mobile comunica l'errore.
  \end{enumerate}
\end{itemize}
% subsub:uc2.5utente (end)

\subsubsection{UC2.6 - Visualizzazione tempo trascorso nell'organizzazione corrente}
\label{subsub:uc2.6utente}
\begin{itemize}
  \item \textbf{Codice:} UC2.6
  \item \textbf{Titolo:} visualizzazione tempo trascorso nell'organizzazione corrente
  \item \textbf{Attori primari:} utente collegato
  \item \textbf{Precondizione:} l'utente deve essere collegato ad un'organizzazione.
  \item \textbf{Postcondizione:} l'utente visualizza il tempo trascorso nell'organizzazione.
  \item \textbf{Scenario Principale:}
  \begin{enumerate}
    \item l'utente è collegato ad un'organizzazione e può accedere ad una schermata dell'applicazione mobile per visualizzare in tempo reale il tempo trascorso all'interno 
    dell'organizzazione corrente.
  \end{enumerate}
  \item \textbf{Estensioni:} 
  \begin{enumerate}
    \item se l'utente cerca di visualizzare il tempo trascorso nell'organizzazione corrente in assenza di rete, allora l'applicazione mobile comunica l'errore.
  \end{enumerate}
\end{itemize}
% subsub:uc2.6utente (end)

\subsubsection{UC2.7 - Visualizzazione errore rete mancante}
\label{subsub:uc2.7utente}
\begin{itemize}
  \item \textbf{Codice:} UC2.7
  \item \textbf{Titolo:} visualizzazione errore rete mancante
  \item \textbf{Attori primari:} utente autenticato
  \item \textbf{Precondizione:} l'utente vuole eseguire un'operazione all'interno dell'applicazione mobile.
  \item \textbf{Postcondizione:} l'applicazione mobile comunica all'utente l'errore.
  \item \textbf{Scenario Principale:}
  \begin{enumerate}
    \item l'utente visualizza il messaggio d'errore, in quanto non si possono eseguire operazioni in mancanza di rete.
  \end{enumerate}
\end{itemize}
% subsub:uc2.7utente (end)

\end{document}
