\documentclass[../../norme-di-progetto.tex]{subfiles}

\begin{document}

\subsubsection{Finalità}%
\label{subs:accertamento_della_qualita/finalita}

GruppOne istanzia il processo di accertamento della qualità per garantire qualità di processo e di prodotto.
Gli esiti dei processi di verifica e validazione saranno indispensabili nello svolgimento del processo di accertamento della qualità.

\subsubsection{Descrizione}%
\label{subs:accertamento_della_qualita/descrizione}

Il processo di accertamento della qualità garantisce che i prodotti e i processi del ciclo di vita del software rispettino i requisiti prestabiliti e che aderiscano ai piani esecutivi prefissati.
Gli obiettivi e le strategie che il team si impegna ad applicare durante lo svolgimento del progetto sono presentati nel \textit{Piano di qualifica (versione \versione)}.

\subsubsection{Attività}%
\label{subs:accertamento_della_qualita/attivita}

Le principali attività coinvolte nel processo di accertamento della qualità sono:

\begin{description}
  \item [Pianificazione] si cercano le metodologie, le procedure, gli strumenti e le attività offerte da altri processi di supporto per organizzare le pianificazione della qualità.
  \item [Implementazione] si utilizza ciò che si è individuato al punto precedente per effettuare controlli di qualità sui processi e sui prodotti in atto.
  \item [Documentazione dei risultati] i risultati dei controlli di qualità vengono documentati.
\end{description}

\paragraph{Accertamento della qualità di prodotto}%
\label{par:accertamento_della_qualita_di_prodotto/attivita}
L'attività di accertamento della qualità di prodotto assicura la qualità dei processi di fornitura del prodotto e promuove controlli continui, che possano garantire i requisiti e le funzionalità concordate, e prevenire danni irreparabili al termine del progetto.
È fortemente dipendente dalla qualità di processo: da processi privi di qualità, infatti, non è possibile ottenere buoni prodotti.

Il riferimento principale che adottiamo per questo asse di qualità è lo standard ISO/IEC 25010:2011, ponendo particolare attenzione alle caratteristiche che hanno maggiore interesse per il proponente del prodotto.

Riferimento alla metrica\@:~\ref{par:percentuale_delle_metriche_soddisfatte}

\paragraph{Accertamento della qualità di processo}%
\label{par:accertamento_della_qualita_di_processo/attivita}

L'attività di accertamento della qualità di processo monitora i processi istanziati, e deve essere perseguita nel corso del ciclo di vita del software in modo che il prodotto soddisfi le richieste del proponente e che i processi convergano con costi ridotti in termini di risorse a pari qualità di prodotto.
Per avere un sistema di accertamento della qualità di processo che funzioni è necessario che:

\begin{itemize}
  \item Si individuino i processi da controllare.
  \item Si stabiliscano le metriche di valutazione del processo.
  \item Si eseguano accertamenti continui sui processi scelti.
  \item In base ai risultati ottenuti, si ricerchi un miglioramento continuo dei processi.
\end{itemize}

Per questo asse di qualità lo standard principale che adottiamo è ISO/IEC 90003:2014, pur considerando che per la natura del progetto la sua applicazione avrà portata limitata allo svolgimento del progetto stesso.

Riferimento alla metrica\@:~\ref{par:percentuale_delle_metriche_soddisfatte}

\paragraph{Accertamento del sistema di qualità}%
\label{par:accertamento_del_sistema_di_qualita}

L'attività di accertamento del sistema di qualità è strettamente legato allo standard ISO 9001:2005 che adotta il modello del ciclo PDCA, o ciclo di Deming.

Il ciclo di Deming o ciclo PDCA è un processo iterativo per il monitoraggio e il miglioramento continuo dei processi. Sfrutta la ripetizione di quattro attività:

\begin{description}
  \item [Plan] definisce gli obiettivi, le attività e i processi necessari per raggiungere i risultati attesi.
  \item [Do] esegue ciò che è stato definito nella fase di pianificazione.
  \item [Check] verifica gli esiti dei processi.
  \item [Act] esegue azioni correttive per migliorare la qualità dei processi.
\end{description}
\begin{figure}[H]
  \includegraphics[width=8cm]{PDCA-process.png}
  \centering
  \caption{ciclo di Deming o PDCA.}
\end{figure}

\subsubsection{Metriche}%
\label{subs:accertamento_della_qualita/metriche}

\paragraph{Denominazione delle metriche}%
\label{par:denominazione_delle_metriche}

Le metriche sono essenziali per avere una misura oggettiva di qualità del nostro prodotto, e vengono definite durante la pianificazione del processo di accertamento della qualità.
Per tale ragione per la definizione delle metriche di qualità è necessario identificare una denominazione comune.
Ogni metrica avrà la seguente struttura:
\begin{center}
  \textbf{M[Tipo][Sigla]}
\end{center}
\begin{description}
  \item [Tipo] indica il tipo di metrica. Può essere:
        \begin{description}
          \item[PS]: metrica di processo
          \item[PR]: metrica di prodotto
          \item[TS]: metrica di test.
        \end{description}
  \item [Sigla] indica una sigla che abbrevia la metrica.
\end{description}

\paragraph{Percentuale delle metriche soddisfatte (MPS-PMS)}%
\label{par:percentuale_delle_metriche_soddisfatte}

La metrica permette di conoscere la percentuale delle metriche di processo soddisfatte rispetto a tutte le metriche di processo disponibili.
La formula della metrica è la seguente:
\[
  PMS = \frac{MSOD}{MTOT} \times 100
\]
dove:
\begin{description}
  \item[PMS] risultato che indica la percentuale di metriche di processo soddisfatte.
  \item[MSOD] numero di metriche soddisfatte, dove per \textit{soddisfatte} si intende tutte le metriche di processo che hanno superato una certa soglia di accettabilità.
  \item[MTOT] numero totale di metriche di processo.
\end{description}

Ecco una lista che indica che significato assume PMS\@:
\begin{itemize}
  \item Se il risultato è pari a 0\%, nessuna metrica è stata superata.
  \item Se il risultato è pari al 100\%, tutte le metriche sono state soddisfatte.
  \item Se il risultato è maggiore di 0\%, ma minore di 100\%, non tutte le metriche sono state soddisfatte.
\end{itemize}

%\subsubsection{Procedure}%
%\label{subs:accertamento_della_qualita/procedure}

%\subsubsection{Strumenti di supporto}%
%\label{subs:accertamento_della_qualita/strumenti_di_supporto}

\end{document}
