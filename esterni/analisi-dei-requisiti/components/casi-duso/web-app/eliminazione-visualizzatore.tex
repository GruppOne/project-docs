\documentclass[../../../analisi-dei-requisiti.tex]{subfiles}

\begin{document}


\begin{figure}[H]
  \centering
  \begin{plantuml}
    @startuml
    !include ../../commons/style/use-cases.pu

    actor :amministratore: as A1
    actor :owner: as A2

    rectangle {
        together {
            usecase (AUC14.1) as "AUC14.1\nInserimento email visualizzatore"
          }
      }

    :A1: -- AUC14.1
    :A2: -- AUC14.1

    @enduml
  \end{plantuml}
  \caption{AUC14: Eliminazione visualizzatore}%
  \label{fig:auc14}
\end{figure}

\begin{description}
  \item[Codice:] AUC14;
  \item[Titolo:] Eliminazione visualizzatore;
  \item[Attori primari:] amministratore, owner;
  \item[Precondizione:] il sistema deve rendere disponibile la pagina di eliminazione visualizzatore;
  \item[Postcondizione:] l'utente non è più visualizzatore;
  \item[Scenario principale:]
        \begin{enumerate}
          \item l'amministratore, oppure l'owner, vuole eliminare i privilegi ad un'utente visualizzatore.
        \end{enumerate}
\end{description}

\subsubsection{AUC14.1: Inserimento email visualizzatore}%
\label{subs:AUC14.1}
\begin{description}
  \item[Codice:] AUC14.1;
  \item[Titolo:] Inserimento email visualizzatore;
  \item[Attori primari:] amministratore, owner;
  \item[Precondizione:] il sistema deve rendere disponibile il campo per l'inserimento della mail del visualizzatore da eliminare;
  \item[Postcondizione:] il campo relativo al visualizzatore viene compilato;
  \item[Scenario principale:]
        \begin{enumerate}
          \item l'amministratore, oppure l'owner, inserisce l'email del visualizzatore da eliminare.
        \end{enumerate}
\end{description}

\end{document}
