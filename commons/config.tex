\usepackage[italian]{babel}
\usepackage[margin=2cm, footskip=5mm]{geometry}
\usepackage{hyperref}
\usepackage{soulutf8}
\usepackage{contour}
\usepackage{float}
\usepackage[T1]{fontenc}
\usepackage{graphicx}
\usepackage{fancyhdr}
\usepackage{longtable}
\usepackage[table]{xcolor}
\usepackage{titling}
\usepackage{lastpage}
\usepackage{ifthen}

% setup della sottolineatura
\setuldepth{Flat}
\contourlength{0.8pt}

\newcommand{\uline}[1]{%
  \ul{{\phantom{#1}}}%
  \llap{\contour{white}{#1}}%
}

% setup dei link
\hypersetup{
    colorlinks=true, %set true if you want colored links
    linktoc=all,     %set to all if you want both sections and subsections linked
    linkcolor=black,  %choose some color if you want links to stand out
}

% shortcut per i placeholder
\newcommand{\plchold}[1]{\textit{\{#1\}}}

% setup di header e footer
\pagestyle{fancy}

\fancyhf{}
\fancyhead[L]{\includegraphics[width=1cm]{\commons/img/logo.png}}
\fancyhead[R]{\thetitle}
\fancyfoot[R]{\thepage\ di \pageref{LastPage}}

\fancypagestyle{nopage}{%
\fancyfoot{}%
}

\setlength{\headheight}{1.2cm}

% definizione del comando \uso
\makeatletter
\newcommand{\setUso}[1]{%
\newcommand{\@uso}{#1}%
}
\newcommand{\uso}{\@uso}
\makeatother