\documentclass{article}

\usepackage[italian]{babel}
\usepackage[margin=2cm, footskip=5mm]{geometry}
% questi package non sono necessari in lualatex; ref https://tex.stackexchange.com/a/413046
% \usepackage[utf8]{inputenc}
% \usepackage[T1]{fontenc}
\usepackage{enumitem}
\usepackage{hyperref}
\usepackage{titlesec}
\usepackage{soulutf8}
\usepackage{contour}
\usepackage{float}
\usepackage{graphicx}
\usepackage{fancyhdr}
\usepackage{longtable}
\usepackage[table]{xcolor}
\usepackage{titling}
\usepackage{lastpage}
\usepackage{ifthen}
\usepackage{calc}
\usepackage{minted}
\usepackage{pgfgantt}
\usepackage{subfiles}

\newlength{\imgwidth}

\newcommand\scalegraphics[1]{%
    \settowidth{\imgwidth}{\includegraphics{#1}}%
    \setlength{\imgwidth}{\minof{\imgwidth}{\textwidth}}%
    \includegraphics[width=\imgwidth]{#1}%
}

% XXX definizione dei percorsi in cui cercare immagini
\graphicspath{ {./}
    {./img/}
}

% esempio di utilizzo: \appendToGraphicspath{./img/} (un comando diverso per ogni path da includere)
% N.B.: ci DEVE essere un forward slash alla fine del path, a indicare che è una cartella.
\makeatletter
\newcommand\appendToGraphicspath[1]{%
  \g@addto@macro\Ginput@path{{#1}}%
}
\makeatother

% setup della sottolineatura
\setuldepth{Flat}
\contourlength{0.8pt}

\newcommand{\uline}[1]{%
  \ul{{\phantom{#1}}}%
  \llap{\contour{white}{#1}}%
}

% setup dei link
\hypersetup{
  colorlinks=true, % set true if you want colored links
  linktoc=all,     % set to all if you want both sections and subsections linked
  linkcolor=black, % choose some color if you want links to stand out
}

% setup di header e footer
\pagestyle{fancy}

\fancyhf{}
\fancyhead[L]{\includegraphics[width=1cm]{logo.png}}
\fancyhead[R]{\thetitle}
\fancyfoot[R]{\thepage\ di~\pageref{LastPage}}

\fancypagestyle{nopage}{%
  \fancyfoot{}%
}

\setlength{\headheight}{1.2cm}

% setup forma \paragraph e \subparagraph
\titleformat{\paragraph}[hang]{\normalfont\normalsize\bfseries}{\theparagraph}{1em}{}
\titleformat{\subparagraph}[hang]{\normalfont\normalsize\bfseries}{\thesubparagraph}{1em}{}

% setup profondità indice di default
\setcounter{secnumdepth}{5}
\setcounter{tocdepth}{5}

% shortcut per i placeholder
\newcommand{\plchold}[1]{\textit{\{#1\}}} % chktex 20

% hook per lo script che genera il glossario
\newcommand{\glossario}[1]{\underline{#1}\textsubscript{g}}

% definizione dei comandi \uso e \stato
\makeatletter
\newcommand{\setUso}[1]{%
  \newcommand{\@uso}{#1}%
}
\newcommand{\uso}{\@uso}

\newcommand{\setStato}[1]{%
  \newcommand{\@stato}{#1}%
}
\newcommand{\stato}{\@stato}

\newcommand{\setVersione}[1]{%
  \newcommand{\@versione}{#1}%
}
\newcommand{\versione}{\@versione}

\newcommand{\setResponsabile}[1]{%
  \newcommand{\@responsabile}{#1}%
}
\newcommand{\responsabile}{\@responsabile}

\newcommand{\setRedattori}[1]{%
  \newcommand{\@redattori}{#1}%
}
\newcommand{\redattori}{\@redattori}

\newcommand{\setVerificatori}[1]{%
  \newcommand{\@verificatori}{#1}%
}
\newcommand{\verificatori}{\@verificatori}

\newcommand{\setDescrizione}[1]{%
  \newcommand{\@descrizione}{#1}%
}
\newcommand{\descrizione}{\@descrizione}

\newcommand{\setModifiche}[1]{%
  \newcommand{\@modifiche}{#1}%
}

\newcommand{\modifiche}{\@modifiche}

\makeatother

% setup delle description
\setlist[description,1]{font=$\bullet$\hspace{1.5mm}, leftmargin=*,labelindent=12.5mm}
\setlist[description,2]{font=$\bullet$\hspace{1.5mm}, leftmargin=*,labelindent=12.5mm}

\appendToGraphicspath{../../commons/img/}

\title{Verbale interno --- 07/04/2020}

\setResponsabile{Alberto Cocco}
\setRedattori{Fabio Scettro}
\setVerificatori{
  Luca Ercole
}
\setUso{Interno}
\setDescrizione{Verbale dell'incontro di GruppOne del 07/04/2020}
\setModifiche{%
\cellcolor{white!80!lightgray!100} & Alberto Cocco & 2020--04--09 & approva documento \\%
\cellcolor{white!80!lightgray!100} & Luca Ercole & 2020--03--08 & verifica verbale \\%
\multirow{-3}{*}{-} & Fabio Scettro & 2020--04--07 & stendi verbale%
}

\disabilitaVersione{}
\disabilitaElencoFigure{}
\disabilitaElencoTabelle{}

\begin{document}

\thispagestyle{empty}
\pagenumbering{gobble}
\begin{center}
	\includegraphics[width=8.5cm]{\commons/img/logo.png}\\
	{\Large GruppOne - progetto "Stalker"}\\
	\vspace{1.5cm}
	{\Huge \thetitle}
	\vspace{1.5cm}
	\begin{table}[H]
		\centering
		\begin{tabular}{r|l}
			\textbf{Versione}&\versione\\
			\textbf{Approvazione}&\responsabile\\
			\textbf{Redazione}&\redattori\\
			\textbf{Verifica}&\verificatori\\
			\textbf{Stato}&\stato\\
			\textbf{Uso}&\uso\\
			\textbf{Destinato a}&Imola Informatica\\
			&GruppOne\\
			&Prof. Tullio Vardanega\\
			&Prof. Riccardo Cardin\\
		\end{tabular}
	\end{table}
	\vspace{3cm}
	\textbf{Descrizione}\\
	\descrizione\\
	\vfill
	\verb|gruppone.swe@gmail.com|
\end{center}
\newpage
\thispagestyle{nopage}
\section*{Registro delle modifiche}
\label{sec:registro_delle_modifiche}
\begin{table}[H]
	\label{tab:registro_delle_modifiche}
	\centering
	\rowcolors{2}{lightgray}{white!80!lightgray!100}
	\begin{longtable}[c]{c c c c l}
		\rowcolor{darkgray!90!}\color{white}{\textbf{Versione}}&\color{white}{\textbf{Data}}&\color{white}{\textbf{Nominativo}}&\color{white}{\textbf{Ruolo}}&\color{white}{\textbf{Descrizione}}\\\endhead
	\end{longtable}
\end{table}
% section registro_delle_modifiche (end)
\newpage
\thispagestyle{nopage}
\pagenumbering{roman}
\tableofcontents
\newpage
\pagenumbering{arabic}

\section{Informazioni logistiche}%
\label{sec:informazioni_logistiche}

\begin{description}
  \item [Luogo] videochiamata su Google Hangouts
  \item [Data] 07/04/2020
  \item [Ora] 15:00 \symbol{8594} 17:00
\end{description}

\subsection{Membri del gruppo presenti}%
\label{sub:membri_del_gruppo_presenti}

\begin{enumerate}
  \item Riccardo Agatea
  \item Riccardo Cestaro
  \item Alberto Cocco
  \item Luca Ercole
  \item Alberto Gobbo
  \item Alessandro Rizzo
  \item Fabio Scettro
\end{enumerate}
% sub:membri_del_gruppo_presenti (end)

\section{Ordine del giorno}%
\label{sec:ordine_del_giorno}

\begin{itemize}
  \item Punto della situazione
  \item Soluzioni individuate
\end{itemize}

\section{Punto della situazione}%
\label{sec:punto_della_situazione}
Il gruppo si è aggiornato sugli sviluppi del proprio lavoro ed è emersa una situazione di difficoltà legata all'uso delle tecnologie scelte per la procedura di autenticazione, che si sono rivelate più complicate del previsto. Questa situazione ha portato ad un progressivo ritardo nel raggiungimento degli obiettivi, fino ad arrivare al termine del primo incremento senza che questi venissero soddisfatti.

Il gruppo ha inoltre discusso sul risultato ottenuto nella Technology Baseline svoltasi precedentemente, sui commenti e sulle perplessità sollevate dal docente circa la mancanza di coesione e collaborazione tra i componenti dei tre team previsti (divisi in team sviluppo server, team sviluppo mobile app e team sviluppo web app).
È emersa infatti la necessità di rompere la separazione troppo rigida che il gruppo aveva inizialmente adottato, in favore di una collaborazione che si avvicinasse maggiormente ad una ad ampio spettro, in modo tale che ogni componente del gruppo potesse avere la possibilità di lavorare ad ognuna delle componenti e di avere un'idea chiara sul funzionamento e sullo sviluppo dell'intero progetto.

\section{Soluzioni individuate}%
\label{sec:soluzioni_individuate}

Nel corso della discussione sono state proposte e accettate due soluzioni:
\begin{itemize}
  \item È stato spostato il secondo incremento dalla settimana del 7/04--14/04 alla settimana successiva alla consegna dei documenti, scelta dettata dalla necessità del gruppo di investire una maggiore quantità di tempo nel periodo di autoapprendimento degli strumenti necessari allo sviluppo della procedura di autenticazione.
  \item È stato inoltre stabilito di intervenire creando una sorta di elenco condiviso di attività, ordinate per priorità, in cui ogni componente possa inserire un compito da svolgere nell'immediato futuro, stabilendone l'urgenza. Ogni componente quindi potrà accedere a questo elenco, scegliere l'attività in cima alla lista, che quindi avrà un'urgenza più alta, e lavorarci indipendentemente dall'ambito di sviluppo (server, mobile app, web app). In questo modo ognuno avrà la possibilità di dedicarsi allo sviluppo di ogni componente software.
\end{itemize}

\newpage
\section{Registro delle decisioni}%
\label{sec:registro_delle_decisioni}

\begin{enumerate}
  \item È stato spostato il secondo incremento alla settimana successiva alla consegna dei documenti.
  \item È stata implementata una project board con l'elenco delle attività da svolgere, ordinate per urgenza.
\end{enumerate}

% sec:registro_delle_decisioni (end)
\end{document}
