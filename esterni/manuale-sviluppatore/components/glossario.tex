\documentclass[../../../analisi-dei-requisiti.tex]{subfiles}

\begin{document}

\subsection{A}

\begin{description}
    \item[ADB] acronimo di Android Debug Bridge, permette la comunicazione tramite shell con il dispositivo Android.
    \item[Android Studio] IDE di Google per la realizzazione di applicazioni Android.
    \item[Angular] framework open source, sviluppato principalmente da Google, per lo sviluppo di applicazioni web nel linguaggio TypeScript.
\end{description}

\subsection{B}

\subsection{C}

\begin{description}
    \item[Container] file immagine che contiene un istanza da avviare in Docker.
\end{description}

\subsection{D}

\begin{description}
    \item[Debug] attività che consiste nell'individuazione di uno o più errori, eseguendo un'analisi dinamica.
\end{description}

\subsection{E}

\subsection{F}

\subsection{G}

\begin{description}
    \item[Gradle] kit di automazione di sviluppo che può essere integrato in diversi ambienti, attraverso plugins.
\end{description}

\subsection{H}

\subsection{I}

\begin{description}
    \item[IDE] acronimo Integrated development environment, è un ambiente di sviluppo con l'obbiettivo di supportare il programmatore durante la codifica.
    \item[IntelliJ IDEA] ambiente di sviluppo integrato per il linguaggio di programmazione Java.
\end{description}

\subsection{J}

\begin{description}
    \item[Java] linguaggio di programmazione ad alto livello, orientato agli oggetti e a tipizzazione statica, progettato per essere il più possibile indipendente dalla piattaforma di esecuzione.
    \item[JavaScript] linguaggio di scripting orientato agli oggetti e agli eventi, comunemente utilizzato nella programmazione web lato client per gestire gli effetti dinamici interattivi.
    \item[JUnit] framework di unit testing per il linguaggio di programmazione Java.
\end{description}

\subsection{K}

\begin{description}
    \item[Kotlin] linguaggio di programmazione general purpose, multi-paradigma, open source sviluppato dall'azienda di software JetBrains.
\end{description}

\subsection{L}

\begin{description}
    \item[LiveReload] tecnica per la compilazione automatica ad ogni modifica del codice sorgente.
\end{description}

\subsection{M}

\begin{description}
    \item[MySQL] relational database management system composto da un client a riga di comando e un server.
\end{description}

\subsection{N}

\subsection{O}

\begin{description}
    \item[OpenAPI] standard open source frequentemente utilizzato per la descrizione delle API\@.
\end{description}

\subsection{P}

\begin{description}
    \item[Port Forwarding] operazione che permette il trasferimento dei dati da un computer ad un altro tramite una specifica porta di comunicazione.
    \item[PhpMyAdmin] applicazione web che consente di amministrare un database MySQL o MariaDB\@.
\end{description}

\subsection{Q}

\subsection{R}

\begin{description}
    \item[Relational Database] database digitale basato sul modello relazionale di dati.
    \item[R2DBC] standard API per il reactive programming con un database SQL\@.
\end{description}

\subsection{S}

\begin{description}
    \item[Spring] Framework Java e contenitore IoC per lo sviluppo di applicazioni web basate su Java EE\@.
\end{description}

\subsection{T}

\begin{description}
    \item[TypeScript] linguaggio di programmazione open source sviluppato da Microsoft, estende il JavaScript rendendolo orientato agli oggetti.
\end{description}

\subsection{U}

\subsection{V}

\begin{description}
    \item[VM] acronimo di virtual machine, indica un software che, attraverso un processo di virtualizzazione, crea un ambiente virtuale che emula tipicamente il comportamento di una macchina fisica.
\end{description}

\subsection{W}

\subsection{X}

\subsection{Y}

\subsection{Z}

\end{document}
