\documentclass[../../../analisi-dei-requisiti.tex]{subfiles}

\begin{document}


\begin{figure}[H]
  \centering
  \begin{plantuml}
  @startuml
  !include ../../commons/style/use-cases.pu

  actor :utente autenticato: as A

  rectangle {
    together {
      usecase (UUC4.1) as "UUC4.1\nAggiornamento\nlista organizzazioni"
      usecase (UUC4.2) as "UUC4.2\nSeleziona organizzazioni"
      usecase (UUC12) as "UUC12\nVisualizzazione errore\nrete mancante"
    }
  }

  :A: -- (UUC4.1)
  :A: -- (UUC4.2)

  (UUC12) .up.> (UUC4.1) : <<extends>>
  (UUC12) .up.> (UUC4.2) : <<extends>>

  @enduml
  \end{plantuml}
  \caption{UUC4: Recupero lista organizzazioni}%
  \label{fig:uuc4}
\end{figure}

\begin{description}
  \item[Caso d’uso:] UUC4;
  \item[Titolo:] Recupero lista organizzazioni;
  \item[Attori primari:] utente autenticato;
  \item[Precondizione:] l'utente si è appena autenticato;
  \item[Postcondizione:] l'utente visualizza una lista di tutte le organizzazioni;
  \item[Scenario principale:]
        \begin{enumerate}
          \item l'utente ha la possibilità di recuperare una lista di organizzazioni alla quale si può collegare.
        \end{enumerate}
  \item[Estensioni:]
        \begin{enumerate}
          \item in caso di rete mancante, non possono essere eseguite queste operazioni e quindi verrà notificato un errore \emph{[UUC12]}.
        \end{enumerate}
\end{description}


\subsubsection{UUC4.1: Aggiornamento lista organizzazioni}%
\label{subs:UUC4.1}
\begin{description}
  \item[Caso d’uso:] UUC4.1;
  \item[Titolo:] Aggiornamento lista organizzazioni;
  \item[Attori primari:] utente autenticato;
  \item[Precondizione:] l'utente visualizza la lista delle organizzazioni;
  \item[Postcondizione:] l'utente ha aggiornato la lista delle organizzazioni;
  \item[Scenario principale:]
        \begin{enumerate}
          \item l'utente ha la possibilità di aggiornare la lista delle organizzazioni, e viene avvisato mediante \glossario{notifica} dell'applicazione mobile.
        \end{enumerate}
\end{description}



\subsubsection{UUC4.2: Seleziona organizzazioni}%
\label{subs:UUC4.1}
\begin{description}
  \item[Caso d’uso:] UUC4.2;
  \item[Titolo:] Seleziona organizzazioni;
  \item[Attori primari:] utente autenticato;
  \item[Precondizione:] l'utente visualizza la lista delle organizzazioni;
  \item[Postcondizione:] l'utente ha selezionato una o più organizzazioni;
  \item[Scenario principale:]
        \begin{enumerate}
          \item l'utente ha la possibilità di selezionare una o più organizzazioni.
        \end{enumerate}
\end{description}

\end{document}
