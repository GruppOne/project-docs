\documentclass[../analisi-dei-requisiti.tex]{subfiles}

\begin{document}

\subsubsection{AUC1 - Sistema di autenticazione}%
\label{subsub:AUC1}

\begin{figure}[h!]
  \centering
  \begin{plantuml}
  @startuml
  !include ../../commons/style/use-cases.pu
  scale 3/4

  actor :utente non autenticato: as A

  rectangle {
    together {
      usecase (AUC1.1) as "AUC1.1\nAutenticazione\n--\nExtension points:\nVisualizzazione errore se\nle credenziali inserite\n non sono corrette"
      usecase (AUC1.2) as "AUC1.2\nVisualizzazione credenziali errate"
    }
  }

  :A: -- AUC1.1

  (AUC1.3) .up.|> (AUC1.1) : <<extends>>

  @enduml
  \end{plantuml}
  \caption{AUC1: Sistema di autenticazione}
  \label{fig:auc1}
\end{figure}

\begin{description}
  \item[Codice:] AUC1
  \item[Titolo:] Sistema di autenticazione
  \item[Attori primari:] utente non autenticato
  \item[Precondizione:] l'utente non è autenticato alla piattaforma.
  \item[Postcondizione:] l'\glossario{utente} ha effettuato correttamente l'autenticazione nel sistema.
  \item[Scenario principale:]
  \begin{enumerate}
    \item l'utente non è ancora autenticato e vuole eseguire l'autenticazione.
  \end{enumerate}
\end{description}
% subsub:AUC1 (end)

\subsubsection{AUC1.1 - Autenticazione}%
\label{subsub:AUC1.1}

\begin{figure}[h!]
  \centering
  \begin{plantuml}
  @startuml
  !include ../../commons/style/use-cases.pu
  scale 3/4

  actor :utente non autenticato: as A

  rectangle {
    together {
    usecase (AUC1.1.1) as "AUC1.1.1\nInserimento email"
    usecase (AUC1.1.2) as "AUC1.1.2\nInserimento password"
    }
  }

  :A: -- AUC1.1.1
  :A: -- AUC1.1.2

  @enduml
  \end{plantuml}
  \caption{AUC1.1: Autenticazione}
  \label{fig:auc1_1}
\end{figure}

\begin{description}
  \item[Codice:] AUC1.1
  \item[Titolo:] Autenticazione
  \item[Attori primari:] utente non autenticato
  \item[Precondizione:] il sistema è raggiungibile e funzionante, l'utente non autenticato deve poter visualizzare la pagina di autenticazione.
  \item[Postcondizione:] l'autenticazione è andata a buon fine e l'utente è autenticato.
  \item[Scenario principale:]
  \begin{enumerate}
    \item  l'utente non autenticato accede alla pagina di autenticazione, e visualizza tutti i campi che deve compilare:
    \begin{enumerate}
      \item inserisce l’email associata all’account \emph{[AUC1.1.1]};
      \item inserisce la password associata all’account \emph{[AUC1.1.2]}.
    \end{enumerate}
    \item
  \end{enumerate}
  \item[Estensioni:]
  \begin{enumerate}
    \item se l'utente inserisce le credenziali in modo errato, verrà visualizzato un messaggio d'errore \emph{[AUC1.2]}.
  \end{enumerate}
\end{description}
% subsub:AUC1.1 (end)

\subsubsection{AUC1.1.1 - Inserimento email}%
\label{subsub:AUC1.1.1}
\begin{description}
  \item[Codice:] AUC1.1.1
  \item[Titolo:] Inserimento email
  \item[Attori primari:] utente non autenticato
  \item[Precondizione:] il sistema ha reso disponibile il campo per l'inserimento della propria email.
  \item[Postcondizione:] l'utente ha compilato il campo relativo alla propria email.
  \item[Scenario principale:]
  \begin{enumerate}
    \item l'utente compila il campo relativo alla propria mail di registrazione.
  \end{enumerate}
\end{description}
% subsub:AUC1.1.1 (end)

\subsubsection{AUC1.1.2 - Inserimento password}%
\label{subsub:AUC1.1.2}
\begin{description}
  \item[Codice:] AUC1.1.2
  \item[Titolo:] Inserimento password
  \item[Attori primari:] utente non autenticato
  \item[Precondizione:] il sistema ha reso disponibile il campo per l'inserimento della password.
  \item[Postcondizione:] l'utente ha compilato il campo relativo alla sua password.
  \item[Scenario principale:]
  \begin{enumerate}
    \item l'utente compila il campo relativo alla propria password di registrazione.
  \end{enumerate}
\end{description}
% subsub:AUC1.1.2 (end)

\subsubsection{AUC1.2 - Visualizzazione credenziali errate}%
\label{subsub:AUC1.2}
\begin{description}
  \item[Codice:] AUC1.2
  \item[Titolo:] Visualizzazione credenziali errate
  \item[Attori primari:] utente non autenticato
  \item[Precondizione:] l'utente ha inviato al server le credenziali per effettuare l'autenticazione.
  \item[Postcondizione:] l'utente non autenticato visualizza un messaggio di credenziali sbagliate.
  \item[Scenario principale:]
  \begin{enumerate}
    \item l'utente cerca di effettuare l'autenticazione con credenziali errate.
  \end{enumerate}
\end{description}
% subsub:AUC1.2 (end)

\subsubsection{AUC2 - Richiesta owner}%
\label{subsub:AUC2}

\begin{figure}[h!]
  \centering
  \begin{plantuml}
  @startuml
  !include ../../commons/style/use-cases.pu
  scale 3/4

  actor :utente autenticato: as A

  rectangle {
    together {
      usecase (AUC2.1) as "AUC2.1\nInvio richiesta owner\n--\nExtension points:\nVisualizzazione errore nel\ncaso in cui la richiesta venga\nrifiutata dall'amministratore"
      usecase (AUC2.2) as "AUC2.2\nVerifica tipologia utente\n--\nExtension points:\nVisualizzazione errore nel\ncaso in cui l'utente\nautenticato sia già owner"
    }
      usecase (AUC2.3) as "AUC2.3\nRichiesta rifiutata"
    }

  :A: -- AUC2.1

  (AUC2.2) .right.|> (AUC2.1) : <<include>>

  (AUC2.3) .up.|> (AUC2.1) : <<extends>>
  (AUC2.3) .up.|> (AUC2.2) : <<extends>>

  @enduml
  \end{plantuml}
  \caption{AUC2: Richiesta owner}
  \label{fig:auc2}
\end{figure}

\begin{description}
  \item[Codice:] AUC2
  \item[Titolo:] Richiesta owner
  \item[Attori primari:] utente autenticato
  \item[Precondizione:] l'utente autenticato non deve essere un \glossario{owner}.
  \item[Postcondizione:] la richiesta viene accettata e l'utente autenticato diventa un owner.
  \item[Scenario principale:]
  \begin{enumerate}
    \item l'utente autenticato vuole iniziare ad utilizzare \emph{Stalker} creando una sua organizzazione.
  \end{enumerate}
\end{description}
% subsub:AUC2 (end)

\subsubsection{AUC2.1 - Invio richiesta owner}%
\label{subsub:AUC2.1}
\begin{description}
  \item[Codice:] AUC2.1
  \item[Titolo:] Invio richiesta owner
  \item[Attori primari:] utente autenticato
  \item[Precondizione:] l'utente autenticato utilizza l'apposito form per inviare la richiesta di diventare owner.
  \item[Postcondizione:] viene inviata la richiesta, in attesa di essere approvata da un amministratore.
  \item[Scenario principale:]
  \begin{enumerate}
    \item l'utente autenticato invia la richiesta per diventare owner.
  \end{enumerate}
  \item[Inclusioni:]
  \begin{enumerate}
    \item viene effettuato un controllo sulla tipologia di utente che vuole effettuare la richiesta \emph{[AUC2.2]};
  \end{enumerate}
  \item[Estensioni:]
  \begin{enumerate}
    \item la richiesta di diventare owner può essere rifiutata \emph{[AUC2.3]}.
  \end{enumerate}
\end{description}
% subsub:AUC2.1 (end)

\subsubsection{AUC2.2 - Verifica tipologia utente}%
\label{subsub:AUC2.2}
\begin{description}
  \item[Codice:] AUC2.2
  \item[Titolo:] Verifica tipologia utente
  \item[Attori primari:] utente autenticato
  \item[Precondizione:] l'utente autenticato visualizza l'apposito form per inviare la richiesta di diventare owner.
  \item[Postcondizione:] il form relativo alla richiesta di owner viene abilitato o disabilitato di conseguenza.
  \item[Scenario principale:]
  \begin{enumerate}
    \item l'utente autenticato può richiedere di diventare un owner, nel caso in cui sia:
    \begin{description}
      \item un \glossario{amministratore},
      \item un \glossario{visualizzatore},
      \item un \glossario{gestore},
      \item oppure un utente senza alcun privilegio;
    \end{description}
    \item ogni tentativo di richiesta da parte di un utente che è già owner viene rifiutato;
    \item \glossario{root} è già un owner;
    \item l'amministratore può accettare la propria richiesta inviata.
  \end{enumerate}
\end{description}
% subsub:AUC2.2 (end)

\subsubsection{AUC2.3 - Richiesta rifiutata}%
\label{subsub:AUC2.3}
\begin{description}
  \item[Codice:] AUC2.3
  \item[Titolo:] Richiesta rifiutata
  \item[Attori primari:] utente autenticato
  \item[Precondizione:] l'utente ha inviato la propria richiesta per diventare owner.
  \item[Postcondizione:] l'utente autenticato non ha ottenuto l'abilitazione a diventare un owner e visualizza un messaggio d'errore.
  \item[Scenario principale:]
  \begin{enumerate}
    \item l'amministratore non ha accettato la richiesta effettuata dall'utente autenticato.
  \end{enumerate}
\end{description}
% subsub:AUC2.2 (end)

\subsubsection{AUC3 - Disconnessione}%
\label{subsub:AUC3}

\begin{figure}[h!]
  \centering
  \begin{plantuml}
  @startuml
  !include ../../commons/style/use-cases.pu
  scale 3/4

  actor :utente autenticato: as A

  rectangle {
    together {
      usecase (AUC3.1) as "AUC3.1\nDisconnessione utente"
    }
  }

  :A: -- AUC3.1

  @enduml
  \end{plantuml}
  \caption{AUC3: Disconnessione}
  \label{fig:auc3}
\end{figure}

\begin{description}
  \item[Codice:] AUC3
  \item[Titolo:] Disconnessione
  \item[Attori primari:] utente autenticato
  \item[Precondizione:] il sistema dispone la possibilità di effettuare la disconnessione.
  \item[Postcondizione:] l'utente non è più autenticato.
  \item[Scenario principale:]
  \begin{enumerate}
    \item l'utente autenticato vuole effettuare la disconnessione dalla piattaforma;
  \end{enumerate}
\end{description}
% subsub:AUC3 (end)

\subsubsection{AUC4 - Creazione owner}%
\label{subsub:AUC4}

\begin{figure}[h!]
  \centering
  \begin{plantuml}
  @startuml
  !include ../../commons/style/use-cases.pu
  scale 3/4

  actor :amministratore: as A

  rectangle {
    together {
      usecase (AUC4.1) as "AUC4.1\nRichiesta accettata"
      usecase (AUC4.2) as "AUC4.2\nRichiesta rifiutata"
    }
    usecase (AUC4.3) as "AUC4.3\nInvio risposta all'utente"
  }

  :A: -- AUC4.1
  :A: -- AUC4.2

  (AUC4.3) .left.|> (AUC4.1) : <<include>>
  (AUC4.3) .up.|> (AUC4.2) : <<include>>

  @enduml
  \end{plantuml}
  \caption{AUC4: Creazione owner}
  \label{fig:AUC4}
\end{figure}

\begin{description}
  \item[Codice:] AUC4
  \item[Titolo:] Creazione owner
  \item[Attori primari:] amministratore
  \item[Precondizione:] un utente ha inviato una richiesta per diventare owner.
  \item[Postcondizione:] l'owner viene creato.
  \item[Scenario principale:]
  \begin{enumerate}
    \item l'amministratore gestisce una richiesta ricevuta da un utente che vuole diventare owner.
  \end{enumerate}
\end{description}


\subsubsection{AUC4.1 - Richiesta accettata}%
\label{subsub:AUC4.1}

\begin{description}
  \item[Codice:] AUC4.1
  \item[Titolo:] Richiesta accettata
  \item[Attori primari:] amministratore
  \item[Precondizione:] l'amministratore verifica la richiesta.
  \item[Postcondizione:] l'amministratore accetta la richiesta.
  \item[Scenario principale:]
  \begin{enumerate}
    \item L'amministratore esamina la richiesta ricevuta e la accetta. L'utente che ha fatto richiesta diventa owner.
  \end{enumerate}
  \item[Inclusioni:]
  \begin{enumerate}
    \item l'accettazione della richiesta viene notificata all'utente \emph{[AUC4.3]}.
  \end{enumerate}
\end{description}
% subsub:AUC4.1 (end)

\subsubsection{AUC4.2 - Richiesta rifiutata}%
\label{subsub:AUC4.2}

\begin{description}
  \item[Codice:] AUC4.2
  \item[Titolo:] Richiesta rifiutata
  \item[Attori primari:] amministratore
  \item[Precondizione:] l'amministratore verifica la richiesta.
  \item[Postcondizione:] l'amministratore rifiuta la richiesta.
  \item[Scenario principale:]
  \begin{enumerate}
    \item L'amministratore esamina la richiesta ricevuta e la rifiuta. L'utente che ha fatto richiesta non può diventare owner.
  \end{enumerate}
  \item[Inclusioni:]
  \begin{enumerate}
    \item il rifiuto della richiesta viene notificata all'utente \emph{[AUC4.3]}.
    \item
  \end{enumerate}
\end{description}
% subsub:AUC4.2 (end)

\subsubsection{AUC4.3 - Invio risposta all'utente}%
\label{subsub:AUC4.3}

\begin{description}
  \item[Codice:] AUC4.3
  \item[Titolo:] Invio risposta all'utente
  \item[Attori primari:] amministratore
  \item[Precondizione:] l'amministratore ha accettato, oppure rifiutato, la richiesta di un utente per diventare owner.
  \item[Postcondizione:] l'utente che ha fatto la richiesta riceve l'esito tramite un messaggio di risposta.
  \item[Scenario principale:]
  \begin{enumerate}
    \item una volta esaminata la richiesta, la risposta dell'amministratore verrà notificata all'utente.
  \end{enumerate}
\end{description}
% subsub:AUC4.3 (end)

\subsubsection{AUC5 - Gestione organizzazione}%
\label{subsub:AUC5}

\begin{figure}[h!]
  \centering
  \begin{plantuml}
  @startuml
  !include ../../commons/style/use-cases.pu
  scale 3/4

  actor :amministratore: as A

  rectangle {
    together {
      usecase (AUC5.5) as "AUC5.5\nInvio richiesta aggiornamento\nlista organizzazioni"
      usecase (AUC5.4) as "AUC5.4\nSeleziona organizzazione"
      usecase (AUC5.3) as "AUC5.3\nModifica organizzazione"
      usecase (AUC5.2) as "AUC5.2\nEliminazione organizzazione"
      usecase (AUC5.1) as "AUC5.1\nCreazione organizzazione"
    }
  }

  :A: -- AUC5.1
  :A: -- AUC5.2
  :A: -- AUC5.3

  (AUC5.4) .up.|> (AUC5.2) : <<include>>
  (AUC5.4) .up.|> (AUC5.3) : <<include>>

  (AUC5.5) .up.|> (AUC5.1) : <<include>>
  (AUC5.5) .up.|> (AUC5.2) : <<include>>

  @enduml
  \end{plantuml}
  \caption{AUC5: Gestione organizzazione}
  \label{fig:AUC5}
\end{figure}

\begin{description}
  \item[Codice:] AUC5
  \item[Titolo:] Gestione organizzazione
  \item[Attori primari:] amministratore
  \item[Precondizione:] il sistema deve rendere disponibile la pagina di gestione dell'organizzazione.
  \item[Postcondizione:] viene gestita un'organizzazione.
  \item[Scenario principale:]
  \begin{enumerate}
    \item sorge la necessità di effettuare operazioni su un'organizzazione;
  \end{enumerate}
\end{description}

  \subsubsection{AUC5.1 - Creazione organizzazione}%
  \label{subsub:AUC5.1}

  \begin{figure}[h!]
    \centering
    \begin{plantuml}
    @startuml
    !include ../../commons/style/use-cases.pu
    scale 3/4

    actor :amministratore: as A1

    rectangle {
      together {
        usecase (AUC5.1.1) as "AUC5.1.1\nInserisci nome organizzazione"
        usecase (AUC5.1.2) as "AUC5.1.2\nInserisci descrizione organizzazione"
        usecase (AUC5.1.3) as "AUC5.1.3\nConfigurazione dettagli server LDAP"
      }
    }

    :A1: -- AUC5.1.1
    :A1: -- AUC5.1.2
    :A1: -- AUC5.1.3
    @enduml
    \end{plantuml}
    \caption{AUC5.1: Creazione organizzazione}
    \label{fig:AUC5_1}
  \end{figure}

  \begin{description}
    \item[Codice:] AUC5.1
    \item[Titolo:] Creazione organizzazione
    \item[Attori primari:] amministratore
    \item[Precondizione:] l'organizzazione non deve esistere nella lista di \emph{Stalker}, deve essere specificato il suo nome.
    \item[Postcondizione:] l'organizzazione viene creata.
    \item[Scenario principale:]
    \begin{enumerate}
      \item sorge la necessità di creare un'organizzazione, senza essere effettivamente richiesta;
    \end{enumerate}
    \item[Inclusioni:]
    \begin{enumerate}
      \item alla fine della procedura di creazione dell'organizzazione, tutte le applicazioni mobile riceveranno una notifica di aggiornamento della lista di organizzazioni \emph{[AUC5.5]};
    \end{enumerate}
  \end{description}
  % subsub:AUC5.1 (end)

  \subsubsection{AUC5.1.1 - Inserisci nome organizzazione}%
  \label{subsub:AUC5.1.1}
  \begin{description}
    \item[Codice:] AUC5.1.1
    \item[Titolo:] Inserisci nome organizzazione
    \item[Attori primari:] amministratore
    \item[Precondizione:] il sistema fornisce il campo di inserimento del nome.
    \item[Postcondizione:] il nome viene opportunamente inserito.
    \item[Scenario principale:]
    \begin{enumerate}
      \item si vuole inserire il nome di un'organizzazione.
    \end{enumerate}

  \end{description}
  % subsub:AUC5.1.1 (end)

  \subsubsection{AUC5.1.2 - Inserisci descrizione organizzazione}%
  \label{subsub:AUC5.1.2}
  \begin{description}
    \item[Codice:] AUC5.1.2
    \item[Titolo:] Inserisci descrizione organizzazione
    \item[Attori primari:] amministratore
    \item[Precondizione:] il sistema fornisce il campo di inserimento della descrizione dell'organizzazione.
    \item[Postcondizione:] il campo relativo alla descrizione viene riempito.
    \item[Scenario principale:]
    \begin{enumerate}
      \item si vuole inserire la descrizione di un'organizzazione.
    \end{enumerate}
  \end{description}
  % subsub:AUC5.1.2 (end)

  \subsubsection{AUC5.1.3 - Configurazione dettagli server LDAP}%
  \label{subsub:AUC5.1.3}
  \begin{description}
    \item[Codice:] AUC5.1.3
    \item[Titolo:] Configurazione dettagli server LDAP
    \item[Attori primari:] amministratore
    \item[Precondizione:] il sistema fornisce i campi per la configurazione del server LDAP.
    \item[Postcondizione:] il server LDAP è stato configurato.
    \item[Scenario principale:]
    \begin{enumerate}
      \item si vogliono configurare i dettagli del \glossario{server LDAP} che le applicazioni mobile dovranno utilizzare per registrarsi ad un'organizzazione.
      \item se l'organizzazione è segnata come pubblica, i parametri del server LDAP non verranno configurati.
    \end{enumerate}
  \end{description}
  % subsub:AUC5.1.3 (end)

\subsubsection{AUC5.2 - Eliminazione organizzazione}%
\label{subsub:AUC5.2}
\begin{description}
  \item[Codice:] AUC5.2
  \item[Titolo:] Eliminazione organizzazione
  \item[Attori primari:] amministratore
  \item[Precondizione:] deve essere stata selezionata l'organizzazione da eliminare, presente nella lista di \emph{Stalker}.
  \item[Postcondizione:] l'organizzazione viene eliminata.
  \item[Scenario principale:]
  \begin{enumerate}
    \item sorge la necessità di eliminare un'organizzazione, senza interagire con il suo owner;
  \end{enumerate}
  \item[Inclusioni:]
  \begin{enumerate}
    \item viene selezionata un'organizzazione \emph{[AUC5.4]};
    \item alla fine della procedura di eliminazione dell'organizzazione, tutte le applicazioni mobile riceveranno una notifica di aggiornamento della lista di organizzazioni \emph{[AUC5.5]};
  \end{enumerate}
\end{description}
% subsub:AUC5.2 (end)

\subsubsection{AUC5.3 - Modifica organizzazione}%
\label{subsub:AUC5.3}

\begin{figure}[h!]
  \centering
  \begin{plantuml}
  @startuml
  !include ../../commons/style/use-cases.pu
  scale 3/4

  actor :amministratore: as A

  rectangle {
    together {
      usecase (AUC5.3.1) as "AUC5.3.1\nModifica nome organizzazione"
      usecase (AUC5.3.2) as "AUC5.3.2\nModifica descrizione organizzazione"
      usecase (AUC5.3.3) as "AUC5.3.3\nGestione luoghi"
      usecase (AUC5.3.4) as "AUC5.3.4\nModifica configurazione dettagli server LDAP"
    }
  }

  :A: -- AUC5.3.1
  :A: -- AUC5.3.2
  :A: -- AUC5.3.3
  :A: -- AUC5.3.4

  @enduml
  \end{plantuml}
  \caption{AUC5.3: Modifica organizzazione}
  \label{fig:AUC5_3}
\end{figure}

\begin{description}
  \item[Codice:] AUC5.3
  \item[Titolo:] Modifica organizzazione
  \item[Attori primari:] amministratore
  \item[Precondizione:] l'amministratore seleziona l'organizzazione da modificare, presente nella lista di \emph{Stalker}.
  \item[Postcondizione:] l'organizzazione viene modificata.
  \item[Scenario principale:]
  \begin{enumerate}
    \item sorge la necessità di modificare un'organizzazione, senza interagire con il suo owner;
  \end{enumerate}
  \item[Inclusioni:]
  \begin{enumerate}
    \item viene scelta un'organizzazione \emph{[AUC5.4]};
  \end{enumerate}
\end{description}
% subsub:AUC5.3 (end)

\subsubsection{AUC5.3.1 - Modifica nome organizzazione}%
\label{subsub:AUC5.3.1}
\begin{description}
  \item[Codice:] AUC5.3.1
  \item[Titolo:] Modifica nome organizzazione
  \item[Attori primari:] amministratore
  \item[Precondizione:] il sistema fornisce il campo di modifica del nome.
  \item[Postcondizione:] il nome viene opportunamente modificato.
  \item[Scenario principale:]
  \begin{enumerate}
    \item l'amministratore vuole modificare il nome di un'organizzazione.
  \end{enumerate}
\end{description}
% subsub:AUC5.3.1 (end)

\subsubsection{AUC5.3.2 - Modifica descrizione organizzazione}%
\label{subsub:AUC5.3.2}
\begin{description}
  \item[Codice:] AUC5.3.2
  \item[Titolo:] Modifica descrizione organizzazione
  \item[Attori primari:] amministratore
  \item[Precondizione:] il sistema fornisce il campo di modifica della descrizione dell'organizzazione.
  \item[Postcondizione:] la descrizione viene opportunamente modificata.
  \item[Scenario principale:]
  \begin{enumerate}
    \item l'amministratore vuole modificare la descrizione di un'organizzazione.
  \end{enumerate}
\end{description}
% subsub:AUC5.3.2 (end)

\subsubsection{AUC5.3.3 - Gestione luoghi}%
\label{subsub:AUC5.3.3}

\begin{figure}[h!]
  \centering
  \begin{plantuml}
  @startuml
  !include ../../commons/style/use-cases.pu
  scale 3/4

  actor :amministratore: as A

  rectangle {
    together {
      usecase (AUC5.3.3.1) as "AUC5.3.3.1\nAggiungi luogo"
      usecase (AUC5.3.3.2) as "AUC5.3.3.2\nEliminazione luogo"
      usecase (AUC5.3.3.3) as "AUC5.3.3.3\nModifica luogo"
    }
    usecase (AUC5.3.3.4) as "AUC5.3.3.4\nSeleziona luogo"
  }

  :A: -- AUC5.3.3.1
  :A: -- AUC5.3.3.2
  :A: -- AUC5.3.3.3

  (AUC5.3.3.4) .up.|> (AUC5.3.3.2) : <<include>>
  (AUC5.3.3.4) .up.|> (AUC5.3.3.3) : <<include>>

  @enduml
  \end{plantuml}
  \caption{AUC5.3.3: Gestione luoghi}
  \label{fig:AUC5_3_3}
\end{figure}

\begin{description}
  \item[Codice:] AUC5.3.3
  \item[Titolo:] Gestione luoghi
  \item[Attori primari:] amministratore
  \item[Precondizione:] il sistema deve rendere disponibile la pagina di gestione dei luoghi di un'organizzazione.
  \item[Postcondizione:] vengono gestiti i luoghi di un'organizzazione.
  \item[Scenario principale:]
  \begin{enumerate}
    \item sorge la necessità di effettuare operazioni sul luogo di un'organizzazione, e viene offerta la possibilità di selezionarlo;
  \end{enumerate}
\end{description}
% subsub:AUC5.3.3 (end)

\subsubsection{AUC5.3.3.1 - Aggiungi luogo}%
\label{subsub:AUC5.3.3.1}

\begin{figure}[h!]
  \centering
  \begin{plantuml}
  @startuml
  !include ../../commons/style/use-cases.pu
  scale 3/4

  actor :amministratore: as A

  rectangle {
    together {
      usecase (AUC5.3.3.1.1) as "AUC5.3.3.1.1\nInserisci coordinate geografiche"
      usecase (AUC5.3.3.1.2) as "AUC5.3.3.1.2\nInserisci indirizzo luogo"
    }
  }

  :A: -- AUC5.3.3.1.1
  :A: -- AUC5.3.3.1.2

  @enduml
  \end{plantuml}
  \caption{AUC5.3.3.1: Aggiungi luogo}
  \label{fig:AUC5_3_3_1}
\end{figure}

\begin{description}
  \item[Codice:] AUC5.3.3.1
  \item[Titolo:] Aggiungi luogo
  \item[Attori primari:] amministratore
  \item[Precondizione:] il \glossario{luogo} da aggiungere nell'organizzazione non deve esistere.
  \item[Postcondizione:] il nuovo luogo viene aggiunto nell'organizzazione.
  \item[Scenario principale:]
  \begin{enumerate}
    \item sorge la necessità di aggiungere un luogo ad un'organizzazione, senza interagire con il suo owner;
  \end{enumerate}
\end{description}
% subsub:AUC5.3.3.1 (end)

\subsubsection{AUC5.3.3.1.1 - Inserisci coordinate geografiche}%
\label{subsub:AUC5.3.3.1.1}
\begin{description}
  \item[Codice:] AUC5.3.3.1.1
  \item[Titolo:] Inserisci coordinate geografiche
  \item[Attori primari:] amministratore
  \item[Precondizione:] il sistema deve fornire i campi relativi alle coordinate geografiche.
  \item[Postcondizione:] le coordinate geografiche sono inserite correttamente nei loro campi.
  \item[Scenario principale:]
  \begin{enumerate}
    \item l'amministratore inserisce le coordinate geografiche di un nuovo luogo;
  \end{enumerate}
\end{description}
% subsub:AUC5.3.3.1.1 (end)

\subsubsection{AUC5.3.3.1.2 - Inserisci indirizzo luogo}%
\label{subsub:AUC5.3.3.1.2}
\begin{description}
  \item[Codice:] AUC5.3.3.1.2
  \item[Titolo:] Inserisci indirizzo luogo
  \item[Attori primari:] amministratore
  \item[Precondizione:] il sistema deve fornire il campo relativo all'indirizzo del luogo.
  \item[Postcondizione:] l'indirizzo è inserito correttamente nei loro campi.
  \item[Scenario principale:]
  \begin{enumerate}
    \item l'amministratore inserisce l'indirizzo di un nuovo luogo;
  \end{enumerate}
\end{description}
% subsub:AUC5.3.3.1.2 (end)

\subsubsection{AUC5.3.3.2 - Eliminazione luogo}%
\label{subsub:AUC5.3.3.2}
\begin{description}
  \item[Codice:] AUC5.3.3.2
  \item[Titolo:] Eliminazione luogo
  \item[Attori primari:] amministratore
  \item[Precondizione:] il luogo dell'organizzazione deve essere presente in \emph{Stalker}.
  \item[Postcondizione:] il luogo dell'organizzazione viene eliminato.
  \item[Scenario principale:]
  \begin{enumerate}
    \item sorge la necessità di eliminare un luogo di un'organizzazione, senza interagire con il suo owner;
  \end{enumerate}
  \item[Inclusioni:]
  \begin{enumerate}
    \item Seleziona luogo\emph{[AUC5.3.3.4]};
  \end{enumerate}
\end{description}
% subsub:AUC5.3.3.2 (end)

\subsubsection{AUC5.3.3.3 - Modifica luogo}%
\label{subsub:AUC5.3.3.3}

\begin{figure}[h!]
  \centering
  \begin{plantuml}
  @startuml
  !include ../../commons/style/use-cases.pu
  scale 3/4

  actor :amministratore: as A

  rectangle {
    together {
      usecase (AUC5.3.3.3.1) as "AUC5.3.3.3.1\nModifica coordinate geografiche"
      usecase (AUC5.3.3.3.2) as "AUC5.3.3.3.2\nModifica indirizzo luogo"
    }
  }

  :A: -- AUC5.3.3.3.1
  :A: -- AUC5.3.3.3.2

  @enduml
  \end{plantuml}
  \caption{AUC5.3.3.3: Modifica luogo}
  \label{fig:AUC5_3_3_3}
\end{figure}

\begin{description}
  \item[Codice:] AUC5.3.3.3
  \item[Titolo:] Modifica luogo
  \item[Attori primari:] amministratore
  \item[Precondizione:] il luogo dell'organizzazione deve essere presente in \emph{Stalker}.
  \item[Postcondizione:] il luogo dell'organizzazione viene modificato.
  \item[Scenario principale:]
  \begin{enumerate}
    \item sorge la necessità di modificare un luogo di un'organizzazione, senza interagire con il suo owner;
  \end{enumerate}
  \item[Inclusioni:]
  \begin{enumerate}
    \item Seleziona luogo\emph{[AUC5.3.3.4]};
  \end{enumerate}
\end{description}
% subsub:AUC5.3.3.3 (end)

\subsubsection{AUC5.3.3.3.1 - Modifica coordinate geografiche}%
\label{subsub:AUC5.3.3.3.1}
\begin{description}
  \item[Codice:] AUC5.3.3.3.1
  \item[Titolo:] Modifica coordinate geografiche
  \item[Attori primari:] amministratore
  \item[Precondizione:] il sistema deve fornire i campi relativi alle coordinate geografiche.
  \item[Postcondizione:] i campi relativi alle coordinate geografiche sono modificati.
  \item[Scenario principale:]
  \begin{enumerate}
    \item l'amministratore vuole modificare le coordinate geografiche di un luogo;
  \end{enumerate}
\end{description}
% subsub:AUC5.3.3.3.1 (end)

\subsubsection{AUC5.3.3.3.2 - Modifica indirizzo luogo}%
\label{subsub:AUC5.3.3.3.2}
\begin{description}
  \item[Codice:] AUC5.3.3.3.2
  \item[Titolo:] Modifica indirizzo luogo
  \item[Attori primari:] amministratore
  \item[Precondizione:] il sistema deve fornire il campo relativo all'indirizzo del luogo.
  \item[Postcondizione:] il campo relativo all'indirizzo del luogo è modificato.
  \item[Scenario principale:]
  \begin{enumerate}
    \item l'amministratore vuole modificare l'indirizzo di un luogo;
  \end{enumerate}
\end{description}
% subsub:AUC5.3.3.3.2 (end)

\subsubsection{AUC5.3.3.4 - Seleziona luogo}%
\label{subsub:AUC5.3.3.4}
\begin{description}
  \item[Codice:] AUC5.3.3.4
  \item[Titolo:] Seleziona luogo
  \item[Attori primari:] amministratore
  \item[Precondizione:] il sistema deve mostrare la lista dei luoghi all'interno di una organizzazione.
  \item[Postcondizione:] viene scelto il luogo desiderato.
  \item[Scenario principale:]
  \begin{enumerate}
    \item sorge la necessità di effettuare operazioni sul luogo di un'organizzazione, e viene offerta la possibilità di selezionarlo;
  \end{enumerate}
\end{description}
% subsub:AUC5.3.3.4 (end)
% subsub:AUC5.3.4 (end)

\subsubsection{AUC5.3.4 - Modifica configurazione dettagli server LDAP}%
  \label{subsub:AUC5.3.4}
  \begin{description}
    \item[Codice:] AUC5.3.4
    \item[Titolo:] Configurazione dettagli server LDAP
    \item[Attori primari:] amministratore
    \item[Precondizione:] il sistema fornisce i campi per la configurazione del server LDAP.
    \item[Postcondizione:] il server LDAP è stato configurato.
    \item[Scenario principale:]
    \begin{enumerate}
      \item l'amministratore vuole modificare la configurazione del server LDAP.
    \end{enumerate}
  \end{description}
  % subsub:AUC5.3.4 (end)

\subsubsection{AUC5.4 - Seleziona organizzazione}%
\label{subsub:AUC5.4}
\begin{description}
  \item[Codice:] AUC5.4
  \item[Titolo:] Seleziona organizzazione
  \item[Attori primari:] amministratore
  \item[Precondizione:] il sistema deve mostrare la lista di organizzazioni in \emph{Stalker}.
  \item[Postcondizione:] viene scelta l'organizzazione desiderata.
  \item[Scenario principale:]
  \begin{enumerate}
    \item sorge la necessità di effettuare operazioni su un'organizzazione, e viene offerta la possibilità di selezionarla;
  \end{enumerate}
\end{description}
% subsub:AUC5.4 (end)

\subsubsection{AUC5.5 - Invio richiesta aggiornamento lista organizzazioni}%
\label{subsub:AUC5.5}
\begin{description}
  \item[Codice:] AUC5.5
  \item[Titolo:] Invio richiesta aggiornamento lista organizzazioni
  \item[Attori primari:] amministratore
  \item[Precondizione:] il sistema mostra la pagina di creazione o eliminazione di un'organizzazione.
  \item[Postcondizione:] la nuova lista delle organizzazioni viene inviata a tutte le applicazioni mobile.
  \item[Scenario principale:]
  \begin{enumerate}
    \item una volta creata o eliminata un'organizzazione, la lista delle organizzazioni viene aggiornata e inviata a tutti gli utenti che hanno installato l'applicazione mobile.
  \end{enumerate}
\end{description}
% subsub:AUC5.5 (end)
% subsub:AUC5 (end)

\subsubsection{AUC6 - Eliminazione account}%
\label{subsub:AUC6}

\begin{figure}[h!]
  \centering
  \begin{plantuml}
  @startuml
  !include ../../commons/style/use-cases.pu
  scale 3/4

  actor :amministratore: as A

  rectangle {
    together {
      usecase (AUC6.1) as "AUC6.1\nSeleziona account"
    }
  }

  :A: -- AUC6.1

  @enduml
  \end{plantuml}
  \caption{AUC6: Eliminazione account}
  \label{fig:auc6}
\end{figure}

\begin{description}
  \item[Codice:] AUC6
  \item[Titolo:] Eliminazione account
  \item[Attori primari:] amministratore
  \item[Precondizione:] deve essere stato selezionato l'\glossario{account} da eliminare, che deve esistere in \emph{Stalker};
  \item[Postcondizione:] l'account selezionato è stato eliminato.
  \item[Scenario principale:]
  \begin{enumerate}
    \item sorge la necessità di eliminare un account;
  \end{enumerate}
\end{description}

\subsubsection{AUC6.1 - Seleziona account}%
\label{subsub:AUC6.1}
\begin{description}
  \item[Codice:] AUC6.1
  \item[Titolo:] Seleziona account
  \item[Attori primari:] amministratore
  \item[Precondizione:] il sistema deve rendere disponibile la lista degli account registrati a \emph{Stalker}.
  \item[Postcondizione:] l'account è stato selezionato.
  \item[Scenario principale:]
  \begin{enumerate}
    \item l'amministratore seleziona l'account da eliminare;
  \end{enumerate}
\end{description}
% subsub:AUC6.1 (end)
% subsub:AUC6 (end)

\subsubsection{AUC7 - Query sull'organizzazione}%
\label{subsub:AUC7}

\begin{figure}[h!]
  \centering
  \begin{plantuml}
  @startuml
  !include ../../commons/style/use-cases.pu
  scale 3/4

  actor :visualizzatore: as A

  rectangle {
    together {
      usecase (AUC7) as "AUC7\nQuery sull'organizzazione"
    }
  }

  :A: -- AUC7

  @enduml
  \end{plantuml}
  \caption{AUC7: Query sull'organizzazione}
  \label{fig:AUC7}
\end{figure}

\begin{description}
  \item[Codice:] AUC7
  \item[Titolo:] Query sull'organizzazione
  \item[Attori primari:] visualizzatore
  \item[Precondizione:] il sistema risponde correttamente alle interrogazioni;
  \item[Postcondizione:] il visualizzatore ottiene le informazioni di cui ha bisogno.
  \item[Scenario principale:]
  \begin{enumerate}
    \item il visualizzatore vuole ottenere delle informazioni riguardo l'organizzazione sulla quale opera;
  \end{enumerate}
\end{description}
% subsub:AUC7 (end)

\subsubsection{AUC8 - Query sul dipendente}%
\label{subsub:AUC8}

\begin{figure}[h!]
  \centering
  \begin{plantuml}
  @startuml
  !include ../../commons/style/use-cases.pu
  scale 3/4

  actor :visualizzatore: as A

  rectangle {
    together {
      usecase (AUC8) as "AUC8\nQuery sul dipendente"
    }
  }

  :A: -- AUC8

  @enduml
  \end{plantuml}
  \caption{AUC8: Query sul dipendente}
  \label{fig:AUC8}
\end{figure}

\begin{description}
  \item[Codice:] AUC8
  \item[Titolo:] Query sul dipendente
  \item[Attori primari:] visualizzatore
  \item[Precondizione:] il sistema risponde correttamente alle interrogazioni;
  \item[Postcondizione:] il visualizzatore ottiene le informazioni di cui ha bisogno.
  \item[Scenario principale:]
  \begin{enumerate}
    \item il visualizzatore vuole ottenere delle informazioni riguardo l'organizzazione sulla quale opera, in particolare riguardo gli accessi di uno specifico dipendente;
  \end{enumerate}
\end{description}
% subsub:AUC8 (end)

\subsubsection{AUC9 - Gestione luoghi del gestore}%
\label{subsub:AUC9}

\begin{figure}[h!]
  \centering
  \begin{plantuml}
  @startuml
  !include ../../commons/style/use-cases.pu
  scale 3/4

  actor :gestore: as A

  rectangle {
    together {
      usecase (AUC9.1) as "AUC10.1\nAggiunta luogo"
      usecase (AUC9.2) as "AUC10.2\nRimozione luogo"
      usecase (AUC9.3) as "AUC10.3\nModifica luogo"
    }
    together {
      usecase (AUC5.3.3.3.1) as "AUC5.3.3.3.1\nModifica coordinate geografiche"
      usecase (AUC5.3.3.3.2) as "AUC5.3.3.3.2\nModifica indirizzo luogo"
      usecase (AUC5.3.3.1.1) as "AUC5.3.3.1.1\nInserisci coordinate geografiche"
      usecase (AUC5.3.3.1.2) as "AUC5.3.3.1.2\nInserisci indirizzo luogo"
    }
  }

  :A: -- AUC9.1
  :A: -- AUC9.2
  :A: -- AUC9.3

  (AUC5.3.3.3.1) .up.|> (AUC9.3) : <<include>>
  (AUC5.3.3.3.2) .up.|> (AUC9.3) : <<include>>
  (AUC5.3.3.1.1) .up.|> (AUC9.1) : <<include>>
  (AUC5.3.3.1.2) .up.|> (AUC9.1) : <<include>>

  @enduml
  \end{plantuml}
  \caption{AUC9: Gestione luoghi del gestore}
  \label{fig:AUC9}
\end{figure}

\begin{description}
  \item[Codice:] AUC9
  \item[Titolo:] Gestione luoghi del gestore
  \item[Attori primari:] gestore
  \item[Precondizione:] il sistema deve rendere disponibile la pagina della gestione dei luoghi.
  \item[Postcondizione:] il gestore è all'interno della pagina di gestione.
  \item[Scenario principale:]
  \begin{enumerate}
    \item il gestore vuole gestire i luoghi dell'organizzazione su cui opera;
  \end{enumerate}
\end{description}

\subsubsection{AUC9.1 - Aggiunta luogo}%
\label{subsub:AUC9.1}
\begin{description}
  \item[Codice:] AUC9.1
  \item[Titolo:] Aggiunta luogo
  \item[Attori primari:] gestore
  \item[Precondizione:] il luogo da aggiungere non deve già esistere;
  \item[Postcondizione:] viene aggiunto un nuovo luogo all'organizzazione corrente.
  \item[Scenario principale:]
  \begin{enumerate}
    \item il gestore vuole aggiungere un nuovo luogo all'organizzazione su cui opera;
  \end{enumerate}
  \item[Inclusioni:]
  \begin{enumerate}
    \item il gestore inserisce le coordinate geografiche da aggiungere \emph{[AUC5.3.3.1.1]};
    \item il gestore inseriscce l'indirizzo del luogo da aggiungere \emph{[AUC5.3.3.1.2]};
  \end{enumerate}
\end{description}
% subsub:AUC9.1 (end)

\subsubsection{AUC9.2 - Rimozione luogo}%
\label{subsub:AUC9.2}
\begin{description}
  \item[Codice:] AUC9.2
  \item[Titolo:] Rimozione luogo
  \item[Attori primari:] gestore
  \item[Precondizione:] il luogo da eliminare deve esistere.
  \item[Postcondizione:] il luogo selezionato è eliminato.
  \item[Scenario principale:]
  \begin{enumerate}
    \item il gestore vuole eliminare un luogo all'organizzazione su cui opera;
  \end{enumerate}
\end{description}
% subsub:AUC9.2 (end)

\subsubsection{AUC9.3 - Modifica luogo}%
\label{subsub:AUC9.3}
\begin{description}
  \item[Codice:] AUC9.3
  \item[Titolo:] Modifica luogo
  \item[Attori primari:] gestore
  \item[Precondizione:] il luogo da modificare deve esistere.
  \item[Postcondizione:] il luogo selezionato è modificato.
  \item[Scenario principale:]
  \begin{enumerate}
    \item il gestore vuole modificare un luogo dell'organizzazione su cui opera;
  \end{enumerate}
  \item[Inclusioni:]
  \begin{enumerate}
    \item il gestore modifica le coordinate geografiche del luogo selezionato \emph{[AUC5.3.3.3.1]};
    \item il gestore modifica l'indirizzo del luogo selezionato \emph{[AUC5.3.3.3.2]};
  \end{enumerate}
\end{description}
% subsub:AUC9.3 (end)
% subsub:AUC9 (end)

\subsubsection{AUC10 - Richiesta modifica organizzazione}%
\label{subsub:AUC10}

\begin{figure}[h!]
  \centering
  \begin{plantuml}
  @startuml
  !include ../../commons/style/use-cases.pu

  actor :owner: as A

  rectangle {
    together {
      usecase (AUC10) as "AUC10\nRichiesta modifica organizzazione"
    }
    together {
      usecase (AUC5.3.4) as "AUC5.3.4\nModifica configurazione dettagli server LDAP"
      usecase (AUC5.3.3) as "AUC5.3.3\nModifica descrizione organizzazione"
      usecase (AUC5.3.2) as "AUC5.3.2\nModifica indirizzo organizzazione"
      usecase (AUC5.3.1) as "AUC5.3.1\nModifica nome organizzazione"
    }
  }

  :A: -- AUC10

  (AUC5.3.1) .up.|> (AUC10) : <<include>>
  (AUC5.3.2) .up.|> (AUC10) : <<include>>
  (AUC5.3.3) .up.|> (AUC10) : <<include>>
  (AUC5.3.4) .up.|> (AUC10) : <<include>>

  @enduml
  \end{plantuml}
  \caption{AUC10: Richiesta modifica organizzazione}
  \label{fig:AUC10}
\end{figure}

\begin{description}
  \item[Codice:] AUC10
  \item[Titolo:] Richiesta modifica organizzazione
  \item[Attori primari:] owner
  \item[Precondizione:] l'organizzazione deve esistere.
  \item[Postcondizione:] la richiesta di modificare un+organizzazione esistente è stata posta.
  \item[Scenario principale:]
  \begin{enumerate}
    \item l'owner vuole modificare un'organizzazione di sua gestione;
  \end{enumerate}
  \item[Inclusioni:]
  \begin{itemize}
    \item In caso di richiesta accettata, sono permesse tutte le operazioni di modifica:
    \begin{enumerate}
      \item inserimento nome organizzazione \emph{[AUC5.3.1]};
      \item inserimento indirizzo organizzazione \emph{[AUC5.3.2]};
      \item inserimento descrizione organizzazione \emph{[AUC5.3.3]};
      \item configurazione dettagli server LDAP \emph{[AUC5.3.4]}.
    \end{enumerate}
  \end{itemize}
\end{description}

\subsubsection{AUC11 - Richiesta creazione organizzazione}%
\label{subsub:AUC11}

\begin{figure}[h!]
  \centering
  \begin{plantuml}
  @startuml
  !include ../../../../../commons/style/use-cases.pu
  scale 3/4

  actor :owner: as A

  rectangle {
    together {
      usecase (AUC11) as "AUC11\nRichiesta creazione organizzazione"
    }
    together {
      usecase (AUC5.1.3) as "AUC5.1.3\nConfigurazione dettagli server LDAP"
      usecase (AUC5.1.2) as "AUC5.1.2\nInserisci descrizione organizzazione"
      usecase (AUC5.1.1) as "AUC5.1.1\nInserisci nome organizzazione"
    }
  }

  :A: -- AUC11

  (AUC5.1.1) .up.|> (AUC11) : <<include>>
  (AUC5.1.2) .up.|> (AUC11) : <<include>>
  (AUC5.1.3) .up.|> (AUC11) : <<include>>

  @enduml
  \end{plantuml}
  \caption{AUC11: Richiesta creazione organizzazione}
  \label{fig:auc11}
\end{figure}

\begin{description}
  \item[Codice:] AUC11
  \item[Titolo:] Richiesta creazione organizzazione
  \item[Attori primari:] owner
  \item[Precondizione:] l'organizzazione non deve già esistere.
  \item[Postcondizione:] la richiesta di creare una nuova organizzazione è stata posta.
  \item[Scenario principale:]
  \begin{enumerate}
    \item l'owner vuole creare una nuova organizzazione.
  \end{enumerate}
  \item[Inclusioni:]
  \begin{itemize}
    \item In caso di richiesta accettata, sono permesse tutte le operazioni di modifica:
    \begin{enumerate}
      \item Inserimento nome organizzazione \emph{[AUC5.1.1]}.
      \item Inserimento descrizione organizzazione \emph{[AUC5.1.2]}.
      \item Configurazione dettagli server LDAP \emph{[AUC5.1.3]}
    \end{enumerate}
  \end{itemize}
\end{description}
% subsub:AUC11 (end)

\subsubsection{AUC12 - Richiesta cessione proprietà organizzazione}%
\label{subsub:AUC12}

\begin{figure}[h!]
  \centering
  \begin{plantuml}
  @startuml
  !include ../../commons/style/use-cases.pu
  scale 3/4

  actor :owner: as A

  rectangle {
    together {
      usecase (AUC12.1) as "AUC12.1\nInserimento email nuovo owner"
    }
  }

  :A: -- AUC12.1

  @enduml
  \end{plantuml}
  \caption{AUC12: Richiesta cessione proprietà organizzazione}
  \label{fig:auc12}
\end{figure}

\begin{description}
  \item[Codice:] AUC12
  \item[Titolo:] Richiesta cessione proprietà organizzazione
  \item[Attori primari:] owner
  \item[Precondizione:] il futuro owner deve esistere.
  \item[Postcondizione:] la richiesta di cedere l'organizzazione è stata posta.
  \item[Scenario principale:]
  \begin{enumerate}
    \item l'owner vuole cedere l'organizzazione ad un'altro owner;
  \end{enumerate}
\end{description}
% subsub:AUC12 (end)

\subsubsection{AUC12.1 - Inserimento email nuovo owner}%
\label{subsub:AUC12.1}
\begin{description}
  \item[Codice:] AUC12.1
  \item[Titolo:] Inserimento email nuovo owner
  \item[Attori primari:] owner
  \item[Precondizione:] il sistema deve rendere disponibile il campo per l'inserimento della email del nuovo owner.
  \item[Postcondizione:] il campo relativo alla mail del nuovo owner è compilato.
  \item[Scenario principale:]
  \begin{enumerate}
    \item l'owner inserisce l'email del nuovo owner;
  \end{enumerate}
\end{description}
% subsub:AUC12.1 (end)

\subsubsection{AUC13 - Richiesta aggiunta visualizzatore}%
\label{subsub:AUC13}

\begin{figure}[h!]
  \centering
  \begin{plantuml}
  @startuml
  !include ../../commons/style/use-cases.pu
  scale 3/4

  actor :owner: as A

  rectangle {
    together {
      usecase (AUC13.1) as "AUC13.1\nInserimento email visualizzatore"
    }
  }

  :A: -- AUC13.1

  @enduml
  \end{plantuml}
  \caption{AUC13: Aggiunta visualizzatore}
  \label{fig:auc13}
\end{figure}

\begin{description}
  \item[Codice:] AUC13
  \item[Titolo:] Aggiunta visualizzatore
  \item[Attori primari:] owner
  \item[Precondizione:] il nuovo super utente non deve già esistere come visualizzatore.
  \item[Postcondizione:] viene aggiunto il visualizzatore specificato dall'owner.
  \item[Scenario principale:]
  \begin{enumerate}
    \item l'owner vuole aggiungere un visualizzatore alla sua organizzazione.
  \end{enumerate}
\end{description}
% subsub:AUC13 (end)

\subsubsection{AUC13.1 - Inserimento email visualizzatore}%
\label{subsub:AUC13.1}
\begin{description}
  \item[Codice:] AUC13.1
  \item[Titolo:] Inserimento email visualizzatore
  \item[Attori primari:] owner
  \item[Precondizione:] il sistema deve rendere disponibile il campo per l'inserimento della mail del nuovo visualizzatore.
  \item[Postcondizione:] il campo relativo al visualizzatore viene compilato.
  \item[Scenario principale:]
  \begin{enumerate}
    \item l'owner inserisce l'email del nuovo visualizzatore;
  \end{enumerate}
\end{description}
% subsub:AUC13.1 (end)

\subsubsection{AUC14 - Aggiunta gestore}%
\label{subsub:AUC14}

\begin{figure}[h!]
  \centering
  \begin{plantuml}
  @startuml
  !include ../../commons/style/use-cases.pu
  scale 3/4

  actor :owner: as A

  rectangle {
    together {
      usecase (AUC14.1) as "AUC14.1\nInserimento email gestore"
    }
  }

  :A: -- AUC14.1

  @enduml
  \end{plantuml}
  \caption{AUC14: Aggiunta gestore}
  \label{fig:auc14}
\end{figure}

\begin{description}
  \item[Codice:] AUC14
  \item[Titolo:] Aggiunta gestore
  \item[Attori primari:] owner
  \item[Precondizione:] il nuovo super utente non deve già esistere come gestore.
  \item[Postcondizione:] viene aggiunto il gestore specificato dall'owner.
  \item[Scenario principale:]
  \begin{enumerate}
    \item l'owner vuole aggiungere un gestore alla sua organizzazione.
  \end{enumerate}
\end{description}
% subsub:AUC14 (end)

\subsubsection{AUC14.1 - Inserimento email gestore}%
\label{subsub:AUC14.1}
\begin{description}
  \item[Codice:] AUC14.1
  \item[Titolo:] Inserimento email gestore
  \item[Attori primari:] owner
  \item[Precondizione:] il sistema deve rendere disponibile il campo per l'inserimento della mail del nuovo gestore.
  \item[Postcondizione:] il campo relativo al gestore viene compilato.
  \item[Scenario principale:]
  \begin{enumerate}
    \item l'owner inserisce l'email del nuovo gestore;
  \end{enumerate}
\end{description}
% subsub:AUC14.1 (end)

\subsubsection{AUC15 - Eliminazione owner}%
\label{subsub:AUC15}

\begin{figure}[h!]
  \centering
  \begin{plantuml}
  @startuml
  !include ../../commons/style/use-cases.pu
  scale 3/4

  actor :amministratore: as A

  rectangle {
    together {
      usecase (AUC15.1) as "AUC15.1\nInserimento email owner"
    }
  }

  :A: -- AUC15.1

  @enduml
  \end{plantuml}
  \caption{AUC15: Eliminazione owner}
  \label{fig:auc15}
\end{figure}

\begin{description}
  \item[Codice:] AUC15
  \item[Titolo:] Eliminazione owner
  \item[Attori primari:] amministratore
  \item[Precondizione:] l'owner deve esistere.
  \item[Postcondizione:] l'utente non è più owner.
  \item[Scenario principale:]
  \begin{enumerate}
    \item l'amministratore vuole eliminare i privilegi ad un'utente owner. Le credenziali dell'utente rimangono.
  \end{enumerate}
\end{description}
% subsub:AUC15 (end)

\subsubsection{AUC15.1 - Inserimento email owner}%
\label{subsub:AUC15.1}
\begin{description}
  \item[Codice:] AUC15.1
  \item[Titolo:] Inserimento email owner
  \item[Attori primari:] amministratore
  \item[Precondizione:] il sistema deve rendere disponibile il campo per l'inserimento della mail dell'owner da eliminare.
  \item[Postcondizione:] il campo relativo all'owner viene compilato.
  \item[Scenario principale:]
  \begin{enumerate}
    \item l'amministratore inserisce l'email dell'owner da eliminare.
  \end{enumerate}
\end{description}
% subsub:AUC15.1 (end)

\subsubsection{AUC16 - Eliminazione gestore}%
\label{subsub:AUC16}

\begin{figure}[h!]
  \centering
  \begin{plantuml}
  @startuml
  !include ../../commons/style/use-cases.pu
  scale 3/4

  actor :amministratore: as A1
  actor :owner: as A2


  rectangle {
    together {
      usecase (AUC16.1) as "AUC16.1\nInserimento email gestore"
    }
  }

  :A1: -- AUC16.1
  :A2: -- AUC16.1

  @enduml
  \end{plantuml}
  \caption{AUC16: Eliminazione gestore}
  \label{fig:auc16}
\end{figure}

\begin{description}
  \item[Codice:] AUC16
  \item[Titolo:] Eliminazione gestore
  \item[Attori primari:] amministratore, owner
  \item[Precondizione:] il gestore deve esistere.
  \item[Postcondizione:] l'utente non è più gestore.
  \item[Scenario principale:]
  \begin{enumerate}
    \item l'amministratore, oppure l'owner, vuole eliminare i privilegi ad un'utente gestore. Le credenziali dell'utente rimangono.
  \end{enumerate}
\end{description}
% subsub:AUC16 (end)

\subsubsection{AUC16.1 - Inserimento email gestore}%
\label{subsub:AUC16.1}
\begin{description}
  \item[Codice:] AUC16.1
  \item[Titolo:] Inserimento email gestore
  \item[Attori primari:] amministratore, owner
  \item[Precondizione:] il sistema deve rendere disponibile il campo per l'inserimento della mail del gestore da eliminare.
  \item[Postcondizione:] il campo relativo al gestore viene compilato.
  \item[Scenario principale:]
  \begin{enumerate}
    \item l'amministratore, oppure l'owner, inserisce l'email del gestore da eliminare.
  \end{enumerate}
\end{description}
% subsub:AUC16.1 (end)

\subsubsection{AUC17 - Eliminazione visualizzatore}%
\label{subsub:AUC17}

\begin{figure}[h!]
  \centering
  \begin{plantuml}
  @startuml
  !include ../../commons/style/use-cases.pu
  scale 3/4

  actor :amministratore: as A1
  actor :owner: as A2

  rectangle {
    together {
      usecase (AUC17.1) as "AUC17.1\nInserimento email visualizzatore"
    }
  }

  :A1: -- AUC17.1
  :A2: -- AUC17.1

  @enduml
  \end{plantuml}
  \caption{AUC17: Eliminazione visualizzatore}
  \label{fig:auc17}
\end{figure}

\begin{description}
  \item[Codice:] AUC17
  \item[Titolo:] Eliminazione visualizzatore
  \item[Attori primari:] amministratore, owner
  \item[Precondizione:] il visualizzatore deve esistere.
  \item[Postcondizione:] l'utente non è più visualizzatore.
  \item[Scenario principale:]
  \begin{enumerate}
    \item l'amministratore, oppure l'owner, vuole eliminare i privilegi ad un'utente visualizzatore. Le credenziali dell'utente rimangono.
  \end{enumerate}
\end{description}
% subsub:AUC17 (end)

\subsubsection{AUC17.1 - Inserimento email visualizzatore}%
\label{subsub:AUC17.1}
\begin{description}
  \item[Codice:] AUC17.1
  \item[Titolo:] Inserimento email visualizzatore
  \item[Attori primari:] amministratore, owner
  \item[Precondizione:] il sistema deve rendere disponibile il campo per l'inserimento della mail del visualizzatore da eliminare.
  \item[Postcondizione:] il campo relativo al visualizzatore viene compilato.
  \item[Scenario principale:]
  \begin{enumerate}
    \item l'amministratore, oppure l'owner, inserisce l'email del visualizzatore da eliminare.
  \end{enumerate}
\end{description}
% subsub:AUC17.1 (end)

\end{document}
