\documentclass[../piano-di-progetto.tex]{subfiles}

\begin{document}
Per ogni periodo appena trascorso, il gruppo presenta il consuntivo delle ore effettivamente lavorate per ruolo, assieme ai corrispettivi costi, indicando le differenze rispetto ai totali riportati nel preventivo e mostrando le cause di tali differenze.
Differenze positive, negative o nulle rappresentano un carico di lavoro superiore, inferiore o uguale a quello preventivato.
\subsection{Fase di Analisi preliminare}%
\label{sub:consuntivo_di_periodo/fase_di_analisi_preliminare}
\begin{table}[H]
  \centering
  \rowcolors{2}{lightgray}{white!80!lightgray!100}
  \renewcommand{\arraystretch}{2}
  \begin{tabular}{c c c}
    \rowcolor{darkgray!90!}\color{white}{\textbf{Ruolo}} & \color{white}{\textbf{Totale ore}} & \color{white}{\textbf{Costo}} \\
    Re&42 (-5)&1 260,00€ (-150,00€)\\
    Am&56 (+3)&1 120,00€ (+60,00€)\\
    An&104 (0)&2 600,00€ (0,00€)\\
    Pt&21 (+9)&462,00€ (+198,00€)\\
    Pr&-&-\\
    Ve&49 (-5)&735,00€ (-75,00€)\\
    \textbf{Totale}&272 (+2)&6 177,00€ (+108,00€)\\
  \end{tabular}
  \caption{Consuntivo di periodo per l'Analisi preliminare}%
~~\label{tab:consuntivo_di_periodo_analisi_preliminare}
\end{table}
Le differenze individuate si possono ricondurre all'inesperienza dei membri del gruppo, che ha causato errori di valutazione nel calcolo delle ore a preventivo e nel carico di lavoro da dedicare alle attività da svolgere.
I 108,00€ eccedenti il preventivo non hanno influenza sulla pianificazione delle fasi successive, in quanto compresi nelle fasi di investimento, e quindi non rendicontati.
% sub:fase_di_analisi_preliminare (end)
\subsection{Fase di Preparazione in entrata alla RR}%
\label{sub:fase_di_preparazione_in_entrata_alla_rr}
\begin{table}[H]
  \centering
  \rowcolors{2}{lightgray}{white!80!lightgray!100}
  \renewcommand{\arraystretch}{2}
  \begin{tabular}{c c c}
    \rowcolor{darkgray!90!}\color{white}{\textbf{Ruolo}} & \color{white}{\textbf{Totale ore}} & \color{white}{\textbf{Costo}} \\
    Re&-&-\\
    Am&24 (0)&480,00€ (0,00€)\\
    An&-&-\\
    Pt&-&-\\
    Pr&-&-\\
    Ve&16 (0)&240,00€ (0,00€)\\
    \textbf{Totale}&40 (0)&720,00€ (0,00€)\\
  \end{tabular}
  \caption{Consuntivo di periodo per la Preparazione in entrata alla RR}%
~~\label{tab:consuntivo_di_periodo_preparazione_in_entrata_alla_rr}
\end{table}
La breve durata di questo periodo ha facilitato il rispetto delle scadenze, permettendo di evitare discostamenti dalle ore pianificate.
% sub:fase_di_preparazione_in_entrata_alla_rr (end)
\subsection{Fase di Progettazione architetturale}%
\label{sub:consuntivo_di_periodo/fase_di_progettazione_architetturale}
\begin{table}[H]
  \centering
  \rowcolors{2}{lightgray}{white!80!lightgray!100}
  \renewcommand{\arraystretch}{2}
  \begin{tabular}{c c c}
    \rowcolor{darkgray!90!}\color{white}{\textbf{Ruolo}} & \color{white}{\textbf{Totale ore}} & \color{white}{\textbf{Costo}} \\
    Re&10 (+4)&300,00€ (+120,00€)\\
    Am&10 (+4)&200,00€ (+80,00€)\\
    An&10 (0)&250,00€ (0,00€)\\
    Pt&72 (+9)&1 584,00€ (+198,00€)\\
    Pr&64 (-7)&960,00€ (-105,00€)\\
    Ve&42 (-10)&630,00€ (-150,00€)\\
    \textbf{Totale}&208 (0)&3 924,00€ (+143,00€)\\
  \end{tabular}
  \caption{Consuntivo di periodo per la Progettazione architetturale}%
  \label{tab:consuntivo_di_periodo_progettazione_architetturale}
\end{table}
\subsubsection{Modifica alla pianificazione}%
\label{subs:modifica_alla_pianificazione}

Dopo la Revisione dei requisiti, il gruppo ha sospeso la propria attività per sostenere i rispettivi esami, portando ad un ritardo rispetto la pianificazione.
In particolare, la conseguenza principale è stata l'impossibilità di progettare e sviluppare il POC in tempo per svolgere il colloquio di Technology Baseline con il committente.
In accordo con il proponente, il giorno 18 febbraio 2020 GruppOne ha deciso di posporre la propria Revisione di progettazione, saltando la revisione di marzo, e ridistribuendo la propria pianificazione sfruttando il mese aggiuntivo.
Il gruppo ha inoltre colto la modifica alla pianificazione come opportunità per correggere l'errore formale sul minimo preventivo imposto dal committente.
Grazie ai tempi più rilassati, il gruppo ha scelto di pianificare gli incrementi ad un ritmo meno serrato, in particolare spostando i primi due al periodo precedente alla Revisione di progettazione.
Inoltre, il gruppo ha pianificato il lavoro su base settimanale, mantenendo i periodi e le fasi indicati in §\ref{sec:pianificazione} come semplice assembramento di alto livello volto a presentare al proponente ed al committente la distribuzione temporale delle attività.
Il gruppo ha infine pianificato degli incontri settimanali atti a fare il punto della situazione ed eventualmente a modificare la pianificazione in caso di discostamenti.

% subs:modifica_alla_pianificazione (end)
\subsubsection{Progettazione architetturale e POC}%
\label{subs:progettazione_architetturale_e_poc}

Nel corso della progettazione architetturale e dello sviluppo del POC, a causa di incomprensioni fra i membri, il gruppo ha speso più tempo del necessario su aspetti che aveva pianificato di abbozzare, e successivamente raffinare durante gli incrementi, e non abbastanza su aspetti che invece erano indispensabili alla produzione di un POC funzionante.
La discrepanza fra le ore pianificate e quelle realmente effettuate non è particolarmente accentuata in quanto parte del gruppo ha frainteso gli obiettivi della fase corrente, ma ha comunque svolto lavoro che risulterà utile nei successivi incrementi, in particolare durante gli incrementi.
La differenza più consistente per quanto riguarda il ruolo di verificatore si può ricondurre in parte alla minore quantità di codice prodotto, ma è troppo accentuata perché questa sia l'unica causa.
Di conseguenza, per i prossimi incrementi le ore da verificatore pianificate saranno ridotte.

% subs:progettazione_architetturale_e_poc (end)
% sub:fase_di_progettazione_architetturale (end)
\subsection{Incremento 1}%
\label{sub:consuntivo_di_periodo/incremento_1}


\begin{table}[H]
  \centering
  \rowcolors{2}{lightgray}{white!80!lightgray!100}
  \renewcommand{\arraystretch}{2}
  \begin{tabular}{c c c}
    \rowcolor{darkgray!90!}\color{white}{\textbf{Ruolo}} & \color{white}{\textbf{Totale ore}} & \color{white}{\textbf{Costo}} \\
    Re&6 (-3)&180,00€ (-90,00€)\\
    Am&6 (0)&120,00€ (0,00€)\\
    An&-&-\\
    Pt&20 (+3)&440,00€ (+66,00€)\\
    Pr&16 (+3)&240,00€ (+45,00€)\\
    Ve&12 (-1)&180,00€ (-15,00€)\\
    \textbf{Totale}&72 (2)&1 220,00€ (+6,00€)\\
  \end{tabular}
  \caption{Consuntivo di periodo per l'Incremento 1}%
  \label{tab:consuntivo_di_periodo_incremento_1}
\end{table}

Nonostante l'eccesso di ore di progettazione e programmazione, gli obiettivi dell'incremento non sono stati completamente raggiunti, principalmente a causa di una sottovalutazione della complessità delle tecnologie da utilizzare.
Di conseguenza, il gruppo ha deciso di unificare l'Incremento 2 e la Preparazione in entrata alla RP, e sfruttare la settimana durante la quale si sarebbe dovuto svolgere l'Incremento 2 per svolgere un autoapprendimento più approfondito, con l'obiettivo di appianare il ritardo rispetto la pianificazione entro la data della Revisione di progettazione.

% sub:incremento_1 (end)
\subsection{Incremento 2 e Preparazione alla RP}%
\label{sub:consuntivo_di_periodo/incremento_2_preparazione_rp}
\begin{table}[H]
  \centering
  \rowcolors{2}{lightgray}{white!80!lightgray!100}
  \renewcommand{\arraystretch}{2}
  \begin{tabular}{c c c}
    \rowcolor{darkgray!90!}\color{white}{\textbf{Ruolo}} & \color{white}{\textbf{Totale ore}} & \color{white}{\textbf{Costo}} \\
    Re&6 (-1)&180,00€ (-30,00€)\\
    Am&22 &440,00€\\
    An&0 (+2)&- (+50,00€)\\
    Pt&24 (-3)&528,00€ (-66,00€)\\
    Pr&21 (+4)&315,00€ (+60,00€)\\
    Ve&28&420,00€\\
    \textbf{Totale}&64 (+3)&1 883,00€ (+14,00€)\\
  \end{tabular}
  \caption{Consuntivo di periodo per l'incremento 2}%
~~\label{tab:consuntivo_incremento_2}
\end{table}
L'eccesso delle ore di programmazione unito al difetto nelle ore di progettazione ci fa capire che l'attività di pianificazione svolta durante la fase di progettazione architetturale non ha richiesto correzioni come avevamo previsto nella pianificazione.
L'attività del programmatore dunque ha potuto proseguire ad un ritmo maggiore aumentando il numero di ore impiegate nell'incremento per questo ruolo.
Questo aumento è dovuto anche ad un debito accumulato nello scorso incremento che in questo periodo, seppur con un aumento delle ore complessive, siamo riusciti a colmare
Diminuire le ore da verificatore in questo incremento si è rivelata una decisione corretta in quanto abbiamo rispettato le ore pianificate.

% sub:incremento_2 (end)
\subsection{Incremento 3}%
\label{sub:consuntivo_di_periodo/incremento_3}
\begin{table}[H]
  \centering
  \rowcolors{2}{lightgray}{white!80!lightgray!100}
  \renewcommand{\arraystretch}{2}
  \begin{tabular}{c c c}
    \rowcolor{darkgray!90!}\color{white}{\textbf{Ruolo}} & \color{white}{\textbf{Totale ore}} & \color{white}{\textbf{Costo}} \\
    Re&3 (-3)&180,00€ (-90,00€)\\
    Am&3 (-3)&120,00€ (-60,00€)\\
    An&-&-\\
    Pt&20 (-2)&440,00€ (-44,00€)\\
    Pr&21 (+ 10)&315,00€ (+150,00€)\\
    Ve&12&180,00€\\
    \textbf{Totale}&64&1 235,00€ (-44,00€)\\
  \end{tabular}
  \caption{Consuntivo di periodo per l'incremento 3}%
~~\label{tab:consuntivo_incremento_3}
\end{table}
La buona coordinazione del gruppo ha reso necessarie un minor numero di ore responsabile, la configurazione predisposta negli incrementi precedenti inoltre si è rivelata completa e ha consentito di utilizzare meno ore da amministratore di quelle pianificate.
Tuttavia le ore programmatore risultano ancora una volta maggiori rispetto a quanto preventivato.
La spiegazione di questo calo nelle ore responsabile è probabilmente da ricercare nel tanto lavoro svolto da questi ruoli nella fase di progettazione architetturale che ci ha permesso di avere a disposizione strumenti per la gestione del lavoro e una configurazione molto efficiente che ha reso meno necessario l'intervento di questi due ruoli nei successivi incrementi.

% sub:incremento_3 (end)
\subsection{Incremento 4}%
\label{sub:consuntivo_di_periodo/incremento_4}
\begin{table}[H]
  \centering
  \rowcolors{2}{lightgray}{white!80!lightgray!100}
  \renewcommand{\arraystretch}{2}
  \begin{tabular}{c c c}
    \rowcolor{darkgray!90!}\color{white}{\textbf{Ruolo}} & \color{white}{\textbf{Totale ore}} & \color{white}{\textbf{Costo}} \\
    Re&4 &120,00€ \\
    Am&4 (-1)&80,00€ (-20,00€)\\
    An&-&-\\
    Pt&20&440,00€\\
    Pr&28 (+2)&420,00€ (+30,00€)\\
    Ve&12 (-1)&180,00€ (-15,00€)\\
    \textbf{Totale}&64&1 240,00€ (+10,00€)\\
  \end{tabular}
  \caption{Consuntivo di periodo per l'incremento 4}%
~~\label{tab:consuntivo_periodo_incremento_4}
\end{table}

Le ore da programmatore occupano ancora una parte consistente dell'incremento, confermando la scelta di ridurre le ore responsabile e amministratore a favore delle suddette.
In questo periodo il gruppo sembra aver trovato in generale un buon equilibrio tra le responsabilità generando solo dei lievi discostamenti orari.
% sub:incremento_4 (end)


\subsection{Incremento 5 e Preparazione RQ}%
\label{sub:consuntivo_di_periodo/incremento_5_preparazione_rq}
\begin{table}[H]
  \centering
  \rowcolors{2}{lightgray}{white!80!lightgray!100}
  \renewcommand{\arraystretch}{2}
  \begin{tabular}{c c c}
    \rowcolor{darkgray!90!}\color{white}{\textbf{Ruolo}} & \color{white}{\textbf{Totale ore}} & \color{white}{\textbf{Costo}} \\
    Re&4 &120,00€ \\
    Am&20&400,00€\\
    An&-&-\\
    Pt&20&440,00€\\
    Pr&28 (+3)&420,00€ (+45,00€)\\
    Ve&28 (-1)&420,00€ (-15,00€)\\
    \textbf{Totale}&64&1 800,00€ (+30,00€)\\
  \end{tabular}
  \caption{Consuntivo di periodo per l'incremento 5 e preparazione RQ}%
~~\label{tab:consuntivo_periodo_incremento_5_preparazione_rq}
\end{table}

Anche in questo incremento si sono verificati solamente lievi discostamenti orari confermando la bontà della pianificazione modificata.
% sub:incremento_4 (end)

\subsection{Incremento 6}%
\label{sub:consuntivo_di_periodo/incremento_6}
\begin{table}[H]
  \centering
  \rowcolors{2}{lightgray}{white!80!lightgray!100}
  \renewcommand{\arraystretch}{2}
  \begin{tabular}{c c c}
    \rowcolor{darkgray!90!}\color{white}{\textbf{Ruolo}} & \color{white}{\textbf{Totale ore}} & \color{white}{\textbf{Costo}} \\
    Re&4 &120,00€ \\
    Am&4 (-1)&80,00€ (-20,00€)\\
    An&-&-\\
    Pt&20 (-6)&440,00€ (-132,00€)\\
    Pr&28 (-1)&420,00€ (-15,00€)\\
    Ve&12 &180,00€ \\
    \textbf{Totale}&64&1 240,00€ (-167,00€)\\
  \end{tabular}
  \caption{Consuntivo di periodo per l'incremento 6}%
~~\label{tab:consuntivo_periodo_incremento_6}
In questo incremento notiamo invece una lieve diminuzione delle ore preventivate, dovuta principalmente ad un rilassamento del gruppo dopo la revisione e l'esame scritto di Ingegneria del Software a cui alcuni dei componenti del gruppo hanno partecipato.
\end{table}


% sub:incremento_4 (end)

\subsection{Incremento 7}%
\label{sub:consuntivo_di_periodo/incremento_7}
\begin{table}[H]
  \centering
  \rowcolors{2}{lightgray}{white!80!lightgray!100}
  \renewcommand{\arraystretch}{2}
  \begin{tabular}{c c c}
    \rowcolor{darkgray!90!}\color{white}{\textbf{Ruolo}} & \color{white}{\textbf{Totale ore}} & \color{white}{\textbf{Costo}} \\
    Re&4 &120,00€ \\
    Am&4 (-1)&80,00€ (-20,00€)\\
    An&-&-\\
    Pt&20 (-7)&440,00€ (-154,00€)\\
    Pr&28 (-1)&420,00€ (-15,00€)\\
    Ve&12 (-1)&180,00€ (-15,00€)\\
    \textbf{Totale}&64&1 240,00€ (-204,00€)\\
  \end{tabular}
  \caption{Consuntivo di periodo per l'incremento 7}%
~~\label{tab:consuntivo_periodo_incremento_7}
\end{table}

Trattandosi dell'ultimo incremento del progetto non erano rimaste molte funzionalità ancora da progettare dunque abbiamo notato un calo delle ore progettista a favore di quelle da programmatore, necessarie per finire di implementare tutte le funzionalità.
% sub:incremento_4 (end)

\subsection{Fase di Verifica e Collaudo finali}%
\label{sub:consuntivo_di_periodo/fase_verifica_collaudo_finali}
\begin{table}[H]
  \centering
  \rowcolors{2}{lightgray}{white!80!lightgray!100}
  \renewcommand{\arraystretch}{2}
  \begin{tabular}{c c c}
    \rowcolor{darkgray!90!}\color{white}{\textbf{Ruolo}} & \color{white}{\textbf{Totale ore}} & \color{white}{\textbf{Costo}} \\
    Re&8&240,00€\\
    Am&8&160,00€\\
    An&-&-\\
    Pt&-&-\\
    Pr&-&-\\
    Ve&24 (+1)&360,00€ (+15,00€)\\
    \textbf{Totale}&40&760,00€ (+15,00€)\\
  \end{tabular}
  \caption{Consuntivo di periodo per la verifica e collaudo finali}%
~~\label{tab:consuntivo_periodo_verifica_collaudo_finali}
\end{table}

Il lavoro in questo periodo è stato intenso ma si è svolto senza intoppi o ritardi anche grazie alla qualità del lavoro fatto finora.
% sub:incremento_4 (end)

\subsection{Fase di preparazione in entrata alla RA}%
\label{sub:consuntivo_di_periodo/incremento_4}
\begin{table}[H]
  \centering
  \rowcolors{2}{lightgray}{white!80!lightgray!100}
  \renewcommand{\arraystretch}{2}
  \begin{tabular}{c c c}
    \rowcolor{darkgray!90!}\color{white}{\textbf{Ruolo}} & \color{white}{\textbf{Totale ore}} & \color{white}{\textbf{Costo}} \\
    Re&-&-\\
    Am&16&320,00€\\
    An&-&-\\
    Pt&-&-\\
    Pr&-&-\\
    Ve&16 (-1)&240,00€ (-15,00€)\\
    \textbf{Totale}&32&560,00€ (-15,00€)\\
  \end{tabular}
  \caption{Consuntivo di periodo per la preparazione in entrata alla RA}%
~~\label{tab:consuntivo_periodo_preparazione_in_entrata_alla_ra}
\end{table}
In quest'ultima fase il gruppo ha preparato come previsto la presentazione finale del progetto oltre a rifinire qualche ultimo dettaglio implementativo che ha causato un eccesso di ore programmatore.
% sub:incremento_4 (end)
\end{document}
