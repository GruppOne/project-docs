\documentclass[../analisi-dei-requisiti]{subfiles}

\renewcommand{\commons}{../../../commons}

\begin{document}
La presente sezione ha lo scopo di descrivere in maniera dettagliata, attraverso il linguaggio \glossario{UML}, le funzionalitá offerte da \glossario{Stalker}.

\subsection{Attori dei casi d'uso}
\label{sub:attori_casi_duso}

\subsubsection{Utenti}
\label{subsub:utenti}

\begin{itemize}
  \item \emph{Utente non registrato}: Si riferisce ad un utente generico che non ha ancora effettuato la registrazione alla piattaforma.
  \item \emph{Utente non autenticato}: Si riferisce ad un utente generico che non ha ancora effettuato l'autenticazione alla piattaforma.
  \item \emph{Utente autenticato}: Si riferisce ad un utente non autenticato che ha effettuato l'autenticazione alla piattaforma.
  \item \emph{Utente non collegato}: Si riferisce ad un utente autenticato che ha non ha effettuato il collegamento ad un'organizzazione.
  \item \emph{Utente collegato}: Si riferisce ad un utente autenticato che ha effettuato il collegamento ad un'organizzazione.
  \item \emph{Utente collegato incognito}: Si riferisce ad un utente collegato che ha deciso di rimanere in modalità incognito all'interno di un'organizzazione.
  \item \emph{Utente collegato noto}: Si riferisce ad un utente collegato che ha deciso di essere noto all'interno di un'organizzazione.
\end{itemize}
% subsub:utenti (end)

\subsubsection{Super Utenti}
\label{subsub:super_utenti}

\begin{itemize}
  \item \emph{Super Utente non autenticato}: si riferisce ad un super utente che non ha ancora effettuato l'accesso.
  \item \emph{Amministratore}: Si riferisce ad un super utente con privilegi avanzati, che gestisce le richieste di \emph{Gestore} e \emph{Owner}.
  \item \emph{Root}: Si riferisce ad un amministratore con privilegi avanzati su tutto il sistema. Gestisce gli amministratori.
  \item \emph{Visualizzatore}: Si riferisce ad un super utente con privilegi di visualizzazione sugli utenti e organizzazione.
  \item \emph{Gestore}: Si riferisce ad un visualizzatore con privilegi di richieste di modifiche all'organizzazione.
  \item \emph{Owner}: Si riferisce ad un gestore con privilegi di gestione utenti e organizzazione. É il proprietario di un'organizzazione.
\end{itemize}
% subsub:super_utenti (end)

\subsection{Elenco casi d'uso Utente}
\label{sub:casi_duso_utente}
\subfile{components/casi-duso-utente.tex}

\subsection{Elenco casi d'uso Super Utente}
\label{sub:casi_duso_superutente}
\subfile{components/casi-duso-superutente.tex}

% sub:attori_casi_duso (end)

\end{document}
