\documentclass[../../../analisi-dei-requisiti.tex]{subfiles}

\begin{document}


\begin{figure}[H]
  \centering
  \begin{plantuml}
  @startuml
  !include ../../commons/style/use-cases.pu

  actor :utente non autenticato: as A

  rectangle {
    together {
      usecase (UUC3.1) as "UUC3.1\nRecupero password"
    }
  }

  :A: -- (UUC3.1)

  @enduml
  \end{plantuml}
  \caption{UUC3: Recupero credenziali}%
  \label{fig:uuc3}
\end{figure}

\begin{description}
  \item[Caso d’uso:] UUC3;
  \item[Titolo:] Recupero credenziali;
  \item[Attori primari:] utente non autenticato;
  \item[Precondizione:] l'utente si trova nella schermata iniziale ed è registrato al servizio;
  \item[Postcondizione:] l'utente si trova nella schermata di recupero credenziali;
  \item[Scenario principale:]
        \begin{enumerate}
          \item l'utente può selezionare la voce per il recupero credenziali e avviene tramite recupero password.
        \end{enumerate}
\end{description}

\subsubsection{UUC3.1: Recupero password}%
\label{subs:UUC3.1}

\begin{figure}[H]
  \centering
  \begin{plantuml}
  @startuml
  !include ../../commons/style/use-cases.pu

  actor :utente non autenticato: as A

  rectangle {
    together {
      usecase (UUC3.1.1) as "UUC3.1.1\nInserimento\nemail di registrazione\n--\nExtension points:\nVisualizzazione errore se l'email\nnon è registrata"
      usecase (UUC3.1.2) as "UUC3.1.2\nReimpostazione password\n--\nExtension points:\nVisualizzazione errore se:\n-nuova password non rispetta vincoli\n-nuova password e sua conferma\nnon coincidono"
      usecase (UUC3.1.3) as "UUC3.1.3\nInformazioni recupero non valide"
    }
  }

  :A: -- (UUC3.1.1)
  :A: -- (UUC3.1.2)

  (UUC3.1.3) .up.|> (UUC3.1.2) : <<extends>>
  (UUC3.1.3) .up.|> (UUC3.1.1) : <<extends>>

  @enduml
  \end{plantuml}
  \caption{UUC3.1: Recupero password}%
  \label{fig:uuc3_1}
\end{figure}

\begin{description}
  \item[Caso d’uso:] UUC3.1;
  \item[Titolo:] Recupero password;
  \item[Attori primari:] utente non autenticato;
  \item[Precondizione:] l'utente si trova nella schermata di recupero credenziali;
  \item[Postcondizione:] l'utente ha recuperato le credenziali e può nuovamente autenticarsi;
  \item[Scenario principale:]
        \begin{enumerate}
          \item tramite questa procedura, l'utente ha la possibilità di recuperare la password.
        \end{enumerate}
\end{description}

\subsubsection{UUC3.1.1: Inserimento email di registrazione}%
\label{subs:UUC3.1.1}
\begin{description}
  \item[Caso d’uso:] UUC3.1.1;
  \item[Titolo:] Inserimento email di registrazione;
  \item[Attori primari:] utente non autenticato;
  \item[Precondizione:] l'utente deve ancora iniziare la procedura di recupero password;
  \item[Postcondizione:] l'utente ha inserito correttamente la email di registrazione;
  \item[Scenario principale:]
        \begin{enumerate}
          \item l'utente inserisce l'email personale per l'autenticazione, e riceve una email con il \glossario{link} per reimpostare la password.
        \end{enumerate}
  \item[Estensioni:]
        \begin{enumerate}
          \item se l'email inserita non è registrata nel database, allora verrà segnalato un errore \emph{[UUC3.1.3]}.
        \end{enumerate}
\end{description}

\subsubsection{UUC3.1.2: Reimpostazione password}%
\label{subs:UUC3.1.2}
\begin{description}
  \item[Caso d’uso:] UUC3.1.2;
  \item[Titolo:] Reimpostazione password;
  \item[Attori primari:] utente non autenticato;
  \item[Precondizione:] l'utente ha ricevuto il link presente nella email personale per reimpostare la password;
  \item[Postcondizione:] l'utente ha inserito correttamente la nuova password e ha recuperato le proprie credenziali;
  \item[Scenario principale:]
        \begin{enumerate}
          \item l'utente reimposta la password in una procedura che avviene in due passaggi:
                \begin{enumerate}
                  \item reset della vecchia password, non visibile all'utente;
                  \item inserimento della nuova password e la conferma della nuova conferma.
                \end{enumerate}
        \end{enumerate}
  \item[Estensioni:]
        \begin{enumerate}
          \item se la nuova password non rispetta determinati vincoli, verrà visualizzato un'errore \emph{[UUC3.1.3]};
          \item se la nuova password e la sua conferma non coincidono tra loro, verrà visualizzato un'errore \emph{[UUC3.1.3]}.
        \end{enumerate}
\end{description}


\end{document}
