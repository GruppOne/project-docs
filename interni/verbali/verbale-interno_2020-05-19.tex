\documentclass{article}

\usepackage[italian]{babel}
\usepackage[margin=2cm, footskip=5mm]{geometry}
% questi package non sono necessari in lualatex; ref https://tex.stackexchange.com/a/413046
% \usepackage[utf8]{inputenc}
% \usepackage[T1]{fontenc}
\usepackage{enumitem}
\usepackage{hyperref}
\usepackage{titlesec}
\usepackage{soulutf8}
\usepackage{contour}
\usepackage{float}
\usepackage{graphicx}
\usepackage{fancyhdr}
\usepackage{longtable}
\usepackage[table]{xcolor}
\usepackage{titling}
\usepackage{lastpage}
\usepackage{ifthen}
\usepackage{calc}
\usepackage{minted}
\usepackage{pgfgantt}
\usepackage{subfiles}

\newlength{\imgwidth}

\newcommand\scalegraphics[1]{%
    \settowidth{\imgwidth}{\includegraphics{#1}}%
    \setlength{\imgwidth}{\minof{\imgwidth}{\textwidth}}%
    \includegraphics[width=\imgwidth]{#1}%
}

% XXX definizione dei percorsi in cui cercare immagini
\graphicspath{ {./}
    {./img/}
}

% esempio di utilizzo: \appendToGraphicspath{./img/} (un comando diverso per ogni path da includere)
% N.B.: ci DEVE essere un forward slash alla fine del path, a indicare che è una cartella.
\makeatletter
\newcommand\appendToGraphicspath[1]{%
  \g@addto@macro\Ginput@path{{#1}}%
}
\makeatother

% setup della sottolineatura
\setuldepth{Flat}
\contourlength{0.8pt}

\newcommand{\uline}[1]{%
  \ul{{\phantom{#1}}}%
  \llap{\contour{white}{#1}}%
}

% setup dei link
\hypersetup{
  colorlinks=true, % set true if you want colored links
  linktoc=all,     % set to all if you want both sections and subsections linked
  linkcolor=black, % choose some color if you want links to stand out
}

% setup di header e footer
\pagestyle{fancy}

\fancyhf{}
\fancyhead[L]{\includegraphics[width=1cm]{logo.png}}
\fancyhead[R]{\thetitle}
\fancyfoot[R]{\thepage\ di~\pageref{LastPage}}

\fancypagestyle{nopage}{%
  \fancyfoot{}%
}

\setlength{\headheight}{1.2cm}

% setup forma \paragraph e \subparagraph
\titleformat{\paragraph}[hang]{\normalfont\normalsize\bfseries}{\theparagraph}{1em}{}
\titleformat{\subparagraph}[hang]{\normalfont\normalsize\bfseries}{\thesubparagraph}{1em}{}

% setup profondità indice di default
\setcounter{secnumdepth}{5}
\setcounter{tocdepth}{5}

% shortcut per i placeholder
\newcommand{\plchold}[1]{\textit{\{#1\}}} % chktex 20

% hook per lo script che genera il glossario
\newcommand{\glossario}[1]{\underline{#1}\textsubscript{g}}

% definizione dei comandi \uso e \stato
\makeatletter
\newcommand{\setUso}[1]{%
  \newcommand{\@uso}{#1}%
}
\newcommand{\uso}{\@uso}

\newcommand{\setStato}[1]{%
  \newcommand{\@stato}{#1}%
}
\newcommand{\stato}{\@stato}

\newcommand{\setVersione}[1]{%
  \newcommand{\@versione}{#1}%
}
\newcommand{\versione}{\@versione}

\newcommand{\setResponsabile}[1]{%
  \newcommand{\@responsabile}{#1}%
}
\newcommand{\responsabile}{\@responsabile}

\newcommand{\setRedattori}[1]{%
  \newcommand{\@redattori}{#1}%
}
\newcommand{\redattori}{\@redattori}

\newcommand{\setVerificatori}[1]{%
  \newcommand{\@verificatori}{#1}%
}
\newcommand{\verificatori}{\@verificatori}

\newcommand{\setDescrizione}[1]{%
  \newcommand{\@descrizione}{#1}%
}
\newcommand{\descrizione}{\@descrizione}

\newcommand{\setModifiche}[1]{%
  \newcommand{\@modifiche}{#1}%
}

\newcommand{\modifiche}{\@modifiche}

\makeatother

% setup delle description
\setlist[description,1]{font=$\bullet$\hspace{1.5mm}, leftmargin=*,labelindent=12.5mm}
\setlist[description,2]{font=$\bullet$\hspace{1.5mm}, leftmargin=*,labelindent=12.5mm}

\appendToGraphicspath{../../commons/img/}

\title{Verbale interno --- 19/05/2020}

\setResponsabile{Fabio Scettro}
\setRedattori{Alberto Gobbo}
\setVerificatori{
  Alberto Cocco
}
\setUso{Interno}
\setDescrizione{Verbale dell'incontro di GruppOne del 19/05/2020}
\setModifiche{%
\cellcolor{white!80!lightgray!100} & Fabio Scettro & 2020--05--20 & approva documento \\% cambiare responsabile se errato
\cellcolor{white!80!lightgray!100} & Alberto Cocco & 2020--05--19 & verifica verbale \\%
\multirow{-3}{*}{-} & Alberto Gobbo & 2020--05--19 & stendi verbale%
}

\disabilitaVersione{}
\disabilitaElencoFigure{}
\disabilitaElencoTabelle{}

\begin{document}

\thispagestyle{empty}
\pagenumbering{gobble}
\begin{center}
	\includegraphics[width=8.5cm]{\commons/img/logo.png}\\
	{\Large GruppOne - progetto "Stalker"}\\
	\vspace{1.5cm}
	{\Huge \thetitle}
	\vspace{1.5cm}
	\begin{table}[H]
		\centering
		\begin{tabular}{r|l}
			\textbf{Versione}&\versione\\
			\textbf{Approvazione}&\responsabile\\
			\textbf{Redazione}&\redattori\\
			\textbf{Verifica}&\verificatori\\
			\textbf{Stato}&\stato\\
			\textbf{Uso}&\uso\\
			\textbf{Destinato a}&Imola Informatica\\
			&GruppOne\\
			&Prof. Tullio Vardanega\\
			&Prof. Riccardo Cardin\\
		\end{tabular}
	\end{table}
	\vspace{3cm}
	\textbf{Descrizione}\\
	\descrizione\\
	\vfill
	\verb|gruppone.swe@gmail.com|
\end{center}
\newpage
\thispagestyle{nopage}
\section*{Registro delle modifiche}
\label{sec:registro_delle_modifiche}
\begin{table}[H]
	\label{tab:registro_delle_modifiche}
	\centering
	\rowcolors{2}{lightgray}{white!80!lightgray!100}
	\begin{longtable}[c]{c c c c l}
		\rowcolor{darkgray!90!}\color{white}{\textbf{Versione}}&\color{white}{\textbf{Data}}&\color{white}{\textbf{Nominativo}}&\color{white}{\textbf{Ruolo}}&\color{white}{\textbf{Descrizione}}\\\endhead
	\end{longtable}
\end{table}
% section registro_delle_modifiche (end)
\newpage
\thispagestyle{nopage}
\pagenumbering{roman}
\tableofcontents
\newpage
\pagenumbering{arabic}

\section{Informazioni logistiche}%
\label{sec:informazioni_logistiche}

\begin{description}
  \item [Luogo] videochiamata su Google Hangouts
  \item [Data] 19/05/2020
  \item [Ora] 18:30 \symbol{8594} 19:30
\end{description}

\subsection{Membri del gruppo presenti}%
\label{sub:membri_del_gruppo_presenti}

\begin{enumerate}
  \item Riccardo Agatea
  \item Tobia Apolloni
  \item Riccardo Cestaro
  \item Alberto Cocco
  \item Luca Ercole
  \item Alberto Gobbo
  \item Alessandro Rizzo
  \item Fabio Scettro
\end{enumerate}
% sub:membri_del_gruppo_presenti (end)

\section{Introduzione}%
\label{sec:introduzione}
L'incontro è avvenuto tramite chiamata Hangouts che, a differenza delle ultime settimane, si è tenuto nell'inconsueta giornata di martedì, in quanto la maggior parte dei componenti era indisponibile il giorno prima, in preparazione per il secondo appello di Ingegneria del Software.
Lo scopo principale era pianificare in breve il lavoro di questo incremento.

\section{Ordine del giorno}%
\label{sec:ordine_del_giorno}

\begin{itemize}
  \item Punto della situazione
  \item Pianificazione lavoro settimanale
\end{itemize}

\section{Punto della situazione}%
\label{sec:punto_della_situazione}

Il gruppo ha discusso sullo stato di completamento di tutte le componenti che sono in fase di sviluppo, soprattutto per confermare le attese della settimana precedente.
È emerso che
\begin{itemize}
  \item per la web application mancano solo alcuni dettagli tecnici che possono essere completati nel giro di un incremento.
  \item per il server vanno completati gli endpoint rimanenti e, in specifici casi, migliorati qualitativamente.
  \item per la mobile application manca l'impianto grafico di alcune pagine non ancora implementate.
\end{itemize}

\section{Pianificazione lavoro settimanale}%
\label{sec:pianificazione_lavoro_settimanale}

A valle del punto della situazione, il gruppo ha deciso di concentrare le proprie energie su due delle tre componenti, e di redistribuire il lavoro come segue:
\begin{itemize}
  \item lo sviluppo della web application rimarrà bloccato fino al prossimo incremento, oppure fino a quando uno o più membri del gruppo non conclude il lavoro assegnatoli prima della fine di questo incremento.
  \item la maggior parte delle risorse umane sarà impiegata nella componente server, per concludere definitivamente l'implementazione degli endpoint mancanti. In questa componente verranno impiegate 6 componenti su 8 componenti, a loro volta suddivise in gruppi da 2 persone.
  \item i restanti 2 membri si focalizzeranno sullo sviluppo della mobile application, in particolare per completare la parte dei test mancanti ed implementare l'impianto grafico per alcune funzionalità.
\end{itemize}

\newpage
\section{Registro delle decisioni}%
\label{sec:registro_delle_decisioni}

\begin{description}
  \item[20200519-int-001] Lo sviluppo della web application rimane momentaneamente sospeso.
  \item[20200519-int-002] 6 componenti lavoreranno nel lato server per l'implementazione definitiva di tutti gli endpoint.
  \item[20200519-int-003] 2 tra le 6 componenti lato server si occuperanno di implementare l'autenticazione LDAP per l'accesso di utenti ad organizzazioni private.
  \item[20200519-int-004] 2 tra le 6 componenti lato server si occuperanno di raffinare ed ultimare l'implementazione dell'autenticazione e dell'autorizzazione per l'accesso a Stalker.
  \item[20200519-int-005] 2 componenti lavoreranno nella mobile application, sia per i test rimanenti che per la parte grafica mancante.
\end{description}

% sec:registro_delle_decisioni (end)
\end{document}
