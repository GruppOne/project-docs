\documentclass[../piano-di-qualifica.tex]{subfiles}
\appendToGraphicspath{../../../commons/img/}

% TODO usare https://en.wikipedia.org/wiki/ISO/IEC_9126?
% TODO usare https://www.praxiom.com/iso-90003.htm?
\begin{document}


\subsection{Specifica dei test}%
\label{sub:test}
    Per misurare la qualità di prodotto inoltre utilizziamo dei test secondo il \glossario{Modello a \textit{V}} per il quale definiamo e sviluppiamo i test in parallelo alle attività di analisi, progettazione architetturale e di sviluppo e verifica incrementi.
    Abbiamo definito i test dividendoli nelle seguenti categorie:
    \begin{description}
      \item [Test di Accettazione (TA):] vengono fatti per verificare che il prodotto soddisfi i requisiti richiesti dal proponente
      \item [Test di Sistema (TS):] vengono effettuati quando si mettono insieme tutte le componenti del software e se ne vuole testare compatibilità e interazioni
      \item [Test di Integrazione (TI):] questi test verificano la compatibilità e le interazioni tra diverse unità del software testate con successo solo singolarmente fino a quel momento tramite Test di Unità
      \item [Test di Unità (TU):] svolgono un'attività mirata di analisi sulle singole unità software, vengono eseguiti con il massimo grado possibile di parallelismo e garantiscono solo il funzionamento della singola unità a cui appartengono.
    \end{description}
    Inoltre classifichiamo i test in base al loro stato:
      \begin{description}
        \item [S]: il test è stato soddisfatto
        \item [NS]: il test non è stato ancora soddisfatto
      \end{description}
  \subsubsection{Test di Accettazione}
  \label{subs:accettazione}
      I test di accettazione hanno lo scopo di verificare che il prodotto soddisfi i requisiti richiesti dal proponente, vengono eseguiti durante la fase di verifica e collaudo finale.
      Essendo tutti i test derivanti da un gruppo di requisiti che hanno una determinata importanza nel prodotto software abbiamo pensato di numerare i Test di Accettazione nel modo seguente:
      \begin{center}
          [TA][requisito][importanza][codice]
      \end{center}
      Dove: \textit{importanza} è un valore numerico ad indicare l'importanza del requisito che deve essere soddisfatto, \textit{codice} è un codice numerico identificativo per il test e \textit{requisito} può assumere i seguenti valori:
      \begin{description}
        \item [F]: requisito funzionale 
        \item [P]: requisito prestazionale
        \item [D]: requisito dichiarativo
      \end{description}
      \begin{centering}
      \rowcolors{2}{lightgray}{white!80!lightgray!100}
      \renewcommand{\arraystretch}{2} % allarga le righe con dello spazio sotto e sopra
      \begin{longtable}[H]{>{\centering\bfseries}p{3cm} >{}p{10cm} >{\centering\arraybackslash}m{3cm}}
        \rowcolor{darkgray!90!}
        \color{white}
        {\textbf{ID test}} & \color{white}{\textbf{Descrizione}} & \color{white}{\textbf{Esito}} \\
        \endhead
        \rowcolor{white}
        \multicolumn{3}{r}{\textit{Continua alla pagina seguente}}
        \endfoot
        \endlastfoot
        TAAFO001      & Al nuovo utente deve essere permesso di registrarsi. \newline 
                        L’utente deve: 
                        \begin{itemize} 
                          \item visualizzare e confermare l’\glossario{EULA};
                          \item inserire password;
                          \item confermare la password;
                          \item inserire i propri dati anagrafici;
                          \item inserire la proria email;
                          \item confermare la registrazione.
                        \end{itemize}
                      & NS \\   
        TAAFO002      & Il sistema deve rifiutare la richiesta di registrazione se i dati inseriti non rispettano i vincoli imposti o se l’email è già presente nel database. \newline 
                        L’utente deve:  
                        \begin{itemize} 
                          \item inserire dati di accesso non validi;
                          \item verificare l'impossibilità di proseguire con la registrazione.
                        \end{itemize}
                      & NS \\ 
        TAAFO003      & Il sistema deve permettere all’utente registrato di autenticarsi. \newline 
                        L’utente deve: 
                        \begin{itemize} 
                          \item inserire la propria email;
                          \item inserire la propria password;
                          \item confermare l'autenticazione.
                        \end{itemize}
                      & NS \\   
        TAAFO004      & Il sistema deve rifiutare la richiesta di autenticazione se i dati inseriti non rispettano i vincoli imposti, o se la combinazione di email e password non è presente nel database. \newline 
                        L’utente deve:  
                        \begin{itemize} 
                          \item inserire dati di accesso non validi;
                          \item verificare l'impossibilità di proseguire con l'autenticazione.
                        \end{itemize}
                      & NS \\ 
        TAAFO005      & Il sistema deve permettere all’utente di recuperare la sua password. \newline 
                        L’utente deve: 
                        \begin{itemize} 
                          \item avviare la procedura di recupero password;
                          \item inserire la propria email;
                          \item confermare la procedura;
                          \item verificare che la password venga ricevuta correttamente;
                        \end{itemize}
                      & NS \\   
        TAAFO006      & Il sistema deve permettere all’utente di reimpostare la sua password. \newline 
                        L’utente deve:  
                        \begin{itemize} 
                          \item avviare la procedura di reimpostazione password;
                          \item inserire la propria email;
                          \item inserire la vecchia password;
                          \item inserire una nuova password;
                          \item confermare la nuova password;
                          \item confermare la procedura;
                          \item verificare che la password sia stata cambiata correttamente;
                        \end{itemize}
                      & NS \\ 
          TAAFO007    & Il sistema deve rifiutare la richiesta di reimpostazione password se la password scelta non rispetta i vincoli imposti. \newline 
                        L’utente deve: 
                        \begin{itemize} 
                          \item avviare la procedura di reimpostazione password;
                          \item inserire la propria email;
                          \item inserire la vecchia password;
                          \item inserire una nuova password non valida;
                          \item confermare la nuova password;
                          \item confermare la procedura;
                          \item visualizzare un messaggio di errore;
                        \end{itemize}
                      & NS \\   
        TAAFO008      & Il sistema deve permettere all’utente autenticato di recuperare dal server una lista di organizzazioni a cui può collegarsi e di aggiornarla. \newline 
                        L’utente deve:  
                        \begin{itemize} 
                          \item autenticarsi;
                          \item visualizzare la lista di organizzazioni;
                          \item aggiornare la lista;
                          \item visualizzare la lista aggiornata.
                        \end{itemize}
                      & NS \\ 
        TAAFO009      & Il sistema deve permettere all’utente autenticato di selezionare una o più organizzazioni dalla lista e collegarsi alle organizzazioni selezionate. \newline 
                        L’utente deve: 
                        \begin{itemize} 
                          \item selezionare una o più organizzazioni dalla lista;
                          \item avviare la procedura di collegamento.
                        \end{itemize}
                      & NS \\   
        TAAFO010      & Il sistema deve permettere all’utente autenticato e collegato ad una o più organizzazioni di scollegarsi da una o più organizzazioni. \newline 
                        L’utente deve:  
                        \begin{itemize} 
                          \item selezionare una o più organizzazioni a cui è collegato;
                          \item avviare la procedura di scollegamento.
                        \end{itemize}
                      & NS \\ 
        TAAFO011      & Il sistema deve permettere all’utente autenticato di visualizzare tramite un filtro la lista delle organizzazioni a cui è collegato oppure quella delle organizzazioni da cui è scollegato. \newline 
                      L’utente deve:  
                      \begin{itemize} 
                        \item impostare il filtro;
                        \item visualizzare la lista richiesta.
                      \end{itemize}
                      & NS \\ 
        TAAFO012      & Il sistema deve permettere all’utente autenticato il passaggio da noto ad incognito e viceversa. \newline 
                      L’utente deve:  
                      \begin{itemize} 
                        \item avviare la procedura di cambio di stato;
                        \item verificare l'effettivo cambio di stato.
                      \end{itemize}
                      & NS \\ 
        TAAFO013      & Il sistema deve permettere all’utente autenticato di visualizzare il proprio storico. \newline 
                      L’utente deve:  
                      \begin{itemize} 
                        \item visualizzare il proprio storico degli accessi;
                        \item visualizzare il proprio tempo trascorso all'interno di ogni organizzazione.
                      \end{itemize}
                      & NS \\ 
        TAAFO014      & Il sistema deve permettere all’utente autenticato di disconnettersi dall’applicazione. \newline 
                      L’utente deve:  
                      \begin{itemize} 
                        \item avviare la procedura di disconnessione;
                        \item confermare la procedura.
                      \end{itemize}
                      & NS \\ 
        TAAFO015      & Il sistema deve permettere all’utente di eliminare il suo account. \newline 
                      L’utente deve:  
                      \begin{itemize} 
                        \item avviare la procedura di eliminazione;
                        \item confermare la procedura.
                      \end{itemize}
                      & NS \\ 
        TAAFO016      & Il sistema deve mostrare un messaggio d’errore e rifiutare le richieste da parte dell’app nel caso mancasse una connessione a internet. \newline 
                      L’utente deve:  
                      \begin{itemize} 
                        \item disconnettere il dispositivo da internet;
                        \item effettuare un'azione che richieda la connessione ad internet;
                        \item visualizzare il messaggio di errore corrispondente.
                      \end{itemize}
                      & NS \\ 
        TAAFO017      & L’interfaccia web deve permettere all’utente non autenticato di autenticarsi. \newline 
                      L’utente deve:  
                      \begin{itemize} 
                        \item compilare il campo email;
                        \item compilare il campo password;
                        \item confermare l'autenticazione.
                      \end{itemize}
                      & NS \\ 
        TAAFO018      & L’interfaccia web deve rifiutare la richiesta di autenticazione se i dati inseriti non rispettano i vincoli imposti, o se la combinazione di email e password non è presente nel database. \newline 
                      L’utente deve:  
                      \begin{itemize} 
                        \item inserire una combinazione di dati non validi;
                        \item verificare la possibilità di proseguire l'autenticazione.
                      \end{itemize}
                      & NS \\ 
      \end{longtable}
      %subs:accettazione (end)
    \end{centering}
    \subsubsection{Test di Sistema}
  \label{subs:sistema}
    I test di sistema verificano il corretto funzionamento tra tutte le componenti del sistema. I test verranno identificati nel seguente modo:
    \begin{center}
      TS[codice]
    \end{center}
    Dove \textit{codice} rappresenta un numero identificativo per il Test di Sistema.
  %subs:sistema (end)
  \subsubsection{Test di Integrazione}
  \label{subs:integrazione}
    I test di sistema verificano la corretta interazione e collaborazione tra un isieme di unità. I test verranno identificati nel seguente modo:
    \begin{center}
      TI[codice]
    \end{center}
    Dove \textit{codice} rappresenta un numero identificativo per il Test di Integrazione.
  %subs:integrazione (end)
  \subsubsection{Test di Unità}
  \label{subs:unita}
    I test di sistema verificano la correttezza e il funzionamento di una singola unità del software. I test verranno identificati nel seguente modo:
    \begin{center}
      TU[codice]
    \end{center}
    Dove \textit{codice} rappresenta un numero identificativo per l'unità a cui il test appartiene.
  %subs:unita (end)
% sub:test (end)
\end{document}
