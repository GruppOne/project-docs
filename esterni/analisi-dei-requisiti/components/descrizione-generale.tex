\documentclass[../analisi-dei-requisiti.tex]{subfiles}

\begin{document}
\subsection{Obiettivo del prodotto}%
\label{sub:obiettivo_del_prodotto}
Il prodotto richiesto da capitolato si pone l'obiettivo di tracciare il numero di persone presenti all’interno di una data organizzazione.\\
Per soddisfare la richiesta, sarà implementata una \glossario{web application} per la gestione delle organizzazioni e il monitoraggio degli utenti, e
di un'\glossario{applicazione mobile} per l'utente finale che gli permetta di essere monitorato dalla specifica organizzazione scelta.


\subsection{Funzionalità del prodotto}%
\label{sub:funzionalita_del_prodotto}
Il sistema di Stalker deve tracciare tutti gli utenti che sono all'interno di specifici luoghi definiti dalle organizzazioni registrate in Stalker.
Affinché questo venga garantito, deve essere presente un server che offra la possibilità di:
\begin{itemize}
  \item creare e gestire più organizzazioni, privilegio concesso ad un sottoinsieme degli utenti utilizzatori della web application;
  \item definire se prevedere una \glossario{tracciatura} nota oppure incognita, al momento della creazione dell'organizzazione;
  \item solo in presenza di autorizzazione specifica, effettuare query di monitoraggio per singolo utente all’interno delle organizzazioni.
\end{itemize}

Per gli utenti che vogliono usufruire del servizio messo a disposizione dal sistema di Stalker, gli utenti devono installare l'applicazione mobile
e, una volta effettuata la registrazione, hanno la possibilità di:
\begin{itemize}
  \item recuperare la lista delle organizzazioni registrate in Stalker;
  \item collegarsi ad un'organizzazione;
  \item utilizzare un pulsante “anonimo” che permetta, all’interno di una organizzazione privata, di risultare presente in maniera anonima;
  \item visualizzare il proprio storico degli accessi;
  \item visualizzare in tempo reale la propria presenza o meno all’interno di un luogo monitorato e il cronometro del tempo trascorso al suo interno.
\end{itemize}
Al momento della registrazione, l'utente deve accettare le condizioni del sistema di Stalker, che prevedono il consenso di tracciare
ogni singolo utente registrato al momento del collegamento ad un'organizzazione.
Le comunicazioni tra applicazione mobile e server avvengono solo al momento d'ingresso ed uscita dai luoghi designati dalle organizzazioni, al fine di garantire
la privacy dell'utente.



\subsection{Caratteristiche degli utenti}%
\label{sub:caratteristiche_degli_utenti}
Nell'ambito di questo progetto, sono presenti due tipologie generiche di utenti, con caratteristiche diverse dovute all'utilizzo del prodotto:
\begin{description}
  \item[Super utente:] è l'utente che detiene dei privilegi avanzati per l'uso della web application, diversi per ogni sottocategoria di super utente;
  \item[Utente generico:] è l'utente utilizzatore dell'applicazione mobile. Può essere un dipendente di un'azienda che monitora la sua presenza, oppure
        un visitatore di un evento pubblico.
\end{description}
Le due tipologie di utenti sono descritte nel dettaglio al capitolo 3, sezione~\ref{sub:attori_casi_duso}.


\subsection{Macroarchitetture del progetto}%
\label{sub:macroarchitetture_del_progetto}
\subparagraph*{Back-end}%
\label{par:back-end}
E' richiesto lo sviluppo \glossario{back-end} del server, dove è richiesto l'utilizzo di protocolli asincroni per le comunicazioni tra applicazione mobile e server.\\
Sarà necessario lo sviluppo di un database che contenga tutti i dati d'interesse per il prodotto, dai dati appartenenti ai vari utenti fino ai dati che appartengono ai
luoghi delle organizzazioni.

\subparagraph*{Front-end}%
\label{par:front-end}
Lo sviluppo \glossario{front-end} sarà costituito:
\begin{itemize}
  \item dalla web application, raggiungibile da qualunque utente (anche generico), per eseguire determinate operazioni in base al tipo di privilegi;
  \item dall'applicazione mobile, scaricabile da ogni utente interessato all'utilizzo del prodotto, che consente di visualizzare le informazioni d'interesse ed eseguire determinate
        operazioni in presenza di rete.
\end{itemize}


\end{document}
