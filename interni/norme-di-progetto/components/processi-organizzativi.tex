\documentclass[../norme-di-progetto.tex]{subfiles}

\begin{document}

\subsection{Descrizione}%
\label{sub:processi_organizzativi/descrizione}

I processi organizzativi stabiliscono le attività interne che il team deve svolgere per garantire l'\glossario{economicità} nel corso dello sviluppo del software.
Rivestono un ruolo fondamentale in quanto permettono di gestire la suddivisione dei ruoli e coordinare i membri del gruppo.
Essi sono trasversali rispetto ai singoli progetti.

Da notare che secondo lo standard ISO/IEC 12207:1995 la gestione dei rischi è un'attività.
Per rispecchiare l'importanza che ricopre nel nostro progetto, facciamo riferimento allo standard ISO/IEC 12207:2008 che la categorizza come processo.

Inoltre, il processo di gestione delle infrastrutture non è rilevante alla nostra situazione, quindi non vi è necessità di normarlo.

I processi organizzativi istanziati da GruppOne sono i seguenti:

\begin{itemize}
  \item Gestione dei rischi
  \item Gestione dei processi
  \item Formazione del personale.
\end{itemize}

Trattandosi di un singolo progetto unito al fatto che, per sua natura, il miglioramento del processo viene effettuato all'interno di un'organizzazione nel corso di molteplici progetti, il gruppo ha deciso di non istanziare il processo organizzativo di miglioramento.


\subsection{Gestione dei rischi}%
\label{sub:gestione_dei_rischi}

\subfile{processi-organizzativi/gestione-dei-rischi}

% questo è il processo "management" nelle slide del professore
\subsection{Gestione di processo}%
\label{sub:gestione_di_processo}

\subfile{processi-organizzativi/gestione-di-processo}

\subsection{Formazione del personale}%
\label{sub:formazione_del_personale}

\subfile{processi-organizzativi/formazione-del-personale}

\end{document}
