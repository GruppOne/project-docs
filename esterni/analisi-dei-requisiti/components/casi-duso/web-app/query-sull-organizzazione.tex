\documentclass[../../../analisi-dei-requisiti.tex]{subfiles}

\begin{document}

\begin{figure}[H]
  \centering
  \begin{plantuml}
  @startuml
  !include ../../commons/style/use-cases.pu

  actor :visualizzatore: as A

  rectangle {
    together {
      usecase (AUC5) as "AUC5\nQuery sull'organizzazione"
    }
  }

  :A: -- AUC5

  @enduml
  \end{plantuml}
  \caption{AUC5: Query sull'organizzazione}%
  \label{fig:AUC5}
\end{figure}

\begin{description}
  \item[Codice:] AUC5;
  \item[Titolo:] Query sull'organizzazione;
  \item[Attori primari:] visualizzatore;
  \item[Precondizione:] il sistema risponde correttamente alle interrogazioni;
  \item[Postcondizione:] il visualizzatore ottiene le informazioni di cui ha bisogno;
  \item[Scenario principale:]
  \begin{enumerate}
    \item il visualizzatore vuole ottenere delle informazioni riguardo l'organizzazione sulla quale opera.
  \end{enumerate}
\end{description}

\end{document}
