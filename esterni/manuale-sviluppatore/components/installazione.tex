\documentclass[../manuale-sviluppatore.tex]{subfiles}

\begin{document}

\subsection{Stalker server}

All'interno della repository è presente una coppia di script (gradlew per linux e gradlew.bat per windows) che permette di eseguire i task necessari 
senza bisogno di installare \glossarioLocale{gradle}.\\
Il comando da utilizzare per eseguire il server è:\\

\emph{./gradlew bootRun}\\

Spring utilizza una funzione chiamata LiveReload: consente una compilazione automatica ad ogni modifica del codice sorgente in modo da velocizzare il lavoro.\\
Il server funziona anche senza bisogno del database attivo sotto, ma chiaramente se si vuole controllare che le cose vadano correttamente è necessario avviarlo.
Il comando per avviarlo (da eseguire posizionandosi sulla root nel server) è:\\

\emph{docker-compose up -d rdb}\\

Per i dettagli bisogna aprire i file \emph{docker-compose.yml} e \emph{docker-compose.override.yml}.\\
RDB (Relational DataBase) è il nome del servizio definito all'interno di quei file. \\
E' anche possibile far partire più di un servizio alla volta con:\\

\emph{docker-compose up -d rdb rdb-gui}\\

(questo comando fa partire anche un istanza di phpmyadmin con cui è possibile ispezionare il database)\\

oppure\\

\emph{docker-compose up -d}\\

Il comando soprastante fa partire tutti i servizi disponibili. -d, che sta per --detached serve a ridare il prompt dopo che docker ha fatto partire il container.\\ 
Si può provare anche ad eseguire gli stessi comandi senza il -d per vedere la differenza.\\

Il teardown completo dei database si può fare con:\\

\emph{docker-compose down --rmi all --remove-orphans --volumes}\\

Il consiglio è di eseguirlo spesso in modo da evitare che il funzionamento del server dipenda da cose che si hanno fatto su una specifica istanza del database.

\subsubsection{Docker Toolbox con Windows Home}

Se si sta utilizzando Windows 10 Pro o Education è conveniente installare Docker Desktop for Windows.

Se invece si utilizza Windows 10 Home è obbligatorio usare Docker Toolbox, che lavora dentro una \glossarioLocale{VM}, 
in questo modo i container non vengono eseguiti su localhost ma sull'IP della VM.% 
La cosa si può risolvere configurando il port forwarding delle porte interessate da VirtualBox:

\begin{itemize}
\item selezionare la VM su cui sta girando docker;
\item andare sui setting della VM;%
\item selezionare la tab network;
\item cliccare su port forwarding (è dentro il menu advanced), e aprire le porte.
\end{itemize}

\subsection{Stalker web app}

\subsubsection{Installazione delle dipendenze}

Prima di effettuare qualsiasi operazione si avvii il comando \emph{npm install}, in questo modo saranno installate tutte le dipendenze necessarie.\\

\subsubsection{Avvio del server}

Utilizzare il comando \emph{ng serve} per ottenere un server di sviluppo e navigare all'indirizzo http://localhost:4200/. \\
La web app si aggiornerà automaticamente dopo qualsiasi cambiamento ai file sorgente.

\subsubsection{Creazione di nuovi componenti}

Utilizzare \emph{ng generate component component-name} per generare un nuovo componente.\\
Si può anche usare \emph{ng generate directive|pipe|service|class|guard|interface|enum|module}.

\subsubsection{Build}

Scrivere \emph{ng build} per effettuare la build del progetto. \\
L'artefatto di build si troverà nella cartella \emph{dist/}. Usare la flag \emph{--prod} per una build di produzione. 

\subsubsection{Eseguire i test di unità}

Scrivere \emph{ng test} per eseguire i test di unità tramite Karma.

\subsubsection{Eseguire i test end-to-end}

Scrivere \emph{ng e2e} per eseguire i test end-to-end tramite Protractor.

\subsubsection{Configurazione}

Per eseguire comandi con una configurazione specifica scrivere \emph{ng build|serve|test --configuration=your-configuration} con \emph{your-configuration} la configurazione desiderata. \\
Le configurazioni possibili sono \emph{localhost|imola|production}, per eseguire la configurazione di default è sufficiente non aggiungere nulla ai comandi originali.


\subsection{Stalker Android app}

Per configurare l'ambiente di lavoro dell'Android app di Stalker, è necessario utilizzare un IDE come IntelliJ IDEA o \glossarioLocale{Android Studio}.\\
Per lo sviluppo dell'applicazione il team ha deciso di utilizzare IntelliJ IDEA, ma è possibile utilizzare senza problemi anche Android Studio.\\
Per il \glossarioLocale{debug} dell'applicazione basterà aprire il codice sorgente di essa in uno dei due IDE sopra citati, e avviare l'esecuzione tramite \glossarioLocale{ADB}. \\


\end{document}
