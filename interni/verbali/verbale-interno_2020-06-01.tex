\documentclass{article}

\usepackage[italian]{babel}
\usepackage[margin=2cm, footskip=5mm]{geometry}
% questi package non sono necessari in lualatex; ref https://tex.stackexchange.com/a/413046
% \usepackage[utf8]{inputenc}
% \usepackage[T1]{fontenc}
\usepackage{enumitem}
\usepackage{hyperref}
\usepackage{titlesec}
\usepackage{soulutf8}
\usepackage{contour}
\usepackage{float}
\usepackage{graphicx}
\usepackage{fancyhdr}
\usepackage{longtable}
\usepackage[table]{xcolor}
\usepackage{titling}
\usepackage{lastpage}
\usepackage{ifthen}
\usepackage{calc}
\usepackage{minted}
\usepackage{pgfgantt}
\usepackage{subfiles}

\newlength{\imgwidth}

\newcommand\scalegraphics[1]{%
    \settowidth{\imgwidth}{\includegraphics{#1}}%
    \setlength{\imgwidth}{\minof{\imgwidth}{\textwidth}}%
    \includegraphics[width=\imgwidth]{#1}%
}

% XXX definizione dei percorsi in cui cercare immagini
\graphicspath{ {./}
    {./img/}
}

% esempio di utilizzo: \appendToGraphicspath{./img/} (un comando diverso per ogni path da includere)
% N.B.: ci DEVE essere un forward slash alla fine del path, a indicare che è una cartella.
\makeatletter
\newcommand\appendToGraphicspath[1]{%
  \g@addto@macro\Ginput@path{{#1}}%
}
\makeatother

% setup della sottolineatura
\setuldepth{Flat}
\contourlength{0.8pt}

\newcommand{\uline}[1]{%
  \ul{{\phantom{#1}}}%
  \llap{\contour{white}{#1}}%
}

% setup dei link
\hypersetup{
  colorlinks=true, % set true if you want colored links
  linktoc=all,     % set to all if you want both sections and subsections linked
  linkcolor=black, % choose some color if you want links to stand out
}

% setup di header e footer
\pagestyle{fancy}

\fancyhf{}
\fancyhead[L]{\includegraphics[width=1cm]{logo.png}}
\fancyhead[R]{\thetitle}
\fancyfoot[R]{\thepage\ di~\pageref{LastPage}}

\fancypagestyle{nopage}{%
  \fancyfoot{}%
}

\setlength{\headheight}{1.2cm}

% setup forma \paragraph e \subparagraph
\titleformat{\paragraph}[hang]{\normalfont\normalsize\bfseries}{\theparagraph}{1em}{}
\titleformat{\subparagraph}[hang]{\normalfont\normalsize\bfseries}{\thesubparagraph}{1em}{}

% setup profondità indice di default
\setcounter{secnumdepth}{5}
\setcounter{tocdepth}{5}

% shortcut per i placeholder
\newcommand{\plchold}[1]{\textit{\{#1\}}} % chktex 20

% hook per lo script che genera il glossario
\newcommand{\glossario}[1]{\underline{#1}\textsubscript{g}}

% definizione dei comandi \uso e \stato
\makeatletter
\newcommand{\setUso}[1]{%
  \newcommand{\@uso}{#1}%
}
\newcommand{\uso}{\@uso}

\newcommand{\setStato}[1]{%
  \newcommand{\@stato}{#1}%
}
\newcommand{\stato}{\@stato}

\newcommand{\setVersione}[1]{%
  \newcommand{\@versione}{#1}%
}
\newcommand{\versione}{\@versione}

\newcommand{\setResponsabile}[1]{%
  \newcommand{\@responsabile}{#1}%
}
\newcommand{\responsabile}{\@responsabile}

\newcommand{\setRedattori}[1]{%
  \newcommand{\@redattori}{#1}%
}
\newcommand{\redattori}{\@redattori}

\newcommand{\setVerificatori}[1]{%
  \newcommand{\@verificatori}{#1}%
}
\newcommand{\verificatori}{\@verificatori}

\newcommand{\setDescrizione}[1]{%
  \newcommand{\@descrizione}{#1}%
}
\newcommand{\descrizione}{\@descrizione}

\newcommand{\setModifiche}[1]{%
  \newcommand{\@modifiche}{#1}%
}

\newcommand{\modifiche}{\@modifiche}

\makeatother

% setup delle description
\setlist[description,1]{font=$\bullet$\hspace{1.5mm}, leftmargin=*,labelindent=12.5mm}
\setlist[description,2]{font=$\bullet$\hspace{1.5mm}, leftmargin=*,labelindent=12.5mm}

\appendToGraphicspath{../../commons/img/}

\title{Verbale interno --- 01/06/2020}

\setResponsabile{Fabio Scettro}
\setRedattori{Alberto Gobbo}
\setVerificatori{
  Alberto Cocco
}
\setUso{Interno}
\setDescrizione{Verbale dell'incontro di GruppOne del 01/06/2020}
\setModifiche{%
\cellcolor{white!80!lightgray!100} & Fabio Scettro & 2020--06--02 & approva documento \\%
\cellcolor{white!80!lightgray!100} & Alberto Cocco & 2020--06--01 & verifica verbale \\%
\multirow{-3}{*}{-} & Alberto Gobbo & 2020--06--01 & stendi verbale%
}

\disabilitaVersione{}
\disabilitaElencoFigure{}
\disabilitaElencoTabelle{}

\begin{document}

\thispagestyle{empty}
\pagenumbering{gobble}
\begin{center}
	\includegraphics[width=8.5cm]{\commons/img/logo.png}\\
	{\Large GruppOne - progetto "Stalker"}\\
	\vspace{1.5cm}
	{\Huge \thetitle}
	\vspace{1.5cm}
	\begin{table}[H]
		\centering
		\begin{tabular}{r|l}
			\textbf{Versione}&\versione\\
			\textbf{Approvazione}&\responsabile\\
			\textbf{Redazione}&\redattori\\
			\textbf{Verifica}&\verificatori\\
			\textbf{Stato}&\stato\\
			\textbf{Uso}&\uso\\
			\textbf{Destinato a}&Imola Informatica\\
			&GruppOne\\
			&Prof. Tullio Vardanega\\
			&Prof. Riccardo Cardin\\
		\end{tabular}
	\end{table}
	\vspace{3cm}
	\textbf{Descrizione}\\
	\descrizione\\
	\vfill
	\verb|gruppone.swe@gmail.com|
\end{center}
\newpage
\thispagestyle{nopage}
\section*{Registro delle modifiche}
\label{sec:registro_delle_modifiche}
\begin{table}[H]
	\label{tab:registro_delle_modifiche}
	\centering
	\rowcolors{2}{lightgray}{white!80!lightgray!100}
	\begin{longtable}[c]{c c c c l}
		\rowcolor{darkgray!90!}\color{white}{\textbf{Versione}}&\color{white}{\textbf{Data}}&\color{white}{\textbf{Nominativo}}&\color{white}{\textbf{Ruolo}}&\color{white}{\textbf{Descrizione}}\\\endhead
	\end{longtable}
\end{table}
% section registro_delle_modifiche (end)
\newpage
\thispagestyle{nopage}
\pagenumbering{roman}
\tableofcontents
\newpage
\pagenumbering{arabic}

\section{Informazioni logistiche}%
\label{sec:informazioni_logistiche}

\begin{description}
  \item [Luogo] videochiamata su Google Hangouts
  \item [Data] 01/06/2020
  \item [Ora] 18:30 \symbol{8594} 19:30
\end{description}

\subsection{Membri del gruppo presenti}%
\label{sub:membri_del_gruppo_presenti}

\begin{enumerate}
  \item Riccardo Agatea
  \item Tobia Apolloni
  \item Riccardo Cestaro
  \item Alberto Cocco
  \item Luca Ercole
  % \item Alberto Gobbo
  \item Alessandro Rizzo
  \item Fabio Scettro
\end{enumerate}
% sub:membri_del_gruppo_presenti (end)

\section{Introduzione}%
\label{sec:introduzione}
L'incontro è avvenuto tramite chiamata Hangouts.
Vista la sempre più imminente Revisione di Accettazione (RA), lo scopo principale era fissare scadenze temporali significative fino alla data di fine progetto.

\section{Ordine del giorno}%
\label{sec:ordine_del_giorno}

\begin{itemize}
  \item Punto della situazione
  \item Pianificazione lavoro settimanale
  \item Pianificazione scadenze temporali fino alla fine del progetto
\end{itemize}

\section{Punto della situazione}%
\label{sec:punto_della_situazione}
Il lavoro che era stato pianificato nell'incremento appena concluso è stato rispetto, eccetto qualche problema tecnico legato ad alcuni endpoint delle organizzazioni che hanno avuto un leggero ritardo. Nonostante ciò, questi problemi verranno risolti nei primi giorni dell'incremento.

\section{Pianificazione lavoro settimanale}%
\label{sec:pianificazione_lavoro_settimanale}
A valle dei risultati ottenuti nell'incremento appena passato, ne consegue che il gruppo pianificherà il lavoro settimanale in questo modo:
\begin{itemize}
  \item il lavoro nella web application impiegherà 2 persone, per concludere le pagine sui report.
  \item nel server verranno impiegati 4 membri del gruppo per concludere i rimanenti endpoint.
  \item il lavoro della mobile application impiegherà 2 membri del gruppo, per implementare le pagine di report per l'utente.
\end{itemize}

\section{Pianificazione scadenze temporali fino alla fine del progetto}%
\label{sec:pianificazione_scadenze_temporali_fino_alla_fine_del_progetto}
Oltre al classico appuntamento di recap di fine incremento (che si svolge come di consueto ogni lunedì), il gruppo ha fissato delle call brevi di recap di metà incremento fino alla data di fine progetto, in modo da conoscere lo stato di avanzamento corrente, fissando più obiettivi in tempi brevi massimizzando la percentuale degli obiettivi soddisfatti. È stato deciso di non verbalizzare questi incontri di metà incremento, in quanto non verranno prese decisioni degne di nota.
Inoltre, sono state fissate delle date coincidenti con le aspettative legate allo stato di avanzamento del prodotto in una certa data. Più precisamente, i gradi di soddisfacimento dello stato di avanzamento sono divisi in \textit{ottimale} e \textit{accettabile}. Ci si aspetta che lo stato di avanzamento sia ottimale per il 07/06/20 e accettabile per il 10/06/20.
Per quanto riguarda la documentazione, è stato deciso che 2 membri del gruppo inizieranno la correzione degli errori segnalati in Revisione di Qualifica (RQ) a partire dal giorno 05/06/20, in modo da concluderli in anticipo (eccetto i manuali e le misurazioni delle metriche nel Piano di Qualifica) e poter poi dedicare poi tutte le energie per limare gli ultimi dettagli nei giorni rimanenti che precedono la fine del progetto.
Infine, è stata fissata una data per iniziare il lavoro in merito alla presentazione per la Revisione di Accettazione (RA). Come buona pratica assodata nelle precedenti revisioni, il gruppo inizierà a lavorarci congiuntamente esattamente una settimana prima della data di fine progetto, coincidente con l'RA\@.

\newpage
\section{Registro delle decisioni}%
\label{sec:registro_delle_decisioni}

\begin{description}
  \item[20200601-int-001] Il gruppo ha fissato le call di recap di metà incremento per i giorni 06/06/20 e 13/06/20.
  \item[20200601-int-002] Il gruppo ha deciso che lo stato di avanzamento globale del progetto sia ottimale per il giorno 07/06/20.
  \item[20200601-int-003] Il gruppo ha deciso che lo stato di avanzamento globale del progetto sia accettabile per il giorno 10/06/20.
  \item[20200601-int-004] Il gruppo inizierà a lavorare sulla presentazione della Revisione di Accettazione (RA) il giorno 11/06/20.
  \item[20200601-int-005] 2 membri del gruppo lavoreranno sulla documentazione a partire dal giorno 05/06/20.
\end{description}

% sec:registro_delle_decisioni (end)
\end{document}
