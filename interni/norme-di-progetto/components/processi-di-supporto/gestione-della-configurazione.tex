\documentclass[../../norme-di-progetto.tex]{subfiles}

\begin{document}

\subsubsection{Finalità}%
\label{subs:gestione_della_configurazione/finalita}

GruppOne istanzia il processo di gestione della configurazione per:
\begin{itemize}
  \item Assicurare completezza e correttezza del prodotto.
  \item Gestire al meglio il lavoro di gruppo.
  \item Identificare procedure per l'organizzazione, gestione e rilascio del prodotto.
  \item Conservare documentazione e software in un luogo accessibile e compatibile con la licenza richiesta dal committente.
\end{itemize}

\subsubsection{Descrizione}%
\label{subs:gestione_della_configurazione/descrizione}

Attraverso il processo di gestione della configurazione vengono tracciati lo stato e l'evoluzione delle singole componenti del prodotto, la relazione complessiva tra le parti e vengono documentate le richieste di modifiche.
Le attività coinvolte sono:

\begin{itemize}
  \item Implementazione del processo.
  \item Identificazione della configurazione.
  \item Controllo della configurazione.
  \item Resoconto dello stato della configurazione.
  \item Valutazione della configurazione.
  \item Gestione dei rilasci e della consegna.
\end{itemize}

\subsubsection{Attività}%
\label{subs:gestione_della_configurazione/attivita}
% un \paragraph per ogni attività

\paragraph{Implementazione del processo}%
\label{par:implementazione_del_processo}

Verrà sviluppato nella fase iniziale del progetto un piano di gestione della configurazione, espresso attraverso procedure documentate nel presente documento (§~\ref{subs:gestione_della_configurazione/procedure}).
I vari membri del gruppo saranno di volta in volta incaricati della gestione puntuale di aspetti specifici della configurazione del prodotto nel rispetto delle norme definite.

% par:implementazione_del_processo (end)

\paragraph{Identificazione della configurazione}%
\label{par:identificazione_della_configurazione}

% TODO scrivere sezione
% GruppOne si impegna a fornire esplicitamente la versione della documentazione che stabilisce univocamente per ogni componente la

% par:identificazione_della_configurazione (end)

\paragraph{Controllo della configurazione}%
\label{par:controllo_della_configurazione}

Verranno identificate e tracciate tutte le richieste di modifiche pervenute da fonti esterne (quali ad es.\ il committente) e interne, insieme all'evoluzione, implementazione e rilascio di tali modifiche.

% par:controllo_della_configurazione (end)

\paragraph{Resoconto dello stato della configurazione}%
\label{par:resoconto_dello_stato_della_configurazione}

Lo stato attuale della configurazione del prodotto è espresso attraverso un numero di versione di prodotto

% par:resoconto_dello_stato_della_configurazione (end)

\paragraph{Valutazione della configurazione}%
\label{par:valutazione_della_configurazione}

Nel corso del progetto verrà documentata all'interno del documento \textit{Piano di Qualifica
  (versione \versione)} la copertura dei requisiti elencati nel documento \textit{Analisi dei Requisiti (versione \versione)} rispetto all'implementazione effettiva dei componenti software richiesti.

% par:valutazione_della_configurazione (end)

\paragraph{Gestione dei rilasci e delle consegne}%
\label{par:gestione_dei_rilasci_e_delle_consegne}

GruppOne effettuerà con cadenza settimanale in corrispondenza degli incrementi definiti nel \textit{Piano di Progetto (versione \versione)} dei rilasci interni dei componenti del prodotto software in via di sviluppo, che andranno a costituire baseline incrementali. In occasione delle consegne di materiale richieste dal committente, le baseline più recenti dei componente insieme alla documentazione corrispondente verranno rese disponibili al committente, per concludersi in un rilascio finale del prodotto compiuto a

% par:gestione_dei_rilasci_e_della_consegna (end)

% \subsubsection{Metriche}%
% \label{subs:gestione_della_configurazione/metriche}

\subsubsection{Procedure}%
\label{subs:gestione_della_configurazione/procedure}

\paragraph{Tracciamento delle richieste di modifiche}%
\label{par:tracciamento_delle_richieste_di_modifiche}


% par:tracciamento_delle_richieste_di_modifiche (end)

\subsubsection{Strumenti di supporto}%
\label{subs:gestione_della_configurazione/strumenti_di_supporto}

\paragraph{Git}%
\label{par:git}

% par:git (end)

\paragraph{GitHub}%
\label{par:github}

% par:github (end)

\end{document}
