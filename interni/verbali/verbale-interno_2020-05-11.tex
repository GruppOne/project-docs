\documentclass{article}

\usepackage[italian]{babel}
\usepackage[margin=2cm, footskip=5mm]{geometry}
% questi package non sono necessari in lualatex; ref https://tex.stackexchange.com/a/413046
% \usepackage[utf8]{inputenc}
% \usepackage[T1]{fontenc}
\usepackage{enumitem}
\usepackage{hyperref}
\usepackage{titlesec}
\usepackage{soulutf8}
\usepackage{contour}
\usepackage{float}
\usepackage{graphicx}
\usepackage{fancyhdr}
\usepackage{longtable}
\usepackage[table]{xcolor}
\usepackage{titling}
\usepackage{lastpage}
\usepackage{ifthen}
\usepackage{calc}
\usepackage{minted}
\usepackage{pgfgantt}
\usepackage{subfiles}

\newlength{\imgwidth}

\newcommand\scalegraphics[1]{%
    \settowidth{\imgwidth}{\includegraphics{#1}}%
    \setlength{\imgwidth}{\minof{\imgwidth}{\textwidth}}%
    \includegraphics[width=\imgwidth]{#1}%
}

% XXX definizione dei percorsi in cui cercare immagini
\graphicspath{ {./}
    {./img/}
}

% esempio di utilizzo: \appendToGraphicspath{./img/} (un comando diverso per ogni path da includere)
% N.B.: ci DEVE essere un forward slash alla fine del path, a indicare che è una cartella.
\makeatletter
\newcommand\appendToGraphicspath[1]{%
  \g@addto@macro\Ginput@path{{#1}}%
}
\makeatother

% setup della sottolineatura
\setuldepth{Flat}
\contourlength{0.8pt}

\newcommand{\uline}[1]{%
  \ul{{\phantom{#1}}}%
  \llap{\contour{white}{#1}}%
}

% setup dei link
\hypersetup{
  colorlinks=true, % set true if you want colored links
  linktoc=all,     % set to all if you want both sections and subsections linked
  linkcolor=black, % choose some color if you want links to stand out
}

% setup di header e footer
\pagestyle{fancy}

\fancyhf{}
\fancyhead[L]{\includegraphics[width=1cm]{logo.png}}
\fancyhead[R]{\thetitle}
\fancyfoot[R]{\thepage\ di~\pageref{LastPage}}

\fancypagestyle{nopage}{%
  \fancyfoot{}%
}

\setlength{\headheight}{1.2cm}

% setup forma \paragraph e \subparagraph
\titleformat{\paragraph}[hang]{\normalfont\normalsize\bfseries}{\theparagraph}{1em}{}
\titleformat{\subparagraph}[hang]{\normalfont\normalsize\bfseries}{\thesubparagraph}{1em}{}

% setup profondità indice di default
\setcounter{secnumdepth}{5}
\setcounter{tocdepth}{5}

% shortcut per i placeholder
\newcommand{\plchold}[1]{\textit{\{#1\}}} % chktex 20

% hook per lo script che genera il glossario
\newcommand{\glossario}[1]{\underline{#1}\textsubscript{g}}

% definizione dei comandi \uso e \stato
\makeatletter
\newcommand{\setUso}[1]{%
  \newcommand{\@uso}{#1}%
}
\newcommand{\uso}{\@uso}

\newcommand{\setStato}[1]{%
  \newcommand{\@stato}{#1}%
}
\newcommand{\stato}{\@stato}

\newcommand{\setVersione}[1]{%
  \newcommand{\@versione}{#1}%
}
\newcommand{\versione}{\@versione}

\newcommand{\setResponsabile}[1]{%
  \newcommand{\@responsabile}{#1}%
}
\newcommand{\responsabile}{\@responsabile}

\newcommand{\setRedattori}[1]{%
  \newcommand{\@redattori}{#1}%
}
\newcommand{\redattori}{\@redattori}

\newcommand{\setVerificatori}[1]{%
  \newcommand{\@verificatori}{#1}%
}
\newcommand{\verificatori}{\@verificatori}

\newcommand{\setDescrizione}[1]{%
  \newcommand{\@descrizione}{#1}%
}
\newcommand{\descrizione}{\@descrizione}

\newcommand{\setModifiche}[1]{%
  \newcommand{\@modifiche}{#1}%
}

\newcommand{\modifiche}{\@modifiche}

\makeatother

% setup delle description
\setlist[description,1]{font=$\bullet$\hspace{1.5mm}, leftmargin=*,labelindent=12.5mm}
\setlist[description,2]{font=$\bullet$\hspace{1.5mm}, leftmargin=*,labelindent=12.5mm}

\appendToGraphicspath{../../commons/img/}

\title{Verbale interno --- 11/05/2020}

\setResponsabile{Fabio Scettro}
\setRedattori{Alberto Gobbo}
\setVerificatori{
  Alberto Cocco
}
\setUso{Interno}
\setDescrizione{Verbale dell'incontro di GruppOne del 04/05/2020}
\setModifiche{%
\cellcolor{white!80!lightgray!100} & Fabio Scettro & 2020--05--12 & approva documento \\% cambiare responsabile se errato
\cellcolor{white!80!lightgray!100} & Alberto Cocco & 2020--05--11 & verifica verbale \\%
\multirow{-3}{*}{-} & Alberto Gobbo & 2020--05--11 & stendi verbale%
}

\disabilitaVersione{}
\disabilitaElencoFigure{}
\disabilitaElencoTabelle{}

\begin{document}

\thispagestyle{empty}
\pagenumbering{gobble}
\begin{center}
	\includegraphics[width=8.5cm]{\commons/img/logo.png}\\
	{\Large GruppOne - progetto "Stalker"}\\
	\vspace{1.5cm}
	{\Huge \thetitle}
	\vspace{1.5cm}
	\begin{table}[H]
		\centering
		\begin{tabular}{r|l}
			\textbf{Versione}&\versione\\
			\textbf{Approvazione}&\responsabile\\
			\textbf{Redazione}&\redattori\\
			\textbf{Verifica}&\verificatori\\
			\textbf{Stato}&\stato\\
			\textbf{Uso}&\uso\\
			\textbf{Destinato a}&Imola Informatica\\
			&GruppOne\\
			&Prof. Tullio Vardanega\\
			&Prof. Riccardo Cardin\\
		\end{tabular}
	\end{table}
	\vspace{3cm}
	\textbf{Descrizione}\\
	\descrizione\\
	\vfill
	\verb|gruppone.swe@gmail.com|
\end{center}
\newpage
\thispagestyle{nopage}
\section*{Registro delle modifiche}
\label{sec:registro_delle_modifiche}
\begin{table}[H]
	\label{tab:registro_delle_modifiche}
	\centering
	\rowcolors{2}{lightgray}{white!80!lightgray!100}
	\begin{longtable}[c]{c c c c l}
		\rowcolor{darkgray!90!}\color{white}{\textbf{Versione}}&\color{white}{\textbf{Data}}&\color{white}{\textbf{Nominativo}}&\color{white}{\textbf{Ruolo}}&\color{white}{\textbf{Descrizione}}\\\endhead
	\end{longtable}
\end{table}
% section registro_delle_modifiche (end)
\newpage
\thispagestyle{nopage}
\pagenumbering{roman}
\tableofcontents
\newpage
\pagenumbering{arabic}

\section{Informazioni logistiche}%
\label{sec:informazioni_logistiche}

\begin{description}
  \item [Luogo] videochiamata su Google Hangouts
  \item [Data] 11/05/2020
  \item [Ora] 18:30 \symbol{8594} 20:00
\end{description}

\subsection{Membri del gruppo presenti}%
\label{sub:membri_del_gruppo_presenti}

\begin{enumerate}
  \item Riccardo Agatea
  \item Tobia Apolloni
  \item Riccardo Cestaro
  \item Alberto Cocco
  \item Luca Ercole
  \item Alberto Gobbo
  \item Alessandro Rizzo
  \item Fabio Scettro
\end{enumerate}
% sub:membri_del_gruppo_presenti (end)

\section{Introduzione}%
\label{sec:introduzione}
L'incontro è avvenuto tramite chiamata Hangouts.
Lo scopo principale era chiarire alcune divergenze tra alcuni membri del gruppo e pianificare il lavoro da svolgere nella settimana in preparazione alla Revisione di Qualifica (RQ).
In base alle decisioni prese il giorno 04/05/2020 riguardo alla pianificazione del lavoro dell'incremento appena concluso, non è stato fatto il consueto aggiornamento del foglio dei requisiti.

\section{Ordine del giorno}%
\label{sec:ordine_del_giorno}

\begin{itemize}
  \item Chiarimento sulle divergenze emerse tra membri del team
  \item Pianificazione lavoro settimanale
\end{itemize}

\section{Chiarimento sulle divergenze emerse tra membri del team}%
\label{sec:chiarimento_sulle_divergenze_emerse_tra_membri_del_team}

Durante il periodo dell'incremento appena concluso, alcuni membri del team hanno avuto delle divergenze riguardo alle modalità di lavoro e su alcuni comportamenti errati derivati da alcune incomprensioni.
In un clima disteso, i membri coinvolti hanno avviato una discussione sui problemi riscontrati, coinvolgendo tutti i membri del gruppo in modo da raccogliere opinioni e soluzioni in merito alle problematiche discusse.
Il principale problema è derivato dalle best practice non assodate alla perfezione dai redattori di alcuni documenti che mostravano diverse lacune tipografiche e ripetute all'interno dei documenti. É risultato che l'addetto alla verifica ha utilizzato toni accesi, mal interpretati dai redattori che non hanno digerito il modo in cui sono stati fatti gli accorgimenti.
Alla fine della discussione si è arrivati ad un accordo comune, dichiarando che i redattori coinvolti si impegnano a non commettere più gli errori che sono stati segnalati, e che il verificatore deve utilizzare un tono professionale e mai al di fuori delle righe, esprimendo un punto di vista oggettivo e mai personale.

\section{Pianificazione lavoro settimanale}%
\label{sec:pianificazione_lavoro_settimanale}

La settimana attuale porta in seno tutte le attività prospettate nel quinto incremento, e principalmente serve a soddisfare alcuni requisiti arretrati per poter disporre di una demo il più possibile soddisfacente per il giorno 18/05/2020 della Revisione di Qualifica.
Grazie al fatto che la web application è in linea con le aspettative ed è la componente più vicina alla sua conclusione, il gruppo ha deciso di spostare tutte le risorse umane sulle componenti server e mobile application, in particolare:
\begin{itemize}
  \item 5 componenti lato server per l'implementazione definitiva degli endpoint ed annessi test.
  \item 3 componenti lato mobile application per l'implementazione dell'impianto grafico delle pagine ed i test.
\end{itemize}
Inoltre, va aggiunto che il gruppo si deve impegnare per la presentazione della Revisione di Qualifica, quindi lavorerà su questa parte nelle serate che precederanno questa data.
Va segnalata che la disponibilità oraria di quattro degli otto componenti del gruppo sarà dimezzata fino all'18/05/2020, in quanto parteciperanno al secondo appello di SWE e quindi impegnati con lo studio personale.

\newpage
\section{Registro delle decisioni}%
\label{sec:registro_delle_decisioni}

\begin{description}
  \item[20200511-int-001] 5 componenti lavoreranno lato server per completare l'implementazione degli endpoint ed annessi test.
  \item[20200511-int-002] 3 componenti lavoreranno lato mobile application per l'implementazione dell'impianto grafico delle pagine ed i test.
  \item[20200511-int-003] La sera del giorno 12/05/2020, tutto il gruppo si riunirà per decidere come impostare la presentazione dell'RQ\@.
  \item[20200511-int-004] Nei giorni 15/05/2020 e 16/05/2020, tutto il gruppo si riunirà per le prove generali per la presentazione dell'RQ\@. Come prerequisito, la presentazione deve essere pronta prima del 15/05/2020.
\end{description}

% sec:registro_delle_decisioni (end)
\end{document}
