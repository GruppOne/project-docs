\documentclass{article}

\usepackage[italian]{babel}
\usepackage[margin=2cm, footskip=5mm]{geometry}
% questi package non sono necessari in lualatex; ref https://tex.stackexchange.com/a/413046
% \usepackage[utf8]{inputenc}
% \usepackage[T1]{fontenc}
\usepackage{enumitem}
\usepackage{hyperref}
\usepackage{titlesec}
\usepackage{soulutf8}
\usepackage{contour}
\usepackage{float}
\usepackage{graphicx}
\usepackage{fancyhdr}
\usepackage{longtable}
\usepackage[table]{xcolor}
\usepackage{titling}
\usepackage{lastpage}
\usepackage{ifthen}
\usepackage{calc}
\usepackage{minted}
\usepackage{pgfgantt}
\usepackage{subfiles}

\newlength{\imgwidth}

\newcommand\scalegraphics[1]{%
    \settowidth{\imgwidth}{\includegraphics{#1}}%
    \setlength{\imgwidth}{\minof{\imgwidth}{\textwidth}}%
    \includegraphics[width=\imgwidth]{#1}%
}

% XXX definizione dei percorsi in cui cercare immagini
\graphicspath{ {./}
    {./img/}
}

% esempio di utilizzo: \appendToGraphicspath{./img/} (un comando diverso per ogni path da includere)
% N.B.: ci DEVE essere un forward slash alla fine del path, a indicare che è una cartella.
\makeatletter
\newcommand\appendToGraphicspath[1]{%
  \g@addto@macro\Ginput@path{{#1}}%
}
\makeatother

% setup della sottolineatura
\setuldepth{Flat}
\contourlength{0.8pt}

\newcommand{\uline}[1]{%
  \ul{{\phantom{#1}}}%
  \llap{\contour{white}{#1}}%
}

% setup dei link
\hypersetup{
  colorlinks=true, % set true if you want colored links
  linktoc=all,     % set to all if you want both sections and subsections linked
  linkcolor=black, % choose some color if you want links to stand out
}

% setup di header e footer
\pagestyle{fancy}

\fancyhf{}
\fancyhead[L]{\includegraphics[width=1cm]{logo.png}}
\fancyhead[R]{\thetitle}
\fancyfoot[R]{\thepage\ di~\pageref{LastPage}}

\fancypagestyle{nopage}{%
  \fancyfoot{}%
}

\setlength{\headheight}{1.2cm}

% setup forma \paragraph e \subparagraph
\titleformat{\paragraph}[hang]{\normalfont\normalsize\bfseries}{\theparagraph}{1em}{}
\titleformat{\subparagraph}[hang]{\normalfont\normalsize\bfseries}{\thesubparagraph}{1em}{}

% setup profondità indice di default
\setcounter{secnumdepth}{5}
\setcounter{tocdepth}{5}

% shortcut per i placeholder
\newcommand{\plchold}[1]{\textit{\{#1\}}} % chktex 20

% hook per lo script che genera il glossario
\newcommand{\glossario}[1]{\underline{#1}\textsubscript{g}}

% definizione dei comandi \uso e \stato
\makeatletter
\newcommand{\setUso}[1]{%
  \newcommand{\@uso}{#1}%
}
\newcommand{\uso}{\@uso}

\newcommand{\setStato}[1]{%
  \newcommand{\@stato}{#1}%
}
\newcommand{\stato}{\@stato}

\newcommand{\setVersione}[1]{%
  \newcommand{\@versione}{#1}%
}
\newcommand{\versione}{\@versione}

\newcommand{\setResponsabile}[1]{%
  \newcommand{\@responsabile}{#1}%
}
\newcommand{\responsabile}{\@responsabile}

\newcommand{\setRedattori}[1]{%
  \newcommand{\@redattori}{#1}%
}
\newcommand{\redattori}{\@redattori}

\newcommand{\setVerificatori}[1]{%
  \newcommand{\@verificatori}{#1}%
}
\newcommand{\verificatori}{\@verificatori}

\newcommand{\setDescrizione}[1]{%
  \newcommand{\@descrizione}{#1}%
}
\newcommand{\descrizione}{\@descrizione}

\newcommand{\setModifiche}[1]{%
  \newcommand{\@modifiche}{#1}%
}

\newcommand{\modifiche}{\@modifiche}

\makeatother

% setup delle description
\setlist[description,1]{font=$\bullet$\hspace{1.5mm}, leftmargin=*,labelindent=12.5mm}
\setlist[description,2]{font=$\bullet$\hspace{1.5mm}, leftmargin=*,labelindent=12.5mm}

\appendToGraphicspath{../../commons/img/}

\title{Verbale interno --- 04/05/2020}

\setResponsabile{Nome Cognome}
\setRedattori{Alberto Gobbo}
\setVerificatori{
  Nome Cognome
}
\setUso{Interno}
\setDescrizione{Verbale dell'incontro di GruppOne del 04/05/2020}
\setModifiche{%
\cellcolor{white!80!lightgray!100} & Nome Cognome & 2020--05--05 & approva documento \\%
\cellcolor{white!80!lightgray!100} & Nome Cognome & 2020--05--04 & verifica verbale \\%
\multirow{-3}{*}{-} & Alberto Gobbo & 2020--05--04 & stendi verbale%
}

\disabilitaVersione{}
\disabilitaElencoFigure{}
\disabilitaElencoTabelle{}

\begin{document}

\thispagestyle{empty}
\pagenumbering{gobble}
\begin{center}
	\includegraphics[width=8.5cm]{\commons/img/logo.png}\\
	{\Large GruppOne - progetto "Stalker"}\\
	\vspace{1.5cm}
	{\Huge \thetitle}
	\vspace{1.5cm}
	\begin{table}[H]
		\centering
		\begin{tabular}{r|l}
			\textbf{Versione}&\versione\\
			\textbf{Approvazione}&\responsabile\\
			\textbf{Redazione}&\redattori\\
			\textbf{Verifica}&\verificatori\\
			\textbf{Stato}&\stato\\
			\textbf{Uso}&\uso\\
			\textbf{Destinato a}&Imola Informatica\\
			&GruppOne\\
			&Prof. Tullio Vardanega\\
			&Prof. Riccardo Cardin\\
		\end{tabular}
	\end{table}
	\vspace{3cm}
	\textbf{Descrizione}\\
	\descrizione\\
	\vfill
	\verb|gruppone.swe@gmail.com|
\end{center}
\newpage
\thispagestyle{nopage}
\section*{Registro delle modifiche}
\label{sec:registro_delle_modifiche}
\begin{table}[H]
	\label{tab:registro_delle_modifiche}
	\centering
	\rowcolors{2}{lightgray}{white!80!lightgray!100}
	\begin{longtable}[c]{c c c c l}
		\rowcolor{darkgray!90!}\color{white}{\textbf{Versione}}&\color{white}{\textbf{Data}}&\color{white}{\textbf{Nominativo}}&\color{white}{\textbf{Ruolo}}&\color{white}{\textbf{Descrizione}}\\\endhead
	\end{longtable}
\end{table}
% section registro_delle_modifiche (end)
\newpage
\thispagestyle{nopage}
\pagenumbering{roman}
\tableofcontents
\newpage
\pagenumbering{arabic}

\section{Informazioni logistiche}%
\label{sec:informazioni_logistiche}

\begin{description}
  \item [Luogo] videochiamata su Google Hangouts
  \item [Data] 04/05/2020
  \item [Ora] 18:00 \symbol{8594} 19:30
\end{description}

\subsection{Membri del gruppo presenti}%
\label{sub:membri_del_gruppo_presenti}

\begin{enumerate}
  \item Riccardo Agatea
  \item Riccardo Cestaro
  \item Alberto Cocco
  \item Luca Ercole
  \item Alberto Gobbo
  \item Alessandro Rizzo
\end{enumerate}
% sub:membri_del_gruppo_presenti (end)

\section{Ordine del giorno}%
\label{sec:ordine_del_giorno}

\begin{itemize}
  \item Pianificazione lavoro settimanale
  \item Aggiornamento stato requisiti per componente
\end{itemize}

\section{Pianificazione lavoro settimanale}%
\label{sec:pianificazione_lavoro_settimanale}

In vista delle imminenti scadenze della Product Baseline (PB) di venerdì 08/08/2020 e della consegna dei documenti per la Revisione di Qualifica (RA) di lunedì 11 maggio 2020, il gruppo ha discusso riguardo alla suddivisione del lavoro settimanale.

A causa di queste scadenze, il gruppo è costretto a riprogrammare il lavoro, lasciando da parte almeno fino a lunedì 11 maggio 2020 la parte relativa allo sviluppo delle tre componenti software, e quindi a rivalutare il lavoro del quinto incremento alla settimana successiva (dal 12/05/2020 al 19/05/2020).

\section{Aggiornamento stato requisiti per componente}%
\label{sec:aggiornamento_stato_requisiti_per_componente}

Come stabilito dalla riunione interna di lunedì 13/04/2020, il gruppo ha svolto un aggiornamento dei requisiti non ancora implementati, WIP (Work In Progress) o soddisfatti.

Una novità apportata nel Google Docs di riferimento, è l'aggiunta di un nuova scheda denominata \textit{Responsabili implementazione endpoint}, che definiscono tutti gli endpoint definiti dalla specifica API, e mostrano informazioni riguardo alla persona che è stata assegnata per implementare un particolare endpoint e al suo stato di implementazione.

Al momento della scrittura, gli endpoint implementati e quindi funzionanti sono 5, ma è ancora carente l'implementazione della componente server, che non è ancora in linea con la pianificazione.

Per quanto riguarda la web application, molti requisiti sono stati soddisfatti con successo ed è in linea con la pianificazione, mentre la mobile application mostra un leggero ritardo rispetto a quanto pianificato.

Per monitorare la situazione attuale, visitare il seguente \href{https://docs.google.com/spreadsheets/d/1J-RbNrb1yN_X1rVlKzpP9rjaiGst8B62k1K6EHY-grU/edit#gid=0}{Google Docs} condiviso con i membri del gruppo su Google Drive.

\newpage
\section{Registro delle decisioni}%
\label{sec:registro_delle_decisioni}

\begin{enumerate}
  \item È stato spostato il quinto incremento alla settimana successiva alla consegna dei documenti.
  \item Dal 05/05/2020 al 06/05/2020, tutto il gruppo si concentra a definire tutti i diagrammi delle classi e di sequenza.
  \item Il giorno 07/05/2020, la maggior parte del gruppo si concentra sulla presentazione della PB, mentre i restanti membri del gruppo si concentrano sullo svolgimento della documentazione.
  \item Dal 08/05/2020 al 11/05/2020, tutto il gruppo si concentra sullo svolgimento della documentazione. 
\end{enumerate}

% sec:registro_delle_decisioni (end)
\end{document}