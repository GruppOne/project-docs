\documentclass[../../../analisi-dei-requisiti.tex]{subfiles}

\begin{document}

\subsubsection{UUC12: Passaggio a noto}%
\label{subs:UUC12}

\begin{figure}[H]
  \centering
  \includegraphics[width=150mm]{passaggio-noto.png}
  \caption{UUC12: Passaggio a noto}%
  \label{fig:UUC12}
\end{figure}

\begin{description}
  \item[Caso d'uso:] UUC12;
  \item[Titolo:] Passaggio a noto;
  \item[Attori primari:] utente collegato, in particolare utente incognito;
  \item[Precondizione:] l'utente deve essere collegato ed incognito;
  \item[Postcondizione:] l'utente è collegato e noto;
  \item[Scenario Principale:]
        \begin{enumerate}
          \item l'utente è incognito in un'organizzazione e vuole diventare noto: in questo modo, sia la sua presenza che la sua identità sono note.
        \end{enumerate}
  \item[Estensioni:]
        \begin{enumerate}
          \item in caso di rete mancante, non possono essere eseguite queste operazioni e quindi verrà notificato un errore~\ref{subs:UUC18};
        \end{enumerate}
\end{description}




\subsubsection{UUC13: Passaggio a incognito}%
\label{subs:UUC13}

\begin{figure}[H]
  \centering
  \includegraphics[width=150mm]{passaggio-incognito.png}
  \caption{UUC13: Passaggio a incognito}%
  \label{fig:UUC13}
\end{figure}

\begin{description}
  \item[Caso d'uso:] UUC13;
  \item[Titolo:] Passaggio a incognito;
  \item[Attori primari:] utente collegato, in particolare utente noto;
  \item[Precondizione:] l'utente deve essere collegato e noto;
  \item[Postcondizione:] l'utente è collegato ed incognito;
  \item[Scenario Principale:]
        \begin{enumerate}
          \item l'utente è noto in un'organizzazione e vuole diventare incognito: in questo modo la sua presenza è nota, ma non lo è la sua identità.
        \end{enumerate}
  \item[Estensioni:]
        \begin{enumerate}
          \item in caso di rete mancante, non possono essere eseguite queste operazioni e quindi verrà notificato un errore~\ref{subs:UUC18};
        \end{enumerate}
\end{description}






\end{document}
