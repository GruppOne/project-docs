\documentclass[../../../analisi-dei-requisiti.tex]{subfiles}

\begin{document}

\begin{figure}[H]
  \centering
  \begin{plantuml}
  @startuml
  !include ../../commons/style/use-cases.pu

  actor :amministratore: as A

  rectangle {
    together {
      usecase (AUC4.1) as "AUC4.1\nRichiesta accettata"
      usecase (AUC4.2) as "AUC4.2\nRichiesta rifiutata"
    }
    usecase (AUC4.3) as "AUC4.3\nInvio risposta all'utente"
  }

  :A: -- AUC4.1
  :A: -- AUC4.2

  (AUC4.3) .left.|> (AUC4.1) : <<include>>
  (AUC4.3) .up.|> (AUC4.2) : <<include>>

  @enduml
  \end{plantuml}
  \caption{AUC4: Creazione owner}%
  \label{fig:AUC4}
\end{figure}

\begin{description}
  \item[Codice:] AUC4;
  \item[Titolo:] Creazione owner;
  \item[Attori primari:] amministratore;
  \item[Precondizione:] il sistema deve rendere disponibile la pagina di creazione owner, che deve contenere:
  \begin{itemize}
    \item la possibilità di creare un owner;
    \item le richieste effettuate da utenti autenticati da poter gestire.
  \end{itemize}
  \item[Postcondizione:] l'owner viene creato.
  \item[Scenario principale:]
  \begin{enumerate}
    \item l'amministratore gestisce la creazione degli owner.
  \end{enumerate}
\end{description}


\subsubsection{AUC4.1: Richiesta accettata}%
\label{subs:AUC4.1}

\begin{description}
  \item[Codice:] AUC4.1;
  \item[Titolo:] Richiesta accettata;
  \item[Attori primari:] amministratore;
  \item[Precondizione:] l'amministratore verifica la richiesta;
  \item[Postcondizione:] l'amministratore accetta la richiesta, e il nuovo owner viene creato;
  \item[Scenario principale:]
  \begin{enumerate}
    \item L'amministratore esamina la richiesta ricevuta e la accetta. L'utente che ha fatto richiesta diventa owner.
  \end{enumerate}
  \item[Inclusioni:]
  \begin{enumerate}
    \item l'accettazione della richiesta viene notificata all'utente \emph{[AUC4.3]}.
  \end{enumerate}
\end{description}

\subsubsection{AUC4.2: Richiesta rifiutata}%
\label{subs:AUC4.2}

\begin{description}
  \item[Codice:] AUC4.2;
  \item[Titolo:] Richiesta rifiutata;
  \item[Attori primari:] amministratore;
  \item[Precondizione:] l'amministratore verifica la richiesta;
  \item[Postcondizione:] l'amministratore rifiuta la richiesta, l'utente non diventerà owner;
  \item[Scenario principale:]
  \begin{enumerate}
    \item L'amministratore esamina la richiesta ricevuta e la rifiuta. L'utente che ha fatto richiesta non può diventare owner.
  \end{enumerate}
  \item[Inclusioni:]
  \begin{enumerate}
    \item il rifiuto della richiesta viene notificata all'utente \emph{[AUC4.3]}.
    \item
  \end{enumerate}
\end{description}

\subsubsection{AUC4.3: Invio risposta all'utente}%
\label{subs:AUC4.3}

\begin{description}
  \item[Codice:] AUC4.3;
  \item[Titolo:] Invio risposta all'utente;
  \item[Attori primari:] amministratore;
  \item[Precondizione:] l'amministratore ha accettato, oppure rifiutato, la richiesta di un utente per diventare owner;
  \item[Postcondizione:] l'utente che ha fatto la richiesta riceve l'esito tramite un messaggio di risposta;
  \item[Scenario principale:]
  \begin{enumerate}
    \item una volta esaminata la richiesta, la risposta dell'amministratore verrà notificata all'utente.
  \end{enumerate}
\end{description}


\subsubsection{AUC4.4: Creazione istantanea owner}%
\label{subs:AUC4.4}

\begin{description}
  \item[Codice:] AUC4.4;
  \item[Titolo:] Creazione istantanea owner;
  \item[Attori primari:] amministratore;
  \item[Precondizione:] il sistema deve rendere disponibile il campo di creazione istantanea;
  \item[Postcondizione:] l'utente indicato diventa owner istantaneamente;
  \item[Scenario principale:]
  \begin{enumerate}
    \item l'amministratore crea un owner, senza che abbia fatto effettivamente richiesta.
  \end{enumerate}
\end{description}

\end{document}
